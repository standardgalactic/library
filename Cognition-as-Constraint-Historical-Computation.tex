\documentclass[11pt]{article}

% ============================================================
% Encoding and Fonts
% ============================================================

\usepackage[T1]{fontenc}
\usepackage[utf8]{inputenc}
\usepackage{lmodern}

% ============================================================
% Page Geometry and Spacing
% ============================================================

\usepackage{geometry}
\geometry{margin=1in}

\usepackage{setspace}
\setstretch{1.15}

% ============================================================
% Mathematics and Symbols
% ============================================================

\usepackage{amsmath,amssymb,amsthm,mathtools}

% ============================================================
% Quotations and References
% ============================================================

\usepackage{csquotes}
\usepackage{hyperref}

\hypersetup{
  colorlinks=true,
  linkcolor=black,
  citecolor=black,
  urlcolor=black
}

% ============================================================
% Tables and Figures (minimal)
% ============================================================

\usepackage{booktabs}

% ============================================================
% Theorem Environments
% ============================================================

\theoremstyle{definition}
\newtheorem{definition}{Definition}

\theoremstyle{plain}
\newtheorem{assumption}{Assumption}
\newtheorem{lemma}{Lemma}
\newtheorem{proposition}{Proposition}
\newtheorem{theorem}{Theorem}

\theoremstyle{remark}
\newtheorem{remark}{Remark}

% ============================================================
% Custom Commands
% ============================================================

\newcommand{\Hist}{\mathsf{Hist}}
\newcommand{\Fut}{\mathsf{Fut}}
\newcommand{\Ev}{\mathsf{Ev}}
\newcommand{\Viab}{\mathrm{Viab}}

% ============================================================
% Title Page
% ============================================================

\title{
Cognition as Constraint-Historical Computation:\\
\large A Formalization of the Info--Computational Framework
}

\author{
Flyxion\\
\small Independent Researcher
}

\date{December 2025}

\begin{document}

\maketitle



\begin{abstract}
The Info--Computational (ICON) framework proposes that information, computation,
and cognition form a continuous natural process grounded in the physical world.
Information is understood as physically instantiated structure, computation as
the lawful transformation of that structure, and cognition as the adaptive
regulation of interaction characteristic of living systems. While ICON offers a
powerful unifying perspective, its core commitments are often presented at a
philosophical level, leaving their formal implications underdeveloped.

This paper provides a rigorous formalization of ICON by integrating
field-theoretic dynamics with an event-historical ontology of irreversible
constraint acquisition. Information is recast as constrained optionality over
admissible futures, computation as the irreversible application of events that
reshape possibility spaces, and morphology as sedimented constraint performing
computational work through physical form. Cognition is defined as the
history-dependent preservation of nonempty viable futures under energetic and
organizational limits.

Using category-theoretic presheaves, natural transformations, and hybrid
dynamical systems, the framework captures learning, development, and commitment
as regime-changing events that rewrite the space of admissible histories. A
worked reconstruction of bacterial chemotaxis demonstrates that minimal
biological cognition can be expressed as event-historical constraint management
without appeal to representation or optimization. The paper further contrasts
this approach with state-based variational models, highlighting the necessity
of explicit regime change for modeling life and cognition.

By integrating ICON with the Spherepop calculus, the paper shows how
informational and computational naturalism can be made formally operational.
The result is a unified account of cognition as constraint-historical
computation, applicable across biological, artificial, and social systems, and
grounded in the irreversible construction of worlds.
\end{abstract}

% ============================================================
\section{Introduction}
% ============================================================

The scientific study of cognition remains fragmented across multiple, often
incompatible explanatory frameworks. Classical computationalism models cognition
as symbolic manipulation; dynamical systems approaches emphasize continuous
state evolution; variational and Bayesian frameworks recast cognition as
inference or optimization; and enactive and embodied theories stress the role of
physical interaction and lived experience. While each of these perspectives has
yielded important insights, none has yet provided a fully satisfactory account
of how cognition arises as a natural process continuous with physics, biology,
and social organization.

A central difficulty lies in the treatment of information and computation.
Traditional cognitive models often assume abstract informational states and
algorithmic computation as primitive, importing mathematical formalisms whose
ontological status remains unclear when applied to living systems. As a result,
cognition is frequently modeled as a special-purpose process layered atop the
physical world, rather than as a phenomenon emerging from the same material and
organizational principles that govern matter and life.

The Info--Computational (ICON) framework, developed by Dodig--Crnkovic, offers a
systematic attempt to resolve this tension by naturalizing both information and
computation. ICON treats information as physically instantiated structure and
computation as the lawful dynamics of that structure, extending the notion of
computation beyond symbolic algorithms to encompass biological, chemical, and
social processes. Cognition, on this view, is not an abstract representational
faculty but a mode of organized interaction characteristic of living systems.

Despite its unifying ambition, ICON is often presented at a level of
philosophical generality that leaves open questions about formal structure,
mechanism, and testability. In particular, the framework requires a precise
account of how informational structure constrains future behavior, how
computational processes accumulate historically, and how learning and
development can irreversibly alter the space of possible actions. Without such
an account, ICON risks remaining an interpretive stance rather than a
computational or dynamical theory.

This paper addresses that gap by providing a formalization of ICON grounded in
event-historical dynamics and constraint-based computation. Rather than modeling
cognition as state evolution within a fixed space, we treat histories of
irreversible events as primary and define information in terms of admissible
futures conditioned on those histories. Computation is identified with the
physical processes that transform these future spaces, while morphology is
understood as accumulated constraint that performs computational work by
restricting possibility.

The formalism integrates continuous field-theoretic dynamics with discrete
regime-changing events, yielding a hybrid ontology capable of representing
development, learning, and commitment as genuine structural transformations.
Category-theoretic tools, including presheaves and natural transformations, are
used to make explicit the relational and invariant aspects of constraint, while
viability theory provides a criterion for cognition grounded in continued
existence rather than optimization or representational accuracy.

The paper further situates this approach relative to prominent contemporary
frameworks. In contrast to classical computationalism, it rejects symbolic
representation as a foundational explanatory primitive. In contrast to purely
dynamical models, it emphasizes irreversibility and historical constraint. In
contrast to variational and Bayesian approaches, including the Free Energy
Principle, it allows the constraint language itself to change through discrete
events, rather than confining adaptation to trajectories within a fixed model
class. At the same time, it aligns with enactive and embodied approaches by
treating cognition as inseparable from physical form and environmental coupling,
while providing a formal apparatus often absent from those traditions.

To demonstrate the concreteness of the framework, the paper integrates ICON with
the Spherepop calculus, an event-sourced formal system in which objects are
defined by their histories of irreversible commitments. Within this setting,
information appears as constrained optionality, computation as irreversible
event application, and cognition as the maintenance of viable futures. A worked
example based on bacterial chemotaxis illustrates how minimal cognition can be
captured without appeal to representation or inference.

Taken together, these developments aim to reposition cognition as a natural,
history-dependent process of constraint construction. By making the
commitments of Info--Computationalism formally explicit, the paper seeks to
provide a foundation for studying cognition, life, and intelligence within a
single, coherent theoretical framework.


% ============================================================
\section{The Info-Computational (ICON) Framework}
% ============================================================

The \emph{Info-Computational (ICON)} framework, developed by \textbf{0}, is a unifying relational perspective for understanding natural phenomena, including living organisms and cognition, through the joint lenses of information and computation. ICON aims to naturalize cognition by situating it within the same physical processes that govern matter, life, and organization, thereby rejecting any principled separation between cognitive science and the natural sciences.

At its core, ICON integrates two complementary theses: \emph{(pan)informationalism} and \emph{(pan)computationalism}. Informationalism characterizes the physical universe as an informational structure, while computationalism interprets the evolution of that structure as computation. These commitments are not metaphysical claims about abstract entities, but methodological constraints on explanation: to explain a phenomenon is to describe its informational organization and the computational processes by which that organization changes.

\subsection{Information as Physical Structure}

Within ICON, information is identified with physically instantiated structure. Information is not symbolic content, semantic representation, or abstract data; rather, it consists in the organization, relations, and constraints of material systems. Consequently, information is inherently relational and observer-dependent, where an observer is itself a physical system embedded within the same informational universe it observes.

Differences count as information only insofar as they are physically realized and causally efficacious for some interacting system. This stance rejects disembodied or purely syntactic notions of information, grounding informational description in material embodiment and interaction.

\subsection{Computation as Natural Dynamics}

Computation, in the ICON framework, is understood as the lawful transformation of informational structures over time. This notion of computation is deliberately broader than classical Turing computation. It encompasses chemical reactions, biological regulation, morphogenesis, neural dynamics, and social interaction as genuine computational processes.

Natural computation is typically distributed rather than centralized, asynchronous rather than clocked, and interaction-driven rather than algorithmically prescribed. Under ICON, computation is not imposed upon nature by an external interpreter; it is a process continuously performed by physical systems themselves.

\subsection{Morphological Computing as the Mediating Mechanism}

A central role within ICON is played by \emph{morphological computing}. Morphological computing refers to the way in which an agent’s physical form—its geometry, topology, material properties, and dynamical constraints—actively contributes to computational processes. The body is not treated as a passive carrier of control algorithms, but as an active participant in computation.

Through its morphology, an agent can filter inputs, stabilize dynamics, offload control, and simplify behavioral coordination. Tasks that would otherwise require complex centralized processing can be partially or fully resolved by the physical structure of the system itself. Morphological computing thus provides the concrete mechanism that links informational structure with computational dynamics, grounding computation firmly in embodiment.

\subsection{Cognition as a Life Process}

ICON adopts the strong naturalization principle that cognition is coextensive with life. Cognition is not defined primarily in terms of representation, language, or abstract reasoning, but in terms of a system’s capacity for self-maintenance, environmental coupling, and adaptive regulation.

From this perspective, cognition consists of networks of natural computational processes operating on informational structures. These processes occur across all levels of biological organization, from single cells to complex nervous systems, and extend naturally to collective and social systems. Cognition is therefore embodied, processual rather than state-based, and irreducibly historical.

\subsection{Relational Perspective and Embodiment}

A defining commitment of ICON is its relational epistemology. Information and computation are always defined relative to an interacting system; there is no privileged, observer-independent description of informational content. Informational relevance arises from histories of interaction and coupling between systems, each with its own morphology and constraints.

This relational stance avoids both naïve realism and subjective idealism by grounding perspective in physical embodiment. Embodiment is not an optional feature of cognition, but a necessary condition for informational access and computational activity.

\subsection{Unconventional and Distributed Computation}

ICON aligns naturally with unconventional and distributed models of computation. In such models, information processing emerges from interaction and message exchange among components of a system, rather than from centralized symbolic manipulation. Distributed agent-based and interaction-oriented computational paradigms provide a closer match to biological, ecological, and social systems, where cognition arises from coordination and coupling rather than from global control.

\subsection{Hierarchical Organization Across Scales}

Morphological and natural computation operate across multiple scales of organization. Molecular self-assembly, cellular regulation, neural organization, organismal behavior, and social coordination all instantiate the same informational-computational principles, albeit with different morphologies and constraints. ICON is thus hierarchical without being reductionist: higher-level cognitive phenomena are not eliminated in favor of lower-level physics, but understood as organizationally emergent regimes grounded in physical computation.

\subsection{Summary}

Within the ICON framework, information, computation, and cognition form a single explanatory continuum. Information is physical structure; computation is physical change; cognition is the embodied, history-dependent regulation of interaction. By integrating these commitments, ICON provides a coherent framework for studying intelligence and cognition as natural physical processes, continuous with life itself and applicable to both biological organisms and artificial autonomous agents.

% ============================================================
\section{Formalizing the ICON Framework: Field-Theoretic and Event-Historical Commitments}
% ============================================================

This section presents a formalization of the Info-Computational (ICON) framework
in terms of a hybrid ontology that combines continuous field-theoretic structure
with discrete, irreversible event histories. The aim is to make explicit the
ontological commitments implicit in ICON—physical information, natural
computation, morphological embodiment, and cognition as life-process—while
providing a mathematical language suitable for analysis, simulation, and
comparison with contemporary dynamical and control-theoretic models.

\subsection{Field-Theoretic Ontology of Information and Computation}

Let $M$ denote spacetime (or an environment manifold), and let $\Lambda$ index
organizational scales (e.g.\ molecular, cellular, neural, organismic, social).
We define a state bundle
\[
\pi : \mathcal{X} \to M \times \Lambda,
\]
whose sections $x \in \Gamma(\mathcal{X})$ represent the physically instantiated
informational structure of the world across scales.

We decompose the system state as
\[
x = (\Phi, V, \Sigma),
\]
where each component captures a distinct but coupled aspect of the system’s
physical and informational organization. The component $\Phi$ denotes
\emph{structural fields}, encompassing morphology, material properties, boundary
conditions, and the geometric constraints that delimit admissible configurations
and interactions. The component $V$ denotes \emph{interaction or flow fields},
encoding transport processes, signaling pathways, coupling relations, and modes
of actuation through which structure is dynamically engaged with its
environment. The component $\Sigma$ denotes \emph{entropy and irreversibility
fields}, representing dissipation, path dependence, and the thermodynamic costs
associated with maintaining organization over time. Together, these components
define a coupled state description in which form, interaction, and irreversible
history are treated as co-evolving and inseparable aspects of natural
computation.


This triadic decomposition captures ICON’s core claim that information is always
physically instantiated ($\Phi$), that computation is the lawful evolution of
that structure ($V$), and that history and irreversibility are fundamental
($\Sigma$).

\subsection{Relational Information as an Agent-Dependent Functional}

In ICON, information is not an intrinsic scalar of the world but a relational
quantity defined with respect to a cognizing agent. Let $A$ be an embodied agent
realized as a subsystem of $x$, with embodiment parameters $\theta_A$ and an
interaction domain $\Omega_A(t) \subset M$.

We define the information available to $A$ at time $t$ as a functional
\[
\mathcal{I}_A[x](t) =
\int_{\Omega_A(t)}
\rho_A\!\left(\Phi, V, \Sigma; \theta_A\right)\, d\mu,
\]
where $\rho_A$ measures differences that make a difference for $A$, given its
morphology and coupling. This formalizes the ICON commitment that information is
observer-relative without being subjective: different agents induce different
informational observables over the same physical state.

\subsection{Natural Computation as Field Evolution}

Natural computation is identified with the time evolution of the physical state:
\[
\frac{d}{dt}x(t) = \mathcal{F}\big(x(t), u(t), \eta(t)\big),
\]
where $u(t)$ represents boundary forcing or control inputs (including actions of
other agents) and $\eta(t)$ represents stochastic perturbations.

To capture morphological computation, we decompose the generator as
\[
\mathcal{F}
=
\mathcal{F}_{\mathrm{flow}}(V)
+
\mathcal{F}_{\mathrm{constraint}}(\Phi, V)
+
\mathcal{F}_{\mathrm{diss}}(\Sigma),
\]
where $\mathcal{F}_{\mathrm{constraint}}$ encodes passive computation performed
by morphology itself, such as geometric feasibility, compliance filtering, and
material memory.

\subsection{Morphological Computing as Constraint-Induced Efficiency}

Let $\mathcal{T}$ be a class of tasks and let $u(t)$ denote active control inputs.
Define a control-effort functional for an agent $A$:
\[
\mathcal{E}_A[u]
=
\int_0^T
\left(
\|u(t)\|^2
+
\alpha\,\mathcal{D}(\Sigma(t))
\right)\, dt,
\]
where $\mathcal{D}$ measures dissipative cost.

Morphological computing is expressed as the existence of structural configurations
$\Phi'$ such that, for all $\tau \in \mathcal{T}$,
\[
\inf_{u}\mathcal{E}_A[u \mid \Phi']
<
\inf_{u}\mathcal{E}_A[u \mid \Phi].
\]
Thus, morphology performs computation by reshaping the constraint geometry of
the system, reducing the energetic and informational burden of active control.

\subsection{Event-Historical Ontology and Irreversibility}

Field evolution alone is typically Markovian. ICON, however, treats cognition
and life as fundamentally historical. To capture this, we introduce an event
space $E$ and define a history as a finite or countable sequence
\[
h = (e_1,t_1)\cdots(e_n,t_n) \in \mathcal{H}.
\]

The system’s evolution is conditioned on its history:
\[
x(t) = \mathcal{U}(t; h)\,x(0).
\]
Each event $e \in E$ acts as an irreversible update on the system’s admissible
future set $\mathcal{A}(x)$:
\[
e : \mathcal{A}(x) \to \mathcal{A}(x'),
\qquad
\mathcal{A}(x') \subsetneq \mathcal{A}(x).
\]
Events do not merely change the state; they eliminate possibilities, encoding
commitments, learning, development, and institutional constraint.

\subsection{Two-Channel Dynamics: Texture and Regime}

We decompose the state as
\[
x = (\xi, \kappa),
\]
where $\xi$ are reversible or weakly constrained \emph{texture} variables and
$\kappa$ are \emph{regime} variables encoding morphology, wiring, norms, and
constraints.

The dynamics take the hybrid form
\[
\dot{\xi} = f(\xi, \kappa),
\qquad
\kappa^{+} = \kappa \oplus e
\quad \text{(at event times)}.
\]
This two-channel structure distinguishes continuous evolution within a regime
from discrete, irreversible regime changes induced by events.

\subsection{Cognition as History-Conditioned Viability}

Let $K \subset \mathcal{X}$ be a viability set corresponding to continued
existence or functional integrity. A system is cognitive in the ICON sense if
there exists a (possibly distributed) policy $\pi$ such that
\[
x(t) \in K \quad \forall t \in [0,T],
\]
where the policy is explicitly history-dependent:
\[
\pi_t = \pi(\mathrm{obs}_{\le t}, h_{\le t}).
\]
Cognition is thus identified with the capacity to maintain viability through
history-sensitive constraint management, rather than with representation or
symbol manipulation.

\subsection{Hybrid Generative Law}

Combining fields and events, the ICON framework can be expressed as a hybrid
system:
\[
\dot{x}(t) = \mathcal{F}(x(t),u(t)),
\]
with discrete updates
\[
\text{if } G_e(x,t) \text{ then } x(t^{+}) = \mathcal{R}_e(x(t^{-})),
\]
where $\mathcal{R}_e$ acts primarily on regime variables $\kappa$. Learning,
development, and social commitment appear naturally as event-triggered rewrites
of constraint structure.

\subsection{Summary}

In this formalization, ICON commits to a world composed of physically
instantiated informational fields whose evolution constitutes natural
computation, and whose cognitive capacities arise from the accumulation of
irreversible, morphology-altering events. Morphological computing appears as
constraint-induced efficiency, while cognition emerges as the historical
management of viable futures under energetic and informational limits.


% ============================================================
\section{Platonic Constraints as Functorial Structure}
% ============================================================

\subsection{Event-Historical Foundations}

We formalize cognition as an event-historical process by treating histories,
rather than instantaneous states, as the primary carriers of constraint. In this
setting, temporal asymmetry and irreversibility are not emergent properties of
dynamics but foundational features of the representational framework itself.

\begin{definition}[History Category]
Let $\mathsf{Ev}$ be a set of event types. The \emph{history category}
$\mathsf{Hist}$ is defined as a category whose objects are finite, admissible
event traces of the form $h = (e_1,\dots,e_n)$, where admissibility is determined
by the prevailing physical, biological, or organizational constraints. A
morphism $i : h \to h'$ exists if and only if $h'$ is obtained from $h$ by
extension with a (possibly empty) sequence of additional events. Composition of
morphisms is given by concatenation of such extensions, reflecting the temporal
ordering of events, and identity morphisms correspond to empty extensions that
leave a history unchanged.
\end{definition}


Intuitively, a morphism represents a permissible continuation of the world.
The categorical structure encodes temporal asymmetry and irreversibility.

\subsection{Admissible Futures as a Presheaf}

Let $\mathsf{Fut}$ be a category whose objects represent possible future
continuations, such as sets of trajectories, policies, or field evolutions, and
whose morphisms encode appropriate notions of restriction or comparison between
such futures.

\begin{definition}[Admissible Futures Presheaf]
An \emph{admissible futures presheaf} is a contravariant functor
\[
\mathcal{A} : \mathsf{Hist}^{op} \to \mathsf{Fut}
\]
with the property that, for each history $h \in \mathsf{Hist}$, the object
$\mathcal{A}(h)$ represents the collection of all future continuations that are
consistent with the commitments encoded in $h$. For any morphism
$i : h \to h'$ corresponding to an extension of histories, the associated map
\[
\mathcal{A}(i) : \mathcal{A}(h') \to \mathcal{A}(h)
\]
acts as a restriction operation that forgets the additional commitments
introduced by the extension from $h$ to $h'$, thereby recovering the larger
space of futures admissible prior to those commitments.
\end{definition}


\begin{assumption}[Monotonic Constraint]
For any $i:h\to h'$, the image of $\mathcal{A}(i)$ is a (typically proper)
subobject of $\mathcal{A}(h)$.
\end{assumption}

This formalizes the principle that longer histories impose strictly stronger
constraints on the future.

\subsection{Platonic Forms as Natural Transformations}

We now reinterpret Platonic forms not as generative templates, but as invariant
constraints on admissible futures.

\begin{definition}[Platonic Constraint]
A \emph{Platonic constraint} is a natural transformation
\[
\mathcal{P} : \mathcal{A} \Rightarrow \mathcal{A}
\]
such that for each history $h$, the component
\[
\mathcal{P}_h : \mathcal{A}(h) \to \mathcal{A}(h)
\]
selects a subset of futures satisfying invariant structural constraints.
\end{definition}

Naturality ensures that for any extension $i:h\to h'$, the square
\[
\begin{CD}
\mathcal{A}(h') @>\mathcal{P}_{h'}>> \mathcal{A}(h') \\
@V\mathcal{A}(i)VV @VV\mathcal{A}(i)V \\
\mathcal{A}(h) @>>\mathcal{P}_h> \mathcal{A}(h)
\end{CD}
\]
commutes.

\begin{assumption}[Idempotence]
Platonic constraints satisfy $\mathcal{P}\circ\mathcal{P}=\mathcal{P}$.
\end{assumption}

Idempotence captures the idea that forms are stable feasibility regions rather
than stepwise rules.

\begin{lemma}[Invariant Feasibility Sub-Presheaf]
An idempotent Platonic constraint $\mathcal{P}$ determines a sub-presheaf
$\mathrm{Fix}(\mathcal{P})\hookrightarrow\mathcal{A}$ defined by
\[
\mathrm{Fix}(\mathcal{P})(h)=\{\,f\in\mathcal{A}(h)\mid \mathcal{P}_h(f)=f\,\}.
\]
\end{lemma}

\subsection{Morphology as Regime Change}

Let $\kappa$ denote a morphological or organizational regime determining which
events are available and how they compose.

A regime change event $e$ induces a functor
\[
\mathcal{M}_e:\mathsf{Hist}_{\kappa}\to\mathsf{Hist}_{\kappa'}
\]
rewriting the space of admissible histories.

Admissible futures are transported along $\mathcal{M}_e$ via pullback and
pushforward, formalizing embodiment as constraint-language modification rather
than state perturbation.

\subsection{Cognition as Viability Preservation}

Let $K\subseteq\mathcal{X}$ be a viability condition.
Define the \emph{viable futures sub-presheaf}
\[
\mathcal{V}(h)=\{\,f\in\mathcal{A}(h)\mid f \text{ remains within } K\,\}.
\]

\begin{definition}[Cognition]
A system is cognitive iff there exists a history-dependent event selection
policy such that for the realized history chain
\[
h_0\to h_1\to \cdots,
\]
the viable futures presheaf satisfies $\mathcal{V}(h_t)\neq\varnothing$ for all
$t$.
\end{definition}

\begin{proposition}[Constraints, Not Blueprints]
Platonic forms correspond to idempotent natural transformations on the
admissible futures presheaf, defining invariant feasibility regions rather than
generative templates. Morphological computation exploits these invariant
constraints to preserve viability under event-historical evolution.
\end{proposition}

% ============================================================
\section{Worked Example: Chemotactic Cell as Event-Historical Cognition}
% ============================================================

We now instantiate the preceding formalism in the simplest nontrivial cognitive
system: a chemotactic unicellular organism. This example demonstrates that
minimal cognition can be characterized as history-dependent constraint
management without appeal to representation or optimization.

\subsection{Environment and History Structure}

Let the environment be a spatial manifold $M=\mathbb{R}^2$ equipped with a
nutrient concentration field
\[
\Phi_c : M \times \mathbb{R}^+ \to \mathbb{R}^+.
\]

Let $\mathsf{Ev}=\{\textsf{Run},\textsf{Tumble},\textsf{Adapt}\}$ be the event
types available to the organism. These events encode changes in motion and
internal organization.

Histories are finite traces of these events. The history category
$\mathsf{Hist}$ is defined as in the previous section, with admissibility
determined by biochemical feasibility (e.g.\ motor saturation, adaptation
timescales).

\subsection{Admissible Futures}

Let $\mathsf{Fut}=\mathsf{Set}$. For each history $h$, we define $\mathcal{A}(h)$
to be the set of admissible future trajectories
\[
\gamma : [t_h,\infty) \to M
\]
that are consistent with the organism’s current motility constraints, its
internal biochemical regime, and the ambient environmental field $\Phi_c$.
Admissibility is therefore determined jointly by bodily capabilities,
organizational state, and environmental coupling, rather than by any abstract
planning or representational criterion.

For any extension $i:h\to h'$, corresponding to the accumulation of additional
events beyond those already recorded in $h$, the associated restriction map
\[
\mathcal{A}(i):\mathcal{A}(h')\to\mathcal{A}(h)
\]
forgets the additional commitments imposed by the events in $h'\setminus h$.
This restriction recovers the larger space of futures that were admissible prior
to those commitments, thereby formalizing the monotonic reduction of
possibility induced by irreversible history extension.

\subsection{Texture and Regime Variables}

Within this framework, we distinguish between two classes of variables that
play different roles in the dynamics. Texture variables, denoted $\xi$,
represent continuously evolving biochemical states such as receptor occupancy,
signaling molecule concentrations, or other fast-changing internal quantities.
These variables evolve smoothly in time according to the organism’s intrinsic
dynamics and its immediate environmental coupling.

Regime variables, denoted $\kappa$, represent slowly changing or irreversible
organizational constraints, including motor bias thresholds, adaptation
parameters, or other structural features that determine which future events and
trajectories are admissible. Unlike texture variables, regime variables are
modified only through discrete events, and their updates correspond to genuine
reorganizations of the system’s constraint structure rather than to incremental
state evolution.


Texture variables evolve according to continuous dynamics
\[
\dot{\xi} = f\big(\xi,\Phi_c(q(t)),\nabla \Phi_c(q(t))\big),
\]
where $q(t)\in M$ is the organism's position.

Regime variables are updated only by events.

\subsection{Events as Constraint-Rewriting Operators}

Events act as operators that rewrite the system’s constraint structure rather
than merely perturbing its instantaneous state. Consider a \textsf{Tumble}
event, which is triggered when a guard condition
\[
G(\xi,\kappa)
\]
defined over the current texture variables $\xi$ and regime variables $\kappa$
is satisfied. The satisfaction of this guard marks a qualitative transition in
the admissible dynamics, indicating that continued evolution within the current
regime threatens viability or coherence.

Upon occurrence, the \textsf{Tumble} event induces a discontinuous change in the
system’s velocity direction, interrupting the smooth evolution of texture
variables and introducing stochastic reorientation. Simultaneously, the event
updates the regime variables according to
\[
\kappa^{+} = \kappa \oplus \delta\kappa,
\]
where $\delta\kappa$ encodes a modification of organizational constraints, such
as altered bias thresholds or sensitivity parameters. This update does not
represent parameter tuning within a fixed behavioral grammar, but a genuine
rewriting of the conditions under which future events become admissible.

In this way, the \textsf{Tumble} event exemplifies how discrete events function
as constraint-rewriting operators. They reshape the space of possible future
histories by modifying regime variables, thereby altering which trajectories and
event sequences are available to the system after the event has occurred.


Categorically, this event induces a functor
\[
\mathcal{M}_{\textsf{Tumble}} :
\mathsf{Hist}_{\kappa} \to \mathsf{Hist}_{\kappa^+},
\]
rewriting the space of admissible histories.

This update does not merely perturb the current state; it modifies which
future event sequences are admissible.

\subsection{Viability and Cognition}

Let $E(t)$ denote the organism's internal energy state, with dynamics
\[
\dot{E} = \alpha\,\Phi_c(q(t)) - \beta,
\]
where $\alpha,\beta>0$.

Define the viability condition
\[
K = \{x \mid E(x) > E_{\min}\}.
\]

For each history $h$, define the viable futures sub-presheaf
\[
\mathcal{V}(h) =
\{\,\gamma\in\mathcal{A}(h)\mid \gamma(t)\in K \ \forall t\ge t_h\,\}.
\]

\begin{definition}[Chemotactic Cognition]
The chemotactic cell is cognitive iff there exists a history-dependent event
selection policy such that along the realized history chain
\[
h_0 \to h_1 \to \cdots,
\]
the viable futures presheaf satisfies $\mathcal{V}(h_t)\neq\varnothing$ for all
$t$.
\end{definition}

\subsection{Interpretation}

In this formulation, gradient sensing is realized as a form of natural
computation instantiated in the continuous evolution of the texture variables.
Rather than constructing or manipulating explicit representations of the
environment, the system exploits the lawful coupling between biochemical
dynamics and external concentration fields to generate behaviorally relevant
structure directly within its physical state evolution.

Behavioral decisions are not modeled as selections among precomputed actions,
but as discrete events that rewrite the set of admissible futures. Each event
functions as an irreversible commitment that reshapes the option--space,
eliminating some continuations while enabling others. Decision-making is thus
identified with constraint modification rather than with choice over a fixed
action set.

Adaptation, on this account, corresponds to regime change rather than to state
estimation or parameter optimization within a fixed model. Events that update
regime variables alter the conditions under which future behaviors are possible,
thereby transforming the grammar of admissible histories instead of refining an
internal estimate of environmental state.

Cognition emerges as the preservation of nonempty viable futures under
irreversible history extension. A system counts as cognitive insofar as its
event-historical dynamics succeed in maintaining at least one admissible
continuation compatible with viability constraints, despite the monotonic
accumulation of commitments imposed by past events.


No internal representation of the nutrient field is required. The organism
computes by exploiting environmental and morphological constraints, updating
them only when viability is threatened.

This example establishes that minimal biological cognition can be formalized as
event-historical constraint management rather than optimization or inference.

% ============================================================
\section{Limits of Variational Inference: A Contrast with the Free Energy Principle}
% ============================================================

We now contrast the event-historical, constraint-based account of cognition
developed above with the Free Energy Principle (FEP). The aim is not to refute
FEP as a modeling framework, but to identify a structural limitation that
prevents it from representing regime change as functorial history rewriting.

\subsection{FEP as State-Based Variational Dynamics}

In the Free Energy Principle, a system is modeled by a state variable $x(t)$
evolving so as to minimize a variational free energy functional
\[
\mathcal{F}(q) = \mathbb{E}_q[\ln q(s) - \ln p(s,o)],
\]
where $q$ is an approximate posterior over hidden states $s$ and $o$ are
observations.

Learning and adaptation are implemented via gradient flows on $\mathcal{F}$,
typically by adjusting parameters $\theta$ of a generative model
$p(o,s\mid\theta)$.

Crucially, both perception and learning are expressed as trajectories within a
fixed model class.

\subsection{Fixed Generative Model Classes}

Let $\Theta$ denote the space of admissible model parameters.
Under FEP, adaptation takes the form:
\[
\dot{\theta} = -\nabla_{\theta}\mathcal{F}.
\]

Although $\theta$ may change over time, the structure of the generative model
itself---its factorization, variable types, and dependency graph---remains
fixed.

Thus, the admissible futures under FEP are determined by:
\[
\mathcal{A}_{\mathrm{FEP}}(t) = \{\text{trajectories consistent with } p(\cdot\mid\theta(t))\}.
\]

All future possibilities are evaluated relative to a single, unchanging
representational framework.

\subsection{Absence of Event-Historical Regime Change}

In the event-historical formalism developed above, regime variables $\kappa$
rewrite the history category itself by altering which event sequences are
admissible. This corresponds to functors
\[
\mathcal{M}_e : \mathsf{Hist}_{\kappa} \to \mathsf{Hist}_{\kappa'}
\]
that modify the space of possible futures.

By contrast, the Free Energy Principle does not admit operators that modify the category of histories itself. Within the FEP formalism, there is no internal mechanism by which a system can introduce genuinely new event types, eliminate previously available transitions, or alter the compositional structure according to which actions and observations are sequenced.

The space of admissible histories is fixed in advance by the generative model class, and all possible trajectories are evaluated relative to this unchanging structural framework. As a consequence, adaptation under FEP is confined to motion within a fixed inferential regime.

Parameter updates may alter which trajectories are favored, but they do not rewrite the grammar of admissible histories. The system can optimize its behavior only with respect to constraints that are already present, and cannot acquire new forms of constraint through irreversible historical commitment. 

\subsection{State Optimization vs.\ Constraint Acquisition}

This distinction can be stated with precision. Under the Free Energy Principle, cognition is modeled as the optimization of trajectories within a fixed constraint structure. The system adapts by adjusting internal state variables and parameters so as to minimize a variational objective, but the underlying form of the generative model remains unchanged.

Learning, in this framework, refines parameter values and improves fit to observations, while preserving the topology of the model and the space of admissible histories it defines.

By contrast, within the event-historical ICON formalism, cognition is identified with the acquisition of constraints through irreversible events. Learning does not merely improve performance within a pre-given space of possibilities; it alters that space itself.

Each commitment event restricts, reorganizes, or rewrites the set of admissible futures, thereby changing what the system can become. Cognition is thus not state optimization within fixed limits, but the historical construction of those limits through irreversible constraint formation. 


Formally, FEP optimizes trajectories within a single admissible-futures
presheaf, whereas ICON permits transitions between distinct presheaves via
regime-changing events.

\subsection{Consequences for Viability}

Let $\mathcal{V}(h)\subseteq\mathcal{A}(h)$ be the viable futures sub-presheaf
defined previously.

In the ICON framework, cognition preserves viability by selecting events that
prevent $\mathcal{V}(h)$ from collapsing under history extension.

In the Free Energy Principle, viability is encoded indirectly through prior preferences and cost functions that bias the system toward particular regions of state space. Survival is thus represented as the maintenance of trajectories that satisfy these predefined expectations. However, because the structure of the admissible future space is fixed by the generative model, the FEP framework cannot represent situations in which continued viability requires altering the constraint language itself rather than merely optimizing behavior within it.

This limitation becomes especially salient in regimes where survival depends on irreversible structural change. Developmental reorganization involves the creation of new organizational constraints that redefine what actions and responses are possible. Morphological plasticity requires the modification of the body or control architecture in ways that permanently reshape future dynamics.

Institutional and social commitments introduce normative constraints that alter the space of admissible collective actions. Irreversible learning events similarly impose lasting restrictions and affordances that cannot be captured as smooth parameter updates within a fixed model class. In all such cases, viability depends on the capacity to rewrite constraints rather than to optimize trajectories under static ones, placing these phenomena outside the expressive scope of state-based variational formulations. 

\subsection{Summary}

The Free Energy Principle provides a powerful account of adaptive behavior as
variational inference within a fixed generative model. However, it lacks the
formal machinery to represent cognition as event-historical regime change.

By treating morphology and organization as functorial rewritings of admissible
history spaces, the ICON-based framework captures a class of cognitive
phenomena---including minimal life, learning, and commitment---that lie outside
the expressive scope of purely state-based variational dynamics.

% ============================================================
\section{Synthesis: Cognition as Constraint-Historical Computation}
% ============================================================

We now synthesize the preceding formal developments into a unified statement of
the Info-Computational (ICON) framework, clarifying its ontological commitments
and its distinctive explanatory power.

\subsection{From Information to Constraint}

Within ICON, information is not an abstract quantity or symbolic encoding, but
the physically instantiated relational structure of a system. When recast in
event-historical terms, informational structure is most naturally understood
not as a space of states, but as a space of admissible continuations.

The admissible futures presheaf $\mathcal{A}:\mathsf{Hist}^{op}\to\mathsf{Fut}$
thus plays a central role. It encodes, for each partial history, the space of
physically and organizationally feasible futures. Informational content is
realized through the restriction of this space by material, morphological, and
historical constraints.

In this sense, information is identified with constraint structure over
possibility space.

\subsection{Platonic Forms as Invariant Feasibility Regions}

Dodig-Crnkovic’s reinterpretation of Platonic forms finds a precise expression
in this framework. Forms are not generative templates or transcendental
blueprints, but invariant constraints that carve out stable regions within the
space of the possible.

Categorically, such forms correspond to idempotent natural transformations
\[
\mathcal{P}:\mathcal{A}\Rightarrow\mathcal{A},
\]
whose fixed sub-presheaves represent feasibility regions preserved across
histories. These constraints are independent of particular material
instantiations, yet only realized through physical processes.

This interpretation preserves the explanatory utility of Platonism while fully
naturalizing it within physical computation.

\subsection{Morphological Computing as Constraint Exploitation}

Morphological computing is now seen not as a metaphor, but as a concrete mode of
computation in which physical structure directly enforces constraints on future
behavior.

Rather than executing algorithms in the conventional sense, morphological systems compute by shaping the space of possible behavior through their physical structure. Computation is realized as the restriction of admissible trajectories, where the geometry and material properties of the system eliminate large classes of otherwise possible motions or responses.

At the same time, morphology stabilizes viable dynamics by passively guiding the system toward attractors that preserve functional integrity. Control is thus offloaded to material geometry and physics, so that coordination and regulation emerge from constraint rather than from explicit calculation. 

In the categorical formulation, morphological change corresponds to functorial
rewriting of the history category, altering which event sequences and futures
are available. Computation thus occurs not only through state evolution, but
through modification of the constraint language itself.

\subsection{Cognition as Event-Historical Viability}

Cognition, under ICON, is neither representation nor inference, but the
historically situated management of constraints in service of continued
existence.

Formally, cognition is the preservation of a nonempty viable futures
sub-presheaf $\mathcal{V}\hookrightarrow\mathcal{A}$ under irreversible history
extension.

Events that rewrite regimes and morphologies are essential, because they enable a system to reshape its future possibility space when existing constraints threaten viability. When continuous adjustment within a given regime is no longer sufficient, irreversible events introduce new structural commitments that alter which futures remain admissible. In this way, cognition is not exhausted by fine-grained adaptation, but includes the capacity for qualitative reorganization of the conditions under which adaptation occurs.

This definition applies uniformly across scales. In unicellular chemotaxis, it appears as irreversible changes in behavioral bias and sensitivity that reshape future motion. In developmental plasticity, it takes the form of structural reorganization that fixes new pathways of growth and interaction. In neural learning, it manifests as lasting changes in connectivity that constrain future dynamics.

In social and institutional contexts, it emerges as commitments and norms that permanently restructure collective possibilities. Across all of these domains, cognition is identified with the historical acquisition and management of constraints that preserve viable futures. 

What distinguishes cognitive systems is not complexity, but the capacity to
acquire and stabilize new constraints over time.

\subsection{Against Blueprint and Optimization Models}

This synthesis clarifies why blueprint-based and purely variational models are
insufficient for a full account of cognition.

Blueprint models fail because they presuppose fixed forms that nature must
instantiate. Variational models fail because they optimize trajectories within
a fixed constraint regime. In both cases, the possibility of irreversible
reorganization of the constraint structure itself is excluded.

ICON, by contrast, treats constraint acquisition as the primary cognitive act.

\subsection{Conclusion}

By unifying field-theoretic structure, event-historical dynamics, and
category-theoretic constraint modeling, the ICON framework provides a
naturalized account of cognition as constraint-historical computation.

Information is physical structure; computation is physical change; cognition is
the embodied, irreversible reshaping of what can happen next.

This perspective situates intelligence squarely within the dynamics of the
natural world, without reducing it to symbol manipulation, optimization, or
representation, and provides a common formal language for biological,
artificial, and social cognition.

% ============================================================
\section{Social and Institutional Cognition as Constraint-Historical Systems}
% ============================================================

The event-historical formalism developed above extends naturally beyond
biological organisms to social and institutional systems. In this extension,
institutions are treated as cognitive systems whose morphology is normative
rather than material, and whose cognitive activity consists in the acquisition,
maintenance, and revision of collective constraints.

\subsection{Institutional Histories}

Let $\mathsf{Ev}_{\mathrm{soc}}$ denote a set of social event types, including
commitments, promises, enactments, violations, sanctions, and revisions. A
social history is a finite trace of such events, ordered temporally and subject
to admissibility conditions determined by existing norms.

Define the social history category $\mathsf{Hist}_{\mathrm{soc}}$ analogously to
the biological case. Objects are admissible sequences of social events, and a
morphism $h \to h'$ exists whenever $h'$ is a permissible extension of $h$ under
the prevailing normative regime. Composition is given by concatenation of event
extensions, and identity morphisms correspond to null extensions.

This categorical structure encodes the irreversibility of social commitments
and the asymmetry between enacted obligations and merely possible actions.

\subsection{Norms as Constraint Operators}

For each social history $h$, let $\mathcal{A}_{\mathrm{soc}}(h)$ denote the set
of admissible future institutional trajectories consistent with that history.
These futures may include sequences of actions by agents, further institutional
acts, or states of collective organization.

Norms are not treated as prescriptions that generate behavior, but as invariant
constraints that restrict the space of admissible futures. Formally, a norm is
represented as a natural transformation
\[
\mathcal{N} : \mathcal{A}_{\mathrm{soc}} \Rightarrow \mathcal{A}_{\mathrm{soc}},
\]
whose components $\mathcal{N}_h$ project admissible futures onto those
consistent with the normative structure.

Idempotence of $\mathcal{N}$ expresses the stability of norms: once a future is
norm-compliant, further application of the norm introduces no additional
restriction. Naturality ensures that normative constraints commute with history
extension, guaranteeing coherence across time.

\subsection{Commitment Events and Regime Change}

Certain social events function not merely as state updates but as regime
changes. Examples include ratification of constitutions, formation of treaties,
or recognition of new authorities. Such events alter which future event
sequences are admissible and how obligations compose.

Formally, a commitment event $e$ induces a functor
\[
\mathcal{M}_e : \mathsf{Hist}_{\kappa} \to \mathsf{Hist}_{\kappa'},
\]
where $\kappa$ and $\kappa'$ denote normative regimes before and after the
commitment. This functor rewrites the space of social histories by modifying the
rules governing admissibility, enforcement, and expectation.

These regime changes are irreversible in the event-historical sense. Once a
constitution is enacted or a treaty signed, the space of possible futures is
permanently altered, even if later revisions occur through additional events.

\subsection{Social Viability and Collective Cognition}

Let $K_{\mathrm{soc}}$ denote a viability condition for an institution, capturing
properties such as legitimacy, stability, solvency, or coherence of collective
expectations. For each history $h$, define the viable futures sub-presheaf
\[
\mathcal{V}_{\mathrm{soc}}(h) \subseteq \mathcal{A}_{\mathrm{soc}}(h)
\]
as those futures that preserve institutional viability.

An institution is cognitive if it possesses mechanisms for selecting and
enacting events such that, along the realized history chain, the viable futures
set remains nonempty. Institutional breakdown corresponds to collapse of this
viable futures presheaf, at which point no admissible continuation preserves the
institution’s defining constraints.

\subsection{Interpretation}

On this account, institutions do not merely regulate behavior; they compute by
restricting collective future spaces. Laws, norms, and commitments function as
morphological constraints in a social medium, shaping feasible trajectories of
action and expectation. Institutional cognition consists in the historically
situated management of these constraints to preserve collective viability over
time.

% ============================================================
\section{Social and Institutional Cognition as Constraint-Historical Systems}
% ============================================================

The event-historical formalism developed above extends naturally beyond
biological organisms to social and institutional systems. In this extension,
institutions are treated as cognitive systems whose morphology is normative
rather than material, and whose cognitive activity consists in the acquisition,
maintenance, and revision of collective constraints.

\subsection{Institutional Histories}

Let $\mathsf{Ev}_{\mathrm{soc}}$ denote a set of social event types, including
commitments, promises, enactments, violations, sanctions, and revisions. A
social history is a finite trace of such events, ordered temporally and subject
to admissibility conditions determined by existing norms.

Define the social history category $\mathsf{Hist}_{\mathrm{soc}}$ analogously to
the biological case. Objects are admissible sequences of social events, and a
morphism $h \to h'$ exists whenever $h'$ is a permissible extension of $h$ under
the prevailing normative regime. Composition is given by concatenation of event
extensions, and identity morphisms correspond to null extensions.

This categorical structure encodes the irreversibility of social commitments
and the asymmetry between enacted obligations and merely possible actions.

\subsection{Norms as Constraint Operators}

For each social history $h$, let $\mathcal{A}_{\mathrm{soc}}(h)$ denote the set
of admissible future institutional trajectories consistent with that history.
These futures may include sequences of actions by agents, further institutional
acts, or states of collective organization.

Norms are not treated as prescriptions that generate behavior, but as invariant
constraints that restrict the space of admissible futures. Formally, a norm is
represented as a natural transformation
\[
\mathcal{N} : \mathcal{A}_{\mathrm{soc}} \Rightarrow \mathcal{A}_{\mathrm{soc}},
\]
whose components $\mathcal{N}_h$ project admissible futures onto those
consistent with the normative structure.

Idempotence of $\mathcal{N}$ expresses the stability of norms: once a future is
norm-compliant, further application of the norm introduces no additional
restriction. Naturality ensures that normative constraints commute with history
extension, guaranteeing coherence across time.

\subsection{Commitment Events and Regime Change}

Certain social events function not merely as state updates but as regime
changes. Examples include ratification of constitutions, formation of treaties,
or recognition of new authorities. Such events alter which future event
sequences are admissible and how obligations compose.

Formally, a commitment event $e$ induces a functor
\[
\mathcal{M}_e : \mathsf{Hist}_{\kappa} \to \mathsf{Hist}_{\kappa'},
\]
where $\kappa$ and $\kappa'$ denote normative regimes before and after the
commitment. This functor rewrites the space of social histories by modifying the
rules governing admissibility, enforcement, and expectation.

These regime changes are irreversible in the event-historical sense. Once a
constitution is enacted or a treaty signed, the space of possible futures is
permanently altered, even if later revisions occur through additional events.

\subsection{Social Viability and Collective Cognition}

Let $K_{\mathrm{soc}}$ denote a viability condition for an institution, capturing
properties such as legitimacy, stability, solvency, or coherence of collective
expectations. For each history $h$, define the viable futures sub-presheaf
\[
\mathcal{V}_{\mathrm{soc}}(h) \subseteq \mathcal{A}_{\mathrm{soc}}(h)
\]
as those futures that preserve institutional viability.

An institution is cognitive if it possesses mechanisms for selecting and
enacting events such that, along the realized history chain, the viable futures
set remains nonempty. Institutional breakdown corresponds to collapse of this
viable futures presheaf, at which point no admissible continuation preserves the
institution’s defining constraints.

\subsection{Interpretation}

On this account, institutions do not merely regulate behavior; they compute by
restricting collective future spaces. Laws, norms, and commitments function as
morphological constraints in a social medium, shaping feasible trajectories of
action and expectation. Institutional cognition consists in the historically
situated management of these constraints to preserve collective viability over
time.

% ============================================================
\section{Constraint-Historical Control and Viability Theory}
% ============================================================

The preceding developments suggest a close relationship between the
event-historical account of cognition and classical control and viability
theory. However, the relationship is not one of simple reduction. Rather, the
ICON framework generalizes control theory by extending control from trajectories
within a fixed state space to the historical management of constraint spaces
themselves.

\subsection{Classical Viability Theory}

In standard viability theory, a system is described by a state space $X$, a set
of admissible controls $U$, and dynamics of the form
\[
\dot{x}(t) = f(x(t), u(t)).
\]
A subset $K \subseteq X$ defines a viability constraint, and the viability
kernel $\mathrm{Viab}(K)$ consists of all initial states from which there exists
a control policy keeping the system within $K$ for all future time.

This formulation presupposes that the state space, control space, and dynamical
laws are fixed. Viability is achieved by selecting appropriate trajectories
within a pre-given constraint structure.

\subsection{From State Spaces to History Spaces}

In the event-historical framework, the primary object of control is no longer
the instantaneous state but the accumulated history of irreversible events.
Accordingly, the state space $X$ is replaced by the history category
$\mathsf{Hist}$, and admissible controls include both continuous control inputs
and discrete events that rewrite the space of admissible futures.

Let $\mathcal{A} : \mathsf{Hist}^{op} \to \mathsf{Fut}$ denote the admissible
futures presheaf as defined previously. The analogue of the state-dependent
viability kernel is now a history-dependent viability condition defined over
$\mathcal{A}$.

\subsection{Trajectory Control and Regime Control}

Controls in this generalized setting decompose into two distinct classes.
Trajectory controls act within a given regime, selecting among admissible
futures without altering the underlying constraint structure. Regime controls,
by contrast, correspond to events that modify which histories and futures are
admissible at all.

Formally, a regime-changing event induces a functor
\[
\mathcal{M}_e : \mathsf{Hist}_{\kappa} \to \mathsf{Hist}_{\kappa'},
\]
thereby altering the domain of the admissible futures presheaf. Such events are
not representable as controls in the classical sense, since they change the
space within which control operates.

\subsection{Historical Viability Kernel}

Define the historical viability kernel $\mathrm{Viab}_{\mathrm{hist}}(K)$ as the
collection of histories from which there exists a sequence of trajectory and
regime controls preserving the nonemptiness of the viable futures presheaf
$\mathcal{V}(h)$ under all admissible extensions.

Unlike the classical viability kernel, $\mathrm{Viab}_{\mathrm{hist}}(K)$ is not
a subset of a state space but a subcategory of $\mathsf{Hist}$. Membership in
this kernel depends on the availability of future constraint-rewriting events,
not merely on current state.

\subsection{Consequences for Cognitive Systems}

This formulation clarifies why biological and social cognition cannot be
adequately modeled by trajectory optimization alone. Systems that rely solely
on trajectory control inevitably exhaust their viability when confronted with
novel or hostile environments. Cognitive systems preserve viability by altering
their own constraint structures, thereby expanding or reshaping the space of
admissible futures.

Cognition, on this view, is a form of control over constraint history rather
than over state evolution. This reframing places learning, development, and
institutional change within a unified control-theoretic perspective while
preserving their irreducibly historical character.

% ============================================================
\section{Simulation-Ready Primitives for Constraint-Historical Systems}
% ============================================================

To support empirical investigation and computational experimentation, the
event-historical framework developed in this work can be instantiated using a
small set of simulation-ready primitives. These primitives are designed to
preserve the conceptual distinctions between texture and regime, trajectory and
event, and state evolution and constraint rewriting, while remaining compatible
with standard hybrid systems implementations.

\subsection{History and Regime Representation}

A system history is represented as a finite, ordered log of events, each event
being a typed record containing a timestamp, a triggering condition, and a
regime update rule. Regime variables encode the current constraint structure,
including available event types, admissibility conditions, and compositional
rules. These regime descriptors are treated as first-class objects rather than
implicit parameters.

The current history determines the active regime, and thus the space of
admissible future histories. This separation ensures that irreversible
commitments are explicitly tracked rather than implicitly encoded in state
variables.

\subsection{Texture Dynamics}

Texture variables represent continuously evolving quantities governed by
ordinary or stochastic differential equations. These variables evolve under
fixed regime constraints between events. Texture dynamics are evaluated
independently of future feasibility considerations, reflecting the
pan-computational substrate of natural dynamics.

Texture evolution is interrupted only when event guard conditions are
satisfied, at which point discrete updates occur.

\subsection{Event Guards and Handlers}

Each event type is associated with a guard condition expressed as a predicate on
the current texture and regime variables. When a guard condition is satisfied,
the corresponding event handler is invoked.

Event handlers perform two distinct operations. First, they may apply
instantaneous updates to texture variables, such as resets or stochastic
perturbations. Second, and more importantly, they update regime variables,
thereby rewriting the space of admissible future events and trajectories.

This mechanism implements regime change as an explicit computational operation,
rather than as an emergent side effect of parameter drift.

\subsection{Constraint Operators}

Platonic constraints are implemented as operators that filter or project sets
of admissible futures. In practice, these operators may take the form of
feasibility masks, constraint satisfaction filters, or pruning rules applied to
candidate trajectories or policies.

Constraint operators are idempotent and history-coherent, ensuring that once a
future is deemed feasible under a given constraint, repeated application does
not further restrict it. This operationalizes the interpretation of forms as
invariant feasibility regions.

\subsection{Viability Monitoring}

Viability is monitored continuously by evaluating whether the current history
admits at least one future that preserves the viability condition. This may be
implemented by estimating the cardinality or measure of the viable futures set,
or by maintaining a representative ensemble of viable continuations.

When viability is threatened, regime-changing events may be triggered to alter
constraints and restore feasible futures. This mechanism distinguishes
constraint-historical control from classical trajectory optimization.

\subsection{Minimal Working Examples}

The framework supports a range of minimal working examples, including
chemotactic agents, soft robotic controllers, and norm-following multi-agent
systems. In each case, cognition is realized not through explicit optimization
or inference, but through the historical management of constraints governing
future possibilities.

These primitives provide a direct pathway from the theoretical framework to
simulation, enabling systematic exploration of cognition as constraint-historical
computation across biological, artificial, and social domains.

% ============================================================
\section{Integration of Info--Computationalism into Spherepop}
% ============================================================

The Info--Computational (ICON) framework proposed by Dodig--Crnkovic presents
cognition as a natural phenomenon arising from the physical interplay of
information and computation. While ICON is often presented as a philosophical
unification of informational ontology and computational dynamics, its core
commitments admit a precise realization within the Spherepop calculus once
information is reinterpreted as constraint and computation as irreversible
event application.

\subsection{Information as Constrained Optionality}

ICON treats information not as an abstract symbol system but as physically
realized structure: differences that make a difference for an embodied agent.
Spherepop formalizes this intuition by identifying information with the shape
of an option--space rather than with any representational content.

At any history prefix $h$, the Spherepop option--space $X_h$ represents the set
of admissible futures consistent with the accumulated commitments encoded in
$h$. Informational content is therefore identified with the exclusion, binding,
and identification of possibilities enacted by events. No additional notion of
information is required. To gain information is to reduce optionality in a
structured way.

This identification renders informational content inherently relational and
agent-relative, since different histories induce different option--spaces even
when embedded in the same physical substrate. ICON’s rejection of abstract,
observer-independent information is thus realized directly as a consequence of
Spherepop’s history dependence.

\subsection{Computation as Event--Historical Transformation}

ICON generalizes computation beyond Turing-style symbol manipulation to include
physical and biological processes. Spherepop provides a rigorous account of this
generalization by treating computation as the irreversible application of
events that transform option--spaces.

Each Spherepop event is a computation in the ICON sense: it is a physical or
semantic process that changes what can happen next. Computation is therefore
identified not with function evaluation over states, but with morphisms in a
category of histories that monotonically restrict future admissibility.

This interpretation aligns ICON’s notion of natural computation with
Spherepop’s axiom of irreversibility. Computation is not repeatable without
cost; it leaves a trace in history that constrains future behavior. As a
result, computational power is not measured by expressivity or universality,
but by the ability to construct and inhabit stable histories.

\subsection{Morphological Computing as Constraint Offloading}

Morphological computing occupies a central role in ICON as the mechanism by
which physical form performs computational work. In Spherepop terms, morphology
is represented as persistent structure in the option--space induced by past
events, particularly bind and refuse operations.

A morphological feature is computational precisely because it eliminates or
orders possibilities without requiring ongoing deliberation. Once a bind is
introduced, entire classes of futures become inaccessible. Once a refusal is
recorded, whole regions of the option--space are sealed. The morphology of an
agent is therefore the sedimented history of such constraint-inducing events.

This realizes ICON’s claim that morphology performs computation by offloading
control effort. The computation has already occurred, in the past, as an
irreversible shaping of the future space. What remains is not calculation, but
navigation within a constrained world.

\subsection{Cognition as Viability-Preserving History Construction}

ICON adopts the principle that cognition is coextensive with life, understood
as the capacity to maintain organization under environmental pressure.
Spherepop renders this principle precise by defining cognition as the ongoing
construction of histories that preserve nonempty viable futures.

Let $K$ be a viability condition expressed as a predicate over option--spaces.
An agent is cognitive if, along its authoritative history $h$, the induced
option--space $X_h$ continues to admit at least one extension satisfying $K$.
Cognitive activity consists in selecting events that preserve this condition,
including regime-changing events that alter the structure of admissible futures
when existing constraints threaten collapse.

This formulation distinguishes cognition from mere physical interaction.
Passive systems evolve within fixed option--spaces; cognitive systems actively
rewrite the space itself. Learning, development, and adaptation correspond to
irreversible changes in the constraint structure governing future action.

\subsection{Relationality and Perspective}

ICON emphasizes that information and computation are always relative to an
agent’s perspective. Spherepop enforces this perspectival dependence formally:
there is no global option--space independent of history. Every admissible future
is indexed by a specific prefix of irreversible commitments.

Different agents, or the same agent at different times, inhabit different
worlds in the strong semantic sense that their option--spaces are not
interchangeable. Perspective is therefore not an epistemic overlay but an
ontological fact: it is constituted by the history one is bound to.

\subsection{Unconventional Computation and Distributed Agency}

ICON often appeals to unconventional models of computation, such as actor-based
or message-passing systems. Spherepop subsumes these models by allowing multiple
histories to evolve in parallel and to interact through bind, meld, and collapse
operations.

Distributed computation corresponds to the construction of partially
overlapping option--spaces whose coherence is enforced through sheaf-theoretic
gluing conditions. When local histories fail to glue, divergence persists. When
commitment enforces compatibility, a shared world emerges.

Thus, distributed cognition is not coordination over shared state, but
coordination through shared constraint.

\subsection{Conclusion}

Viewed through the lens of Spherepop, the ICON framework is not an external
philosophical overlay but a semantic interpretation of event--historical
calculus. Information appears as constrained optionality, computation as
irreversible event application, morphology as sedimented constraint, and
cognition as viability-preserving history construction.

Spherepop therefore provides ICON with what it lacks: a precise formalism in
which its commitments are not merely asserted but enacted. Conversely, ICON
provides Spherepop with a naturalized interpretive context that situates its
calculus within the broader landscape of biological and physical cognition. The
two frameworks coincide at the level that matters most: the ontology of
irreversible world construction.

% ============================================================
\section{Future Directions: Chemotaxis as a Spherepop Program}
% ============================================================

A natural next step for the Spherepop calculus is the explicit reconstruction of
canonical minimal cognitive systems as executable event-histories. Among such
systems, bacterial chemotaxis occupies a privileged position. It is widely
regarded as the simplest form of cognition that nonetheless exhibits adaptation,
history dependence, and viability preservation. Recasting chemotaxis directly
in Spherepop terms would both validate the expressive adequacy of the calculus
and provide a concrete bridge between biological cognition and event-sourced
computation.

\subsection{Chemotactic Agency as an Event-Sourced Object}

Let $X$ be a Spherepop object representing a single motile cell. The authoritative
state of $X$ at any moment is not given by instantaneous variables, but by its
event history $\Hist(X)$. This history determines the current option--space
$X_h$, the set of admissible continuations consistent with accumulated
commitments.

We distinguish two internal components of $X$ as merged sub-objects:
\[
X \;\cong\; X_{\xi} \;\mathrm{MERGE}\; X_{\kappa},
\]
where $X_{\xi}$ represents fast-changing biochemical texture and $X_{\kappa}$
represents slowly changing regime constraints governing behavior. This
distinction is not ontological but operational: both components are realized
entirely through events, but only a restricted subset of events is permitted to
modify $X_{\kappa}$.

\subsection{Primitive Event Types}

The chemotactic repertoire is generated by a small set of typed events. A
$\mathrm{POP}(X,\textsf{Run})$ event extends the current history by maintaining
directional motion under existing constraints. A $\mathrm{POP}(X,\textsf{Tumble})$
event introduces a discontinuous reorientation, corresponding to stochastic
reset of motion direction. A $\mathrm{POP}(X,\textsf{Adapt})$ event modifies
internal thresholds governing the relative frequency of runs and tumbles.

In Spherepop terms, \textsf{Run} and \textsf{Tumble} primarily update texture,
while \textsf{Adapt} is a regime event that rewrites the future event grammar by
altering guard conditions under which \textsf{Tumble} becomes admissible.

\subsection{Environmental Coupling and Guards}

Let $\Phi_c(q)$ denote the nutrient concentration at spatial position $q$. The
environment is not represented symbolically within $X$; instead, it enters only
through guard predicates evaluated at event boundaries. For example, a guard
condition for \textsf{Tumble} may take the form
\[
G_{\textsf{Tumble}}(h) \;:\; \Delta\Phi_c(h) < 0,
\]
where $\Delta\Phi_c(h)$ denotes the change in concentration experienced along
the most recent run segment encoded in the history $h$.

Guards are evaluated relative to history, not state. The relevant comparison is
not between absolute concentrations but between successive experiential
segments, which are themselves defined by prior events.

\subsection{Option--Space Evolution}

Each event strictly reduces or reshapes the option--space. After a
$\textsf{Run}$ event, futures that require immediate reorientation are excluded.
After a $\textsf{Tumble}$ event, futures that depend on maintaining previous
directional commitments are eliminated. After an $\textsf{Adapt}$ event, entire
classes of future histories become inaccessible due to revised admissibility
conditions.

Formally, if $h$ is the current history and $e$ an event, then
\[
X_{h \cdot e} \;\subsetneq\; X_h,
\]
with strict inclusion whenever $e$ encodes a commitment or refusal. Chemotactic
behavior emerges not from trajectory optimization but from monotonic
restriction of future possibilities under environmental coupling.

\subsection{Viability as a Persistent Nonempty Future}

Let $E(h)$ be an energy functional derived from the cumulative history, defined
implicitly by nutrient intake events encoded in $\textsf{Run}$ segments. Define
a viability condition $K$ by $E(h) > E_{\min}$. The cell is viable at history $h$
iff there exists at least one continuation $h'$ extending $h$ such that
$E(h') > E_{\min}$ for all further extensions.

Cognition, in this setting, consists in the selection of events such that the
option--space $X_h$ never collapses to a set containing only nonviable futures.
The chemotactic cell does not optimize a path; it preserves the existence of
paths.

\subsection{Interpretive Significance}

Written in this way, chemotaxis appears as a minimal Spherepop program whose
correctness criterion is viability rather than efficiency. All computational
work is performed by irreversible events that reshape the future, not by
internal inference or representation. Morphological computation is realized as
the long-term effect of past events that have already paid the cost of
constraint.

This example suggests a broader research program. Other biological and
artificial systems can be reconstructed as Spherepop programs by identifying
their primitive events, guard conditions, and viability predicates. In doing so,
the calculus offers a unified language for cognition, control, and embodiment,
grounded not in state or symbol, but in the irreversible construction of worlds.

% ============================================================
\section{Conclusion: Worldhood, Constraint, and Cognition}
% ============================================================

This essay has developed a unified account of information, computation, and
cognition by synthesizing the Info--Computational (ICON) framework with an
event-historical and constraint-based formalism. The central claim has been
that cognition is not adequately understood in terms of representation,
optimization, or state-based inference, but rather as the irreversible
construction and maintenance of constrained worlds.

ICON provides the philosophical grounding for this view by insisting that
information is always physically instantiated, that computation is the natural
dynamics of such structure, and that cognition is continuous with life itself.
What this work has contributed is a precise articulation of these commitments in
terms of admissible futures, event histories, and constraint evolution. In doing
so, it renders ICON not merely interpretive, but formally operational.

Information has been reconceived as constrained optionality: the structured
restriction of what can happen next. Computation has been reinterpreted as the
irreversible application of events that reshape future possibility spaces.
Morphology has appeared not as passive structure but as accumulated constraint,
encoding prior computational work in the very geometry of the system. Cognition
has emerged as the preservation of nonempty viable futures under ongoing
history extension.

This perspective dissolves several long-standing dichotomies. It removes the
divide between computation and physics by treating computation as physical
change. It removes the divide between body and mind by showing that morphology
itself performs computation. It removes the divide between perception, action,
and learning by locating them within a single event-historical process of
constraint acquisition. It also removes the divide between biological,
artificial, and social cognition by providing a common formal language in which
all may be described.

By contrasting this framework with state-based variational and optimization
models, the essay has shown that a crucial dimension of cognition is often
overlooked: the ability to alter the constraint language itself. Development,
learning, commitment, and institutional formation are not well-described as
trajectories within fixed spaces. They are regime changes that rewrite what
counts as a possible future. Any theory that cannot represent such rewritings
is necessarily incomplete.

The integration with the Spherepop calculus has demonstrated that these ideas
are not merely abstract. Spherepop provides a minimal, event-sourced substrate
in which ICON’s commitments can be enacted rather than assumed. In this setting,
information is not stored, computation is not executed, and cognition is not
represented. Instead, worlds are built, option spaces are narrowed, and
viability is maintained through irreversible commitment.

The worked example of chemotaxis illustrates that even the simplest living
systems instantiate this logic. Cognition appears not at the point where
symbols arise, but at the point where history matters—where past events
constrain future possibilities in ways that must be actively managed to
persist.

Taken together, these results support a strong conclusion. Cognition is not a
special faculty layered atop the physical world. It is a mode of participation
in that world, characterized by the capacity to accumulate constraints without
foreclosing viability. To understand cognition is therefore to understand how
systems construct and inhabit worlds that they cannot simply step outside of.

In this sense, the deepest continuity revealed by the ICON framework is not
between information and computation, but between cognition and worldhood
itself.

% ============================================================
\appendix
\section{Appendix A: Measure-Theoretic Structure of Admissible Futures}
% ============================================================

This appendix endows the admissible futures presheaf with explicit
measure-theoretic structure, allowing viability, collapse, and constraint
severity to be treated quantitatively rather than purely qualitatively.

For each history $h$, let $(\Omega_h,\mathcal{F}_h)$ be a measurable space whose
points $\omega \in \Omega_h$ represent complete future continuations compatible
with $h$. The $\sigma$-algebra $\mathcal{F}_h$ is generated by cylinder sets over
finite extensions of $h$, so that measurability corresponds to dependence on
finite future commitments.

For any admissible extension $i:h\to h'$, the restriction map
\[
\mathcal{A}(i): \Omega_{h'} \to \Omega_h
\]
is assumed measurable. This induces a pullback map on $\sigma$-algebras
\[
\mathcal{A}(i)^{-1} : \mathcal{F}_h \to \mathcal{F}_{h'},
\]
ensuring that admissibility respects history extension.

\subsection{Constraint Operators as Measurable Projectors}

A Platonic constraint $\mathcal{P}$ is realized, for each $h$, as a measurable
map
\[
\mathcal{P}_h : \Omega_h \to \Omega_h
\]
satisfying idempotence almost everywhere,
\[
\mathcal{P}_h \circ \mathcal{P}_h = \mathcal{P}_h.
\]

The feasible future set induced by $\mathcal{P}$ is defined as
\[
\Omega_h^{\mathcal{P}} = \{\omega \in \Omega_h \mid \mathcal{P}_h(\omega)=\omega\}.
\]

Naturality of $\mathcal{P}$ implies that for any extension $i:h\to h'$,
\[
\mathcal{A}(i)\big(\Omega_{h'}^{\mathcal{P}}\big) \subseteq \Omega_h^{\mathcal{P}},
\]
so invariant feasibility regions are preserved under forgetting future
commitments.

\subsection{Viability Measures and Collapse}

Let $\mu_h$ be a reference measure on $(\Omega_h,\mathcal{F}_h)$, interpreted as
a weighting over admissible futures. Define the viable futures set
$\mathcal{V}(h)\subseteq \Omega_h$ as in the main text, and define the viability
measure
\[
\nu(h) = \mu_h(\mathcal{V}(h)).
\]

A history $h$ is viable if and only if $\nu(h)>0$. Viability collapse occurs
precisely when $\nu(h)=0$, at which point no admissible continuation preserves
the viability condition.

Cognition, in this measure-theoretic sense, corresponds to the existence of
history-dependent event-selection policies that maintain $\nu(h)>0$ under all
admissible extensions. This formulation makes precise the idea that cognition
preserves not a particular future, but a nonzero measure of futures.

% ============================================================
\section{Appendix B: Hybrid System Semantics and Regime Rewrite}
% ============================================================

This appendix provides a formal hybrid-systems semantics for the dynamics
introduced in the main text, with particular emphasis on the distinction between
trajectory evolution within a regime and regime rewrite as an irreducible
structural operation.

Let $\Xi$ denote the space of continuous texture variables and let $\mathcal{K}$
denote the space of discrete regime variables. The instantaneous configuration
of the system is given by a pair
\[
x = (\xi,\kappa) \in \Xi \times \mathcal{K}.
\]

Between discrete events, the system evolves according to a regime-indexed
dynamical law
\[
\dot{\xi}(t) = f_{\kappa}(\xi(t)),
\]
where $f_{\kappa}$ is assumed to be locally Lipschitz on $\Xi$ for each fixed
$\kappa$. This evolution defines a flow $\varphi^{\kappa}_t$ on $\Xi$, valid
until an event guard condition is satisfied.

\subsection{Event Structure}

An event type $e$ is specified by a triple $(G_e,R_e,\kappa_e)$, where $G_e$ is a
guard predicate
\[
G_e : \Xi \times \mathcal{K} \to \{0,1\},
\]
$R_e$ is a reset map
\[
R_e : \Xi \times \mathcal{K} \to \Xi,
\]
and $\kappa_e$ is a regime update map
\[
\kappa_e : \mathcal{K} \to \mathcal{K}.
\]

When $G_e(\xi,\kappa)=1$, the event $e$ is admissible. Its occurrence produces an
instantaneous transition
\[
(\xi,\kappa) \mapsto (\xi^{+},\kappa^{+}),
\qquad
\xi^{+} = R_e(\xi,\kappa),
\quad
\kappa^{+} = \kappa_e(\kappa).
\]

While $R_e$ may introduce discontinuities in the texture variables, the crucial
feature is that $\kappa_e$ alters the regime under which all subsequent
dynamics and admissibility conditions are evaluated.

\subsection{Regime Rewrite as Categorical Change}

Let $\mathsf{Conf}_{\kappa}$ denote the category whose objects are flows
generated by $f_{\kappa}$ on $\Xi$ and whose morphisms are time reparametrized
flow segments. For a fixed regime $\kappa$, this category captures all
trajectory-level behavior available without structural change.

A regime update $\kappa \mapsto \kappa'$ induces a functor
\[
\mathcal{R}_e : \mathsf{Conf}_{\kappa} \to \mathsf{Conf}_{\kappa'},
\]
mapping trajectories admissible under $\kappa$ to trajectories admissible under
$\kappa'$, where this mapping may be partial or undefined for trajectories that
violate the new regime constraints.

This functorial action is not representable as an endomorphism within
$\mathsf{Conf}_{\kappa}$. It strictly changes the object of discourse by
replacing the governing flow, admissibility conditions, and event grammar.

\subsection{Irreducibility of Regime Change}

No continuous deformation of $f_{\kappa}$ within $\mathsf{Conf}_{\kappa}$ can
implement the effect of $\mathcal{R}_e$, since regime rewrite alters which
trajectories are admissible at all. In particular, regime change may introduce
or eliminate event types, modify guard predicates, or change compositional
rules for future events.

This establishes that regime-changing events are formally irreducible to
parameter updates, control inputs, or state perturbations. They require a
distinct semantic layer corresponding to irreversible modification of the
constraint structure governing future evolution.

\subsection{Hybrid Executions}

A complete system execution is therefore a sequence
\[
(\xi_0,\kappa_0)
\;\xrightarrow{\varphi^{\kappa_0}}
(\xi_1,\kappa_0)
\;\xrightarrow{e_1}\;
(\xi_1^{+},\kappa_1)
\;\xrightarrow{\varphi^{\kappa_1}}
\cdots
\]
in which continuous flows and discrete regime rewrites alternate. The execution
is well-defined only relative to the accumulated history of regime updates, not
merely to the current texture state.

This hybrid semantics makes precise the claim that cognition involves not only
the control of trajectories, but the historical modification of the dynamical
laws and admissibility conditions under which trajectories are defined.

% ============================================================
\section{Appendix C: Spherepop Correspondence and Algebraic Properties}
% ============================================================

This appendix establishes a precise correspondence between the event-historical
ICON formalism developed in the main text and the Spherepop calculus, showing
that the latter provides an algebraic realization of constraint-historical
computation.

Let $\mathsf{SP}$ be the category whose objects are option--spaces and whose
morphisms are irreversible constraint applications. For a given history $h$,
define the Spherepop object
\[
X_h \;\coloneqq\; \mathcal{A}(h),
\]
where $\mathcal{A}$ is the admissible futures presheaf. Inclusion relations
between option--spaces encode the partial order induced by history extension.

\subsection{Spherepop Events as Morphisms}

Each Spherepop primitive event corresponds to a morphism in $\mathsf{SP}$. A
$\mathrm{POP}(e)$ operation extending history $h$ to $h\cdot e$ induces a
monotone map
\[
X_h \xrightarrow{e} X_{h\cdot e},
\]
satisfying
\[
X_{h\cdot e} \subsetneq X_h,
\]
with strict inclusion whenever $e$ introduces a genuine commitment or refusal.
This captures the axiom of irreversibility: events monotonically restrict the
space of admissible futures.

\subsection{Merge and Compatibility}

Given two Spherepop objects $X_h$ and $Y_k$ representing partially independent
histories, a $\mathrm{MERGE}(X_h,Y_k)$ operation enforces compatibility between
their future commitments. Algebraically, this is represented as a pullback
\[
X_h \times_Z Y_k,
\]
where $Z$ encodes shared constraints or interaction interfaces. Only those
futures that are jointly admissible survive the merge, and incompatibility
appears as empty pullbacks.

\subsection{Collapse as Quotienting}

A $\mathrm{COLLAPSE}$ operation identifies histories that differ only by
inessential distinctions while preserving admissible futures up to
isomorphism. Formally, this corresponds to quotienting an option--space by an
equivalence relation $\sim$ such that
\[
X_h / {\sim} \;\cong\; X_{h'},
\]
whenever $h$ and $h'$ induce the same admissible futures. Collapse is therefore
not forgetting in the epistemic sense, but the elimination of historical detail
that no longer constrains the future.

\subsection{Order-Enriched Structure}

The category $\mathsf{SP}$ is enriched over a preorder defined by future
inclusion. For objects $X_h$ and $X_{h'}$, we write
\[
X_{h'} \preceq X_h
\]
if and only if $X_{h'} \subseteq X_h$. Composition of Spherepop morphisms is
monotone with respect to this order, reflecting the irreversible accumulation of
constraint.

\subsection{Fixed Points and Dead Systems}

An object $X_h$ is a fixed point of the Spherepop dynamics if no nontrivial
event is admissible from $h$. In this case, $X_h$ is minimal with respect to
$\preceq$. Such fixed points correspond either to fully constrained systems
with a unique future or to dead systems with no viable continuation.

\subsection{Algebraic Characterization}

Spherepop computation is thus monotone, history-indexed, and non-idempotent,
except at fixed points. Unlike classical computational models, expressivity is
not measured by the ability to simulate arbitrary functions, but by the ability
to construct stable chains of irreversible morphisms that preserve nonempty
option--spaces.

This establishes Spherepop as an algebraic semantics for the ICON framework, in
which information is constraint, computation is irreversible morphism
application, and cognition is the sustained construction of viable histories.

% ============================================================
\section{Appendix D: Sheaf-Theoretic Gluing of Local Histories}
% ============================================================

This appendix develops a sheaf-theoretic formulation of constraint-historical
systems, making precise how locally constructed histories and option--spaces
compose into coherent global worlds. This formalization underwrites the claims
in the main text concerning distributed cognition, multi-agent interaction, and
institutional coherence.

Let $\mathsf{Loc}$ be a category whose objects are localized interaction domains,
such as spatial regions, subsystems, or agent-relative perspectives, and whose
morphisms represent inclusion or refinement of domains. For each object
$U \in \mathsf{Loc}$, let $\mathsf{Hist}(U)$ denote the history category
restricted to events observable or admissible within $U$.

\subsection{Local Admissible Futures}

For each $U \in \mathsf{Loc}$, define a presheaf
\[
\mathcal{A}_U : \mathsf{Hist}(U)^{op} \to \mathsf{Fut},
\]
assigning to each local history $h_U$ the set of admissible futures consistent
with the constraints accumulated in $U$ alone.

Restriction along inclusions $V \hookrightarrow U$ induces natural
transformations
\[
\rho_{UV} : \mathcal{A}_U \Rightarrow \mathcal{A}_V,
\]
which forget commitments not observable in $V$. These maps formalize the fact
that different agents or subsystems have access to different portions of the
global constraint structure.

\subsection{Sheaf Condition}

The collection $\{\mathcal{A}_U\}_{U \in \mathsf{Loc}}$ defines a presheaf of
option--spaces over $\mathsf{Loc}$. We impose the sheaf condition as follows.

Given a covering family $\{U_i \to U\}$ and a family of local futures
$f_i \in \mathcal{A}_{U_i}(h_i)$ such that for all overlaps
$U_i \cap U_j$ the restrictions agree,
\[
\rho_{U_i,U_i\cap U_j}(f_i)
=
\rho_{U_j,U_i\cap U_j}(f_j),
\]
there exists a unique global future
\[
f \in \mathcal{A}_U(h)
\]
whose restriction to each $U_i$ is $f_i$.

This gluing condition expresses global coherence of futures as compatibility of
local constraint-histories.

\subsection{Failure of Gluing and Constraint Conflict}

If the sheaf condition fails for a given family of local futures, no global
future exists that realizes all local commitments simultaneously. Such failures
correspond to genuine constraint conflicts rather than epistemic uncertainty.
In biological systems, this may appear as incompatible local regulatory demands;
in social systems, as norm conflict; in multi-agent systems, as coordination
failure.

Collapse events can be understood as operations that modify local histories so
as to restore the sheaf condition, eliminating distinctions that obstruct
gluing while preserving future viability.

\subsection{Distributed Cognition}

A distributed cognitive system is one in which viable futures arise not from a
single global controller, but from the successful gluing of locally viable
histories. Cognition, in this setting, consists in maintaining sheaf coherence
over time, ensuring that local adaptations do not destroy the existence of
global futures.

This formulation clarifies why distributed systems fail not when local
optimization fails, but when local constraint acquisition becomes mutually
incompatible.

\subsection{Relation to Spherepop}

In Spherepop terms, each local history induces an option--space, and merge
operations correspond to attempts at sheaf gluing. A successful merge produces
a nonempty pullback, corresponding to a global section of the sheaf. Failure of
merge corresponds to obstruction to gluing, signaling irreconcilable
commitments.

Thus, Spherepop merge and collapse operations implement sheaf-theoretic
mechanisms for maintaining global coherence across distributed,
event-historical systems.

\subsection{Summary}

The sheaf-theoretic perspective completes the formal picture by showing how
constraint-historical cognition scales from individual agents to distributed
and institutional systems. Local histories generate local option--spaces; global
worlds exist only when these spaces glue coherently. Cognition, at scale, is the
maintenance of this gluing under irreversible constraint acquisition.

% ============================================================
\section{Appendix E: Topos-Theoretic Semantics and Internal Logic of Futures}
% ============================================================

This appendix develops a topos-theoretic interpretation of the admissible
futures presheaf, making explicit the internal logic governing constraint,
viability, and cognition. The purpose is to show that event-historical cognition
induces a non-classical, intuitionistic logic of possibility that cannot be
reduced to truth over states.

Let $\mathsf{Hist}$ be the history category defined in the main text. Consider
the presheaf topos
\[
\widehat{\mathsf{Hist}} \;\coloneqq\; \mathbf{Set}^{\mathsf{Hist}^{op}}.
\]
Objects of this topos are contravariant functors assigning sets to histories,
and morphisms are natural transformations. The admissible futures presheaf
$\mathcal{A}$ is therefore an object of $\widehat{\mathsf{Hist}}$.

\subsection{Subobjects and Propositions}

In $\widehat{\mathsf{Hist}}$, propositions about futures are represented as
subobjects of $\mathcal{A}$. A predicate such as viability corresponds to a
monomorphism
\[
\mathcal{V} \hookrightarrow \mathcal{A}.
\]

Crucially, such propositions are history-indexed. Whether a future satisfies a
predicate is not an absolute matter, but depends on the accumulated constraints
encoded by the history at which the proposition is evaluated.

The internal logic of $\widehat{\mathsf{Hist}}$ is intuitionistic. In
particular, the law of excluded middle does not hold in general. For a given
history $h$, it need not be the case that every admissible future is either
viable or nonviable; instead, viability may be undecidable until further events
occur.

\subsection{Truth Values as Sieves}

In the presheaf topos, truth values are given by sieves. For a history $h$, a
truth value is a set of extensions $S \subseteq \mathrm{Ext}(h)$ closed under
further extension. Intuitively, a proposition is true at $h$ if it remains true
under all admissible continuations in $S$.

For example, the proposition “viability can be preserved” is true at $h$ if
there exists a sieve of extensions of $h$ along which the viable futures
presheaf remains nonempty. This captures the idea that cognition concerns
robust future maintainability rather than instantaneous satisfaction.

\subsection{Necessity and Possibility}

Modal operators arise naturally. A proposition is necessary at $h$ if it holds
for all admissible extensions of $h$, and possible if it holds for at least one
extension. These modalities are history-relative and evolve as events restrict
the extension order.

As constraints accumulate, possibilities may collapse into necessities or
vanish entirely. This dynamic modal structure formalizes the intuition that
learning and commitment transform what is merely possible into what is fixed.

\subsection{Cognition as Internal Consistency}

Within this topos-theoretic setting, cognition can be characterized as the
maintenance of internal consistency of the object $\mathcal{V}$. A cognitive
system is one for which the internal logic does not collapse to falsity; that
is, the bottom object is never forced at the current history.

Extinction or breakdown corresponds to the forcing of falsity, where the
viability predicate becomes false at all extensions. This reframes survival as
a logical condition internal to the topos of histories.

\subsection{Relation to Classical Semantics}

Classical state-based models assume a Boolean logic evaluated pointwise over
states. The topos-theoretic semantics here replaces this with a temporal,
constraint-sensitive logic evaluated over histories. This shift is essential
for representing irreversible learning, commitment, and morphological change,
none of which admit classical truth conditions at a single time slice.

\subsection{Summary}

The topos-theoretic interpretation shows that event-historical cognition
naturally induces an intuitionistic logic of futures. Truth is not a property
of states but of histories; propositions are preserved or destroyed by events;
and cognition is the maintenance of nontrivial internal logic over time.

% ============================================================
\section{Appendix F: Obstruction Theory and Irreversible Collapse}
% ============================================================

This appendix introduces an obstruction-theoretic characterization of
irreversible collapse in constraint-historical systems. The aim is to make
precise when and why a system loses the capacity to preserve viable futures,
and to show that such loss is generically not detectable by local state
analysis alone.

Let $\mathsf{Hist}$ be the history category and let
$\mathcal{V} \hookrightarrow \mathcal{A}$ be the viable futures sub-presheaf.
For each history $h$, $\mathcal{V}(h)$ represents the set of admissible futures
that preserve the viability condition.

\subsection{Extension Obstructions}

Given a history $h$ and an admissible extension $i:h\to h'$, viability fails to
extend along $i$ if the restriction map satisfies
\[
\mathcal{V}(h') = \varnothing
\quad\text{while}\quad
\mathcal{V}(h) \neq \varnothing.
\]
Such an extension constitutes a local obstruction to viability preservation.

Define the obstruction indicator
\[
\mathfrak{o}(i) =
\begin{cases}
1 & \text{if } \mathcal{V}(h')=\varnothing,\\
0 & \text{otherwise}.
\end{cases}
\]

An obstruction is irreversible if for all further extensions
$j:h'\to h''$, one has $\mathcal{V}(h'')=\varnothing$. Irreversibility thus
corresponds to terminal failure in the extension order.

\subsection{Global Obstructions and Cohomology}

Consider the nerve $N(\mathsf{Hist})$ of the history category and interpret
viability as a presheaf over this simplicial object. Obstructions to global
viability correspond to nontrivial cohomology classes in the derived functor
cohomology of $\mathcal{V}$.

In particular, if there exists a cycle of histories whose local viable futures
cannot be glued into a global viable future, then $\mathcal{V}$ admits a
nontrivial obstruction class. Such obstructions signal that no sequence of
future events can restore viability without regime change.

\subsection{Collapse as Obstruction Realization}

A collapse event occurs when an obstruction class becomes realized at the
current history. At this point, the viable futures presheaf collapses to the
initial object, and the internal logic of futures forces falsity.

Importantly, this collapse may occur even when all local state variables remain
within acceptable bounds. Collapse is therefore a property of the global
constraint structure, not of instantaneous system health.

\subsection{Regime Change as Obstruction Resolution}

Certain regime-changing events act by modifying the extension structure of
$\mathsf{Hist}$ itself. Formally, a regime rewrite induces a functor
\[
\mathcal{M} : \mathsf{Hist} \to \mathsf{Hist}'
\]
such that the pullback of $\mathcal{V}$ along $\mathcal{M}$ has trivial
obstruction classes.

This corresponds to resolving obstructions by changing the constraint
language. Biological examples include developmental transitions and metabolic
switches; social examples include constitutional revision and norm change.

\subsection{Relation to Classical Control Failure}

In classical control theory, failure is detected when trajectories exit a
viability set. In the obstruction-theoretic account, failure occurs when no
admissible continuation remains viable, even if the current state is still
well-behaved.

This explains why certain collapses appear sudden and inexplicable from a
state-based perspective. The obstruction has accumulated historically and is
realized only when the last extension is consumed.

\subsection{Summary}

Obstruction theory provides a precise mathematical account of irreversible
collapse in constraint-historical systems. Viability failure is not merely the
crossing of a boundary in state space, but the realization of a global
obstruction in the space of admissible futures. Cognition consists, in part, in
anticipating and resolving such obstructions through regime-changing events
before collapse becomes inevitable.

% ============================================================
\section{Appendix G: Failure of Bellman Optimality in Constraint-Historical Systems}
% ============================================================

This appendix establishes that constraint-historical systems generically violate
Bellman optimality and therefore cannot, in principle, be reduced to dynamic
programming or Markov decision process formulations without loss of expressive
power.

\subsection{Bellman Optimality}

Bellman optimality presupposes that a decision problem admits a decomposition
into subproblems indexed by states. Let $X$ be a state space, let $U$ be a set of
actions, and let the dynamics be Markovian. A value function
\[
V : X \to \mathbb{R}
\]
is Bellman-optimal if it satisfies
\[
V(x) = \sup_{u \in U(x)} \left( r(x,u) + \mathbb{E}[V(x') \mid x,u] \right),
\]
where $x'$ denotes the successor state. Crucially, the optimality of future
decisions depends only on the current state, not on the path by which that state
was reached.

\subsection{History Dependence of Admissibility}

In the constraint-historical framework, admissibility of actions and futures is
history-dependent. Two histories $h$ and $h'$ may induce the same instantaneous
state $x$ while yielding distinct option--spaces $X_h \neq X_{h'}$. In such
cases, the set of admissible future events differs despite identical state
descriptions.

Formally, the projection
\[
\pi : \mathsf{Hist} \to X
\]
is many-to-one, and the admissible futures presheaf $\mathcal{A}$ does not factor
through $X$. Therefore, no value function defined purely on $X$ can represent
future feasibility.

\subsection{Non-Decomposability}

Let $h = h_1 \cdot h_2$ be a decomposition of a history into prefix and suffix.
Bellman optimality requires that the optimal continuation from $h$ can be
computed solely from the terminal state of $h_1$. However, in a
constraint-historical system, the commitments encoded in $h_1$ constrain which
continuations of $h_2$ are admissible, independently of state.

Thus, optimality does not decompose along temporal boundaries. Decisions that
are locally optimal within a given regime may destroy the possibility of future
regime change, leading to global nonviability.

\subsection{Formal Counterexample}

Consider a system with two admissible regime-changing events $e$ and $f$,
available only before a commitment event $c$. Executing $c$ first yields a state
identical to executing $e$ or $f$ first, but eliminates the availability of $e$
and $f$ thereafter.

No Bellman-consistent value function can assign a single value to the state
reached after $c$ that correctly accounts for the lost future options. The loss
is not reflected in state variables but in the history-indexed option--space.

\subsection{Consequences for Control and Learning}

Because Bellman optimality fails, optimal control policies cannot be computed by
backward induction over states. Learning algorithms that assume Markovian value
functions necessarily misestimate long-term viability when regime change and
irreversibility are present.

This explains the systematic brittleness of optimization-based approaches in
domains involving irreversible commitment, learning, or development.

\subsection{Summary}

Constraint-historical systems violate the structural assumptions required for
Bellman optimality. Cognition, when understood as the management of irreversible
constraints over future possibility spaces, cannot be captured by state-based
dynamic programming. This establishes a fundamental separation between
optimization and cognition as treated in the ICON and Spherepop frameworks.


\begin{thebibliography}{99}

\bibitem{dodig-crnkovic2006}
G.~Dodig--Crnkovic,
\newblock ``Investigations into Information Semantics and Ethics of Computing,''
\newblock \emph{Mälardalen University Press}, 2006.

\bibitem{dodig-crnkovic2012}
G.~Dodig--Crnkovic,
\newblock ``Information and Computation -- Omnipresent and Pervasive,''
\newblock in \emph{Information and Computation}, Springer, 2012, pp.~13--32.

\bibitem{dodig-crnkovic2018}
G.~Dodig--Crnkovic,
\newblock ``Cognition as Embodied Morphological Computation,''
\newblock in \emph{Philosophy and Theory of Artificial Intelligence}, Springer, 2018, pp.~19--23.

\bibitem{dodig-crnkovic2020}
G.~Dodig--Crnkovic,
\newblock ``Nature as a Network of Morphological Computations,''
\newblock \emph{Information}, vol.~11, no.~1, 2020.

\bibitem{dodig-crnkovic2023}
G.~Dodig--Crnkovic,
\newblock ``Platonic Space as Cognitive Construct,''
\newblock \emph{Philosophies}, vol.~8, no.~2, 2023.

\bibitem{pfeifer2006}
R.~Pfeifer, F.~Iida,
\newblock ``Morphological Computation: Connecting Body, Brain and Environment,''
\newblock \emph{Japanese Scientific Monthly}, vol.~58, no.~2, 2006.

\bibitem{pfeifer2007}
R.~Pfeifer, J.~Bongard,
\newblock \emph{How the Body Shapes the Way We Think},
\newblock MIT Press, 2007.

\bibitem{brooks1991}
R.~A.~Brooks,
\newblock ``Intelligence Without Representation,''
\newblock \emph{Artificial Intelligence}, vol.~47, 1991, pp.~139--159.

\bibitem{varela1991}
F.~J.~Varela, E.~Thompson, E.~Rosch,
\newblock \emph{The Embodied Mind},
\newblock MIT Press, 1991.

\bibitem{maturana1980}
H.~R.~Maturana, F.~J.~Varela,
\newblock \emph{Autopoiesis and Cognition},
\newblock D.~Reidel Publishing Company, 1980.

\bibitem{ashby1956}
W.~R.~Ashby,
\newblock \emph{An Introduction to Cybernetics},
\newblock Chapman \& Hall, 1956.

\bibitem{aubin1991}
J.-P.~Aubin,
\newblock \emph{Viability Theory},
\newblock Birkhäuser, 1991.

\bibitem{friston2010}
K.~Friston,
\newblock ``The Free-Energy Principle: A Unified Brain Theory?''
\newblock \emph{Nature Reviews Neuroscience}, vol.~11, 2010, pp.~127--138.

\bibitem{friston2019}
K.~Friston,
\newblock ``A Free Energy Principle for a Particular Physics,''
\newblock \emph{Entropy}, vol.~21, no.~10, 2019.

\bibitem{hewitt2012}
C.~Hewitt,
\newblock ``The Actor Model of Computation for Scalable Robust Interactive Systems,''
\newblock \emph{arXiv:1008.1459}, 2012.

\bibitem{rosen1985}
R.~Rosen,
\newblock \emph{Anticipatory Systems},
\newblock Pergamon Press, 1985.

\bibitem{anderson1972}
P.~W.~Anderson,
\newblock ``More Is Different,''
\newblock \emph{Science}, vol.~177, no.~4047, 1972.

\bibitem{flyxion2025}
F.~Guimond,
\newblock ``Spherepop: An Event-Historical Calculus of Constraint and Worldhood,''
\newblock manuscript in preparation, 2025.

\end{thebibliography}

\end{document}