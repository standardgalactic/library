\documentclass{article}
\usepackage{amsmath}
\usepackage{amssymb}
\usepackage{amsthm}
\theoremstyle{definition}
\newtheorem{definition}{Definition}[section]
\newtheorem{assumption}{Assumption}[section]
\usepackage{mathpartir}
\usepackage{tikz}
\usepackage{tikz-cd}
\usepackage{float}
\usepackage{tabularx}
\usepackage{booktabs}
\usepackage{cleveref}
\usepackage{hyperref}
\usepackage{boxedminipage}

\title{Extrapolated Riemannian Curvature of Semantic Manifolds}
\author{Flyxion}

\date{\today}

\newtheorem{theorem}{Theorem}[section]
\newtheorem{lemma}[theorem]{Lemma}
\newtheorem{corollary}[theorem]{Corollary}
\newtheorem{proposition}[theorem]{Proposition}
\newtheorem{observation}[theorem]{Observation}

\begin{document}

\maketitle

\begin{abstract}

This essay develops a formal framework for understanding semantic manifolds through extrapolated Riemannian curvature, integrating differential geometry, active inference, and second-person neuroscience. We explore how curvature distortions in high-dimensional representational spaces underpin information loss in multimodal models and neural synchrony in social interactions. The manifold hypothesis serves as a foundational principle, positing that real-world data concentrate on low-dimensional submanifolds, enabling interpolation and generalization. We map empirical findings from embedding connectors to an RSVP field theory, where scalar capacity, vector flows, and entropy govern semantic fidelity. Applications to therapy emphasize affective inference as a regulatory mechanism for rupture and repair, with curvature entropy as a sociomarker for interpersonalized psychiatry. Extensions consider humor as frame-shift resolution of mismatched manifolds, complex emotions as higher-order inoculations, geometric hyperscanning as interbrain curvature dynamics, and the ontogenetic parade as developmental curvature trajectories. Future directions include simulations of dyadic agents and ethical considerations for real-time relational tracking. By synthesizing these elements, we advance a unified geometry of meaning across computation, cognition, and interaction.
\end{abstract}

\tableofcontents

\part{Theoretical Foundations}

\section{Introduction}

High-dimensional data in machine learning and neuroscience often exhibit surprising structure: despite their apparent complexity, they frequently lie along low-dimensional latent manifolds within the ambient space. This \emph{manifold hypothesis} \cite{fefferman2016testing,gorban2018blessing,olah2014blog,cayton2005algorithms} explains the efficacy of dimensionality reduction techniques and the generalization capabilities of deep learning models \cite{chollet2021deep}. It posits that data requiring many variables for initial description can be captured by fewer variables tied to the local coordinates of an underlying manifold, facilitating continuous interpolation between samples—a key to robust inference.

However, traditional similarity metrics in embeddings (e.g., cosine distance) and correlation-based synchrony in neuroscience overlook the deeper geometric distortions that arise when mapping between manifolds. Curvature and entropy provide critical insights into these processes, not merely as mathematical abstractions but as socially relevant measures. In artificial intelligence, they explain interpretability failures and robustness issues; in psychiatry and therapy, they quantify relational attunement or rupture, offering tools for mental health interventions.

This essay extrapolates Riemannian curvature to semantic manifolds, viewing curvature as a measure of distortion in representational flows. This extrapolation reveals how mappings between manifolds—such as connectors in vision-language models (VLMs) or coupling in interbrain networks—induce geometric shear and entropy production. Inspired by recent geometric hyperscanning \cite{hinrichs2025geometry}, we model social interactions as dynamic reconfigurations of neural manifolds, where transitions in synchrony reflect affective regulation rather than dysfunction.

Our framework integrates active inference \cite{friston2017graphical}, where affect signals coherence in coupled generative systems, with an RSVP field theory (scalar capacity $\Phi$, vector flows $\mathbf{v}$, entropy $S$). We formalize connector losses as curvature-induced entropy, sheaf gluing as patch consistency, and Bayesian comparison as manifold interference. Extended explanations draw connections to therapy, where curvature entropy serves as a sociomarker for rupture-repair cycles, advancing interpersonalized psychiatry \cite{adel2025systematic}. Beyond clinical settings, we extend the same formal machinery to humor as the resolution of mismatched manifolds, to complex emotions as higher-order recursive inoculations, to geometric hyperscanning as interbrain curvature dynamics, and to the ontogenetic parade as developmental modulation of curvature regulation.

The main contributions are:
\begin{enumerate}
  \item A new definition of extrapolated curvature for semantic mappings, with entropy bounds and sheaf-theoretic interpretations.
  \item A mapping of embedding losses to RSVP fields, yielding design principles for geometry-preserving multimodal systems.
  \item Extensions to higher-order affective phenomena: humor as frame-shift resolution of mismatched manifolds, complex emotions as recursive inoculation operators, geometric hyperscanning as interbrain curvature dynamics, and the ontogenetic parade as developmental trajectories of curvature regulation.
  \item Applications to therapy, viewing affective inference as curvature-guided regulation of relational manifolds.
\end{enumerate}


The essay proceeds by first introducing the manifold hypothesis and defining extrapolated curvature, then developing its application to interbrain synchrony through geometric hyperscanning. We next turn to affective inference, showing how curvature regulation underlies rupture and repair in relational contexts, before mapping these dynamics into RSVP fields. Applications to therapy illustrate how curvature entropy can act as a sociomarker of co-regulation, while broader extensions connect the framework to humor as frame-shift resolution, to complex emotions as higher-order recursive inoculations, and to the ontogenetic parade of developmental fears as predictable curvature flows widened through learning and play. We close with future directions, conclusions, and a review of related work.

\section{Semantic Manifolds and the Manifold Hypothesis}

Semantic representations—whether in language models, neural activations, or cognitive processes—can be formalized as Riemannian manifolds equipped with additional structures to capture dynamics and information flow.

\subsection{Core Definitions}

Before presenting the formal definition, consider that a semantic manifold is
like a curved landscape where points represent meanings or states, distances
reflect semantic similarity, and the overall shape encodes how information
flows. This intuitive picture motivates the mathematical structure.

\begin{definition}[Semantic Manifold]
A semantic manifold is a quadruple $\mathfrak{M} = (X, g, \Psi, \mu)$, where $X$ is a smooth manifold, $g$ is a Riemannian metric, $\Psi$ is a field bundle (e.g., scalar-vector fields), and $\mu$ is a probability measure with density bounded on compact subsets.
\end{definition}

The manifold hypothesis asserts that high-dimensional data concentrate on such low-dimensional structures \cite{fefferman2016testing,gorban2018blessing}. This concentration reduces effective complexity: machine learning fits subspaces rather than the full ambient space, enabling interpolation via continuous paths \cite{chollet2021deep}. Extensions like the union of manifolds \cite{brown2023union} account for heterogeneous data, aligning with sheaf gluing for overlapping submanifolds.

In information geometry, these manifolds carry the Fisher metric $g_F$, quantifying sensitivity to parameter changes \cite{caticha2015geometry}. Under the free energy principle, manifolds are demarcated by Markov blankets, separating internal states from external environments \cite{kirchhoff2018markov}.

Expanded, this hypothesis implies that generalization arises from preserving manifold geometry under mappings. Distortions—measured by curvature—signal entropy production, linking to RSVP fields where $\Phi$ represents capacity, $\mathbf{v}$ flows, and $S$ dissipation.

\subsection{Examples of Semantic Manifolds}

In machine learning, token embeddings in large language models form low-dimensional manifolds where semantically similar concepts cluster. For instance, the embedding space of GPT-like models can be visualized as a curved surface where paths between tokens correspond to interpolation in meaning.

In neuroscience, EEG or fMRI time series are projected into latent manifolds, where neural states evolve along trajectories governed by dynamical systems. Conceptual categories in cognition, such as emotions or beliefs, form semantic manifolds where local neighborhoods represent related ideas.

Topology provides further insight: sheaf theory models how local charts (e.g., patch embeddings) glue into global coherence. The union of manifolds hypothesis \cite{brown2023union} extends this, suggesting data lie on intersecting submanifolds, with gluing conditions ensuring consistency. Geometric analyses of autoencoders \cite{lee2023geometric} show encoder-decoder pairs as approximate isometries, preserving manifold structure.

\subsection{Summary}

The manifold hypothesis bridges high-dimensional data to low-dimensional
geometric structures, providing the foundation for our extrapolated curvature
analysis. In the following sections, we build on this to formalize distortions
in semantic mappings and their social implications.

\section{Extrapolated Riemannian Curvature}

To quantify distortions in semantic mappings, we define extrapolated curvature as the deviation induced by projections between manifolds. Intuitively, think of this as measuring how much a map ``bends'' the meaning from one space to another, like stretching a rubber sheet until it tears.

\subsection{Formalization}

Let $F: (X, g) \to (Y, h)$ be a smooth map (e.g., VLM connector). The pullback metric is $F^* h$, and the distortion tensor is $\mathsf{D}_F = F^* h - g$. The extrapolated curvature tensor is $\mathcal{K}_F = \mathrm{Ric}_{F^* h} - \mathrm{Ric}_g$, with scalar $\kappa_F = \mathrm{Scal}(F^* h) - \mathrm{Scal}(g)$.

\begin{proposition}[Curvature and Entropy Bound]
Under bounded sectional curvature and positive reach, extrapolated curvature bounds connector entropy production $\sigma[F|\mu]$.
\end{proposition}

\begin{proof}[Sketch]
Curvature integrates metric distortions; entropy follows from transport inequalities linking $\kappa_F$ to distributional changes.
\end{proof}

This extrapolation extends discrete curvatures like Forman-Ricci \cite{forman2003bochner} to continuous manifolds, measuring how projections shear semantic geometry.

\subsection{Mathematical Expansion}

The Riemann tensor $R$ captures intrinsic geometry; Ricci $\mathrm{Ric}$ averages it over directions; scalar $\mathrm{Scal}$ contracts further. For embeddings, extrapolated curvature $\mathcal{K}_F$ quantifies how $F$ deforms the source manifold's geometry to match the target's.

\begin{lemma}[Distortion and Curvature]
For small $\|\mathsf{D}_F\|$, $\kappa_F \approx \Delta_g \|\mathsf{D}_F\| + O(\|\mathsf{D}_F\|^2)$, where $\Delta_g$ is the Laplace-Beltrami operator.
\end{lemma}

Proofs involve linearization of the curvature operator under metric perturbations. Category-theoretically, $F$ is a functor between manifold categories, with faithfulness reflecting information preservation.

\begin{tikzcd}
(X, g) \arrow[r, "F"] \arrow[d, "g_F"'] & (Y, h) \arrow[d, "g_F"] \\
(\mathcal{M}_X, g_F) \arrow[r, "\cong"'] & (\mathcal{M}_Y, g_F)
\end{tikzcd}

This diagram shows curvature as a natural invariant.

\section{Mapping to RSVP Field Theory}

RSVP models representations as fields: scalar $\Phi$ (capacity), vector $\mathbf{v}$ (flows), entropy $S$ (dissipation). Connectors are entropy-respecting functors; KNOR estimates global shear ($S$ increase), patch-loss local tears.

\subsection{Mathematical Correspondences}

Bi-Lipschitz bounds align with Lyapunov stability; rate-distortion reflects $S$ budgets; restricted isometry preserves negentropic corridors.

Empirically, 40-60\% neighbor divergence signals entropic shear; Procrustes failures indicate irreversible $S$.

Design: Minimize $S$ with curvature regularization; route negentropically task-aware.

Predictions: Conditional KNOR better predicts errors; corridor ablations reduce relevant losses.

Implications: Dual-use in interpretability and media; humans as lossy projectors.

\subsection{Functorial Correspondence}

Connectors as functors between RSVP categories:

\begin{tikzcd}
(X, g, \Phi, \mathbf{v}, S) \arrow[r, "F"] & (Y, h, \tilde{\Phi}, \tilde{\mathbf{v}}, \tilde{S})
\end{tikzcd}

Entropy production $S - \tilde{S}$ bounds curvature distortion.

\part{Social Applications}

\section{Geometric Hyperscanning and Interbrain Networks}

Hyperscanning simultaneously records neural signals from interacting individuals, revealing interbrain synchrony \cite{montague2002hyperscanning}. Traditional metrics are descriptive; geometric approaches offer mechanistic insights \cite{hinrichs2025geometry}.

\subsection{Discrete Curvature in Networks}

Interbrain graphs link regions across agents, weighted by synchrony. Forman-Ricci curvature quantifies expansion/contraction; negative values indicate bridges, positive dense regions. Entropy of curvature distributions detects phase transitions:
\[
H_{RC}(G_t) = -\int f^t_{RC}(x) \log f^t_{RC}(x) \, dx.
\]

Divergences in $H_{RC}$ signal rupture-repair, extending intra-brain analyses \cite{weber2019curvature,chatterjee2021detecting}.

Expanded, this geometry views dyads as coupled manifolds, with curvature flows routing information between shortest-path traversal and diffusion \cite{avena2019spectrum}.

\subsection{Simulation Example}

Consider a toy dyad modeled as small-world graphs with rewiring probability $p$. As $p$ increases from 0 (lattice) to 1 (random), $H_{RC}$ diverges around $p \approx 10^{-2}$, marking a transition from segregated to integrated topology. This mirrors social shifts from misalignment to attunement.

Topological data analysis adds persistent homology to distinguish transient vs. lasting structures, complementing curvature entropy.

\section{Affective Inference in Relational Dynamics}

Affect regulates dyadic coherence, signaling narrative alignment \cite{hinrichs2025hyperscanning}. Under active inference, curvature entropy informs belief updates about shared states.

In therapy, entropy peaks mark ruptures; repairs restore low-entropy gluing. This extends to groups via hierarchical manifolds, fusing multimodal cues in generative models.

\subsection{Expanded on Psychotherapy}

Rupture-repair cycles are foundational to therapeutic alliance. Curvature entropy quantifies misattunement \cite{bolis2017dialectical}, with peaks predicting breakdown and declines indicating resolution. Case study: therapist-client dyad where entropy spikes during conflict, guiding intervention.

Mathematically, two agents minimize variational free energy coupled by curvature signals:
\[
F[\pi] = \mathbb{E}_\pi [D_{KL}(q||p)] + H[\tilde{q}],
\]
where curvature contributes to the entropy term $H$.

Ethically, real-time detection requires consent and interpretability, avoiding reduction of relational autonomy to geometric metrics.

\subsection{Additional Worked Examples}

\paragraph{Group Therapy Vignette.}
In a family session, hierarchical manifolds model subgroup alignments. Entropy spikes in parent-child dyads propagate to group-level curvature, signaling systemic rupture. Intervention smooths these via targeted repair.

\paragraph{Social Anxiety Example.}
Social fears manifest as high curvature in interpersonal manifolds. Simulation shows how exposure therapy flattens ridges, reducing entropy.

\paragraph{Grief Processing.}
Grief as persistent gluing failure; therapy resolves via recursive inoculation, lowering cohort entropy.

\part{Affective Extensions}

\section{Humor as Resolution of Mismatched Manifolds}

Humor can be formalized as the resolution of mismatched semantic manifolds via frame-shift pattern matching. 
A joke establishes a primary interpretive manifold and then abruptly induces a shift to a competing manifold, 
forcing the cognitive system to reconcile divergent metrics. Laughter is modeled as the entropy release 
that accompanies this reconciliation.

\subsection{Manifold Interference}

Let semantic context be a Riemannian manifold $(M,g)$ with probability measure $\mu$ over interpretations.
A joke sets up a primary manifold $M_1$ with metric $g_1$, then abruptly induces a shift to $M_2$ with metric $g_2$.
The \emph{humor event} occurs at the interference region
\[
\mathcal{H} = M_1 \pitchfork M_2 
= \{ x \in M_1 \cap M_2 : g_1(x) \neq g_2(x) \}.
\]

Resolution requires a mapping $F: M_1 \to M_2$ minimizing distortion while preserving incongruity:
\[
\Delta g = F^\* g_2 - g_1.
\]

The laughter response is modeled as entropy release:
\[
L = \sigma[F|\mu] \propto \int_{\mathcal{H}} 
\log \det (I + g_1^{-1} \Delta g)\, d\mu.
\]

\subsection{Frame Shift as Pattern Matching}

Let $\{ \mathcal{F}_i \}$ denote interpretive frames, each a sheaf of local patches glued into a manifold of meaning.
The punchline acts as a functor
\[
P : \mathsf{Sheaf}(M_1) \to \mathsf{Sheaf}(M_2),
\]
reinterpreting a section $s$ under a different gluing law.

Humor arises when the transition function is non-trivial but still recognizable:
\[
t_{12}(s) \neq s, \quad d(s, t_{12}(s)) < \epsilon.
\]
That is, the shifted pattern is divergent yet matchable.

\subsection{RSVP Mapping}

In RSVP notation:
\begin{itemize}
  \item $\Phi$: scalar capacity --- potential to hold multiple manifold interpretations.
  \item $\mathbf{v}$: vector flows --- trajectory following one manifold then redirected to another.
  \item $S$: entropy --- mismatch cost when $g_1 \neq g_2$.
\end{itemize}

Humor is a \emph{negentropic corridor} where divergent trajectories re-align, releasing entropy as affective resolution:
\[
\text{Humor}(M_1,M_2) =
\min_{F} \big\{ \|\mathsf{D}_F\| : \Delta S(F) > 0 \big\},
\]
where $\mathsf{D}_F$ is the distortion tensor and $\Delta S$ the entropy gain.

\subsection{Psychological Implication}

\begin{itemize}
  \item Setup: low-entropy expectation on $M_1$.
  \item Punchline: sudden high curvature between $M_1$ and $M_2$.
  \item Resolution: recognition of overlap, entropy released as laughter.
\end{itemize}

This reframes incongruity theory in geometric-information terms: humor is the controlled rupture and repair 
of semantic manifold coherence, where the ``funny'' intensity corresponds to the curvature–entropy spike 
and its subsequent dissipation.

\section{Complex Emotions as Higher-Order Surprise Minimization}

We extend the recursive inoculation framework to model complex emotions as 
\emph{meta-inoculations} against anticipated classes of surprise. Let 
$\mathcal{I}$ denote the inoculation operator acting on a prior $q$, and 
$\mathcal{I}^d$ its $d$-fold composition:

\[
q^{(d)}(x) = \mathcal{I}^d(q)(x),
\]

where the recursion depth $d$ encodes the degree of preparation for 
higher-order uncertainties. Basic emotions correspond to $d=1$ 
(first-order inoculation), while complex emotions emerge for $d \geq 2$.

\subsection{Examples of Complex Emotions}

\paragraph{Guilt.}  
Formally, guilt is a depth-2 inoculation conditioned on 
counterfactual priors $\mu'$ over actions not taken:

\[
q_{\text{guilt}}(x) 
= \mathcal{I}^2 \big( q(x) \,\big|\, \mu' \neq \mu \big).
\]

Interpretation: guilt contracts action manifolds via negative curvature, 
redirecting flows $\mathbf{v}$ toward reparative pathways.

\paragraph{Awe.}  
Awe corresponds to high-capacity expansion of $\Phi$, producing 
singular curvature and volumetric expansion:

\[
\kappa_{\text{awe}} \to -\infty, 
\quad \mathrm{Vol}(\mathfrak{M}) \uparrow.
\]

Interpretation: awe arises from epistemic shock, reorganizing the 
semantic manifold under low predictability.

\paragraph{Nostalgia.}  
Nostalgia is recursive inoculation against surprise in temporal 
reconstructions:

\[
q^{(d)}_{\text{nost}}(x_t) 
= \mathcal{I}^d \big( q(x_{t-k}) \big), 
\quad k>0.
\]

Interpretation: nostalgia retroactively glues present states to 
past embeddings, reducing entropy by aligning current priors 
with remembered distributions.

\subsection{RSVP Field Mapping}

\begin{itemize}
  \item $\Phi$ (capacity): Upregulated in awe, downregulated in guilt.  
  \item $\mathbf{v}$ (flows): Retrocausal in nostalgia, contractive in guilt.  
  \item $S$ (entropy): Transiently increased in awe, suppressed in nostalgia, 
        rupture–repair dynamics in guilt.  
\end{itemize}

\subsection{Category-Theoretic View}

Let $\mathcal{E}$ be the category of emotional states, 
with objects = manifolds indexed by recursion depth $d$, 
and morphisms = inoculation operators:

\[
\mathcal{I}^d : \mathfrak{M}_i \to \mathfrak{M}_j.
\]

\begin{itemize}
  \item Guilt = morphism conditioned on counterfactual sheaves.  
  \item Awe = colimit expansion in $\mathcal{E}$.  
  \item Nostalgia = pullback functor along temporal fibrations.  
\end{itemize}

\subsection{Sheaf-Theoretic Integration}

Sheaves $\mathcal{F}$ over $\mathfrak{M}_{\text{emo}}$ encode 
local patches of affect:

\begin{itemize}
  \item Guilt = failed gluing, where local coherence cannot extend 
        to a global section.  
  \item Awe = successful gluing of disjoint patches via curvature blow-up.  
  \item Nostalgia = retroactive gluing aligning past and present stalks 
        through temporal restriction maps.  
\end{itemize}

\subsection{Taxonomy of Emotions}

\begin{table}[ht]
\centering
\caption{Recursive inoculation operators and their affective correspondences.}
\label{tab:inoculation}
\renewcommand{\arraystretch}{1.3}
\begin{tabularx}{\textwidth}{@{}l>{\centering\arraybackslash}X>{\centering\arraybackslash}X>{\centering\arraybackslash}X@{}}
\toprule
\textbf{Emotion} & \textbf{Recursive depth $d$} & \textbf{Operator form} & \textbf{Interpretation} \\
\midrule
Fear & $d=1$ &
$\;q^{(1)}(x) = \mathcal{I}(q)(x)\;$ &
First-order inoculation against immediate surprise. \\
\addlinespace
Guilt & $d=2$ &
$\;q^{(2)}(x) = \mathcal{I}^2\!\big(q(x)\mid \mu' \neq \mu\big)\;$ &
Second-order inoculation conditioned on counterfactual priors. \\
\addlinespace
Awe & variable $d$ &
$\;\kappa \to -\infty,\; \mathrm{Vol}(\mathfrak{M}) \uparrow\;$ &
Curvature singularity inducing expansion of semantic capacity. \\
\addlinespace
Nostalgia & $d>1$ &
$\;q^{(d)}(x_t) = \mathcal{I}^d\big(q(x_{t-k})\big),\; k>0\;$ &
Recursive inoculation over past states, retroactive temporal gluing. \\
\bottomrule
\end{tabularx}
\end{table}

\part{Empirical Methods}

\section{Simulation Details}

We provide pseudocode and Python implementations for constructing interbrain
graphs, computing curvatures, estimating entropy, and detecting phase
transitions. These can be adapted for EEG, fNIRS, or fMRI data.

\subsection{Pseudocode Pipeline}

[Insert the pseudocode from earlier response here]

\subsection{Python Implementation}

[Insert the Python snippets from earlier response here]

\subsection{Validation Methodology}

To validate curvature entropy as a sociomarker, compute effect sizes for entropy
divergences during known rupture events (e.g., Cohen's $d > 1.2$ for large
effects). Statistical power analysis suggests $N=20$ dyads suffice for 80\% power
at $\alpha=0.05$. Compare against baseline synchrony metrics to demonstrate
superior sensitivity.

\section{Hyperscanning Modality Comparison}

The interpretation of curvature signatures in interbrain networks is constrained
by the spatiotemporal sampling properties of different hyperscanning modalities.
Hinrichs et al.~\cite{hinrichs2025geometry} provide canonical ranges for
edge-weight magnitudes across electroencephalography (EEG), functional near-infrared
spectroscopy (fNIRS), and functional magnetic resonance imaging (fMRI), paired
with task vs.\ resting conditions. These ranges frame expectations for the
distribution of curvature values and their entropy across modalities.

\begin{table}[H]
\centering
\caption{Illustrative edge-weight ranges and implications for hyperscanning modalities under task and resting conditions (adapted from Hinrichs et al.~\cite{hinrichs2025geometry}).}
\label{tab:modality-comparison}
\begin{tabularx}{\textwidth}{@{}p{1.5cm} p{1.8cm} p{2.5cm} p{2cm} X@{}}
\toprule
\textbf{Modality} & \textbf{Condition} & \textbf{Edge-weight range} & \textbf{Timescale} & \textbf{Empirical implication} \\
\midrule
EEG   & Task    & PLV $\approx 0.2$--$0.6$ & tens--hundreds ms & Captures rapid, transient behaviour \\
EEG   & Resting & PLV $\approx 0.1$--$0.4$ & tens--hundreds ms & Spontaneous background activity \\
\addlinespace
fNIRS & Task    & Corr.\ $\approx 0.1$--$0.3$ & $\sim 0.1$--1 s & Suited to slower, block-like tasks \\
fNIRS & Resting & Corr.\ $< 0.2$ & $\sim 0.1$--1 s & Long-term spontaneous fluctuations \\
\addlinespace
fMRI  & Task    & Coh.\ $\approx 0.2$--$0.5$ & 1--2 s & Captures sustained blocks, too slow for fast events \\
fMRI  & Resting & Coh.\ $< 0.2$ & 1--2 s & Long-term resting-state networks \\
\bottomrule
\end{tabularx}
\end{table}


\paragraph{Interpretation.}
Because EEG yields higher temporal resolution, its curvature distributions are
expected to show sharper divergences in entropy during fast rupture--repair
episodes, whereas fNIRS and fMRI capture only slower topological reconfigurations.
Consequently, the entropy of Forman--Ricci curvature distributions should be
interpreted in light of the modality’s resolution: rapid synchrony shifts
manifest in EEG, gradual meso-scale reorganization in fNIRS, and long-term
resting-state topology in fMRI. This comparison highlights that curvature-based
hyperscanning must be modality-aware, with expectations for edge-weight
magnitudes and entropy divergences conditioned on measurement scale.

\section{Proof Sketches}
\label{sec:proof-sketches}

We sketch two complementary arguments: (i) Lyapunov stability of
\emph{negentropic corridors}---regions where a connector is near-isometric and
curvature production is small; (ii) entropy bounds obtained from
rate--distortion theory for mappings that incur nonzero distortion on
task-relevant patches.

\subsection{Lyapunov Stability for Negentropic Corridors}

Let $(X,g)$ be a semantic manifold and $F:(X,g)\to(Y,h)$ a $C^2$ connector.
Fix a compact, task-relevant patch $U\subset X$. We call $U$ a
\emph{negentropic corridor} for $F$ if the following hold for some
constants $0<\alpha\le \beta<\infty$ and $\varepsilon,\eta>0$:
\begin{equation}
\label{eq:corridor-conditions}
\alpha\,g_x(v,v)\ \le\ F^\!h_x(v,v)\ \le\ \beta\,g_x(v,v)
\quad\text{and}\quad
\|\mathcal{K}_F(x)\|_{\mathrm{op}}\le \eta,\ \ \|\mathrm{II}_F(x)\|\le \varepsilon,
\quad \forall x\in U,\,v\in T_xX,
\end{equation}
where $\mathcal{K}_F=\mathrm{Ric}_{F^\*h}-\mathrm{Ric}_g$ is the extrapolated
Ricci tensor and $\mathrm{II}_F$ the second fundamental form of the graph of
$F$ in $X\times Y$.

\begin{proposition}[Local Lyapunov function]
\label{prop:local-lyapunov}
Define $V(x):=\tfrac12\,d_h\!\big(F(x),\mathcal{M}\big)^2$, where
$\mathcal{M}\subset Y$ is a smooth embedded target submanifold encoding the
task constraint (e.g., an answer manifold or policy surface). Suppose $\mathcal{M}$
is \emph{$\lambda$-geodesically convex} in $(Y,h)$ on $F(U)$:
$\mathrm{Hess}_h\big(\tfrac12 d_h(\cdot,\mathcal{M})^2\big)\succeq
\lambda\, I$ on $F(U)$ for some $\lambda>0$. Consider the gradient flow on $X$
with respect to the pullback potential $V\circ F$:
\[
\dot x\ =\ -\,\nabla_g (V\circ F)(x).
\]
If the corridor inequalities \eqref{eq:corridor-conditions} hold with
$\alpha>0$ and $\|\mathcal{K}_F\|,\|\mathrm{II}_F\|$ sufficiently small, then
for all $x\in U$,
\[
\dot V(x)\ =\ \langle \nabla_h V(F(x))\,,\,dF_x\dot x\rangle_h
\ \le\ -\,\alpha\,\lambda\, \|\nabla_h V(F(x))\|_h^2\ \le\ 0.
\]
Hence $V$ is a strict Lyapunov function on $U$, and the set
$F^{-1}(\mathcal{M})\cap U$ is locally asymptotically stable.
\end{proposition}

\begin{proof}[Sketch]
By the chain rule,
$\nabla_g(V\circ F)=dF^\*\,\nabla_h V$, and the flow gives
$\dot x=-\,dF^\*\,\nabla_h V$. The metric sandwich
$F^\*h\simeq G$ with $G\in[\alpha,\beta]$ (Assumption \eqref{eq:corridor-conditions})
implies
$\|dF\,\dot x\|_h^2=\langle dF\,\dot x,dF\,\dot x\rangle_h
= \langle \dot x,\dot x\rangle_{F^\*h}\ge \alpha\,\|\dot x\|_g^2$.
Convexity of $\mathcal{M}$ yields
$\langle \nabla_h V, \nabla_h V\rangle_h\ge \lambda\, V$, so
$\dot V=-\langle \nabla_h V,\,dF\,dF^\*\,\nabla_h V\rangle_h
\le -\alpha \|\nabla_h V\|_h^2 \le -\alpha\lambda\, V$.
Small $\|\mathcal{K}_F\|,\|\mathrm{II}_F\|$ ensure stability of these
inequalities on $U$ (no curvature-induced loss of convexity).
\end{proof}

\paragraph{Input-to-state robustness.}
If $F$ is time-varying, $F_t$, with $\|dF_t-dF\|,\ \|\partial_t F_t\|$
bounded by $\delta$, the same argument yields
$\dot V\le -\alpha\lambda V + c\,\delta$ for some $c>0$, i.e.\ ISS with respect
to connector drift; thus negentropic corridors are \emph{robustly} attractive.

\begin{corollary}[Restricted isometry $\Rightarrow$ corridor stability]
\label{cor:ri-corridor}
If $F$ satisfies a restricted isometry on $U$,
$(1-\epsilon)\|v\|_g^2 \le \|dF_x v\|_h^2 \le (1+\epsilon)\|v\|_g^2$
with $\epsilon<1$, then $\alpha=1-\epsilon,\ \beta=1+\epsilon$ in
\eqref{eq:corridor-conditions}; hence $V$ is a Lyapunov function and
$F^{-1}(\mathcal{M})\cap U$ is locally asymptotically stable.
\end{corollary}

\subsection{Entropy Bounds from Rate--Distortion}

Let $(X,\mu)$ be a source with random variable $X\sim \mu$, and let
$Y=F(X)$ be the connector output on $(Y,h)$. Fix a nonnegative distortion
$d:Y\times Y\to\mathbb{R}_{\ge 0}$ and a target (reconstruction) variable
$\widehat Y$ with conditional law $q(\widehat y|y)$. The (Shannon) rate--distortion
function is
\[
R(D)\ :=\ \inf_{p(\widehat y|x)\,:\,\mathbb{E}[d(Y,\widehat Y)]\le D}\ I(X;\widehat Y).
\]
We connect $R(D)$ to curvature-induced metric distortion on patches.

\begin{assumption}[Patch-wise distortion budget]
\label{ass:patch-D}
Let $\{U_k\}$ be a cover of task-relevant regions with $\mu(U_k)=w_k$ and
local isometry constants $\alpha_k,\beta_k$ (as in \eqref{eq:corridor-conditions}).
Assume that on each $U_k$, any decoder achieves at best average distortion
$D_k\ge D^\ast_k(\alpha_k,\beta_k,\eta_k)$, where $\eta_k$ encodes the
local curvature/II bounds.
\end{assumption}

\begin{proposition}[Lower bound on information and entropy]
\label{prop:RD-lower}
Under Assumption~\ref{ass:patch-D},
\[
I(X;\widehat Y) \ \ge\ \sum_k w_k\, R_k(D_k),
\qquad
h(Y)\ \le\ h(X) + \mathbb{E}[\log J_F(X)],
\]
and if $R_k(\cdot)$ are strictly convex, the bound tightens to the Jensen
envelope $I(X;\widehat Y)\ge R(\sum_k w_k D_k)$.
\end{proposition}

\begin{proof}[Sketch]
Decompose $I(X;\widehat Y)=\sum_k \Pr[X\in U_k]\,
I(X;\widehat Y\,|\,X\in U_k)$ and apply the definition of $R_k(D_k)$ on each
patch. The entropy identity follows from change of variables; cf. Prop.
\ref{prop:entropy-jacobian}. The two statements combine to relate the
log-Jacobian budget to the required mutual information for a given per-patch
distortion.
\end{proof}

\paragraph{Geometric $R(D)$ estimates.}
On a $d$-dimensional Riemannian manifold,
quadratic distortion $d(y,\widehat y)=\|y-\widehat y\|_h^2$ yields
(in high-resolution regime) $R(D)\gtrsim \tfrac{d}{2}\log\!\big(\sigma_Y^2/D\big)$,
with $\sigma_Y^2$ the per-dimension variance of $Y$.
If $F$ contracts variance on a corridor ($\beta\approx 1$) but expands elsewhere,
the mixture lower bound
\(
I(X;\widehat Y)\ge \sum_k w_k \tfrac{d}{2}\log(\sigma_{Y,k}^2/D_k)
\)
forces larger information for patches with poor corridor geometry (large $D_k$).

\begin{corollary}[Connector entropy production vs.\ $R(D)$]
\label{cor:sigma-RD}
Let $\sigma[F\mid \mu]=h(Y)-h(X)$ be the entropy production.
Then for any decoder achieving average distortion $D=\sum_k w_k D_k$,
\[
\sigma[F\mid \mu]\ \ge\ -\,\sum_k w_k\, \log J_{F,k}^{-}\ -\ C
\quad\Longrightarrow\quad
I(X;\widehat Y)\ \ge\ R(D)\ \gtrsim\ \frac{d}{2}\log\!\frac{\sigma_Y^2}{D},
\]
where $J_{F,k}^{-}$ is the geometric mean of the reciprocal Jacobian over $U_k$,
and $C$ collects curvature-dependent constants (via Prop.~\ref{prop:graph-entropy}).
Hence larger curvature/metric distortion (smaller corridors) increases the
required rate for a target distortion budget.
\end{corollary}

\paragraph{Sheaf-consistent gluing penalty.}
If local decoders $\widehat Y_k$ violate overlap consistency on $U_i\cap U_j$,
a sheaf penalty $\|\widehat Y_i-\widehat Y_j\|^2$ induces an \emph{effective}
distortion $\widetilde D \ge D + \lambda \sum_{i<j} \mathbb{E}
[\|\widehat Y_i-\widehat Y_j\|^2\,\mathbf{1}_{U_i\cap U_j}]$, pushing $R(\widetilde D)$
upward. Thus, sheaf-consistent reconstruction \emph{lowers} the information
requirement at fixed accuracy.

\subsection{Takeaways}

Negentropic corridors---near-isometric regions with small extrapolated curvature
and bounded second fundamental form---admit a Lyapunov function that certifies
local asymptotic stability of task manifolds under gradient-like flows, and this
stability is robust to mild connector drift. Conversely, where corridors are
absent (large Jacobian/curvature distortion), rate--distortion lower bounds
force higher mutual information to attain a given accuracy, and change-of-variables
implies larger entropy production. Together these sketches justify the design
heuristics: curvature control, bi-Lipschitz regularization, and sheaf-consistent
patching minimize entropy and stabilize inference.

\newpage
\bibliographystyle{plain}
\bibliography{references}

\end{document}
