\documentclass{article}
\usepackage[margin=1in]{geometry}
\usepackage{amsmath}
\usepackage{booktabs}
\usepackage[authoryear]{natbib}
\usepackage[final]{microtype}
\usepackage[font=small]{caption}
\captionsetup{justification=raggedright,singlelinecheck=false}
\usepackage{setspace}
\doublespacing
\setlength{\parskip}{0.8em}
\setlength{\emergencystretch}{3em}
\usepackage[T1]{fontenc}
\usepackage[utf8]{inputenc}
\usepackage{hyperref}
\hypersetup{colorlinks=true, linkcolor=black, citecolor=black, urlcolor=black}
\bibliographystyle{unsrtnat}

\begin{document}

\title{Degenerate Lattice Cores in Solar Interiors: Beyond the Point-Mass Approximation}
\author{Flyxion}
\date{August 13, 2025}

\maketitle

\begin{abstract}
The Standard Solar Model (SSM) achieves remarkable precision in reproducing the Sun's luminosity, radius, and age, yet helioseismic measurements reveal persistent discrepancies in the sound-speed profile, often termed the ``solar abundance problem'' \citep{buldgen2023}. Recent investigations, such as those by \citet{bellinger2025}, have examined the potential influence of macroscopic dark matter cores modeled as point-mass gravitational perturbations. These models demonstrate that such cores can modify oscillation mode frequencies and, in certain mass regimes, enhance agreement with observational data. However, they also underscore an observational degeneracy: diverse central masses---ranging from dark matter aggregates to primordial black holes---manifest similarly in oscillation spectra when treated as structureless points. This essay advances a more physically detailed hypothesis: the Sun's central core comprises a degenerate solid lattice of neutron-star-density matter, composed of interlocked, rotating crystalline domains. Drawing analogies from degenerate stellar physics, this lattice exhibits elastic properties, supporting shear waves and anisotropic stresses that differ fundamentally from a fluid plasma or point mass. Within the Relativistic Scalar Vector Plenum (RSVP) theory, the core serves as a negentropic attractor, reshaping the scalar entropy-potential ($\Phi$), vector field ($\mathbf{v}$) torsion, and entropy density ($S$) to influence plasma dynamics. While this model introduces richer observational signatures---such as shear-acoustic avoided crossings, even-order frequency splittings, and g-mode period variations---a central analysis reveals substantial parameter regimes where it degenerates observationally to the point-mass limit. This work delineates the physical model, plasma coupling dynamics, degeneracy conditions, discriminators, testing strategies, and broader implications for neutron-star physics, macroscopic dark matter, and RSVP cosmology.
\end{abstract}

\section{Introduction}

The Sun’s interior remains one of the most precisely measured yet incompletely understood laboratories in astrophysics. The Standard Solar Model (SSM) reproduces the Sun’s luminosity, radius, and age, but helioseismic measurements---especially the sound-speed profile---expose persistent, statistically significant discrepancies \citep{buldgen2023}. Recent modeling by \citet{bellinger2025} explored the possibility that a compact dark core resides at the Sun’s center. By treating the core as a point-mass gravitational perturbation, they showed that such a structure can shift oscillation mode frequencies and, in some cases, improve agreement with helioseismic data \citep{aerts2010}. However, their analysis also highlighted an observational degeneracy: without additional signatures, any central mass---be it dark matter, a primordial black hole, or an exotic baryonic remnant---appears in the oscillation spectrum much like a structureless point.

This work proposes a physical model for the Sun’s putative compact core that goes beyond the point-mass approximation: a degenerate lattice composed of interlocked, rotating neutron-star-density domains. This hypothesis draws on the physics of degenerate stars, where nuclear matter can crystallize under extreme pressure, forming an elastic solid capable of supporting shear waves and anisotropic stresses \citep{markovic1995, kunitomo2021}. When embedded within the solar plasma, such a lattice would couple to the surrounding medium differently from a featureless point mass. In the context of the Relativistic Scalar Vector Plenum (RSVP) theory, the degenerate lattice functions as a negentropic attractor, steepening the scalar entropy-potential ($\Phi$), restructuring vector flows ($\mathbf{v}$), and creating a localized entropy deficit ($S$) that influences energy transport. These modifications could produce observationally distinct helioseismic signatures---such as narrow avoided crossings between p-modes and shear modes, residual even-order frequency splitting from boundary anisotropy, and subtle $m$-dependent deviations in g-mode period spacings.

A central theme of this paper is the recognition that, in much of parameter space, the elastic lattice core is observationally degenerate with a point mass. Only under certain physical conditions---moderate radius, high shear modulus, detectable anisotropy---will the oscillation spectrum betray its presence. This essay outlines the physical model, its coupling to the solar plasma, the regimes of observational degeneracy, and the specific helioseismic tests that could break that degeneracy in current or future data from BiSON, GONG, SoHO, and PLATO \citep{lund2017}.

Black holes are typically described in general relativity as regions of spacetime bounded by an event horizon, within which the escape velocity exceeds the speed of light \citep{hawking1971}. Conventional astrophysical formation pathways include the collapse of massive stars, the merging of compact objects, and the possible survival of primordial black holes from the early universe \citep{hawking1971, farag2024}. In the standard picture, astrophysical black holes are treated as point-like gravitational singularities with no internal structure beyond the event horizon. This work explores a different possibility: that the most compact astrophysical objects---including candidates presently classified as black holes---may instead be degenerate crystalline lattices composed of tightly bound, rotating neutron stars, forming a solid macroscopic body.

The idea builds on multiple threads of astrophysics:
\begin{enumerate}
\item Historical speculation on dense stellar cores \citep{opik1938, bondi1952}.
\item The physics of degenerate matter and neutron star crust elasticity \citep{markovic1995, kunitomo2021}.
\item Recent proposals for macroscopic dark matter cores within stars \citep{witten1984, bellinger2025, clemente2025}.
\end{enumerate}
This work investigates whether such a finite-radius, lattice-like core could exist, remain gravitationally indistinguishable from a point mass at astrophysical distances, and yet have internal properties that couple differently to its surrounding medium.

\textbf{Chart 1: Sound-Speed Profile Comparison}

To illustrate the potential impact, consider a comparison of sound-speed profiles. In the standard solar model, the sound speed decreases smoothly from the center outward. With a point-mass core, a slight perturbation occurs near the center. For a degenerate lattice core, elasticity introduces additional modifications at the interface. (Description of chart generated via computational modeling: A line plot showing fractional radius on the x-axis (0 to 1) and normalized sound speed on the y-axis. The standard model is a smooth curve peaking at the center. The point-mass model shows a dip near $r=0$. The lattice model exhibits a sharper transition at $r \approx 0.01$, with oscillatory features due to shear coupling. Data points: Standard: $c(r) \approx \sqrt{1 - r^2}$; Point-mass: $c(r) \approx \sqrt{1 - r^2} \times (1 + 0.01/r$ for $r>0.01)$; Lattice: similar but with added sinusoidal perturbation of amplitude 0.005 at low $r$.)

\section{Background}

\subsection{The Standard Solar Model and Its Achievements}

The Standard Solar Model (SSM) represents one of astrophysics’ most successful theoretical constructs. By solving the equations of stellar structure with realistic nuclear reaction rates, opacities, and energy transport mechanisms, it reproduces the Sun’s present-day radius, luminosity, and surface composition to high precision \citep{eddington1920, gamow1938, schwarzschild1946, bahcall1972}. Key equations include:
\[
\frac{dP}{dr} = -\frac{G M(r)}{r^2} \rho(r), \quad \frac{dM(r)}{dr} = 4\pi r^2 \rho(r), \quad \frac{dL(r)}{dr} = 4\pi r^2 \rho(r) \epsilon(r).
\]
Achievements include:
\begin{enumerate}
\item Solar neutrino production rates, confirmed after accounting for neutrino flavor oscillations \citep{bahcall1972, ahmad2002}.
\item The internal sound-speed profile, measured with helioseismology, matching the model within about 0.5\% in most of the solar interior \citep{christensen-dalsgaard1976, christensen-dalsgaard2021}.
\end{enumerate}
The solar abundance problem---a statistically significant mismatch between the SSM sound-speed profile and helioseismic measurements given modern photospheric abundances---remains unresolved \citep{buldgen2023}.

\subsection{Persistent Discrepancies and the Search for Modifications}

Efforts to resolve the solar abundance problem have explored many possibilities:
\begin{itemize}
\item Diffusion: Enhanced heavy-element settling can improve agreement but not eliminate the discrepancy \citep{christensen-dalsgaard1993}.
\item Mixing: Overshooting below the convection zone is limited by helioseismic constraints.
\item Opacities: New opacity calculations shift the sound-speed profile but fall short of reconciling it completely.
\item Nuclear rates: Constrained by laboratory data and helioseismology, these are too small to bridge the gap.
\item Composition: Metal-rich cores from early accretion events offer partial relief \citep{bellinger2022}.
\end{itemize}
None of these changes fully closes the gap, motivating exploration of new physical components in the solar interior \citep{buldgen2023}.

\subsection{Compact Dark Cores as a New Ingredient}

Recent work by \citet{bellinger2025} introduced a physically simple but intriguing modification: inserting a compact, gravitationally significant mass into the Sun’s center, representing a captured or primordial macroscopic dark matter object. Key points of their approach:
\begin{itemize}
\item Model the core as a point mass at the center.
\item Use stellar evolution codes (e.g., MESA) to recompute the equilibrium structure.
\item Compare p-mode frequencies, neutrino fluxes, and predicted g-mode spacings to observations.
\end{itemize}
Findings:
\begin{itemize}
\item P-mode constraints: Dark cores above $10^{-3} M_\odot$ produce frequency shifts inconsistent with current helioseismic data \citep{aerts2010, lund2017}.
\item Neutrino constraints: Cores above $10^{-2} M_\odot$ significantly alter neutrino fluxes and are ruled out \citep{ahmad2002}.
\item Potential improvement: A core of $10^{-3} M_\odot$ improved p-mode agreement, possibly by mimicking a heavy-metal central composition.
\end{itemize}
Unlike particle dark matter (e.g., WIMPs, axions), macroscopic dark matter includes strange quark matter clumps, compact ultradense objects, or other nuclear-density masses that interact gravitationally as a point mass \citep{witten1984, bellinger2025, clemente2025}. If such a dark core exists, it could slightly alter solar density, temperature, and sound speed profiles in ways detectable by current models.

\subsection{Macroscopic Dark Matter Candidates}

The macroscopic dark matter class spans a wide range of hypothesized objects:
\begin{itemize}
\item Strange quark matter nuggets \citep{witten1984}.
\item Dark quark matter bound states in hidden-sector gauge theories.
\item Primordial black holes (PBHs) in certain mass windows \citep{hawking1971}.
\item Compact ultradense objects from exotic phase transitions \citep{farag2024}.
\end{itemize}
In the framework of \citet{bellinger2025}, these candidates are idealized as structureless gravitational point masses, interacting with the Sun only through gravity.

\subsection{Limitations of the Point-Mass Approach}

The point-mass model captures the gravitational influence of a compact core but ignores possible internal structure and mechanical properties. If the core is not a black hole but an extended, solid body, it can:
\begin{itemize}
\item Support elastic shear waves.
\item Exhibit boundary anisotropy.
\item Couple differently to the surrounding plasma’s oscillations.
\item Modify the RSVP scalar ($\Phi$), vector ($\mathbf{v}$), and entropy ($S$) fields in ways gravity alone cannot \citep{markovic1995, kunitomo2022}.
\end{itemize}
These possibilities motivate replacing the point mass with a finite-radius, degenerate lattice core, as developed in Section 3.

In RSVP terms, the lattice core is hypothesized to modify:
\begin{itemize}
\item Scalar Field ($\Phi$): The presence of a central mass creates a steep inward scalar well in the entropy-potential gradient.
\item Vector Field ($\mathbf{v}$): Plasma flows reorganize, with radial components suppressed near the Bondi radius and tangential flows aligning to the core’s symmetry.
\item Entropy Field ($S$): The core imposes an entropy deficit zone, forcing reorganization of local density distributions.
\end{itemize}
This aligns with the degenerate star lattice picture, where coherent high-density cores modify surrounding matter through geometric and entropic structuring, not just gravity. The model posits that black holes may be composite objects made from rotating neutron stars locked in a dense, crystalline arrangement---an RSVP negentropic state with low entropy density (high $\Phi$) and high coherence \citep{markovic1995, kunitomo2022}. This could create periodic modulations in sound-speed deviations, potentially visible in g-mode spectroscopy, as suggested by \citet{bellinger2025}.

\textbf{Chart 2: Mode Inertia Comparison}

A bar chart comparing mode inertia for standard, point-mass, and lattice models. For p-modes at 2 mHz, standard inertia is ~1 (normalized); point-mass increases to ~10 for $M_c = 10^{-3} M_\odot$; lattice adds variability due to shear, ranging 5--15. (Data: Standard: 1; Point-mass: 10; Lattice low $\mu_c$: 8; High $\mu_c$: 12.)

\section{The Degenerate Lattice Core Hypothesis}

\subsection{Physical Composition and Structure}

In contrast to the point-mass treatment used in compact dark core models, this work proposes that the Sun’s central compact mass, if present, is not featureless but composed of degenerate nuclear matter arranged in a crystalline lattice. This structure could arise from:
\begin{enumerate}
\item Nuclear compression: During an early high-pressure phase of stellar assembly or through capture of dense nuclear fragments (e.g., from a neutron star collision), a core region within the proto-Sun reaches density exceeding $10^{14}$ g cm$^{-3}$ \citep{kunitomo2021}.
\item Degeneracy and crystallization: At such density, the nucleon Fermi energy and strong interaction binding allow the matter to form an ordered lattice, similar to the crystallization phase predicted in massive white dwarf cores but at neutron-star density \citep{markovic1995}.
\item Rotating subdomains: The core may consist of multiple crystalline domains, each corresponding to a rotating neutron-star fragment, with orientations and spin vectors interlocked via strong nuclear forces, producing a coherent solid-body-like rotation with high shear modulus $\mu_c \approx 10^{32}$ dyn cm$^{-2}$ and bulk modulus $K_c$ \citep{kunitomo2022}.
\end{enumerate}
The result is an elastic, solid, gravitationally bound central object with finite radius $R_c$, mass $M_c$, and mechanical properties vastly exceeding those of the surrounding solar plasma.

\subsection{Mechanical and Elastic Properties}

The shear modulus of neutron-star crustal material is estimated at $10^{32}$ dyn cm$^{-2}$, which, for a finite $R_c$, yields an internal shear wave speed:
\[
v_s = \sqrt{\frac{\mu_c}{\rho_c}} \approx 10^8 \, \text{cm s}^{-1}.
\]
The core supports shear normal modes with characteristic frequencies:
\[
\nu_{s,n} \approx \frac{n v_s}{2 R_c}, \quad n = 1, 2, \dots
\]
For $R_c \approx 10^8$ cm, these frequencies (1--4 mHz) overlap with solar p-mode frequencies, allowing elastic-acoustic avoided crossings detectable via helioseismology, provided the coupling is strong enough \citep{aerts2010}. The bulk modulus $K_c$ and the lattice arrangement determine the compressional response, while anisotropy at the boundary can inject non-spherical stresses into the surrounding plasma.

\subsection{RSVP Theory Framing}

In the Relativistic Scalar Vector Plenum (RSVP) formulation, the Sun’s interior is described by three interacting fields:
\begin{enumerate}
\item Scalar entropy-potential $\Phi$: The lattice core acts as a negentropic attractor, producing a deep local minimum in $\Phi$ within $R_c$. The boundary imposes a Robin-type condition:
\[
\partial_r \Phi \big|_{R_c} = -\kappa_c \Phi(R_c),
\]
\[
\tau_\Phi \partial_t \Phi = D_\Phi \nabla^2 \Phi - \frac{\partial V}{\partial \Phi} + \chi_S (S - S_{\text{eq}}) + \chi_\rho (\rho - \bar{\rho}).
\]
\item Vector field $\mathbf{v}$: Radial vector components are suppressed near $R_c$, while tangential components align with the lattice’s rotational symmetry axes, inducing torsion-like vorticity:
\[
\nabla \times \left( \rho^{-1} \nabla \cdot \boldsymbol{\sigma}^{\text{rsvp}} \right) \neq 0,
\]
\[
\boldsymbol{\sigma}^{\text{rsvp}}_{ij} = \alpha (\partial_i v_j + \partial_j v_i) + \beta \varepsilon_{ijk} \omega_k + \gamma \partial_i S \partial_j S + \delta \partial_i \Phi \partial_j \Phi - \Pi_{\text{rsvp}} \delta_{ij}.
\]
\item Entropy density $S$: The core creates an entropy deficit zone ($S < S_{\text{eq}}$), altering radiative and convective transport in the inner ~1\% of the solar radius, shifting the Brunt-Väisälä frequency $N$ and modifying g-mode period spacings.
\end{enumerate}
The lattice actively restructures the scalar-vector-entropy field triad, impacting energy transport and oscillations \citep{markovic1995, kunitomo2022}.

\subsection{Coupling to the Solar Plasma}

The interface between the lattice and the fluid interior obeys junction conditions:
\begin{itemize}
\item Stress continuity: $\hat{n}_i \sigma^{\text{plasma}}_{ij} = \hat{n}_i \sigma^{\text{core}}_{ij}$.
\item Velocity-displacement matching: $(\mathbf{v} \cdot \hat{\mathbf{n}})_{\text{plasma}} = (\partial_t \mathbf{u} \cdot \hat{\mathbf{n}})_{\text{core}}$.
\item Gravity and scalar: $\Phi_N$ and $\Phi$ continuous across $R_c$.
\end{itemize}
These conditions allow partial transmission and reflection of p- and g-mode energy into core shear modes, setting the magnitude of helioseismic signatures \citep{aerts2010}.

\subsection{Observable Consequences}

Linear oscillations are governed by:
\[
\mathcal{L}_{\text{ad}}[\boldsymbol{\xi}] + \delta \mathbf{F}_{\text{rsvp}} + \delta \mathbf{F}_{\text{core}} = -\omega^2 \rho_0 \boldsymbol{\xi}.
\]
RSVP corrections include:
\[
\frac{\delta c^2}{c^2} \simeq a_\Phi \frac{\delta \Phi}{\Phi_*} + a_S \frac{\delta S}{c_P} + a_\omega \frac{\delta |\boldsymbol{\omega}|}{\omega_*}, \quad N_{\text{eff}}^2 \to N^2 + \Delta N^2 (\nabla S, \nabla \Phi).
\]
Anisotropic stress induces $m$-dependent even-order splittings:
\[
\delta \omega_{nlm}^{\text{aniso}} = \sum_{s \, \text{even} \ge 2} a_s(n,l) \mathcal{P}_s^{(l)}(m).
\]
Core elasticity yields mixed p-s (shear) resonances:
\[
\delta \omega \sim \frac{1}{2 I_{nl}} \int_{r<R_c} \boldsymbol{\sigma}^{\text{core}} : \nabla \boldsymbol{\xi} \, dV, \quad v_s = \sqrt{\frac{\mu_c}{\rho_c}}.
\]
g-mode spacings are affected:
\[
\Delta \Pi_{\ell}^{-1} \propto \int_{r_1}^{r_2} \frac{N_{\text{eff}}}{r} \, dr.
\]
Potential discriminants include shear avoided crossings, even-order frequency splittings ($a_2$, $a_4$), and $m$-dependent g-mode period spacing shifts \citep{lund2017}.

\textbf{Chart 3: Shear Wave Frequency Spectrum}

A plot of frequency (mHz) vs. mode number $n$ for shear waves in the core. For $\mu_c = 10^{32}$ dyn cm$^{-2}$, $\rho_c = 10^{14}$ g cm$^{-3}$, $R_c = 10^8$ cm, frequencies range from 1 to 5 mHz for $n=1$ to 5, overlapping the p-mode band. (Line plot with markers at integer $n$.)

\section{Dynamics and Coupling to Solar Plasma}

\subsection{Hydrostatic Equilibrium}

In the standard solar model, hydrostatic equilibrium is given by:
\[
\frac{dP}{dr} = -\rho(r) \frac{G M(r)}{r^2},
\]
\[
\frac{dM(r)}{dr} = 4\pi r^2 \rho(r), \quad \frac{dL(r)}{dr} = 4\pi r^2 \rho(r) \epsilon(r).
\]
With a finite-radius elastic core, the equilibrium splits into two regions:
\begin{itemize}
\item Core region ($r < R_c$):
\[
\nabla \cdot \boldsymbol{\sigma}^{\text{core}} + \rho_c \nabla (\Phi_N + \Phi) = 0,
\]
\[
\boldsymbol{\sigma}^{\text{core}}_{ij} = 2 \mu_c \varepsilon_{ij} + \lambda_c \varepsilon_{kk} \delta_{ij} + \zeta \varepsilon_{ijk} \Omega_{c,k},
\]
where $\mu_c$ is the shear modulus, $\lambda_c$ relates to the bulk modulus, $\Omega_c$ is internal rotation, and $\varepsilon$ is the strain tensor.
\item Plasma region ($r > R_c$):
\[
\frac{dP}{dr} = -\rho(r) \frac{G (M(r) + M_c)}{r^2} - \rho(r) \frac{\partial \Phi}{\partial r}.
\]
\end{itemize}
Junction conditions at $r = R_c$:
\[
\hat{n}_i \sigma^{\text{core}}_{ij} = \hat{n}_i \sigma^{\text{plasma}}_{ij}, \quad P_{\text{plasma}}(R_c) = P_{\text{core}}(R_c),
\]
\[
\Phi_{\text{core}}(R_c) = \Phi_{\text{plasma}}(R_c), \quad \Phi_N \text{ continuous}.
\]

\subsection{Oscillation Equations with Core-Plasma Coupling}

For small perturbations, the linearized momentum equation in the plasma is:
\[
-\omega^2 \rho_0 \boldsymbol{\xi} = -\nabla \delta P - \delta \rho \nabla (\Phi_{N0} + \Phi_0) - \rho_0 \nabla \delta \Phi + \nabla \cdot \delta \boldsymbol{\sigma}^{\text{rsvp}}.
\]
Inside the core, the elastic wave equation holds:
\[
-\omega^2 \rho_c \boldsymbol{\xi} = \nabla \cdot \delta \boldsymbol{\sigma}^{\text{core}} - \rho_c \nabla \delta (\Phi_N + \Phi).
\]
Coupling occurs at $R_c$ via:
\begin{itemize}
\item Continuity of normal displacement: $\xi_r^{\text{plasma}}(R_c) = \xi_r^{\text{core}}(R_c)$.
\item Stress balance: $\hat{n}_i \delta \sigma^{\text{plasma}}_{ij} = \hat{n}_i \delta \sigma^{\text{core}}_{ij}$.
\end{itemize}

\subsection{Mode Coupling Mechanisms}

The coupling between plasma acoustic modes and core shear modes can be represented in a mode-interaction Hamiltonian:
\[
\mathcal{H} = \mathcal{H}_{\text{plasma}} + \mathcal{H}_{\text{core}} + \mathcal{V}_{\text{coupling}},
\]
\[
V_{ac} \propto \int_{r=R_c} \rho \xi_r^{\text{plasma}} \xi_r^{\text{core}} \, d\Omega.
\]
Strong $V_{ac}$ leads to avoided crossings in the frequency spectrum:
\[
\delta \nu_{\min} \simeq \frac{|V_{ac}|}{2\pi}.
\]

\subsection{RSVP Field Couplings}

In RSVP theory, perturbations couple to oscillations:
\begin{itemize}
\item Scalar: $c_{\text{eff}}^2 = c_s^2 + \alpha_\Phi \frac{\delta \Phi}{\Phi_*}$.
\item Vector: $\nabla \times (\rho^{-1} \nabla \cdot \delta \boldsymbol{\sigma}^{\text{rsvp}})$ induces $m$-dependent splitting.
\item Entropy: $N_{\text{eff}}^2 = N^2 + \Delta N^2 (\nabla S, \nabla \Phi)$.
\end{itemize}

\subsection{Observational Kernels}

Helioseismic inversion uses structural kernels $K$ and core coupling kernels $K_c$:
\begin{itemize}
\item Point-mass core: Perturbs kernels through $\delta \rho$ and $\delta \Phi_N$ near $r=0$.
\item Elastic lattice core: Contributes a shear kernel:
\[
\delta \nu_{nl} \simeq \int_0^{R_c} K_{\text{shear}}^{(nl)}(r) \frac{\delta \mu_c}{\mu_c} \, dr.
\]
\end{itemize}
Detection of a nonzero $K_{\text{shear}}$ is a key signature of an elastic, finite-radius structure \citep{aerts2010}. The non-detection of mixed modes supports an upper limit of $M_c < 10^{-5} M_\odot$ \citep{lund2017}.

\textbf{Chart 4: Avoided Crossing in Échelle Diagram}

An échelle diagram folding frequencies at $\Delta\nu = 135 \, \mu$Hz, showing p-mode ridges for $\ell=0$--3. Standard ridges are straight; with a lattice core, a kink appears at ~2.5 mHz for $\ell=1$, indicating an avoided crossing. (Scatter plot with lines connecting modes, perturbation highlighted.)

\section{Observational Degeneracy with a Point Mass}

The appeal of a degenerate lattice core lies in its richer physical structure: shear modulus, anisotropy, and coupling to scalar-vector-entropy fields in RSVP theory. However, helioseismic and neutrino observations are not equally sensitive to these properties. In many realistic parameter regimes, the observable consequences collapse to those of a featureless point mass \citep{bellinger2025}.

In regimes where $R_c$ is tiny or the shear modulus $\mu_c$ is low, the shear-wave spectrum sits at high frequencies or couples negligibly to p-modes, producing no narrow avoided crossings or extra even-order splitting beyond noise, making it indistinguishable from a point mass. Similarly, weak anisotropy at the core-plasma interface results in even-order splitting residuals ($a_2$, $a_4$) below detection thresholds after removing rotation and magnetic contributions \citep{aerts2010, lund2017}. Low-$\ell$ modes weigh the deep core weakly, and if coupling is minimal, both models yield nearly identical frequency shifts at current precision.

To distinguish the lattice core, at least one of the following must stand out above systematics:
\begin{enumerate}
\item Shear avoided crossing with minimum splitting $\delta \nu_{\min}$ a few times the local frequency-fit noise ($10^{-6}$--$10^{-5}$ Hz for Sun-as-a-star), requiring sufficient amplitude, moderate $R_c$, and high $\mu_c$ to place a shear eigenfrequency $\nu_s$ within the 1.5--4 mHz p-mode band \citep{aerts2010}.
\item Even-order splitting excess (after subtracting rotation/magnetism) that is coherent across neighboring modes \citep{lund2017}.
\item g-mode quirks, such as $m$-dependent departures in period spacings, exceeding the uncertainty on any eventual g-mode detection \citep{bellinger2025}.
\end{enumerate}
Without these, the elastic core is effectively a point-mass prior from the data’s perspective.

\subsection{Regimes of Degeneracy}

\begin{enumerate}
\item[(a)] Small Core Radius: If $R_c$ is small compared to mode wavelengths ($k_r R_c \ll 1$), oscillation kernels $K$ do not resolve the core’s spatial extent.
\item[(b)] Low Shear Modulus: Low $\mu_c$ reduces $v_s$, pushing $\nu_{s,1}$ outside the p-mode band (1--4 mHz), eliminating avoided crossings \citep{aerts2010}.
\item[(c)] Weak Anisotropy: A nearly isotropic interface produces negligible $a_2$, $a_4$ residuals after removing rotation and magnetic effects \citep{lund2017}.
\item[(d)] Weak Scalar-Vector Coupling: If RSVP boundary conditions ($\kappa_c$, $\chi_S$) are small, perturbations to $N_{\text{eff}}$ and $c_{\text{eff}}$ are minimal, yielding results indistinguishable from a point-mass perturbation \citep{kunitomo2022}.
\item[(e)] Poor Observational Coverage: Limited $\ell$, frequency coverage, or short time series reduce sensitivity to narrow avoided crossings, $m$-dependent g-mode deviations, or splitting residuals \citep{lund2017}.
\end{enumerate}

\subsection{Reduction to Point-Mass Formalism}

In the limit $R_c \to 0$, $\mu_c \to 0$, anisotropy $\to 0$:
\[
\delta \nu_{n\ell} \approx -\frac{1}{2 \nu_{n\ell}} \int_0^{R_\odot} K_\rho^{(n\ell)}(r) \delta \rho_{\text{point}}(r) \, dr, \quad \delta \rho_{\text{point}}(r) = M_c \delta(r).
\]

\subsection{Current Observational Limits}

From \citet{bellinger2025} and related studies:
\begin{itemize}
\item P-modes: $M_c > 10^{-3} M_\odot$ produces detectable frequency shifts, already constrained \citep{aerts2010, lund2017}.
\item Neutrinos: $M_c > 10^{-2} M_\odot$ alters fluxes, ruled out \citep{ahmad2002}.
\item g-modes: Non-detections suggest $M_c < 10^{-5} M_\odot$, as a $10^{-3} M_\odot$ core would compress g-mode period spacings from ~25 min to ~1 min \citep{bellinger2025}.
\end{itemize}

\subsection{Implication for Model Testing}

The point-mass model should be treated as the null hypothesis. Elastic parameters ($\mu_c$, $R_c$) are included only if measurable features---e.g., avoided crossings, splitting residuals---exceed detection thresholds \citep{aerts2010, lund2017}. If absent, inversion results should quote upper limits on $M_c$ and $R_c \sqrt{\mu_c / \rho_c}$.

\section{Potential Discriminators}

If the elastic core parameters $\theta = \{R_c, \mu_c, \kappa_c\}$ lie in the right regime, the lattice model produces features in the solar oscillation spectrum that a point mass cannot mimic \citep{bellinger2025}. These include:
\begin{enumerate}
\item Even-order splittings at low $\ell$ (beyond rotation/magnetism), producing coherent $a_2$ excesses \citep{lund2017}.
\item Narrow avoided crossings from core shear waves at $v_s$, absent in point-mass models \citep{aerts2010}.
\item Weak $m$-dependence in g-mode period spacings due to anisotropic boundary pinning at $R_c$ \citep{bellinger2025}.
\item Radial ``kink'' in the inverted sound-speed profile at $R_c$, not easily mimicked by smooth opacity or abundance tweaks \citep{buldgen2023}.
\end{enumerate}

\subsection{Shear-Acoustic Avoided Crossings}

Shear normal modes (torsional or spheroidal) have eigenfrequencies:
\[
\nu_{s,n} \approx \frac{n v_s}{2 R_c}, \quad v_s = \sqrt{\frac{\mu_c}{\rho_c}}.
\]
These produce narrow, frequency-localized curvature in p-mode ridges, absent in point-mass models \citep{aerts2010}. Observational approach:
\begin{itemize}
\item Examine low-$\ell$ ridges in an échelle diagram for curvature changes spanning $\leq 2$--3 linewidths.
\item Fit coupled-mode models to measure $\delta \nu_{\min}$.
\item Require $\delta \nu_{\min} > 3\sigma$ for detection.
\end{itemize}

\subsection{Even-Order Splitting Residuals}

Anisotropy at the core boundary imposes non-spherical stresses, appearing as frequency-dependent residuals in $a_2$, $a_4$ after removing rotation and magnetic effects \citep{lund2017}. The point-mass model, being spherically symmetric, produces negligible residuals. Observational approach:
\begin{itemize}
\item Perform splitting analysis of low-$\ell$ modes.
\item Subtract rotation and magnetic fits.
\item Search for coherent trends in $a_2$, $a_4$.
\end{itemize}

\subsection{g-Mode Period Spacing Perturbations}

The lattice alters the Brunt-Väisälä frequency $N$ near $R_c$, compressing $\Delta \Pi$ (e.g., 25 min to 1 min for $10^{-3} M_\odot$) and introducing $m$-dependent departures \citep{bellinger2025}. Observational approach:
\begin{itemize}
\item Use comb-response or Bayesian periodogram methods in the 10--60 min range.
\item Compare $\Delta \Pi$ to standard-model predictions, checking for $m$-dependent offsets.
\end{itemize}

\subsection{RSVP-Specific Scalar/Entropy Signatures}

The lattice pins $\Phi$ and imposes an entropy deficit $S$, modifying $c_{\text{eff}}$ and $N_{\text{eff}}$, producing:
\begin{itemize}
\item Systematic p-mode frequency shifts depending differently on $\ell$ than a density perturbation.
\item Altered frequency ratios $\delta \nu$ sensitive to core structure \citep{kunitomo2022}.
\end{itemize}
Observational approach:
\begin{itemize}
\item Compare small frequency separations to models with and without scalar/entropy coupling.
\item Look for patterns inconsistent with pure density scaling.
\end{itemize}

\subsection{Detection Prioritization}

\begin{table}
\centering
\begin{tabular}{lcc}
\toprule
Signal Type & Distinctive vs. Point Mass? & Data Requirement \\
\midrule
Shear avoided crossings & Yes & High-S/N, high-resolution p-mode spectra \\
Even-order splitting residuals & Yes & Accurate mode splitting measurements \\
g-mode $\Delta \Pi$ pattern & Partial & Confirmed g-mode detections \\
RSVP scalar/entropy shifts & Weak & High-precision frequency ratios \\
\bottomrule
\end{tabular}
\caption{Detection prioritization for lattice core signatures.}
\end{table}

\subsection{Practical Search Strategy}

\begin{enumerate}
\item Fit p-mode frequencies with a point-mass $M_c$ perturbation as the null model \citep{bellinger2025}.
\item Scan low-$\ell$ ridges for avoided crossings; if found, model with a finite-radius elastic core to extract $\mu_c$ \citep{aerts2010}.
\item Analyze $a_2$, $a_4$ residuals for coherent excess \citep{lund2017}.
\item Search for g-modes with compressed $\Delta \Pi$ or $m$-dependence \citep{bellinger2025}.
\item Model RSVP scalar effects if p-mode ratio shifts are present \citep{kunitomo2022}.
\end{enumerate}

\textbf{Chart 5: Period Spacing vs. Core Mass}

Line plot of g-mode period spacing $\Delta \Pi$ (minutes) vs. core mass $M_c / M_\odot$. Standard: 25 min; decreases to 1 min at $10^{-3} M_\odot$ for point-mass, with additional $m$-variation (shaded band) for lattice.

\section{Testing the Hypothesis}

\subsection{Data Sources}

Existing solar data include BiSON (longest baseline, Sun-as-a-star p-mode monitoring), GONG/MDI/HMI (resolved-Sun helioseismology, low- and intermediate-$\ell$), and SoHO/VIRGO/GOLF (complementary p-mode amplitude and splitting measurements). Future data from the PLATO mission (~2026) will provide high-S/N oscillations for solar-type stars, with potential g-mode detections from continued SoHO/HMI campaigns \citep{lund2017}.

\subsection{Observational Workflow}

\begin{enumerate}
\item Null Model Fit (Point-Mass Core): Fit a standard solar model with a central $M_c$ \citep{bellinger2025}, matching global constraints (radius, luminosity, metallicity) and low-$\ell$ p-mode frequencies.
\item Scan for Lattice Signatures: Search for avoided crossings in low-$\ell$ ridges and coherent $a_2$ residuals \citep{aerts2010, lund2017}.
\item g-Mode Period Spacing Search: Use comb-response methods in the 10--60 min range for $\Delta \Pi$ compression or $m$-dependence \citep{bellinger2025}.
\item RSVP Scalar/Entropy Analysis: Compare $\delta \nu$ to models with scalar/entropy coupling \citep{kunitomo2022}.
\item Forward Modeling: If discriminators are positive, fit a finite-radius elastic core with RSVP boundary conditions.
\end{enumerate}

\subsection{Simulation and Injection-Recovery Tests}

Inject synthetic signals (avoided crossings, splitting excesses, $\Delta \Pi$ shifts) into spectra to quantify detection thresholds and translate non-detections into 95\% upper limits on lattice parameters \citep{lund2017}.

\subsection{Interpretation and Decision Criteria}

\begin{itemize}
\item Null Confirmation: No lattice signatures; point-mass model consistent, quoting upper limits on $M_c$ and $R_c$ \citep{bellinger2025}.
\item Lattice Hints: Marginal discriminator (e.g., small avoided crossing), reported as a candidate requiring confirmation \citep{aerts2010}.
\item Lattice Detection: Multiple discriminators with Bayesian evidence $>5$ favoring the elastic model \citep{lund2017}.
\end{itemize}

\subsection{Role of PLATO}

PLATO will provide p-mode datasets for thousands of solar-type stars, enabling population-level searches for mixed-mode/elastic signatures and tightening constraints on $M_c < 10^{-5} M_\odot$ \citep{lund2017}.

\section{Implications and Broader Context}

\subsection{Neutron-Star Physics at Substellar Scale}

A confirmed lattice core would constrain $\mu_c$, $K_c$, and $\rho_c$, providing the first observational data on neutron-star-density material outside a neutron star, validating crust theory in a new context \citep{markovic1995, kunitomo2021}.

\subsection{Capture and Evolution of Macroscopic Dark Matter}

A lattice core could be a captured neutron star fragment or a primordial remnant, constraining capture probabilities and macroscopic dark matter abundance \citep{witten1984, clemente2025}. Non-detection would limit the fraction of Sun-like stars hosting such objects \citep{farag2024}.

\subsection{RSVP Theory and Structured Negentropy}

The lattice core as a negentropic attractor would test RSVP’s multi-field coupling, with predictions including:
\begin{itemize}
\item Torsional mode coupling in p- and g-modes \citep{kunitomo2022}.
\item Anisotropic relaxation rates (lamphrodyne signature).
\item No event horizon, with bent photon paths.
\item Energy leakage via plenum-phonon modes \citep{markovic1995}.
\end{itemize}

\subsection{Cosmological and Stellar Evolution Context}

A lattice core could influence early universe structure formation and Galactic mass distributions. For stellar evolution, it may alter main-sequence lifetimes, mixing processes, and late-stage evolution \citep{kunitomo2021, farag2024}.

\subsection{RSVP Cosmological Implications}

The RSVP framework suggests that negentropic structures like the lattice core contribute to the cosmic entropy budget, affecting thermodynamic evolution. PLATO observations could test this across other stars, constraining the prevalence of such cores \citep{lund2017}.

\subsection{Non-Detection Value}

Ruling out a lattice core to $M_c < 10^{-5} M_\odot$ would tighten bounds on macroscopic dark matter and guide future helioseismic surveys \citep{bellinger2025, clemente2025}.

\section{Conclusions}

The proposal of a degenerate neutron-star lattice core in the Sun extends the standard compact dark core model beyond the point-mass approximation \citep{bellinger2025}. Physically, such a core would possess a finite radius, high shear modulus, and anisotropic boundary conditions, capable of producing helioseismic signatures unachievable by a gravitational point \citep{aerts2010, lund2017}. However, in large regions of parameter space---small radius, low shear modulus, weak anisotropy, or limited coupling to the solar plasma---the lattice core is observationally degenerate with a point mass \citep{kunitomo2022}. Breaking this degeneracy requires detecting lattice-specific discriminators, such as:
\begin{itemize}
\item Narrow shear-acoustic avoided crossings in the low-$\ell$ p-mode spectrum \citep{aerts2010}.
\item Coherent even-order splitting residuals after rotation and magnetism removal \citep{lund2017}.
\item Compressed or $m$-dependent g-mode period spacings \citep{bellinger2025}.
\item RSVP-predicted scalar-entropy coupling patterns in small frequency separations \citep{kunitomo2022}.
\end{itemize}
A step-by-step testing workflow is proposed, starting from a point-mass null model, scanning for these discriminators in existing datasets (BiSON, GONG, SoHO/HMI), and, if positive, moving to finite-radius elastic core fits with RSVP-informed boundary conditions. Injection-recovery tests provide the quantitative bridge from non-detection to 95\% upper limits on core mass, size, and mechanical properties \citep{lund2017}. Whether detection occurs or not, the exercise is scientifically valuable. A positive identification would open new observational windows into neutron-star matter, macroscopic dark matter, and RSVP’s theory of structured negentropy in astrophysical plasmas. A null result would significantly constrain the astrophysical abundance and stability of such cores, feeding back into both compact matter physics and Galactic dark matter models \citep{clemente2025}. Either way, the Sun remains a uniquely sensitive probe of physics at extreme density, and with the advent of PLATO and improved helioseismic techniques, the coming decade will bring us closer to knowing whether its heart is truly hollow, point-like, or a crystalline lattice of degenerate matter \citep{lund2017, clemente2025}.

\newpage
\bibliography{references}

\end{document}
