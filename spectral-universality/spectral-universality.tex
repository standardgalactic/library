\documentclass[11pt]{article}

\usepackage{luatextra}
\usepackage{fontspec}
\setmainfont{Latin Modern Roman}

\usepackage{amsmath,amssymb,amsthm,mathtools,bm}
\usepackage{geometry}
\usepackage{setspace}
\usepackage{csquotes}
\usepackage{hyperref}
\usepackage{enumitem}
\usepackage{physics}
\usepackage{tikz}
\usetikzlibrary{arrows.meta, positioning, shapes, decorations.pathreplacing}

\geometry{margin=1in}
\setstretch{1.15}

\usepackage{float}
\setcounter{topnumber}{10}
\setcounter{bottomnumber}{10}
\setcounter{totalnumber}{20}
\renewcommand{\topfraction}{0.95}
\renewcommand{\bottomfraction}{0.95}
\renewcommand{\textfraction}{0.05}
\renewcommand{\floatpagefraction}{0.85}


\usepackage[
  backend=biber,
  style=authoryear,
  sorting=nyt
]{biblatex}

\addbibresource{references.bib}


\tikzset{
  opnode/.style={
    rectangle,
    rounded corners=6pt,
    draw=black!80,
    thick,
    minimum width=3.6cm,
    minimum height=1cm,
    align=center,
    inner sep=4pt,
    font=\small
  },
  annot/.style={font=\scriptsize, align=center},
  arrow/.style={-{Latex[length=3mm]}, thick}
}

\title{\Large\bfseries Spectral Universality in Complex Systems:\\
Random Matrices, Cortical Resonance, and RSVP Dynamics}

\author{Flyxion}
\date{\today}

\begin{document}
\maketitle

\begin{abstract}
This article develops a unified theory of complex systems grounded in the
phenomenon of spectral universality. We show that nuclear Hamiltonians, the
zeros of the Riemann zeta function, and the oscillatory dynamics of the
mammalian cortex all exhibit universal spectral statistics that are insensitive
to microscopic detail. Random Matrix Theory (RMT) provides the mathematical
bridge linking these systems: Wigner--Dyson statistics describe both the energy
levels of excited nuclei and the spacings of zeta zeros.

Recent advances in systems neuroscience demonstrate that the cortex functions as
a structured resonance chamber whose standing and traveling waves organize
cognition and consciousness. Building on the work of Earl K.~Miller and
ultrafast fMRI evidence from Cabral, Shemesh, and collaborators, we argue that
cortical computation is fundamentally wave-based and analog.

We integrate these domains within the Relativistic Scalar--Vector--Plenum
(RSVP) framework. A linearized RSVP operator contains nuclear, arithmetical, and
cortical operators as symmetry-reduced subcases, providing a field-theoretic
unification of random matrices, brain waves, and number theory. Appendices
present mathematical foundations, operator derivations, symmetry reductions,
numerical methods, and empirical evidence.
\end{abstract}

% ======================================================================
% SECTION 1: Introduction
% ======================================================================

\section{Introduction}

Heavy atomic nuclei, the prime numbers, and the mammalian cortex appear to have
little in common. Yet all three systems exhibit an astonishing and still poorly
understood property: despite enormous microscopic complexity, each produces
simple, universal spectral patterns.

In nuclear physics, Wigner's bold proposal to model heavy nuclei by random
symmetric matrices succeeded spectacularly: the Wigner--Dyson spacing laws
predicted the empirical distributions of nuclear energy levels with striking
accuracy \parencite{wigner1957,dyson1962}. In analytic number theory, Montgomery
and Odlyzko discovered that the pair-correlation statistics of the nontrivial
zeros of the Riemann zeta function match those of the Gaussian Unitary Ensemble
(GUE), suggesting a deep connection between primes and quantum chaos
\parencite{montgomery1973,odlyzko1987,berry1986,FirkMiller2009NucleiPrimesRMT}.

At the same time, neuroscience has undergone a paradigm shift. Traditional
connectionist models resemble digital machine-learning architectures: synaptic
weights encode information, and networks compute by discrete spiking. Earl
K.~Miller and others have overturned this view. The cortex operates as a
dynamic, wave-based analog computer whose large-scale oscillations coordinate
activity across millions of neurons. Ultrafast fMRI and electrophysiological
recordings reveal coherent standing and traveling waves on the cortical sheet
\parencite{cabral2023intrinsic,Trongnetrpunya2022TravelingWavesPFC,Han2024RotatingWavesAttention}.

These observations suggest a striking unification:
\begin{quote}
\emph{Many complex systems --- nuclei, primes, cortical tissue, and field
models --- reduce to spectral systems whose behavior is governed not by
microscopic details but by universal invariants encoded in the spectra of their
dynamical operators.}
\end{quote}

The aim of this article is to make this unification explicit, culminating in the
RSVP (Relativistic Scalar--Vector--Plenum) field-theoretic view. The core steps
of the argument are as follows:

\begin{enumerate}
    \item Random Matrix Theory (RMT) captures the universal spectral
    phenomenology of complex quantum systems and the zeros of $L$-functions.
    \item The cortex behaves as a resonance chamber whose eigenmodes define
    low-dimensional manifolds controlling cognition.
    \item Wave-based analog computation organizes cortical dynamics and
    consciousness, as argued by Miller.
    \item RSVP provides a field-theoretic operator whose symmetry reductions
    recover the nuclear, zeta, and cortical operators.
\end{enumerate}

We begin with a systematic reformulation of complex systems as spectral systems.
Then we trace the connections from nuclei to primes to cortical waves, and
finally embed the entire picture inside the RSVP operator hierarchy.

% ======================================================================
% SECTION 2: Abstract Spectral Systems
% ======================================================================

\section{Abstract Spectral Systems}

To place systems as varied as heavy nuclei, $L$-functions, cortical resonance patterns, and RSVP fields within a single mathematical framework, we begin with an abstract template for a spectral system. Let $X$ denote an arbitrary physical, biological, or mathematical entity whose behavior can be captured through the spectral properties of an operator. We associate to $X$ a quadruple
\[
    (\mathcal{H}_X,\, \mathcal{A}_X,\, L_X,\, \mu_X),
\]
in which each component encodes a different structural layer of the system.

The space $\mathcal{H}_X$ is a Hilbert space representing the admissible states or modes of the system—vibrational states in a nucleus, automorphic states in analytic number theory, cortical eigenmodes in neuroscience, or perturbative RSVP field configurations in the plenum. The algebra $\mathcal{A}_X$ consists of observables acting on $\mathcal{H}_X$, providing the measurable or dynamically relevant quantities. The operator $L_X$ governs the intrinsic dynamics or constraints: it may be a Hamiltonian in nuclear physics, a hypothetical differential operator whose spectral data encode zeta zeros, a reduced operator capturing cortical wave propagation, or the full linearized RSVP operator. Finally, the spectral measure $\mu_X$ derived from $L_X$ determines how eigenvalues distribute and how energy, variance, or probability are resolved across scales.

This abstract template encompasses the concrete examples of interest:
\[
\begin{aligned}
    X = \text{heavy nucleus},  &\qquad L_X = H_{\mathrm{nuclear}}, \\
    X = \text{zeta system},    &\qquad L_X = L_{\zeta} \; (\text{hypothetical}), \\
    X = \text{cortex},         &\qquad L_X = L_{\mathrm{cortex}}, \\
    X = \text{RSVP field},     &\qquad L_X = L_{\mathrm{RSVP}}.
\end{aligned}
\]
In each case, the essential statistical fingerprint of the system is encoded in the $k$-point eigenvalue correlation functions,
\[
    R_k^{(X)}(\lambda_1,\dots,\lambda_k),
\]
which describe the joint fluctuations of eigenvalues at arbitrary order. Two systems belong to the same universality class precisely when these correlation functions coincide after unfolding the spectrum to unit mean density. In practice, such agreement reveals that the systems share the same fluctuation statistics despite differing profoundly in their microscopic dynamics.

Under this formulation, universality becomes a structural property of the operator $L_X$ rather than an accident of physical resemblance. Heavy nuclei, zeta zeros, cortical oscillations, and RSVP modes can therefore be compared, classified, and ultimately integrated through the common language of operator spectra. 


The foundational hypothesis is that many systems with radically different microscopic mechanisms nevertheless give rise to identical local spectral statistics. Phenomena that appear unrelated at the level of their physical substrates can, at the level of eigenvalue structure, fall into the same universality class. This is why nuclear energy levels, the zeros of the Riemann zeta function, cortical standing waves, and even the modal structure of the RSVP field all exhibit the same correlation functions and spacing laws. 

What emerges is a genuine mathematical bridge: a shared spectral language capable of linking physics, number theory, neuroscience, and field theory. Universality makes it possible to treat these domains not as isolated disciplines but as different manifestations of a single underlying order.


% ======================================================================
% SECTION 3: Random Matrices and Nuclear Spectra
% ======================================================================

\section{Random Matrices and Nuclear Spectra}

\subsection{Wigner’s Model of Heavy Nuclei}

Heavy nuclei like $^{238}\mathrm{U}$ have extremely complex many-body
Hamiltonians with billions of interacting terms. Wigner's insight was that
fine-grained nuclear interactions are effectively unpredictable; only the
symmetries matter. He therefore replaced the nuclear Hamiltonian by a random
matrix drawn from one of the Gaussian ensembles (GOE, GUE, GSE).

Let $H$ be an $N \times N$ Hermitian random matrix with entries drawn from a
Gaussian distribution with variance $\sigma^2$. As $N \to \infty$, the empirical
spectral distribution converges to the semicircle law:

\[
    \rho_{\mathrm{sc}}(x)
    = \frac{1}{2\pi\sigma^2}
      \sqrt{4\sigma^2 - x^2}\,\mathbf{1}_{|x|\le 2\sigma}.
\]

\subsection{Wigner--Dyson Spacing}

At the microscopic scale, the spacing between consecutive eigenvalues exhibits
level repulsion. In the GOE and GUE cases, the spacing distributions are:

\[
    P_{\mathrm{GOE}}(s)
    = \frac{\pi}{2} s \exp\left(-\frac{\pi}{4}s^2\right),
\]
\[
    P_{\mathrm{GUE}}(s)
    = \frac{32}{\pi^2} s^2 \exp\left(-\frac{4}{\pi}s^2\right).
\]

These spacing laws are universal: they match experiments measuring the neutron
resonance energies of heavy nuclei.

\subsection{Universality}

The key result is that universality does not depend on:
\begin{itemize}
    \item the distribution of matrix entries,
    \item the physical details of the system,
    \item the microscopic interactions.
\end{itemize}

Only the symmetry class matters. This is the foundation for the bridge to
number theory and cortical dynamics.

% ======================================================================
% SECTION 4: Zeta Zeros and the Polya--Hilbert Framework
% ======================================================================

\section{Zeta Zeros and the Polya--Hilbert Framework}

\subsection{Montgomery’s Pair-Correlation}

In 1973, Montgomery computed the pair-correlation function of the nontrivial
zeros $\frac{1}{2} + i\gamma_n$ of the Riemann zeta function. Assuming the
Riemann Hypothesis, the normalized ordinates $\gamma_n$ satisfy:

\[
    R_2^{(\zeta)}(s)
    = 1 - \left( \frac{\sin \pi s}{\pi s} \right)^2
    + \delta(s),
\]

which is precisely the GUE pair-correlation law discovered by Dyson.

\subsection{Odlyzko’s Numerical Evidence}

Odlyzko’s computations of billions of zeros confirmed that:
\[
    \text{spacing statistics of zeta zeros}
    = \text{spacing statistics of the GUE}.
\]

Thus:
\[
    [\zeta]_{\mathrm{spec}} = [\mathrm{GUE}]_{\mathrm{spec}}.
\]

This identifies the distribution of prime numbers with the universal spectral
laws of quantum-chaotic Hamiltonians.

\subsection{The Polya--Hilbert Conjecture}

The Polya--Hilbert philosophy posits a self-adjoint operator $L_\zeta$ whose
spectrum reproduces the imaginary parts of zeta zeros:
\[
    \sigma(L_\zeta) = \{\gamma_n\}.
\]
If such an operator exists, then the primes become a spectral phenomenon:
\[
    \text{primes} \;\longleftrightarrow\;
    \text{eigenvalues of a quantum-chaotic operator}.
\]

\subsection{Firk--Miller Integration}

Firk and Miller \parencite{FirkMiller2009NucleiPrimesRMT} emphasize that:
\begin{itemize}
    \item heavy nuclei,
    \item zeta zeros,
    \item random matrices
\end{itemize}
form a unified spectral family governed by Wigner--Dyson laws.

Their exposition demonstrates that number theory and nuclear physics occupy the
same universality class. This universality is the bridge we will carry into the
neuroscientific and RSVP domains.

% ======================================================================
% SECTION 5: Intrinsic Resonant Modes in the Cortex
% ======================================================================

\section{Intrinsic Resonant Modes in the Cortex}

Recent advances in ultrafast imaging and large-scale electrophysiology have begun to reveal a striking picture of the resting-state cortex: rather than behaving as a stochastic, unconstrained network of local interactions, it operates as a structured resonance chamber whose intrinsic modes shape the flow of information. The cortex’s spontaneous activity is organized not merely by connectivity but by the spectral geometry of the sheet itself.

\subsection{Cabral–Shemesh Standing Waves}

Ultrafast fMRI, with temporal resolution on the order of \(38\,\mathrm{ms}\), has demonstrated that low-frequency macroscale oscillations spontaneously arrange themselves into coherent spatial eigenmodes that span the cortical mantle \parencite{cabral2023intrinsic}. These modes recur with remarkable reliability across time and individuals, forming stable spatial patterns reminiscent of the vibrational modes of a physical resonator. Their organization is not arbitrary: the shape and coherence of these modes shift systematically under anesthesia, indicating that they are tightly coupled to the global dynamical state of the brain.

Moreover, these standing-wave structures appear to underwrite functional connectivity itself. The large-scale network architecture typically inferred from correlations in BOLD or electrophysiological signals emerges naturally from the interference and superposition of the cortex’s intrinsic eigenmodes. Functional networks are not imposed on the cortex; they arise as a secondary expression of its operator spectrum.

Taken together, these observations strongly suggest that a well-defined operator \(L_{\mathrm{cortex}}\) governs the macroscopic wave dynamics of the brain, and that its eigenmodes—rather than individual synaptic events—provide the true basis of cortical organization. The resonance properties of this operator offer a principled explanation for both the stability of functional networks and the systematic alterations observed during changes of consciousness.

\subsection{Traveling Waves in Working Memory}

Electrophysiological studies have shown that during working-memory engagement, the prefrontal cortex does not simply maintain information through persistent firing. Instead, it generates coherent traveling waves in the beta and gamma bands \parencite{Trongnetrpunya2022TravelingWavesPFC}. These waves sweep across prefrontal circuits in structured trajectories, carrying content-specific information as they propagate. Their motion coordinates spatially distributed cell assemblies, binding them into unified functional units. In effect, the waves act as dynamic traffic signals, opening and closing channels of communication as required by the task. Rather than storing memory as static patterns of activation, the PFC maintains information through evolving waveforms whose geometry encodes both content and context.

\subsection{Rotating and Spreading Waves in Attention}

A complementary line of evidence comes from studies of attention and post-distraction recovery. Rotating cortical waves—circular, spiral, or spreading patterns of activity—have been observed to modulate the rapid reallocation of attentional resources \parencite{Han2024RotatingWavesAttention}. When attention is disrupted, these waves sweep across cortical regions, re-establishing coherent patterns of activity and stabilizing the system into a new focus. Their functional role resembles that of control fields: they orchestrate reset operations, re-synchronize distributed ensembles, and steer the cortex back into task-relevant configurations. These patterns reveal that attentional control is mediated not by isolated top-down commands but by large-scale wave dynamics that reorganize the cortical state.

\subsection{Spectral Interpretation}

Together, these observations motivate a shift in how cortical computation is conceptualized. The waves observed in working memory, attention, and resting-state dynamics are not incidental epiphenomena; they are eigenfunctions of an underlying neural-field operator whose spectral properties determine the frequencies and propagation characteristics of cortical oscillations. Cognitive processes arise from the interaction, interference, and modulation of these eigenmodes.

\begin{quote}
Cortical waves are eigenfunctions of a neural-field operator; their eigenvalues
encode oscillatory frequencies, and cognition emerges from the structured
interaction of these modes.
\end{quote}

On this view, the cortex is fundamentally a spectral system. Its computational capacity derives not from the combinatorics of synaptic graphs but from the geometry and dynamics of its operator spectrum.

% ======================================================================

\section{Consciousness as a Spectral Phase}

The wave-based framework suggests that consciousness corresponds not to a
localized mechanism but to a global spectral phase of the cortical operator.
Let $L_{\mathrm{cortex}}$ have eigenpairs
\[
    L_{\mathrm{cortex}} \psi_k = \lambda_k \psi_k.
\]

Define a coherence order parameter
\[
    \mathcal{C} =
    \frac{
        \left| \sum_{k \in B} e^{i \phi_k} \right|
    }{
        \sum_{k \in B} 1
    },
\]
where $B$ indexes frequency bands and $\phi_k$ are band-specific phases.

Empirically:
\[
    \mathcal{C}_{\mathrm{conscious}} \approx 1,
    \qquad
    \mathcal{C}_{\mathrm{anesthetized}} \ll 1.
\]

Thus consciousness corresponds to a high-coherence spectral regime across
multiple bands, while unconsciousness corresponds to a spectral phase collapse.
This formalizes the empirical findings that anesthesia drives the cortex into a
low-dimensional, incoherent mode distribution.


% ======================================================================

\section{Spectral Geometry of the Plenum Manifold}

The RSVP framework assumes that the scalar, vector, and entropy fields evolve on
a geometric substrate $(M,g)$ whose curvature influences the structure of
eigenmodes. This situates RSVP within the broader domain of spectral geometry:
the study of how the geometry of a manifold determines the spectrum of its
differential operators.

In classical settings, the Laplace--Beltrami operator $\Delta_g$ provides the
canonical link between shape and spectrum. Its eigenfunctions form a basis for
oscillatory modes on $M$, and the associated eigenvalues encode the geometric
and topological features of the domain. Weyl’s law gives the asymptotic density
of eigenvalues and provides a spectral signature of the manifold's volume and
dimension.

In RSVP, the operator $L_{\mathrm{RSVP}}$ generalizes $\Delta_g$ by introducing
cross-field couplings and anisotropic propagation speeds. Nevertheless, its
spectrum retains geometric sensitivity: curvature induces mode splitting,
gyri–sulci geometry modifies cortical eigenmodes, and global topology shapes
the long-range structure of the plenum’s resonant modes.

Thus RSVP unifies cortical spectral geometry and field-theoretic dynamics in a
single analytic framework.


% ======================================================================
% SECTION 6: Earl K. Miller’s Wave-Based Analog Computation
% ======================================================================

\section{Earl K.~Miller’s Wave-Based Analog Computation}

We now integrate the full theoretical synthesis of Miller’s framework, expanded
with explicit parallels to physics and number theory.

\subsection{From Synaptic Computation to Wave Computation}

The classical neural paradigm treats the brain as a kind of synaptic digital network, a system in which information is stored in synaptic weights and computation unfolds through discrete local operations. In this view, perception feeds into cognition, cognition feeds into action, and the entire architecture can be understood as essentially feedforward with modulatory feedback superimposed on top.

Miller’s critique is that this picture cannot account for the most salient features of real cortical processing. It struggles to explain the rapid and global coordination that occurs across widely separated regions, the remarkable flexibility with which the brain reconfigures itself in response to shifting context, and the mechanisms underlying conscious access and executive control. Synaptic transmission alone is too slow, too local, and too granular to generate the large-scale, fast, and smoothly integrated dynamics that cognition clearly requires.

The alternative paradigm reframes the brain not as a digital network but as a spatiotemporal resonance system. In this framework, oscillatory electric fields—rather than synapses—constitute the primary currency of computation. Waves, not weights, organize the flow of information.

\subsection{Three Levels of the Wave-Based Paradigm}

\subsubsection*{1. Waves Coordinate the Cortex}

On the first level, oscillating electric fields provide a mechanism for coordinating cortical activity. These waves synchronize widely distributed neuronal ensembles, regulate which communication pathways are open or gated at any given moment, and dynamically assemble the transient coalitions that underlie perception, memory, and action. Rather than a static wiring diagram, the cortex becomes an actively tuned resonant medium, continually re-shaping itself through the propagation and interference of rhythmic activity.

\subsubsection*{2. PFC as a Generative Model Builder}

Miller and Cohen’s account of prefrontal function emphasizes that the PFC does far more than store rules or maintain working-memory items. It constructs abstract generative models of tasks, inferring the structure of goals and contingencies, and using these models to predict which future actions are most likely to succeed. Crucially, the PFC does not implement this control through isolated synaptic commands; rather, it modulates the activity of other cortical regions by shaping their oscillatory dynamics. Through selective amplification and suppression of particular rhythms, it steers the brain into the configurations required for planning, reasoning, and flexible decision-making.

\subsubsection*{3. Mixed Selectivity and Dimensional Reduction}

Even though individual neurons exhibit highly heterogeneous and mixed selectivity, the collective activity of large neural populations collapses onto a surprisingly low-dimensional manifold. What appears microscopically intricate often resolves, at the population level, into only three to five dominant latent modes. These modes are not arbitrary abstractions but correspond directly to the major spectral components of the underlying cortical waves. In effect, the brain’s apparent complexity is continuously compressed by its own oscillatory dynamics, yielding a manageable set of resonant degrees of freedom that support robust and flexible cognition.

\subsection{The Brain as an Analog Computer}

From the wave-based perspective, the cortex functions less like a digital circuit and more like an analog computer whose primitives are patterns of interference. Oscillations combine, cancel, and reshape one another according to the geometry of their overlap, implementing operations that in digital architectures would require explicit addition, subtraction, filtering, and routing. In place of symbolic manipulation, the brain performs computation through the constructive and destructive interference of continuous fields:
\[
    \text{wave\;overlap\;geometry}
    \;\longrightarrow\;
    \text{analog\;operations\;on\;information}.
\]

This mode of computation confers several distinctive advantages. Because the underlying signals are continuous and graded, the representational space of the cortex is both smooth and high-dimensional. Because waves superpose, the system naturally supports massively parallel processing, with numerous patterns evolving simultaneously without the need for discrete time-stepped updates. And because the substrate is biological tissue operating near thermodynamic limits, the system achieves extraordinary energy efficiency: the brain performs on the order of $10^{15}$ operations per second while consuming only about twenty watts.

By contrast, contemporary digital AI systems are fundamentally discrete and input-driven. They operate through static arrays of parameters rather than dynamically evolving fields, and they rely on expensive high-precision numerical updates rather than low-energy physical resonance. The difference is not merely one of hardware but of computational ontology. Digital AI is fast but power-hungry; the brain is slow in its components but extraordinarily efficient in its collective dynamics. Analog resonance provides a path to forms of intelligence that are fluid, adaptive, and deeply integrated across scales.

\subsection{Consciousness as a Spectral Regime}

Miller’s anesthesia research underscores that consciousness itself is best understood not as a localized function but as a global spectral regime. Across anesthetics with widely varying molecular mechanisms, the transition into unconsciousness displays a common pattern: the collapse of large-scale coherence and the breakdown of multi-band oscillatory structure. Functional consciousness corresponds to a state in which multiple frequency bands maintain coordinated, stable relationships:
\[
    \text{Conscious} \;=\;
    \text{high-coherence, multi-band, cross-scale spectral organization}.
\]

Under anesthesia, these relationships deteriorate. The spectral landscape becomes dominated by slow, unstable rhythms that fail to support the integration of distributed information:
\[
    \text{Unconscious} \;=\;
    \text{low-coherence, slow, incoherent spectral structure}.
\]

The striking universality of this transition—observable across drugs, species, and experimental protocols—parallels the universality of spectral statistics found in random matrices, nuclear systems, and the zeros of the Riemann zeta function. Just as these mathematically and physically distinct systems fall into the same universality classes of fluctuations, states of consciousness appear to inhabit distinct spectral regimes characterized by invariant statistical signatures. This analogy is not merely suggestive: it points toward a spectral theory of consciousness in which global coherence, not localized firing, plays the defining computational role.

% ======================================================================

\section{Operator Reductions and Symmetry Sectors}

The hierarchy
\[
    L_{\mathrm{nucleus}} \subseteq L_{\zeta}
    \subseteq L_{\mathrm{cortex}} \subseteq L_{\mathrm{RSVP}}
\]
arises from symmetry reduction rather than literal operator containment.
Each operator corresponds to a restriction of $L_{\mathrm{RSVP}}$ to a specific
invariant subspace determined by physical or mathematical symmetry.

For example, nuclear operators correspond to high-entropy limits in which
microscopic degrees of freedom fluctuate chaotically, yielding a GOE/GUE
universal regime. The hypothetical zeta operator corresponds to a sector with
arithmetic symmetries but no spatial geometry. Cortical operators correspond to
weakly nonlinear neural fields embedded in a curved two-dimensional manifold.

Formally, write $L_{\mathrm{RSVP}}$ acting on a Hilbert space $\mathcal{H}$.
A symmetry group $G_X \subseteq \mathrm{Aut}(\mathcal{H})$ defines an invariant
subspace $\mathcal{H}_X \subseteq \mathcal{H}$, and the reduced operator
is the restriction
\[
    L_X = L_{\mathrm{RSVP}} \big|_{\mathcal{H}_X}.
\]

This provides a mathematically precise description of how diverse physical and
biological systems emerge as sectors of a single field-theoretic operator.

% ======================================================================

\section{Renormalization and Spectral Scaling}

Spectral universality is intimately connected with renormalization. As a system
is coarse-grained, fine-scale variations in its operator $L_X$ become irrelevant
to the long-range spectral structure. Dyson's Brownian-motion picture makes this
precise: perturbing matrix elements by small random fluctuations drives the
spectrum toward a universal equilibrium distribution, independent of the
microscopic details.

RSVP inherits this phenomenon. Under entropic relaxation, the operator
$L_{\mathrm{RSVP}}$ flows along a renormalization trajectory toward fixed points
corresponding to universality classes such as GOE, GUE, or coherent-wave regimes
depending on the coupling parameters $(c_i, \alpha_i, \beta_i, \gamma_i)$.
High-entropy regions approach random-matrix universality, while low-entropy
regions preserve coherent geometric modes.

This provides a mechanism for understanding why nuclear spectra, zeta zeros, and
cortical eigenmodes exhibit strikingly similar statistical fluctuations: they
are fixed points of the same spectral renormalization flow.


% ======================================================================
% SECTION 7: Integration with RSVP Dynamics
% ======================================================================

\section{Integration with RSVP Dynamics}

Having developed the parallel strands of number theory, nuclear physics, and cortical wave dynamics, we now integrate these domains within the mathematical structure of the RSVP field theory. The central premise is that each system realizes a restricted sector of a more general operator dynamics, and that RSVP provides the field-theoretic space in which these relations become explicit.

\subsection{RSVP Fields}

The RSVP framework is formulated on a manifold $M$ equipped with a background metric $g$. On this geometric substrate evolve three tightly coupled fields:
\[
    \Phi(x,t) \quad \text{a scalar potential governing local plenary tension},
\]
\[
    v(x,t) \quad \text{a vector flow mediating directed transport and influence},
\]
\[
    S(x,t) \quad \text{an entropy density encoding local degrees of disorder}.
\]

Together, these fields form a triplet $(\Phi, v, S)$ whose interactions are described by a Lagrangian action functional,
\[
    \mathcal{S}
    = \int_{M \times \mathbb{R}}
        \mathcal{L}\big(
            \Phi,\nabla\Phi,
            v,\nabla v,
            S,\nabla S,
            \partial_t \cdot
        \big)
        \sqrt{|g|}\,dx\,dt .
\]
The specific form of $\mathcal{L}$ governs the coupling between scalar, vector, and entropic components, and thereby determines the emergent dynamical regimes ranging from lamphrodic smoothing to coherent wave-like propagation. In this formalism, the dynamical content of RSVP is encoded in the Euler–Lagrange equations derived from $\mathcal{S}$, which produce a nonlinear, multi-field generalization of wave and diffusion equations.

\subsection{Linearized RSVP Operator}

To study the spectral properties central to this essay, we consider small fluctuations about a background solution,
\[
    \Psi = (\delta\Phi,\,\delta v,\,\delta S).
\]
Linearizing the full field equations yields a second-order temporal evolution equation of the form
\[
    \partial_t^2 \Psi + L_{\mathrm{RSVP}} \Psi = 0,
\]
where the operator $L_{\mathrm{RSVP}}$ encodes the coupling structure of the linearized dynamics.

In block form, the operator takes the schematic structure
\[
L_{\mathrm{RSVP}} =
\begin{pmatrix}
    -c_1^2 \nabla^2 + \alpha_1      & -\beta_1 \nabla\!\cdot           & -\gamma_1 \\
    -\beta_2 \nabla                & -c_2^2 \nabla^2 + \alpha_2       & -\gamma_2 \nabla \\
    -\gamma_3                      & -\beta_3 \nabla\!\cdot           & -c_3^2 \nabla^2 + \alpha_3
\end{pmatrix}.
\]
The diagonal terms describe scalar, vector, and entropic wave operators with distinct propagation speeds; the off-diagonal terms mediate cross-field couplings. The eigenfunctions of this operator are the standing and traveling RSVP modes—solutions whose spectral characteristics reflect the multi-field geometry of the plenum.

\subsection{RSVP as a Universal Operator Framework}

The value of this construction becomes clear when situating the domain-specific operators introduced earlier. Each previously discussed operator—whether arising from nuclear physics, analytic number theory, or cortical wave dynamics—can be obtained as a symmetry-reduced or parameter-restricted suboperator of $L_{\mathrm{RSVP}}$:
\[
    L_{\mathrm{nucleus}},
    \;L_{\zeta},
    \;L_{\mathrm{cortex}}
    \subseteq
    L_{\mathrm{RSVP}}.
\]

This inclusion expresses the idea that RSVP encompasses, in a single field-theoretic architecture, the principal operators responsible for spectral behavior in these disparate systems. The corresponding hierarchical relationship can be expressed succinctly:
\[
    L_{\mathrm{nucleus}}
    \;\subseteq\;
    L_{\zeta}
    \;\subseteq\;
    L_{\mathrm{cortex}}
    \;\subseteq\;
    L_{\mathrm{RSVP}}.
\]

At the level of eigenvalue statistics, this operator chain unifies the spectrum of physical, mathematical, and biological systems:
\[
\begin{aligned}
    \mathcal{E}(L_{\mathrm{nucleus}}) &= \text{Wigner--Dyson (GOE/GUE)}, \\
    \mathcal{E}(L_{\zeta})           &= \text{GUE (Montgomery--Odlyzko)}, \\
    \mathcal{E}(L_{\mathrm{cortex}}) &= \text{coherent, multi-band cortical spectra}, \\
    \mathcal{E}(L_{\mathrm{RSVP}})   &= \text{general mixtures across these universality classes}.
\end{aligned}
\]

In this view, what makes these systems comparable is not their material substrate but the operator-theoretic structure of their fluctuations. RSVP provides the most general context in which these operators coexist, interact, and can be studied under a unified spectral lens.


\subsection{Interpretation}

Taken together, these observations suggest that nuclei, primes, cortical waves, and RSVP fields are not isolated curiosities but distinct manifestations of a single underlying spectral landscape. Each domain realizes its own physical or mathematical dynamics, yet all of them converge onto the same statistical structures when examined through their eigenmodes.

\begin{quote}
Spectral universality is the bridge connecting matter, mathematics, and mind.
RSVP provides the field-theoretic substrate in which this universality becomes a
single coherent principle.
\end{quote}

In this view, the apparent diversity of systems dissolves into a shared spectral grammar. The same fluctuations that govern nuclear excitations reappear in the spacing of zeta zeros, in the standing-wave modes of cortex, and in the coupled scalar–vector–entropy dynamics of the RSVP plenum. Universality is not merely a pattern we notice; it is the structural fact that allows these domains to be treated within one overarching theoretical framework.

\section{Spectral Equivalence and Universality Classes}

Central to this work is the notion that very different systems may be regarded
as equivalent when their eigenvalue fluctuations obey the same statistical
laws. To formalize this, let $L_X$ and $L_Y$ be densely defined operators on
Hilbert spaces $\mathcal{H}_X$ and $\mathcal{H}_Y$, with unfolded spectra
$\{\lambda_i^{(X)}\}$ and $\{\lambda_i^{(Y)}\}$ normalized to unit mean spacing.

\begin{definition}[Spectral Equivalence]
Two systems $X$ and $Y$ are spectrally equivalent, denoted
$X \sim_{\mathrm{spec}} Y$, if their $k$-point correlation functions agree for
all $k$:
\[
    R_k^{(X)}(\lambda_1,\dots,\lambda_k)
    =
    R_k^{(Y)}(\lambda_1,\dots,\lambda_k).
\]
\end{definition}

Because $k$-point functions determine all fluctuation statistics, spectral
equivalence implies shared universality class membership. The GOE, GUE, and
Poisson classes serve as canonical examples. In this formalism, nuclear spectra,
zeta zeros, cortical oscillatory modes, and RSVP fluctuations may all be
compared through a single mathematical lens.

Spectral universality therefore acts as the categorical unifier: distinct
operators with unrelated microstructure nonetheless converge to identical
fluctuation laws under appropriate normalization.


\section{From Neural Fields to Operators}

Neural-field models describe cortical activity through coupled integro-
differential equations of the form
\[
    \tau \partial_t u(x,t)
    = -u(x,t)
    + \int_{\Omega} W(x,y) f(u(y,t))\,dy
    + I(x,t),
\]
where $u(x,t)$ is local activity, $W$ is a connectivity kernel, and $f$ is a
nonlinear gain function.

Linearizing around a background state $u_0(x)$ gives
\[
    \tau \partial_t \delta u(x,t)
    =
    -\delta u(x,t)
    + \int_{\Omega} W(x,y) f'(u_0(y))\,\delta u(y,t)\,dy.
\]

Define the linear operator
\[
    (L_{\mathrm{NF}} \psi)(x)
    =
    -\psi(x)
    + \int_{\Omega} K(x,y)\psi(y)\,dy,
\quad
K = W f'(u_0).
\]

Under mild symmetry assumptions on $K$, the operator reduces to a differential
form expressible as
\[
    L_{\mathrm{cortex}}
    = -c^2 \Delta_g + V(x),
\]
where $\Delta_g$ is the Laplace--Beltrami operator of the cortical surface. This
derivation provides the operator-theoretic foundation for treating cortical
waves as eigenfunctions of $L_{\mathrm{cortex}}$.



\section{Empirical Spectral Signatures Across Consciousness States}

Consciousness correlates with robust spectral invariants spanning multiple
frequency bands. Table~\ref{tab:spectralstates} summarizes characteristic
patterns across physiological and pharmacological regimes.

\begin{table}[H]
\centering
\renewcommand{\arraystretch}{1.2}
\begin{tabular}{|l|c|c|c|}
\hline
\textbf{State} & \textbf{Coherence} & \textbf{Bandwidth} & \textbf{Mode Structure} \\ \hline
Waking & High & Multi-band & Hierarchical, stable \\ \hline
REM Sleep & Moderate & Broad & Rapidly shifting modes \\ \hline
Deep Sleep & Low & Narrow (slow) & Weak coupling, large-scale modes \\ \hline
Propofol & Very low & Delta-dominant & Fragmented, incoherent \\ \hline
Ketamine & Mixed & High gamma spikes & Nonlinear, disrupted \\ \hline
Post-distraction Reset & Increasing & Restored bands & Re-synchronizing rotating waves \\ \hline
\end{tabular}
\caption{Spectral invariants under different consciousness states.}
\label{tab:spectralstates}
\end{table}


These empirical invariants support the view that consciousness corresponds to a
spectral phase, defined by coherence and cross-band coupling patterns, rather
than any localizable neural mechanism.


\section{Limitations and Open Problems}

While the operator-theoretic framework developed here suggests a deep unity
across nuclear physics, analytic number theory, cortical neuroscience, and
RSVP field dynamics, several limitations must be acknowledged.

First, the Pólya--Hilbert conjecture remains unresolved: no explicit operator
$L_{\zeta}$ is known whose spectrum corresponds exactly to the nontrivial zeros
of the Riemann zeta function. All spectral comparisons involving $L_{\zeta}$
therefore remain conjectural.

Second, empirical measurement of cortical eigenmodes is constrained by spatial
and temporal resolution limits in both electrophysiology and fMRI. Existing
observations support the existence of standing and traveling waves, but a full
spectral decomposition of $L_{\mathrm{cortex}}$ has not yet been achieved.

Third, the nonlinear regime of RSVP is not fully classified spectrally. While
linearization reveals universal features, the nonlinear sector may exhibit
phase transitions, bifurcations, or chaotic dynamics not captured by current
methods.

Fourth, boundaries between universality classes remain mathematically subtle.
Determining when a system transitions between GOE, GUE, Poisson, or coherent-
wave behavior requires a more detailed renormalization analysis.

These limitations point toward a broad range of open mathematical, empirical,
and conceptual questions. Addressing them will require interdisciplinary
collaboration across physics, number theory, neuroscience, and field theory.

\section{Conclusion: Toward a Unified Spectral Science}

The analyses presented throughout this work reveal a striking pattern: systems
that differ radically in ontology, scale, and physical realization nonetheless
exhibit deeply homologous spectral behavior. Heavy nuclei governed by
many-body quantum chaos, the nontrivial zeros of the Riemann zeta function, the
macroscale oscillatory modes of the cortex, and the coupled scalar--vector--
entropy fields of the RSVP plenum all give rise to eigenvalue statistics that
fall within the same small family of universality classes. This convergence is
not accidental. It reflects the fact that spectra are shaped primarily by
symmetry, conservation structure, and coarse-grained dynamical constraints, not
by the fine details of microscopic mechanism.

The central claim argued here is that \emph{spectral universality} provides the
unifying mathematical principle behind this convergence, while the RSVP
framework provides its natural physical interpretation. Random matrix theory,
originally developed for nuclear Hamiltonians, resurfaces in number theory
because the symmetries of the integers generate the same fluctuation laws as
chaotic quantum systems. The cortex adopts resonant eigenmodes because its
geometry and effective connectivity induce operator structures that mirror those
seen in quantum chaotic systems and arithmetic models. RSVP fields inherit these
same signatures because their linearized dynamics reduce to a block operator
whose symmetry classes subsume those of nuclei, zeta systems, and cortical
waves.

Taken together, these correspondences support a broader thesis: spectral
behavior constitutes a privileged descriptive level at which matter,
mathematics, and mind can be studied within a single coherent framework. By
focusing on operators, their eigenfunctions, and the invariants of their
spectral measures, one uncovers forms of organization that transcend domain
boundaries. This perspective dissolves traditional disciplinary silos: nuclear
physics, analytic number theory, computational neuroscience, and field theory
become mutually illuminating expressions of the same underlying mathematical
grammar.

The RSVP plenum offers a concrete realization of this unification. As a
field-theoretic substrate equipped with scalar, vector, and entropy degrees of
freedom, it encompasses the operator classes of the systems surveyed here as
symmetry-reduced sectors. In doing so, RSVP provides not only a mechanism for
why spectral universality appears across these domains but also a fertile
platform for extending the principle further—toward a physics of cognition, a
geometry of meaning, and a systematic science of coherent structure across
scales.

If spectral universality is the grammar of coherence, RSVP furnishes its
semantics. Their integration suggests that the deep structures of nature—from
atomic nuclei to prime numbers to conscious experience—may be understood as
expressions of one continuous spectral order.


\appendix

\section*{Appendix A: Random Matrix Theory, Zeta Zeros, and Spectral Universality} \addcontentsline{toc}{section}{Appendix A: Random Matrix Theory, Zeta Zeros, and Spectral Universality}

This appendix provides a rigorous mathematical account of spectral universality, beginning with classical results in random matrix theory (RMT), moving through the spectral interpretation of zeta zeros, and ending with a formal bridge to RSVP operator theory. The goal is to show how heavy nuclei, the prime numbers, and RSVP fields can be treated within a unified operator-spectral framework.

\subsection*{A.1 Wigner Ensembles and the Semicircle Law}

Let $\mathsf{W}_N$ be an $N\times N$ real symmetric matrix whose entries satisfy:

\mathsf{W}_N(i,j) = \mathsf{W}_N(j,i),\qquad \mathbb{E}[\mathsf{W}_N(i,j)] = 0,\qquad \mathbb{E}[\mathsf{W}_N(i,j)^2] = \begin{cases} \sigma^2 & i\neq j,\\ 2\sigma^2 & i=j. \end{cases} 

Define the normalized matrix

H_N := \frac{1}{\sqrt{N}}\,\mathsf{W}_N. 

Let $\lambda_1^{(N)},\dots,\lambda_N^{(N)}$ denote the eigenvalues of $H_N$, and define the empirical spectral measure:

\mu_N := \frac{1}{N} \sum_{i=1}^N \delta_{\lambda_i^{(N)}}. 

\begin{theorem}[Wigner Semicircle Law] As $N\to\infty$, $\mu_N$ converges weakly (in probability) to

\rho_{\mathrm{sc}}(x)\,dx = \frac{1}{2\pi\sigma^2}\sqrt{4\sigma^2 - x^2}\;\mathbf{1}_{|x|\le 2\sigma}\, dx. 

This law is universal: it holds under very weak assumptions on the distribution of matrix entries, provided they have zero mean, identical variance, and finite moments.

\subsection*{A.2 Local Statistics and $k$-Point Correlation Functions}

Global eigenvalue distributions (e.g.\ the semicircle law) describe coarse spectral geometry. Local statistics are described by $k$-point correlation functions.

Let $\rho_k^{(N)}(x_1,\dots,x_k)$ be the joint density of finding eigenvalues near $(x_1,\dots,x_k)$. Upon unfolding to remove density variation, the scaled limit exists:

R_k(x_1,\dots,x_k) = \lim_{N\to\infty} \rho_k^{(N)}(\tilde x_1,\dots,\tilde x_k). 

For the GUE ensemble,

R_k(x_1,\dots,x_k) = \det\!\left[ K(x_i,x_j) \right]_{i,j=1}^k, K(x,y) = \frac{\sin\pi(x-y)}{\pi(x-y)}. 

The $k$-point correlations are the universal fingerprints of the GUE universality class.

\subsection*{A.3 The Montgomery--Dyson Match}

Let $\gamma_n$ denote the imaginary parts of the nontrivial zeros of the Riemann zeta function:

\zeta\!\left(\frac{1}{2}+i\gamma_n\right)=0. 

Assuming the Riemann Hypothesis, these are real. Montgomery proved that the pair-correlation of the \emph{normalized} zeros satisfies:

R_2^{(\zeta)}(x) = 1 - \left(\frac{\sin \pi x}{\pi x}\right)^2 + o(1), 

\begin{equation} R_2^{(\zeta)}(x) = R_2^{(\mathrm{GUE})}(x). \end{equation}

This constitutes one of the great discoveries of 20th-century mathematics:
\emph{the primes appear to lie in the same universality class as eigenvalues of large random Hermitian matrices.}

\subsection*{A.4 Nuclear Hamiltonians as Pseudorandom Operators}

Heavy nuclei (e.g.\ $^{238}\mathrm{U}$) are many-body quantum systems with $\sim 200$ nucleons. The true Hamiltonian is:

H_{\mathrm{nuclear}} = T + V_{\mathrm{strong}} + V_{\mathrm{Coulomb}} + V_{\mathrm{spin-orbit}} + \cdots 

Wigner’s idea was to model $H_{\mathrm{nuclear}}$ as a GOE matrix. Empirically:

\begin{itemize} \item neutron scattering data show level repulsion, \item spacing distributions match the Wigner surmise, \item $k$-point correlations match GOE predictions. \end{itemize}

Thus nuclear Hamiltonians empirically fall into GOE/GUE universality classes.

\subsection*{A.5 The Hilbert--Pólya Conjecture}

The Hilbert--Pólya idea posits the existence of a self-adjoint operator $L_\zeta$ such that:

\sigma(L_\zeta) = \{\gamma_n\}. 

If such an operator exists, the Riemann Hypothesis follows as a corollary. The spectral data of $\zeta$ would then come from a Hermitian operator with GUE-like statistics.

This places nuclear spectra, random matrices, and zeta zeros into a single operator-theoretic framework.

\subsection*{A.6 Spectral Invariants and Universality Classes}

Let $X$ be any system (nucleus, GUE matrix, $\zeta$ zeros, RSVP operator). Define the $k$-point invariants:

\mathcal{I}_k(X) := R_k^{(X)}(x_1,\dots,x_k). 

Two systems $X,Y$ belong to the same universality class if

\mathcal{I}_k(X) = \mathcal{I}_k(Y) \quad \forall k. 

Thus:

\text{Heavy nuclei} \sim \text{GUE matrices} \sim \text{zeros of }\zeta(s). 

\subsection*{A.7 Diagram: Cross-Domain Spectral Matching}

\begin{center}
\begin{tikzpicture}[node distance=18mm, >=stealth, thick]

% Nodes
\node[rectangle,draw,rounded corners,align=center,minimum width=3.6cm] (A)
  {Heavy nuclei\\[-1mm]{\footnotesize GOE-like spectra}};

\node[rectangle,draw,rounded corners,align=center,minimum width=3.6cm,
      right=25mm of A] (B)
  {Random matrices\\[-1mm]{\footnotesize GUE/GOE}};

\node[rectangle,draw,rounded corners,align=center,minimum width=3.6cm,
      below=18mm of B] (C)
  {Zeta zeros\\[-1mm]{\footnotesize Montgomery--Dyson}};

% Arrows
\draw[->] (A) -- node[above]  {\footnotesize level spacing}   (B);
\draw[->] (B) -- node[right]  {\footnotesize pair correlation} (C);
\draw[->] (A) -- node[left]   {\footnotesize empirical match} (C);

\end{tikzpicture}
\end{center}


\subsection*{A.8 Connection to RSVP Linear Operators}

In the RSVP formalism, perturbations of the field system obey:

\partial_t \Psi = L_{\mathrm{RSVP}} \Psi.  The operator $L_{\mathrm{RSVP}}$ is typically non-self-adjoint but admits a Hermitian part whose spectral data govern stability, oscillatory modes, and wave propagation. Under regimes of high entropy or chaotic vector-field alignment, the effective operator approaches a pseudorandom ensemble. Thus the semicircle law and GUE-type universality emerge naturally as limits of disordered RSVP field configurations.


\subsection*{A.9 The Spectral Form Factor and Dynamical Chaos}

A powerful diagnostic of universality is the \emph{spectral form factor}, defined for any system $X$ by

K_X(t) := \left| \sum_{n} e^{i \lambda_n t} \right|^2, 

For random matrices in the GUE class,

K_{\mathrm{GUE}}(t) = \begin{cases} t & 0 < t < 1,\\[4pt] 1 & t \ge 1. \end{cases} 

Nuclear spectra, zeta zeros, and many classically chaotic quantum systems display the same spectral form factor.

In the RSVP formalism, linearization yields

\partial_t \Psi = L_{\mathrm{RSVP}} \Psi, 

Thus the presence of a ramp in the RSVP spectral form factor identifies chaotic plenum states.

\subsection*{A.10 Trace Formulas: From Primes to Orbits}

Random matrices, nuclear spectra, and zeta zeros are linked through \emph{trace formulas} that connect discrete spectra to classical or arithmetic structures.
For the Riemann zeta function, the explicit formula gives:

\sum_{\gamma} f(\gamma) = \frac{1}{2\pi} \int_{-\infty}^{\infty} f(t)\, \left( \log\frac{t}{2\pi} \right)\, dt \;-\; \sum_p \sum_{k=1}^{\infty} \frac{\log p}{p^{k/2}}\, g(k\log p), 

This is the number-theoretic analogue of the Gutzwiller trace formula for quantum chaotic systems:

\text{Tr}\, e^{-itH/\hbar} \sim \sum_{\gamma\in \text{PO}} A_\gamma\, e^{i S_\gamma/\hbar}, 

The primes play the role of periodic orbits.
This reinforces the interpretation of the zeta function as encoding the dynamics of a chaotic classical system.

Within RSVP, stationary-phase approximations of the plenum action functional lead to analogous trace formulas, where the “periodic orbits’’ are field-theoretic configurations or vortex–torsion cycles.

\subsection*{A.11 Convergence of Universality Across Domains}

The remarkable conclusion from A.1–A.10 is:

\begin{center} \emph{ Random matrices, heavy nuclei, primes, and RSVP fields share the same universality classes because they are governed by formally similar spectral operators. } \end{center}

This is the central idea of spectral universality:
systems with wildly different microscopic rules behave identically at the level of operator spectra.

\subsection*{A.12 RSVP Operators Approaching Random Ensembles}

Consider a background plenary configuration $(\Phi_0,\mathbf{v}_0,S_0)$ with high entropy and rapidly fluctuating vector field $\mathbf{v}_0$. Linearizing the RSVP dynamics yields:

L_{\mathrm{RSVP}} = L_0 + \delta L, 

If the correlation length of $\delta L$ is short relative to the system scale, and the fluctuations are approximately symmetric, then:

\delta L \approx \text{Wigner matrix ensemble}. 

Thus,

L_{\mathrm{RSVP}} \;\longrightarrow\; \text{GOE/GUE universality class}. 

This gives a theoretical grounding for viewing chaotic RSVP regimes through random matrix universality.

\subsection*{A.13 Cortical Universality Classes as RSVP Phases}

The cortex, modeled as a 2D RSVP manifold with structured vector fields, has different operator regimes:

\begin{itemize} \item \textbf{Conscious state:} coherent, multi-frequency oscillatory modes; mixed selectivity supported by several dominant eigenmodes. \item \textbf{Unconscious state:} dominance of slow modes; reduced dimensionality; collapse of beta–gamma coupling. \item \textbf{Chaotic or desynchronized state:} high-entropy RSVP operator approximates a random ensemble. \end{itemize}

Precisely as nuclei and zeta zeros fall into GOE/GUE classes, brain states fall into operator-defined universality classes determined by $L_{\mathrm{RSVP}}$.

\subsection*{A.14 Diagram: RSVP Embedding of Three Universality Domains}

\begin{center} \begin{tikzpicture}[node distance=3cm,>=stealth,thick]

\node[ellipse, draw, align=center, minimum width=6cm, minimum height=2cm] (RMT) {Random Matrices\GOE/GUE Universality)};

\node[ellipse, draw, align=center, below left of=RMT, yshift=-1cm, xshift=0.5cm, minimum width=6cm, minimum height=2cm] (Nuc) {Nuclear Spectra\(Empirical GOE)};

\node[ellipse, draw, align=center, below right of=RMT, yshift=-1cm, xshift=-0.5cm, minimum width=6cm, minimum height=2cm] (Z) {Zeta Zeros\(Montgomery--Dyson GUE)};

\node[rectangle, draw, align=center, above of=RMT, yshift=1.2cm, minimum width=9cm, minimum height=2.2cm] (RSVP) {RSVP Operator $L_{\mathrm{RSVP}}$\(Unified Spectral System)};

\draw[->] (RSVP) -- (RMT); \draw[->] (RMT) -- (Nuc); \draw[->] (RMT) -- (Z);

\end{tikzpicture} \end{center}

This shows how RSVP serves as a unifying supersystem in which all three spectral phenomena are embedded as distinct regimes.

\subsection*{A.15 Semiclassical Limits and RSVP Analogues}

Just as the semiclassical limit of quantum mechanics ($\hbar\to 0$) leads to classical trajectories and trace formulas, RSVP possesses semiclassical limits where:

\begin{itemize} \item the scalar field $\Phi$ behaves like a potential, \item the vector field $\mathbf{v}$ defines flows analogous to classical trajectories, \item the entropy field $S$ modulates noise and damping. \end{itemize}

Stationary-phase approximations of the RSVP action yield periodic orbit expansions analogous to Gutzwiller or explicit formulas, showing that:

\text{RSVP periodic orbits} \leftrightarrow \text{prime-like invariants}.  This deepens the analogy between arithmetic, quantum chaos, and cortical resonance.



\subsection*{A.16 Operator Norms, Resolvents, and Universality Transitions}

For a linear operator $L$ acting on a Hilbert space $\mathcal{H}$, the \emph{resolvent} is defined by

R_L(z) := (L - zI)^{-1}, \qquad z \in \mathbb{C} \setminus \sigma(L). G_L(z) := \frac{1}{N}\mathrm{Tr}\, R_L(z) = \frac{1}{N} \sum_{k=1}^N \frac{1}{\lambda_k - z}. 

In random matrix theory (RMT), the semicircle law arises from a fixed-point equation for $G_L(z)$:

G_L(z) = \frac{1}{-z - G_L(z)}. 

For RSVP operators linearized around a background configuration, we have

L_{\mathrm{RSVP}} = L_{\mathrm{geom}} + \delta L, 

If the operator norm $|\delta L|$ is large relative to the curvature-scale norm $|L_{\mathrm{geom}}|$, the resolvent satisfies an RMT-type fixed-point equation in the limit:

G_{L_{\mathrm{RSVP}}}(z) \approx G_{\mathrm{GUE}}(z). 

\subsection*{A.17 Mapping RSVP Perturbations to Random Ensembles}

Consider the field perturbations

(\delta\Phi,\, \delta \mathbf{v},\, \delta S) \partial_t \Psi = L_{\mathrm{RSVP}} \Psi L_{\mathrm{RSVP}} = \begin{pmatrix} A & B & C\\ D & E & F\\ G & H & J \end{pmatrix}, 

If the background fields contain random microstructure or turbulent components, the off-diagonal blocks behave statistically like random matrices:

B, C, D, F, G, H \approx \text{i.i.d.\ random operators with symmetry constraints}. 

Thus the RSVP operator directly interpolates between:

\text{structured (geometric) spectral regimes} \quad\text{and}\quad \text{chaotic (random matrix) regimes}. 

\subsection*{A.18 Universality of Level Repulsion}

In all three systems—nuclear spectra, zeta zeros, and cortical resonances—level repulsion takes the generic form

p(s) \sim s^\beta, \beta = \begin{cases} 1 & \text{GOE},\\ 2 & \text{GUE},\\ 4 & \text{GSE}. \end{cases} 

In RSVP, the exponent $\beta$ is determined by the symmetry of the perturbation:

\beta_{\mathrm{RSVP}} = \begin{cases} 1 & \text{real-symmetric (time-reversal symmetric)},\\ 2 & \text{complex-Hermitian (broken symmetry)},\\ 4 & \text{quaternionic (spinful symplectic)}. \end{cases} 

In cortical systems, empirical data suggest: \begin{itemize} \item anesthesia induces approximate $\beta\approx 1$ (higher symmetry, lower complexity), \item conscious states exhibit $\beta\approx 2$ (broken symmetry via heterogeneous coupling), \item pathological states may deviate toward $\beta\approx 0$ (Poisson-like, for seizures or deep coma). \end{itemize}

Thus, cortical states correspond to distinct universality classes in the RSVP operator spectrum.

\subsection*{A.19 Spectral Manifolds and Cortical Geometry}

Let ${u_k(x)}$ be eigenfunctions of the cortical wave operator:

\mathcal{L}_{\text{cortex}} u_k = \lambda_k u_k. 

Empirical ultrafast fMRI shows that: \begin{enumerate} \item only a small number of modes dominate global brain activity, \item these modes correspond to geometric features (e.g., gradients, curvature, long-range connectivity), \item transitions between states correspond to bifurcations in the spectral manifold. \end{enumerate}

Define the cortical spectral embedding:

\Phi_{\text{spec}} : M \rightarrow \mathbb{R}^d,\qquad x \mapsto (u_1(x),u_2(x),\dots,u_d(x)). 

Then: \begin{itemize} \item consciousness corresponds to a stable region in $\Phi_{\text{spec}}(M)$, \item anesthesia corresponds to collapse onto a lower-dimensional subset, \item hallucinations correspond to distorted embeddings under altered coupling. \end{itemize}

This geometric picture matches the RSVP view of cognition as movement on a dynamic, entropy-weighted manifold.

\subsection*{A.20 Spectral Flow, Entropy, and RSVP Phase Transitions}

Define the spectral flow under a parameter $\theta$:

\lambda_k(\theta) \quad\text{solving}\quad L_{\mathrm{RSVP}}(\theta) u_k(\theta) = \lambda_k(\theta) u_k(\theta). 

Let $\theta$ encode: \begin{itemize} \item global excitability, \item anesthesia depth, \item disorder/noise level, \item entropy field magnitude. \end{itemize}

The RSVP entropy field $S$ defines a parameter family $L(\theta)$ so that:

\theta \uparrow \quad\Rightarrow\quad \text{greater randomness in off-diagonal blocks}\quad\Rightarrow\quad \text{spectral statistics → RMT-like}. \theta \downarrow \quad\Rightarrow\quad \text{lower entropy}\;\Rightarrow\; \text{geometric coherence and low-dimensional manifolds}. 

Thus:

\text{RSVP phase transitions} \;=\; \text{changes in spectral universality class as a function of entropy}. 

\subsection*{A.21 Diagram: RSVP Spectral Phase Diagram}

\begin{center} \begin{tikzpicture}[scale=1.0, >=stealth, thick]

% Axes \draw[->] (0,0) -- (7,0) node[right] {Entropy / Disorder}; \draw[->] (0,0) -- (0,5) node[above] {Spectral Coherence};

% Regions \fill[blue!10] (0,3.5) rectangle (2.5,5); \fill[green!10] (2.5,1.5) rectangle (5,3.5); \fill[red!10] (5,0) rectangle (7,1.5);

% Labels \node at (1.25,4.2) {Coherent Phase}; \node at (1.25,3.7) {(Consciousness)}; \node at (3.75,2.5) {Critical Phase}; \node at (3.75,2.0) {(Cognitive Transition)}; \node at (6.0,0.7) {Chaotic Phase}; \node at (6.0,0.3) {(RMT-like)};

% RSVP operator markers \node[circle,fill=black,inner sep=1.5pt] at (1,4.5) {}; \node at (1,4.8) {$L_{\mathrm{RSVP}}^{\text{coherent}}$};

\node[circle,fill=black,inner sep=1.5pt] at (4,2.8) {}; \node at (4,3.1) {$L_{\mathrm{RSVP}}^{\text{critical}}$};

\node[circle,fill=black,inner sep=1.5pt] at (6,0.9) {}; \node at (6,1.2) {$L_{\mathrm{RSVP}}^{\text{chaotic}}$};

\end{tikzpicture} \end{center}

This phase diagram summarizes the universality transitions inside the RSVP spectral operator space.

\subsection*{A.22 Summary: Universality as the Organizing Principle}

Appendix A establishes the following:

\begin{enumerate} \item Random matrices, nuclei, primes, cortical waves, and RSVP fields all reduce to spectral systems. \item Their universal behavior is determined by the spectrum of the associated operator. \item Universality classes correspond to entropy-weighted regimes of $L_{\mathrm{RSVP}}$. \item Consciousness appears as a coherent spectral phase of the RSVP operator. \item Anesthesia corresponds to a symmetry-restored, low-complexity spectral phase. \item Chaotic RSVP regimes reproduce RMT statistics (Wigner–Dyson universality). \item Structured RSVP regimes reproduce cortical resonance patterns. \end{enumerate}

Thus, spectral universality is not merely an analogy—it is a deep structural identity across mathematics, physics, and neuroscience.


\section*{Appendix B: Operator Hierarchies and Symmetry Reductions}

This appendix formalizes the operator hierarchy used throughout the main text and establishes, with mathematical precision, how symmetry reductions in the RSVP operator generate distinct universality classes—including those associated with nuclear spectra, zeta zeros, and cortical waves. Whereas Appendix A developed spectral foundations, Appendix B develops the \emph{algebraic and geometric} structure governing transitions between regimes.

The objective is to demonstrate that: \begin{enumerate} \item Random matrix ensembles, zeta operators, cortical wave operators, and the RSVP linearization operator all fit into a nested hierarchy. \item Symmetry reductions determine spectral universality classes. \item Breaking symmetry corresponds to cognitive, dynamical, or entropic transitions. \end{enumerate}

\subsection*{B.1 Operator Hierarchy}

Let each complex system $X$ (nuclear, number-theoretic, cortical, plenary) be associated with an operator $L_X$ acting on a Hilbert space $\mathcal{H}_X$. We define an \emph{operator hierarchy} as a chain of embeddings

L_{\mathrm{nucleus}} \;\subseteq\; L_{\zeta} \;\subseteq\; L_{\mathrm{cortex}} \;\subseteq\; L_{\mathrm{RSVP}}, 

Interpretation: \begin{itemize} \item $L_{\mathrm{nucleus}}$ captures microscopic chaotic quantum systems. \item $L_{\zeta}$ captures arithmetic dynamics via a conjectural spectral operator. \item $L_{\mathrm{cortex}}$ captures wave-driven neural field dynamics. \item $L_{\mathrm{RSVP}}$ captures full scalar–vector–entropy dynamics on a manifold. \end{itemize}

These inclusions do not require the operators to act on the same space; instead, each embedding extends the previous operator into a richer algebra of observables or higher-dimensional manifold.

\subsection*{B.2 Algebraic Structure of the Hierarchy}

We construct each level as an extension of the previous by adjunction of new operator components.

\paragraph{Level 0: Nuclear Operators} Random Hamiltonians $H$ with Wigner ensembles satisfy:

H = H^\dagger,\qquad H \in \mathrm{GOE/GUE/GSE}. 

\paragraph{Level 1: Zeta Operator} The hypothetical Pólya–Hilbert operator $L_\zeta$ satisfies:

\sigma(L_\zeta) = \{\gamma_n\}, 

Formally,

L_\zeta = H + \Delta A, 

\paragraph{Level 2: Cortical Wave Operator}

L_{\mathrm{cortex}} = -\nabla^\ast \nabla + V(x) + \mathcal{C}[\cdot], 

\paragraph{Level 3: RSVP Operator}

L_{\mathrm{RSVP}} \;=\; \begin{pmatrix} L_{\Phi\Phi} & L_{\Phi v} & L_{\Phi S} \\ L_{v\Phi} & L_{vv} & L_{vS} \\ L_{S\Phi} & L_{Sv} & L_{SS} \end{pmatrix}, 

Each block is a differential operator on the manifold $M$, with off-diagonal terms introducing torsion, entropy flow, and advection.

\subsection*{B.3 Symmetry Groups at Each Level}

Each operator $L$ admits a symmetry group $G(L)$ such that:

U^\dagger L\,U = L, \qquad U \in G(L). 

\paragraph{Nuclear Level:}

G(H) = O(N),\ U(N),\ \mathrm{USp}(2N), 

\paragraph{Zeta Operator Level:} The conjectured operator satisfies an analogue of ``time-reversal symmetry breaking'':

G(L_\zeta) \cong U(N), 

\paragraph{Cortical Level:} The symmetry group depends on connectivity patterns:

G(L_{\mathrm{cortex}}) \subseteq \mathrm{Diff}(M), 

\paragraph{RSVP Level:}

G(L_{\mathrm{RSVP}}) \subseteq \mathrm{Diff}(M) \ltimes \mathrm{Aut}(\mathcal{F}), \mathcal{F} = \Phi \oplus v \oplus S. 

\subsection*{B.4 Symmetry Breaking and Universality Class Transitions}

A symmetry reduction

G(L_1) \to G(L_2) 

Examples:

\paragraph{GOE $\to$ GUE:} Broken time-reversal symmetry.
Corresponds to systems with directed flow or complex coupling.

\paragraph{GUE $\to$ Cortical Wave Regime:} Geometry introduces structured coupling; universality transitions from random-matrix to wave-manifold statistics.

\paragraph{Cortex $\to$ RSVP Cognitive Regime:} Entropy-field coupling breaks geometrical symmetries and produces non-Hermitian components:

L_{\mathrm{RSVP}} = L_{\mathrm{cortex}} + \text{dissipative terms}. 

\paragraph{Consciousness $\to$ Anesthesia:} Empirically, anesthesia restores symmetry:

G_{\text{conscious}} \subset G_{\text{anesthetized}}. 

\subsection*{B.5 Hierarchical Projection Maps}

Define maps

\pi_{k+1 \to k} : \mathcal{H}_{k+1} \to \mathcal{H}_k, \pi_{k+1\to k} \circ L_{k+1} = L_k \circ \pi_{k+1\to k}. 

This makes $(L_0,L_1,L_2,L_3)$ a \emph{tower of compatible operators}, analogous to: \begin{itemize} \item renormalization group flows, \item spectral coarse-graining, \item dimensional reduction in Kaluza–Klein theory. \end{itemize}

\subsection*{B.6 Diagram: The Operator Hierarchy}

\begin{center}
\begin{tikzpicture}[node distance=10mm, >=stealth, thick, scale=0.85, every node/.style={transform shape}]

\node (nuc) [rectangle, draw, rounded corners=4pt,
             minimum width=3.4cm, minimum height=9mm]
   {$L_{\mathrm{nucleus}}$ \\[-1mm] {\footnotesize RMT / GOE/GUE}};

\node (zeta) [rectangle, draw, rounded corners=4pt,
              minimum width=3.4cm, minimum height=9mm,
              right=1.8cm of nuc]
   {$L_\zeta$ \\[-1mm] {\footnotesize Zeta zeros / arithmetic}};

\node (cortex) [rectangle, draw, rounded corners=4pt,
                minimum width=3.4cm, minimum height=9mm,
                right=1.8cm of zeta]
   {$L_{\mathrm{cortex}}$ \\[-1mm] {\footnotesize waves / geometry}};

\node (rsvp) [rectangle, draw, rounded corners=4pt,
              minimum width=3.4cm, minimum height=9mm,
              right=1.8cm of cortex]
   {$L_{\mathrm{RSVP}}$ \\[-1mm] {\footnotesize scalar–vector–entropy}};

\draw[->] (nuc)    -- node[above] {\footnotesize $+$ arithmetic structure}  (zeta);
\draw[->] (zeta)   -- node[above] {\footnotesize $+$ cortical geometry}     (cortex);
\draw[->] (cortex) -- node[above] {\footnotesize $+$ plenary coupling}     (rsvp);

\end{tikzpicture}
\end{center}


\subsection*{B.7 RSVP as a Universal Extension}

RSVP extends all previous operators by adding: \begin{enumerate} \item geometric coupling, \item vector-field flows, \item entropy-based dissipation, \item nonlinear cross-terms. \end{enumerate}

Thus,

L_{\mathrm{RSVP}} = \text{universal extension capturing all known spectral regimes}. 

In the limit of: \begin{itemize} \item no geometry, no entropy:→ RMT. \item arithmetic constraints only:→ zeta operator. \item wave geometry only:→ cortical operator. \item full coupling:→ RSVP dynamics. \end{itemize}

\subsection*{B.8 Summary of Appendix B}

Appendix B establishes: \begin{itemize} \item a rigorous hierarchy unifying random matrices, number theory, cortical waves, and RSVP, \item symmetry groups as classifiers of spectral universality, \item symmetry breaking as the mechanism for cognitive and physical phase transitions, \item RSVP as a universal operator encompassing the others via structured extensions. \end{itemize}


\section*{Appendix C: Numerical Methods for RSVP and Neural Operators}

This appendix presents numerical frameworks for approximating the operators and spectral regimes studied throughout the main text. Whereas Appendix A established foundations of spectral universality and Appendix B formalized operator hierarchies, Appendix C provides computational realizations of these structures.

The goal is to illustrate how: \begin{enumerate} \item random-matrix ensembles, \item cortical neural-field operators, \item linearized RSVP operators, \end{enumerate} can all be approximated within a unified numerical architecture that preserves symmetry classes, spectral invariants, and dynamical regimes.

\subsection*{C.1 Numerical Representation of Operators}

Throughout, let $M$ be a compact 2D manifold discretized as a mesh or grid of $N$ points ${x_i}$, and let fields be represented as vectors in $\mathbb{R}^N$ or $\mathbb{C}^N$. Operators $L$ acting on smooth functions are approximated by matrices $L_N$ acting on these finite-dimensional representations.

Given an operator $L$, the numerical approximation is:

L_N = \mathcal{D}_N[L] + \mathcal{E}_N, 

\begin{itemize} \item $\mathcal{D}_N[L]$ is the discretized differential operator, \item $\mathcal{E}_N$ represents boundary-condition and discretization errors. \end{itemize}

Spectral approximation seeks eigenpairs $(\lambda_k^{(N)}, u_k^{(N)})$ satisfying:

L_N u_k^{(N)} = \lambda_k^{(N)} u_k^{(N)}. 

We detail numerical strategies separately for each operator class.

\subsection*{C.2 Random Matrix Numerical Methods}

Random Matrix Theory (RMT) simulations use the following ensembles: \begin{itemize} \item Gaussian Orthogonal Ensemble (GOE) for real-symmetric $N \times N$ matrices, \item Gaussian Unitary Ensemble (GUE) for Hermitian matrices, \item Gaussian Symplectic Ensemble (GSE) for quaternionic-self-adjoint matrices. \end{itemize}

\paragraph{Construction:}

H_{ij} = \begin{cases} X_{ij}, & i<j, \\ X_{ii}, & i=j, \\ \overline{X_{ji}}, & i>j, \end{cases} 

The numerical eigenvalues ${\lambda_k^{(N)}}$ are then unfolded via

\tilde{\lambda}_k^{(N)} = N\int_{-\infty}^{\lambda_k^{(N)}} \rho_{\mathrm{sc}}(x)\, dx, 

\subsection*{C.3 Numerical Discretization of Cortical Operators}

The cortical wave operator has the form

L_{\mathrm{cortex}} u = -D\nabla^2 u + V(x) u + \int_M W(x,y) u(y)\, dy. 

We discretize the components as follows.

\subsubsection*{C.3.1 Laplacian} Finite difference or finite element methods (FEM) approximate the Laplacian:

(\nabla^2 u)(x_i) \approx \sum_{j \sim i} \frac{u(x_j) - u(x_i)}{h_{ij}^2}, 

\subsubsection*{C.3.2 Potential Term} The potential $V(x)$ is discretized as a diagonal matrix:

(Vu)(x_i) = V(x_i) u(x_i). 

\subsubsection*{C.3.3 Long-Range Coupling Kernel}

(Wu)(x_i) = \sum_{j=1}^N W(x_i,x_j) u(x_j) \Delta V_j. 

For biological plausibility, $W$ is often:

W(x_i,x_j) = A \exp(-\|x_i - x_j\|^2/\sigma^2) + B \exp(-\|x_i - x_j\|/\ell). 

\paragraph{Full Operator:}

L_{\mathrm{cortex},N} = -D \nabla^2_N + V_N + W_N. 

We compute eigenvalues via Lanczos or Arnoldi iterations for sparse matrices.

\subsection*{C.4 Numerical Linearization of RSVP}

The RSVP block operator is:

L_{\mathrm{RSVP}} = \begin{pmatrix} L_{\Phi\Phi} & L_{\Phi v} & L_{\Phi S} \\ L_{v\Phi} & L_{vv} & L_{vS} \\ L_{S\Phi} & L_{Sv} & L_{SS} \end{pmatrix}. 

Each block is treated separately:

\paragraph{Scalar block:}

L_{\Phi\Phi} = -c_1^2\nabla^2 + \alpha_1. 

\paragraph{Vector block:} Vector fields require edge-based finite element methods or staggered grids:

L_{vv} = -c_2^2\nabla^2 + \alpha_2. 

\paragraph{Entropy block:}

L_{SS} = -c_3^2\nabla^2 + \alpha_3. 

\paragraph{Coupling operators:} Divergence and gradient operators discretized via:

\nabla_N = (D_x, D_y), \qquad \nabla\cdot_N = D_x^\top + D_y^\top. 

\paragraph{Full discretized operator:}

L_{\mathrm{RSVP},N} = \begin{pmatrix} -c_1^2\nabla^2_N + \alpha_1 & -\beta_1 (\nabla\cdot_N) & -\gamma_1 I \\ -\beta_2 \nabla_N & -c_2^2\nabla^2_N + \alpha_2 & -\gamma_2 \nabla_N \\ -\gamma_3 I & -\beta_3 (\nabla\cdot_N) & -c_3^2\nabla^2_N + \alpha_3 \end{pmatrix}. 

The structure ensures approximate self-adjointness under symmetric coupling.

\subsection*{C.5 Time Integration for RSVP and Cortical Dynamics}

We simulate the PDE

\partial_t U = L U + N(U), 

\subsubsection*{C.5.1 Linear Case} Explicit schemes:

U^{n+1} = U^n + \Delta t\, L U^n 

Implicit schemes:

(I - \Delta t L) U^{n+1} = U^n 

Crank–Nicolson offers second-order symmetry:

U^{n+1} = (I - \tfrac{1}{2}\Delta t\, L)^{-1} (I + \tfrac{1}{2}\Delta t\, L) U^n. 

\subsubsection*{C.5.2 Nonlinear Case} We use operator splitting:

U^{n+1} = \exp(\Delta t L) \circ \exp(\Delta t N)(U^n) + O(\Delta t^2). 

\subsection*{C.6 Numerical Extraction of Spectral Universality}

Given the finite matrix $L_N$ for any operator regime (nuclear, zeta-like, cortical, RSVP), we compute:

\paragraph{1. Eigenvalues} Compute $\lambda_1^{(N)},\dots,\lambda_N^{(N)}$.

\paragraph{2. Unfolding} Let $\bar{\rho}(\lambda)$ be the estimated density via kernel smoothing or polynomial fit. Define unfolded eigenvalues:

\tilde{\lambda}_k = \int_{-\infty}^{\lambda_k} \bar{\rho}(x) dx. 

\paragraph{3. Gap spacings}

s_k = \tilde{\lambda}_{k+1} - \tilde{\lambda}_k. 

\paragraph{4. Compare $P(s)$ with universality classes} GOE:

P_{\mathrm{GOE}}(s) = \frac{\pi s}{2}\exp(-\tfrac{\pi s^2}{4}). P_{\mathrm{GUE}}(s) = \frac{32}{\pi^2} s^2\exp(-\tfrac{4s^2}{\pi}). 

\paragraph{5. Plot form factor}

K(t) = \left\langle \bigg|\sum_k e^{it\tilde{\lambda}_k}\bigg|^2 \right\rangle 

\subsection*{C.7 Unified Numerical Experiment}

We propose a demonstration combining all operator regimes on the same discretized manifold:

\begin{enumerate} \item Generate $L_{\mathrm{RMT},N}$ (random). \item Generate $L_{\mathrm{cortex},N}$ (structured kernel). \item Generate $L_{\mathrm{RSVP},N}$ (coupled PDE operator). \item Interpolate:

L_\alpha = (1-\alpha) L_{\mathrm{RMT},N} + \alpha L_{\mathrm{cortex},N}. 

\end{enumerate}

This numerically illustrates the operator hierarchy of Appendix B and the universality theory of Appendix A.

\subsection*{C.8 Summary of Appendix C}

Appendix C presented: \begin{itemize} \item numerical construction of all operators in the hierarchy, \item discretization of scalar, vector, and entropy fields for RSVP, \item wave-operator discretization for cortical dynamics, \item random matrix ensembles for universality benchmarking, \item numerical extraction of spectral signatures, \item an explicit unified experiment demonstrating transitions across universality classes. \end{itemize}



\section*{Appendix D: Empirical Evidence for Spectral Universality in Cortex}

This appendix surveys and systematizes experimental evidence demonstrating that cortical dynamics obey universal spectral laws analogous to those found in random matrix theory and quantum chaotic systems. We collect data from electrophysiology, MEG/EEG recordings, microelectrode arrays, ultrafast fMRI, and anesthesia studies. Together these results show that the cortex exhibits structure-preserving spectral regimes: coherent wave patterns, universal state transitions, and invariant oscillatory signatures across modalities.

Our goal is to show that the cortical operator $L_{\mathrm{cortex}}$ introduced in the main text empirically displays: \begin{enumerate} \item eigenmode structure consistent across species and recording modalities, \item universal dynamic regimes under perturbations (e.g.\ anesthesia), \item low-dimensional attractor structure despite high-dimensional physiology, \item stable but flexible spectral manifolds underlying cognition, \item transitions between universal classes analogous to GOE/GUE--Poisson crossovers. \end{enumerate}

This constitutes empirical justification for the operator-hierarchy and spectral universality claims of the main text.

\subsection*{D.1 Empirical Evidence from Resting-State fMRI}

Recent ultrafast fMRI studies (Cabral, Fernandes, Shemesh; 2023) reveal spontaneous standing waves in resting-state rodent cortex. Using frame rates $\sim 38$ ms, they report: \begin{itemize} \item a discrete set of macroscale modes with stable spatial profiles across time, \item long-range coherence between distant cortical and subcortical structures, \item a consistent frequency spectrum with clustered modes, \item transitions in spectral occupancy depending on anesthesia level. \end{itemize}

Standing waves were extracted as eigenvectors of a covariance operator:

C u_k = \lambda_k u_k, \qquad C(x,y) = \langle u(x,t) u(y,t) \rangle_t. 

The modes $u_k$ exhibit: \begin{itemize} \item smooth, domain-shape-dependent eigenfunctions, \item approximately sinusoidal phase relationships, \item spectral spacing patterns stable across subjects. \end{itemize}

These correspond to low-lying eigenfunctions of $L_{\mathrm{cortex}}$ and form an empirical basis for the operator model developed in the main text.

\subsection*{D.2 Traveling Waves and Coherent Propagation}

Electrophysiological studies using multi-electrode arrays (Trongnetrpunya et al.\ 2022) observe robust traveling waves in prefrontal cortex during working memory tasks. These waves exhibit: \begin{itemize} \item directionally coherent propagation, \item content-encoding modulations of phase patterns, \item stable propagation velocities, \item wavefront curvature consistent with geometric constraints of the cortical sheet. \end{itemize}

Traveling waves correspond to complex eigenmodes of the linearized operator:

L_{\mathrm{cortex}} u_k = \lambda_k u_k, 

\subsection*{D.3 Rotating Waves and Attentional Reset}

Rotating waves, observed in human MEG (e.g.\ Han et al.\ 2024), reveal an additional spectral phenomenon: \begin{itemize} \item global rotational phase patterns spanning large cortical regions, \item reset of attention following distraction, \item re-establishment of metastable workspace oscillations. \end{itemize}

Rotating waves behave as eigenmodes of $L_{\mathrm{cortex}}$ with azimuthal symmetry breaking. Their recurrence following task transitions suggests that attention is implemented by selecting and stabilizing specific rotating-mode subspaces.

\subsection*{D.4 Universality Under Anesthesia}

Multiple anesthetic agents (propofol, ketamine, isoflurane) produce a shared canonical signature:

\text{loss of consciousness} \;\Longleftrightarrow\; \text{slowing and destabilization of cortical spectral modes.} 

Across EEG, MEG, LFP, and fMRI: \begin{itemize} \item beta and gamma coherence collapse, \item high-frequency modes disappear from the spectrum, \item slow delta waves dominate with high amplitude, \item cortical phase relationships become unstable and non-stationary. \end{itemize}

This is consistent with a transition from a structured operator spectrum to a Poisson-like (weakly coupled) regime. In operator language, anesthesia reduces effective coupling terms in $L_{\mathrm{cortex}}$, pushing the system toward an integrable/weakly chaotic phase.

\subsection*{D.5 Spectral Coherence and Consciousness}

Experimental work by Miller, Brown, and others documents that consciousness corresponds to: \begin{itemize} \item the presence of multiple interacting frequency bands (beta, gamma), \item strong cross-frequency coupling, \item stable global coherence in phase relationships, \item distributed high-dimensional but low-rank spectral structure. \end{itemize}

These are the hallmarks of a spectral phase:

\text{conscious state} = \text{coherent multi-frequency oscillatory regime}. 

The loss of coherence in unconsciousness resembles universality transitions in random operators as coupling parameters vanish.

\subsection*{D.6 Mixed Selectivity and Spectral Dimensionality Reduction}

Despite the high dimensionality of neural populations, neural trajectories lie close to low-dimensional manifolds. Empirical findings include: \begin{itemize} \item top neural principal components explain up to $80%$ of variance, \item task-specific trajectories lie on smooth manifolds parametrized by latent variables, \item spectral decomposition of neural activity often reveals a small number of dominant modes. \end{itemize}

These observations support the claim that the cortex implements a low-rank approximation:

u(t,x) \approx \sum_{k=1}^d a_k(t) u_k(x), \qquad d \ll N, 

\subsection*{D.7 Cross-Species Spectral Invariance}

Remarkably, analogous spectral phenomena appear in: \begin{itemize} \item rodents (fMRI, electrophysiology), \item nonhuman primates (LFP, multi-area recordings), \item humans (MEG, EEG, ECoG), \item bats, songbirds, ferrets (various modalities). \end{itemize}

In all species: \begin{itemize} \item cortex-like structures exhibit discrete spectral modes, \item spectral hierarchies persist across evolutionary scales, \item dynamic regimes follow similar transitions (sleep, anesthesia, attention). \end{itemize}

This cross-species invariance is consistent with universality classes of operators determined by structure and symmetry, not by biological detail.

\subsection*{D.8 Field Operators and Spectral Phenotypes}

For empirical classification, one may define a \emph{spectral phenotype} for each brain state:

\Theta_{\mathrm{state}} = \left( \rho(\lambda),\; P(s),\; \text{coherence metrics},\; \text{cross-frequency coupling},\; \text{eigenmode occupancy} \right). 

Comparisons of spectral phenotypes across states reveal: \begin{enumerate} \item fixed-point-like attractors (deep anesthesia), \item limit-cycle-like manifolds (non-REM sleep), \item broadband meta-stable manifolds (conscious cognition), \item structured subspace decompositions (task-specific working memory). \end{enumerate}

These correspond directly to spectral regimes of the cortical operator $L_{\mathrm{cortex}}$.

\subsection*{D.9 Empirical Universality Classes}

The cortex exhibits three primary universality classes: \begin{description} \item[Class I (coherent chaotic regime):] conscious, metastable, broadband, cross-frequency coupled.

\item[Class II (structured integrable regime):] task-specific cortical resonances, low-dimensional manifolds, rotating waves.

\item[Class III (Poisson-like regime):] deep anesthesia, suppressed connectivity, uncorrelated delta oscillations. \end{description}

These classes parallel: \begin{itemize} \item GOE/GUE universality, \item weakly chaotic semiclassical regimes, \item Poisson statistics of integrable systems. \end{itemize}

This analogy provides a bridge between neuroscience and spectral universality in mathematical physics.

\subsection*{D.10 Synthesis}

The empirical data demonstrate that: \begin{enumerate} \item cortical activity is fundamentally spectral, \item eigenmode structure underlies cognition, \item consciousness corresponds to a coherent spectral phase, \item perturbations induce universal spectral transitions, \item spectral patterns persist across species and modalities, \item operator-based models capture core features of cortical dynamics. \end{enumerate}


\section*{Appendix E: Experimental Predictions and Testable Consequences of the RSVP Spectral Framework}

This appendix articulates empirically falsifiable predictions derived from the operator-theoretic and spectral-universality interpretation of cortical dynamics presented in the main text. These predictions connect the mathematical properties of the RSVP operator $L_{\mathrm{RSVP}}$ with measurable spectral signatures across EEG, MEG, electrophysiology, and ultrafast fMRI.

Our goal is to provide a clear correspondence:

\text{RSVP operator} \quad \Longleftrightarrow \quad \text{observable spectral regime}. 

The predictions are divided into:
\begin{enumerate} \item structural spectral predictions,
\item dynamical (state-transition) predictions,
\item perturbation and anesthesia predictions,
\item connectivity and geometry predictions,
\item cross-species universality predictions. \end{enumerate}

These allow empirical validation or falsification of the framework.

\subsection*{E.1 Prediction I — Cortical Modes Form a Discrete Low-Dimensional Basis}

The RSVP hypothesis predicts that the cortical operator $L_{\mathrm{cortex}}$ possesses a discrete sequence of low-lying eigenmodes whose spatial profiles are stable across time.

Formally:

L_{\mathrm{cortex}} u_k = \lambda_k u_k, \qquad 0 < \lambda_1 \le \lambda_2 \le \cdots, \text{Var}[u(t,\cdot)] \approx \sum_{k=1}^{d} a_k^2(t), \qquad d \ll N_{\mathrm{neurons}}. 

Thus: \begin{itemize} \item independent of recording modality (LFP, MEG, fMRI), \item the same handful of eigenmodes should dominate variance, \item conscious cognition corresponds to occupation of multiple interacting modes. \end{itemize}

\emph{Test:} PCA/ICA/GLM decompositions across modalities should reconstruct the same spatial bases $u_k$ at different time scales.

\subsection*{E.2 Prediction II — Consciousness as a Broadband Coherent Spectral Phase}

The RSVP operator predicts a specific signature for consciousness:

\text{Consciousness} = \text{broadband coherence across} \; \{u_1, u_2, \ldots, u_d\}. 

Empirically: \begin{enumerate} \item cross-frequency coupling between beta and gamma bands increases, \item phase-locking values across distant regions rise, \item spectral entropy remains intermediate (neither maximal nor minimal), \item high-frequency modes remain populated. \end{enumerate}

\emph{Test:} Perturbation experiments (TMS, optogenetics) should show a return to the same coherent regime after perturbation if the subject remains conscious.

\subsection*{E.3 Prediction III — Anesthesia is a Universality-Class Transition}

Anesthesia should push the cortical operator into a Poisson-like spectral regime:

\text{Unconsciousness} = \text{Poisson-like spacing + low-rank spectral support}. 

Predictions: \begin{itemize} \item all anesthetics induce the same spacing statistics $P(s)$, \item beta--gamma coupling collapses universally, \item eigenmodes become spatially smoother (low-frequency dominated), \item spectral entropy decreases sharply. \end{itemize}

\emph{Test:} Fit spacing distributions $P(s)$ to GOE/GUE vs.~Poisson forms; anesthesia should reliably increase similarity to Poisson.

\subsection*{E.4 Prediction IV — Traveling and Rotating Waves Correspond to Degenerate Eigenspaces}

Traveling waves arise when complex eigenvalues of $L_{\mathrm{cortex}}$ appear in conjugate pairs:

L_{\mathrm{cortex}} u_k = (\alpha + i \omega_k) u_k, \qquad L_{\mathrm{cortex}} \overline{u}_k = (\alpha - i \omega_k) \overline{u}_k. 

Predictions: \begin{itemize} \item attention resets should correspond to temporary selection of rotating-mode subspaces, \item speed and direction of traveling waves should be determined by eigenvalue imaginary parts $\omega_k$, \item mixed selectivity emerges from nonlinear coupling between degenerate subspaces. \end{itemize}

\emph{Test:} Compare the directionality of traveling waves in PFC across tasks; eigen-directions should be conserved.

\subsection*{E.5 Prediction V — Universal Spectral Collapse at Task Onset and Offset}

The operator model implies:

\text{Task onset} \;\Rightarrow\; \text{temporary reduction in spectral entropy}, 

\text{Task offset} ;\Rightarrow; \text{return to metastable spectral manifold}. 

Empirical observations should include: \begin{itemize} \item transient synchronization of dominant modes at task onset, \item divergence of modes into metastable ensembles after task completion, \item preservation of spectral manifold topology across repetitions of the same task. \end{itemize}

\emph{Test:} Repeated-task MEG or LFP recordings should reconstruct identical spectral trajectories modulo reparameterization.

\subsection*{E.6 Prediction VI — Spectral Modes Scale with Anatomy in a Lawful Manner}

The RSVP operator predicts that eigenfrequencies scale with cortical surface geometry:

\omega_k \propto \sqrt{\lambda_k(M)}, 

Consequences: \begin{itemize} \item larger brains exhibit proportionally more low-frequency modes, \item cortical folding modifies mode shapes but preserves frequency ordering, \item interspecies variation arises from geometric parameters, not microcircuit differences. \end{itemize}

\emph{Test:} Cross-species MEG/fMRI eigenmodes should align with eigenfunctions of the cortical surface Laplacian.

\subsection*{E.7 Prediction VII — RSVP Predicts a “Spectral Equation of State” for Brain States}

Define spectral observables:

\Xi = \Big(H_{\mathrm{spec}},\; \mathrm{PLV},\; P(s),\; \rho(\lambda),\; \Gamma_{\mathrm{CFC}}\Big), 

\begin{itemize} \item $H_{\mathrm{spec}}$ is spectral entropy, \item $\mathrm{PLV}$ is mean phase-locking value, \item $P(s)$ is spacing distribution, \item $\rho(\lambda)$ is spectral density, \item $\Gamma_{\mathrm{CFC}}$ is cross-frequency coupling. \end{itemize}

RSVP predicts a state equation:

\mathcal{F}(\Xi) = 0, 

\begin{description} \item[Conscious] intermediate entropy, high PLV, broadband $\rho(\lambda)$, \item[Sedated] low entropy, weak PLV, narrow $\rho(\lambda)$, \item[Chaotic/Seizure] high entropy, low PLV, broadband but unstable $\rho(\lambda)$. \end{description}

\emph{Test:} Brain states across conditions (sleep, anesthesia, seizure, psychedelic states) should lie on the same constraint surface $\mathcal{F}(\Xi)=0$.

\subsection*{E.8 Prediction VIII — RMT-Like Behavior in High-Noise or High-Temperature Regimes}

Under high neural noise, the RSVP operator predicts:

L_{\mathrm{cortex}} + \sigma \eta(x,t) \quad \Longrightarrow \quad \text{RMT-like spectral statistics as } \sigma \to \infty. 

Thus: \begin{itemize} \item psychedelic states with increased neural noise may show RMT-like spacing, \item early development (infant EEG) may resemble GOE/GUE spectra, \item cortical injury may produce RMT-like expansions of the spectral bulk. \end{itemize}

\emph{Test:} Compare spacing distributions under psychedelics or high excitability to RMT ensembles.

\subsection*{E.9 Prediction IX — RSVP Implies Coupled-Field Signatures Detectable via Multimodal Recording}

The block operator structure of RSVP:

L_{\mathrm{RSVP}}= \begin{pmatrix} L_{\Phi} & L_{\Phi v} & L_{\Phi S}\\ L_{v\Phi} & L_v & L_{v S}\\ L_{S\Phi} & L_{S v} & L_S \end{pmatrix} 

\begin{itemize} \item vector fields (local currents) correlate with phase gradients of scalar modes, \item entropy field $S$ mediates damping and excitability shifts, \item scalar--vector cross-correlation predicts top-down modulation pathways. \end{itemize}

\emph{Test:} Multimodal fMRI--MEG studies should reveal correlated scalar and vector spectral manifolds.

\subsection*{E.10 Prediction X — Spectral Universality Under Structural Perturbation}

Lesions, microstimulation, or pharmacological manipulations should move the cortex between spectral classes by modulating operator terms.

Predictions: \begin{itemize} \item small structural lesions reorganize spatial eigenmodes but preserve spacing distribution, \item microstimulation can drive selection of specific eigenmodes, \item topological invariants of spectral manifolds remain stable under smooth deformations. \end{itemize}

\emph{Test:} Compare pre- and post-lesion spectral embeddings; topology should be preserved even if spatial layouts differ.

\subsection*{E.11 Summary}

The RSVP spectral framework yields a highly structured set of falsifiable predictions: \begin{enumerate} \item discrete eigenmode bases across modalities, \item consciousness as a coherent spectral phase, \item anesthesia as a Poisson-like universality transition, \item geometric scaling of eigenfrequencies, \item RMT behavior under high-noise regimes, \item stable operator-based state equations for neural spectra. \end{enumerate}

Together, these predictions provide a testbed for validating the operator hierarchy connecting nuclei, primes, cortex, and RSVP dynamics.

\section*{Appendix F: Variational Derivation of the RSVP Field Equations}

In this appendix we derive the RSVP field equations from a variational principle, starting from a Lagrangian density for the scalar field $\Phi$, the vector field $v_\mu$, and the entropy field $S$. We then obtain the corresponding Euler–Lagrange equations, discuss the associated conserved currents, and show how the linearized dynamics can be written in operator form, leading to the RSVP operator $L_{\mathrm{RSVP}}$ used throughout the main text and in the earlier appendices.

\subsection*{F.1 Geometric and Field-Theoretic Setup}

Let $(\mathcal{M}, g_{\mu\nu})$ be a $(d+1)$-dimensional Lorentzian manifold, with metric signature $(-,+,\dots,+)$, and let $\nabla_\mu$ denote the Levi–Civita covariant derivative associated with $g_{\mu\nu}$. Greek indices $\mu,\nu,\dots$ run over spacetime coordinates $0,\dots,d$, while Latin indices $i,j,\dots$ (if needed) run over spatial coordinates $1,\dots,d$.

We consider three fields: \begin{itemize} \item a scalar field $\Phi : \mathcal{M} \to \mathbb{R}$, \item a vector field $v_\mu : \mathcal{M} \to T^*\mathcal{M}$ (co-vector; we may also work with $v^\mu$ via $v^\mu = g^{\mu\nu} v_\nu$), \item an entropy-like scalar field $S : \mathcal{M} \to \mathbb{R}$. \end{itemize}

The RSVP framework treats these as coupled degrees of freedom encoding potential, flow, and entropic structure in a single plenum.

\subsection*{F.2 RSVP Action Functional}

We postulate an action of the form \begin{equation} \label{eq:F_action} \mathcal{S}[\Phi, v_\mu, S]

\int_{\mathcal{M}} \mathcal{L}(\Phi, v_\mu, S; \nabla_\alpha \Phi, \nabla_\alpha v_\mu, \nabla_\alpha S) ,\sqrt{-g}, d^{d+1}x, \end{equation} where $g = \det(g_{\mu\nu})$ and $\mathcal{L}$ is the Lagrangian density. A convenient and sufficiently general ansatz that captures the dynamics discussed in the main text is: \begin{align} \mathcal{L} &= \mathcal{L}\Phi + \mathcal{L}v + \mathcal{L}S + \mathcal{L}{\mathrm{int}}, \4pt] \mathcal{L}\Phi &= -\frac{1}{2} g^{\mu\nu} (\nabla\mu \Phi)(\nabla_\nu \Phi) -\frac{1}{2} m_\Phi^2 \Phi^2, \[4pt] \mathcal{L}v &= -\frac{1}{4} F{\mu\nu} F^{\mu\nu} -\frac{1}{2} m_v^2 v_\mu v^\mu, \qquad F_{\mu\nu} := \nabla_\mu v_\nu - \nabla_\nu v_\mu, \[4pt] \mathcal{L}S &= +\frac{1}{2} \kappa_S, g^{\mu\nu} (\nabla\mu S)(\nabla_\nu S) -\frac{1}{2} m_S^2 S^2, \[4pt] \mathcal{L}_{\mathrm{int}} &=

\alpha_1 ,\Phi, \nabla_\mu v^\mu

\alpha_2 ,S, \nabla_\mu v^\mu

\beta_1 ,\Phi, S

\beta_2 ,g^{\mu\nu} (\nabla_\mu \Phi) (\nabla_\nu S), \end{align} where $m_\Phi, m_v, m_S$ are effective mass parameters, $\kappa_S>0$ is a diffusion-like coefficient for $S$, and $\alpha_1,\alpha_2,\beta_1,\beta_2$ are coupling constants.


This form is chosen so that: \begin{itemize} \item $\mathcal{L}_\Phi$ is a Klein–Gordon-type term for $\Phi$, \item $\mathcal{L}_v$ is a Proca-type term for a massive vector field, \item $\mathcal{L}S$ encodes entropic diffusion and a potential well, \item $\mathcal{L}{\mathrm{int}}$ encodes physically motivated couplings between scalar, vector, and entropy fields. \end{itemize}

\subsection*{F.3 Euler–Lagrange Equations}

The Euler–Lagrange equation for a generic field $\psi$ is: \begin{equation} \label{eq:F_EL_general} \frac{\partial \mathcal{L}}{\partial \psi}

\nabla_\mu \left( \frac{\partial \mathcal{L}}{\partial (\nabla_\mu \psi)} \right) = 0. \end{equation}

\subsubsection*{F.3.1 Scalar Field \texorpdfstring{$\Phi$}{Phi}}

Compute: \begin{align} \frac{\partial \mathcal{L}}{\partial \Phi} &=

m_\Phi^2 \Phi

\alpha_1 \nabla_\mu v^\mu

\beta_1 S, \[4pt] \frac{\partial \mathcal{L}}{\partial(\nabla_\mu \Phi)} &=

g^{\mu\nu} \nabla_\nu \Phi

\beta_2 g^{\mu\nu} \nabla_\nu S. \end{align} Therefore: \begin{align} \nabla_\mu \left( \frac{\partial \mathcal{L}}{\partial (\nabla_\mu \Phi)} \right) &= \nabla_\mu \left(- g^{\mu\nu} \nabla_\nu \Phi - \beta_2 g^{\mu\nu} \nabla_\nu S \right) \ &=

\nabla_\mu \nabla^\mu \Phi

\beta_2 \nabla_\mu \nabla^\mu S =

\Box \Phi - \beta_2 \Box S, \end{align} where $\Box := \nabla_\mu \nabla^\mu$ is the d'Alembertian.


Plugging into \eqref{eq:F_EL_general} with $\psi = \Phi$: \begin{equation}

m_\Phi^2 \Phi

\alpha_1 \nabla_\mu v^\mu

\beta_1 S


\Box \Phi + \beta_2 \Box S = 0, \end{equation} or equivalently: \begin{equation} \label{eq:F_Phi_EL} \Box \Phi - m_\Phi^2 \Phi

\beta_2 \Box S


\alpha_1 \nabla_\mu v^\mu

\beta_1 S = 0. \end{equation}


\subsubsection*{F.3.2 Vector Field \texorpdfstring{$v_\mu$}{v}}

For the vector field: \begin{align} \frac{\partial \mathcal{L}}{\partial v_\sigma} &=

m_v^2 v^\sigma, \[4pt] \frac{\partial \mathcal{L}}{\partial (\nabla_\mu v_\sigma)} &=

\frac{1}{2} \frac{\partial}{\partial (\nabla_\mu v_\sigma)} (F_{\alpha\beta} F^{\alpha\beta})

\alpha_1 \Phi g^{\mu\sigma}

\alpha_2 S g^{\mu\sigma}. \end{align} Since


F_{\alpha\beta} = \nabla_\alpha v_\beta - \nabla_\beta v_\alpha,

one obtains the standard expression:
\[
\frac{\partial}{\partial (\nabla_\mu v_\sigma)} (F_{\alpha\beta} F^{\alpha\beta})
=
4 F^{\mu\sigma},

\frac{\partial \mathcal{L}}{\partial (\nabla_\mu v_\sigma)}
=
- F^{\mu\sigma}
- \alpha_1 \Phi g^{\mu\sigma}
- \alpha_2 S g^{\mu\sigma}.

\begin{align} \nabla_\mu \left( \frac{\partial \mathcal{L}}{\partial (\nabla_\mu v_\sigma)} \right) &=

\nabla_\mu F^{\mu\sigma}

\alpha_1 \nabla^\sigma \Phi

\alpha_2 \nabla^\sigma S. \end{align} The Euler–Lagrange equation \eqref{eq:F_EL_general} gives: \begin{equation}

m_v^2 v^\sigma


\nabla_\mu F^{\mu\sigma}

\alpha_1 \nabla^\sigma \Phi

\alpha_2 \nabla^\sigma S = 0, \end{equation} or \begin{equation} \label{eq:F_v_EL} \nabla_\mu F^{\mu\sigma}


m_v^2 v^\sigma


\alpha_1 \nabla^\sigma \Phi

\alpha_2 \nabla^\sigma S = 0. \end{equation}


\subsubsection*{F.3.3 Entropy Field \texorpdfstring{$S$}{S}}

We compute: \begin{align} \frac{\partial \mathcal{L}}{\partial S} &=

m_S^2 S

\alpha_2 \nabla_\mu v^\mu

\beta_1 \Phi, \4pt] \frac{\partial \mathcal{L}}{\partial (\nabla_\mu S)} &= \kappa_S g^{\mu\nu} \nabla_\nu S

\beta_2 g^{\mu\nu} \nabla_\nu \Phi. \end{align} Then: \begin{align} \nabla_\mu \left( \frac{\partial \mathcal{L}}{\partial (\nabla_\mu S)} \right) &= \nabla_\mu \left( \kappa_S \nabla^\mu S - \beta_2 \nabla^\mu \Phi \right) \ &= \kappa_S \Box S

\beta_2 \Box \Phi. \end{align} Plugging into \eqref{eq:F_EL_general}: \begin{equation}

m_S^2 S

\alpha_2 \nabla_\mu v^\mu

\beta_1 \Phi


\kappa_S \Box S


\beta_2 \Box \Phi = 0, \end{equation} or \begin{equation} \label{eq:F_S_EL} \kappa_S \Box S - m_S^2 S

\alpha_2 \nabla_\mu v^\mu

\beta_1 \Phi

\beta_2 \Box \Phi = 0. \end{equation}


Equations \eqref{eq:F_Phi_EL}, \eqref{eq:F_v_EL}, and \eqref{eq:F_S_EL} are the coupled RSVP field equations derived from the action \eqref{eq:F_action}.

\subsection*{F.4 Stress–Energy Tensor and Conservation Laws}

The canonical stress–energy tensor is obtained by varying the action with respect to the metric: \begin{equation} T_{\mu\nu} := -\frac{2}{\sqrt{-g}}, \frac{\delta \mathcal{S}}{\delta g^{\mu\nu}}, \end{equation} which can be expressed as: \begin{equation} T_{\mu\nu}

2 \frac{\partial \mathcal{L}}{\partial g^{\mu\nu}}

g_{\mu\nu} \mathcal{L}


2 \frac{\partial \mathcal{L}}{\partial (\nabla^\mu \psi)} \nabla_\nu \psi, \end{equation} summing over all fields $\psi \in {\Phi, v_\alpha, S}$ with appropriate index handling. Since the Lagrangian is a scalar density built from $g_{\mu\nu}$, $\Phi$, $v_\mu$, $S$, and their derivatives, we have the covariant conservation law \begin{equation} \nabla_\mu T^{\mu\nu} = 0 \end{equation} whenever the Euler–Lagrange equations are satisfied (on-shell).


Additionally, any continuous internal symmetry of the Lagrangian yields a conserved Noether current. For instance, if there is a global shift symmetry in $S$ or in a combination of fields, one obtains a conserved entropic or informational charge.

\subsection*{F.5 Linearization and the RSVP Operator}

To connect the continuum field equations with the operator form used in the main text and in Appendix~C, we linearize around a background solution $(\Phi_0, v_{0\mu}, S_0)$ that satisfies the Euler–Lagrange equations.

Write \begin{align} \Phi &= \Phi_0 + \delta \Phi, \ v_\mu &= v_{0\mu} + \delta v_\mu, \ S    &= S_0 + \delta S. \end{align}

Substituting into \eqref{eq:F_Phi_EL}, \eqref{eq:F_v_EL}, \eqref{eq:F_S_EL} and retaining only linear terms in $(\delta\Phi, \delta v_\mu, \delta S)$ yields a coupled linear system: \begin{equation} \mathcal{E}_\Phi^{(1)}(\delta\Phi, \delta v, \delta S) = 0, \qquad \mathcal{E}_v^{(1)}(\delta\Phi, \delta v, \delta S) = 0, \qquad \mathcal{E}_S^{(1)}(\delta\Phi, \delta v, \delta S) = 0. \end{equation}

In compact form, define the perturbation multiplet

\Psi
:=
\begin{pmatrix}
\delta \Phi \\ \delta v_\mu \\ \delta S
\end{pmatrix}.

Then the linearized equations can be written as:
\begin{equation}
\mathcal{D} \Psi = 0,
\end{equation}
where $\mathcal{D}$ is a linear differential operator. In many cases of interest, one can factor $\mathcal{D}$ in a `wave-operator $\times$ mass-like' structure. For instance, in a flat background with $g_{\mu\nu} = \eta_{\mu\nu}$ and $v_{0\mu} = 0$, $S_0 = \text{const}$, the temporal and spatial derivatives decouple, leading to a second-order-in-time system that can be recast as:
\begin{equation}
\partial_t^2 \Psi
+ L_{\mathrm{RSVP}} \Psi
= 0,
\end{equation}
where $L_{\mathrm{RSVP}}$ is a spatial (and possibly parameter-dependent) operator acting on the perturbation fields.

For illustrative purposes, in a simple Minkowski background and neglecting mixed spatial derivatives beyond those already in the Lagrangian, $L_{\mathrm{RSVP}}$ takes a block form reminiscent of that used in the main text:
\begin{equation}
\label{eq:F_L_RSVP_block}
L_{\mathrm{RSVP}}
=
\begin{pmatrix}
- c_\Phi^2 \Delta + m_\Phi^2 & \gamma_{\Phi v} \nabla\cdot & \gamma_{\Phi S} \\
\gamma_{v\Phi} \nabla & - c_v^2 \Delta + m_v^2 & \gamma_{v S} \nabla \\
\gamma_{S\Phi} & \gamma_{S v} \nabla\cdot & - c_S^2 \Delta + m_S^2
\end{pmatrix},
\end{equation}
where $\Delta$ is the spatial Laplacian and the constants $c_\Phi, c_v, c_S, \gamma_{\cdot\cdot}$ are effective parameters derived from the couplings $(\alpha_1,\alpha_2,\beta_1,\beta_2,\kappa_S)$ and from the background configuration $(\Phi_0, v_{0\mu}, S_0)$.

In the presence of curvature or nontrivial background fields, $\Delta$ is replaced by the spatial Laplace–Beltrami operator on the spatial slices, and the covariant derivatives inherit additional geometric contributions. Nevertheless, the key idea persists: the RSVP field equations reduce to a linearized operator $L_{\mathrm{RSVP}}$ whose spectral properties (eigenvalues, eigenfunctions, spacing statistics) encode the dynamical repertoire of the system.

\subsection*{F.6 Special Cases and Symmetry-Reduced Models}

\paragraph{F.6.1 Pure Scalar Limit.}
If $v_\mu$ and $S$ are frozen (or decoupled), we obtain a standard Klein–Gordon equation:
\[
\Box \Phi - m_\Phi^2 \Phi = 0,

\paragraph{F.6.2 Vector–Scalar Model with Fixed Entropy.}
Fix $S = S_0$ and retain $\Phi$ and $v_\mu$.  
With the entropy field frozen, the coupled system reduces to

\begin{equation}
\Box \Phi - m_\Phi^2 \Phi
- \alpha_1 \nabla_\mu v^\mu
- \beta_1 S_0 = 0,
\label{eq:Phi_fixedS}
\end{equation}

and

\begin{equation}
\nabla_\mu F^{\mu\sigma}
+ m_v^2 v^\sigma
+ \alpha_1 \nabla^\sigma \Phi
+ \alpha_2 \nabla^\sigma S_0
= 0.
\label{eq:v_fixedS}
\end{equation}

Since $S_0$ is constant, $\nabla^\sigma S_0 = 0$, and the second equation simplifies to

\begin{equation}
\nabla_\mu F^{\mu\sigma}
+ m_v^2 v^\sigma
+ \alpha_1 \nabla^\sigma \Phi
= 0.
\label{eq:v_fixedS_simplified}
\end{equation}

Linearization around a constant background then yields a simplified block operator
acting on $(\delta\Phi,\, \delta v_\mu)$.

\paragraph{F.6.3 Diffusive Entropic Limit.} Assume $\Phi$ and $v_\mu$ are slow or suppressed, and focus on entropic dynamics:

\kappa_S \Box S - m_S^2 S = 0.

\paragraph{F.6.4 Cortical Neural-Field Reduction.} In the effective cortical limit discussed in the main text, one takes a spatially 3D domain with a curved metric approximating the cortical sheet, neglects relativistic effects, and focuses on slowly varying fields in time. The leading terms in $L_{\mathrm{RSVP}}$ reduce to a neural-field-like operator $L_{\mathrm{cortex}}$ with diffusion, wave, and coupling terms. The standing, traveling, and rotating wave solutions studied experimentally correspond to eigenfunctions of this reduced operator.

\subsection*{F.7 Summary}

Starting from the RSVP action \eqref{eq:F_action}, we derived the coupled Euler–Lagrange equations for the scalar field $\Phi$, the vector field $v_\mu$, and the entropy field $S$. Linearization around a background solution yields a block operator $L_{\mathrm{RSVP}}$ whose spectral properties control the wave dynamics and emergent modes of the plenum. Symmetry reductions and parameter limits recover standard field equations (Klein–Gordon, Proca, diffusive scalar) as well as the neural-field operators used to model cortical resonance. This variational foundation justifies treating $L_{\mathrm{RSVP}}$ as the core operator whose spectrum underlies the spectral universality phenomena explored in the main text.

% ======================================================================
% APPENDIX G: Operator Algebras and Symmetry Reductions
% ======================================================================

\section*{Appendix G: Operator Algebras and Symmetry Reductions}
\addcontentsline{toc}{section}{Appendix G: Operator Algebras and Symmetry Reductions}

This appendix formalizes the operator-algebraic structure underlying the
inclusion chain
\[
    L_{\mathrm{nucleus}}
        \;\subseteq\;
    L_{\zeta}
        \;\subseteq\;
    L_{\mathrm{cortex}}
        \;\subseteq\;
    L_{\mathrm{RSVP}},
\]
by treating each operator as the generator of a $C^*$- or von Neumann algebra.

\subsection*{G.1 Operator Algebras}

For a densely defined self-adjoint operator $L$ acting on a Hilbert space
$\mathcal{H}$, define its operator algebra
\[
    \mathcal{A}(L)
    =
    C^*(L)
    =
    \text{the smallest $C^*$-algebra containing $L$}.
\]

For the systems studied in this work, we obtain:
\[
\begin{aligned}
    \mathcal{A}_{\mathrm{nuc}}       &= C^*(H_{\mathrm{nuc}}), \\
    \mathcal{A}_{\zeta}              &= C^*(L_{\zeta}), \\
    \mathcal{A}_{\mathrm{cortex}}    &= C^*(L_{\mathrm{cortex}}), \\
    \mathcal{A}_{\mathrm{RSVP}}      &= C^*(L_{\mathrm{RSVP}}).
\end{aligned}
\]

Spectral universality arises whenever these algebras—despite very different
physical interpretations—induce identical unfolded correlation functions.

\subsection*{G.2 Symmetry Reductions}

Let $G$ be a symmetry group acting unitarily on $\mathcal{H}$.  
A symmetry-reduced operator is obtained by restricting $L$ to the invariant
subspace:
\[
    L|_{\mathcal{H}^G},
    \qquad
    \mathcal{H}^G
    =
    \{ \psi \in \mathcal{H} : g\psi = \psi \ \forall g\in G \}.
\]

This yields:
\[
    L_{\mathrm{nucleus}}
        = L_{\mathrm{RSVP}}\Big|_{\mathcal{H}^{G_{\mathrm{nuc}}}},
\]
and similarly for $L_{\zeta}$ and $L_{\mathrm{cortex}}$.

Thus the operator hierarchy derives from symmetry restriction, and universality
emerges because short-range spectral statistics are invariant under such
reductions.

% ======================================================================
% APPENDIX H: Spectral Geometry and the Plenum Manifold
% ======================================================================

\section*{Appendix H: Spectral Geometry and the Plenum Manifold}
\addcontentsline{toc}{section}{Appendix H: Spectral Geometry and the Plenum Manifold}

RSVP fields evolve on a differentiable manifold $(M,g)$ whose geometry shapes
the spectra of the governing operators. The geometric Laplacian is
\[
    \Delta_g \psi
    =
    \frac{1}{\sqrt{|g|}}\,
    \partial_i\!\left(
         \sqrt{|g|}\,g^{ij}\partial_j \psi
    \right).
\]

\subsection*{H.1 Eigenmodes on Curved Manifolds}

Eigenfunctions of the Laplace–Beltrami operator satisfy
\[
    -\Delta_g \psi_n = \lambda_n \psi_n,
\]
with asymptotics governed by Weyl’s law:
\[
    N(\lambda)
    \sim
    \frac{\mathrm{Vol}(M)}{(4\pi)^{d/2}}
    \lambda^{d/2}.
\]

Curvature induces mode splitting and modifies the nodal structure of
eigenfunctions. On highly folded surfaces such as cortex, this produces
multi-lobed standing-wave patterns consistent with empirical observations.

\subsection*{H.2 RSVP Operators on Curved Geometries}

The linearized RSVP operator takes the geometric form
\[
    L_{\mathrm{RSVP}}
    = A\Delta_g + B\nabla + C,
\]
with $A,B,C$ matrices derived from background fields.

Chaotic geodesic flow on $(M,g)$ induces random-matrix-like spectral
fluctuations at high frequencies. Thus curvature naturally pushes RSVP
fluctuations toward GOE/GUE universality, explaining parallels with nuclear
spectra and zeta zeros.

% ======================================================================
% APPENDIX I: Renormalization and Spectral Scaling Limits
% ======================================================================

\section*{Appendix I: Renormalization and Spectral Scaling Limits}
\addcontentsline{toc}{section}{Appendix I: Renormalization and Spectral Scaling}

Spectral universality emerges through renormalization: small-scale structure
becomes irrelevant under repeated coarse-graining.

Let $U_\ell$ be a dilation by scale $\ell$. Define the renormalized operator:
\[
    L_\ell = \ell^{-2}U_\ell L U_\ell^{-1}.
\]

\subsection*{I.1 Renormalization Flow}

Define the flow equation
\[
    \frac{d}{d\ell} L_\ell
    =
    \beta(L_\ell),
\]
whose fixed points satisfy $\beta(L^\star)=0$.

Hermitian random matrix ensembles are known fixed points of the flow of
coarse-grained Hamiltonians—explaining their remarkable universality.

\subsection*{I.2 Application to RSVP}

When the RSVP fields enter a high-entropy regime, the coefficients of
$L_{\mathrm{RSVP}}$ flow toward isotropic, weakly correlated limits:
\[
    A \to cI,
    \qquad
    B, C \to 0.
\]

Hence:
\[
    L_{\mathrm{RSVP}}
    \quad\longrightarrow\quad
    \text{GOE/GUE fixed point},
\]
recovering nuclear and zeta universality classes as limiting behaviors.

% ======================================================================
% APPENDIX J: Nonlinear RSVP Modes and Phase Structure
% ======================================================================

\section*{Appendix J: Nonlinear RSVP Modes and Phase Structure}
\addcontentsline{toc}{section}{Appendix J: Nonlinear RSVP Modes and Phase Structure}

The full RSVP dynamics include nonlinear couplings such as
\[
\mathcal{N}(\Phi,v,S)
  =
  \lambda_1 \Phi^3
  + \lambda_2 (v^\mu v_\mu)\Phi
  + \lambda_3 S\nabla_\mu v^\mu
  + \cdots
\]

\subsection*{J.1 Bifurcations and Spectral Phase Transitions}

Nonlinearities give rise to:

\begin{itemize}
    \item Hopf bifurcations generating oscillatory modes,
    \item symmetry-breaking pitchfork bifurcations,
    \item chaotic attractors when entropy dominates,
    \item soliton-like wane modes in $\Phi$.
\end{itemize}

This yields transitions between spectral phases:
\[
    \text{Poisson}
        \;\rightleftarrows\;
    \text{GOE/GUE}
        \;\rightleftarrows\;
    \text{Coherent resonant}.
\]

\subsection*{J.2 Cognitive Interpretation}

Such transitions may correspond to:

\begin{itemize}
    \item sleep--wake cycles,
    \item anesthesia onset and emergence,
    \item attentional reset events,
    \item spread of global workspace activation.
\end{itemize}

Thus nonlinear RSVP dynamics supply a unified spectral account of physiological
and cognitive phase transitions.

% ======================================================================
% APPENDIX K: Computational Experiments for RSVP Spectra
% ======================================================================

\section*{Appendix K: Computational Experiments for RSVP Spectra}
\addcontentsline{toc}{section}{Appendix K: Computational Experiments}

This appendix presents computational techniques for validating the spectral
predictions of RSVP through numerical discretizations.

\subsection*{K.1 Discrete RSVP Operator}

Given a lattice with spacing $h$, approximate the Laplacian by
\[
\nabla^2 \psi_i
    \approx
    \frac{1}{h^2}
    \sum_{j\in N(i)} (\psi_j - \psi_i).
\]

The discretized linear RSVP operator is a block matrix
\[
    L_{\mathrm{RSVP}}^{(h)}
    =
    \begin{pmatrix}
        A^{(h)} & B^{(h)} & C^{(h)} \\
        D^{(h)} & E^{(h)} & F^{(h)} \\
        G^{(h)} & H^{(h)} & J^{(h)}
    \end{pmatrix}.
\]

\subsection*{K.2 Spectral Diagnostics}

Compute:

\begin{enumerate}
    \item eigenvalue spacings $s_i$,
    \item unfolded spacing histograms,
    \item two- and three-point correlation functions,
    \item spectral rigidity $\Delta_3(L)$.
\end{enumerate}

Compare these to GOE, GUE, Poisson, and empirical cortical spectra.

\subsection*{K.3 Predictions}

RSVP predicts:

\[
\begin{aligned}
  &\text{High-entropy: GOE/GUE universality}, \\
  &\text{Low-entropy: coherent multi-band modes}, \\
  &\text{Intermediate: zeta-like rigidity}.
\end{aligned}
\]

These provide concrete, falsifiable tests for the theory.

% ======================================================================
% APPENDIX L: Index of Symbols and Glossary
% ======================================================================

\section*{Appendix L: Index of Symbols and Glossary}
\addcontentsline{toc}{section}{Appendix L: Index of Symbols and Glossary}

This appendix summarizes the principal symbols, operators, fields, and
mathematical objects appearing throughout the text.  
It is organized into four categories: spectral systems, RSVP field theory,
operator structures, and neuroscientific constructs.

% ------------------------------------------------------------
% L.1 Spectral Systems
% ------------------------------------------------------------
\subsection*{L.1 Spectral Systems}

\begin{description}[leftmargin=2.2cm,style=nextline]

\item[$\mathcal{H}_X$]  
Hilbert space associated with a system $X$ (nucleus, zeta operator, cortex, RSVP).

\item[$\mathcal{A}_X$]  
Operator algebra generated by the system operator:  
$\mathcal{A}_X = C^*(L_X)$.

\item[$L_X$]  
Primary linear operator governing dynamics or constraints for the system $X$.

\item[$\mu_X$]  
Spectral measure induced by $L_X$ on $\mathbb{R}$.

\item[$R_k^{(X)}$]  
$k$-point correlation functions for the unfolded spectrum of $L_X$.

\item[$\mathcal{E}(L)$]  
Spectrum (eigenvalues) of the operator $L$.

\item[\textbf{GOE}, \textbf{GUE}]  
Gaussian Orthogonal/Unitary Ensembles; universality classes for random spectra.

\end{description}

% ------------------------------------------------------------
% L.2 RSVP Fields and Plenum Dynamics
% ------------------------------------------------------------
\subsection*{L.2 RSVP Fields and Plenum Dynamics}

\begin{description}[leftmargin=2.2cm,style=nextline]

\item[$\Phi(x,t)$]  
Scalar potential field of the RSVP plenum.

\item[$v_\mu(x,t)$]  
Vector flow field (momentum-, drift-, or current-like component).

\item[$S(x,t)$]  
Entropy density or entropic potential.

\item[$g_{\mu\nu}$]  
Metric tensor on the plenum manifold $M$.

\item[$\mathcal{S}$]  
Action functional for RSVP dynamics:
$\mathcal{S}=\int \mathcal{L}\sqrt{|g|}\,dx\,dt$.

\item[$L_{\mathrm{RSVP}}$]  
Linearized RSVP operator acting on perturbations $(\delta\Phi,\delta v,\delta S)$.

\item[$\mathcal{N}(\Phi,v,S)$]  
Nonlinear interaction functional governing higher-order coupling.

\end{description}

% ------------------------------------------------------------
% L.3 Differential and Operator Structures
% ------------------------------------------------------------
\subsection*{L.3 Differential and Operator Structures}

\begin{description}[leftmargin=2.2cm,style=nextline]

\item[$\nabla$, $\nabla_\mu$]  
Covariant derivative induced by $g_{\mu\nu}$.

\item[$\Delta_g$]  
Laplace–Beltrami operator on $(M,g)$:
$\Delta_g = \frac{1}{\sqrt{|g|}}\partial_i(\sqrt{|g|}g^{ij}\partial_j)$.

\item[$F_{\mu\nu}$]  
Field-strength-like tensor associated with $v_\mu$:
$F_{\mu\nu} = \nabla_\mu v_\nu - \nabla_\nu v_\mu$.

\item[$A,B,C$]  
Coefficient matrices in the block form of $L_{\mathrm{RSVP}}$.

\item[$U_\ell$]  
Scaling operator implementing geometric dilation by factor $\ell$.

\item[$\beta(L_\ell)$]  
RG beta-function governing renormalization flow of operators.

\end{description}

% ------------------------------------------------------------
% L.4 Neuroscientific and Cognitive Constructs
% ------------------------------------------------------------
\subsection*{L.4 Neuroscientific and Cognitive Constructs}

\begin{description}[leftmargin=2.2cm,style=nextline]

\item[$L_{\mathrm{cortex}}$]  
Effective cortical wave operator governing standing and traveling waves.

\item[Cabral–Shemesh modes]  
Macroscale standing-wave eigenmodes observed with ultrafast fMRI.

\item[Traveling waves]  
Propagating beta/gamma oscillations supporting working memory.

\item[Rotating waves]  
Cortical rotational modes controlling attentional resetting.

\item[Mixed selectivity]  
High-dimensional neural coding reduced to low-dimensional latent wave modes.

\item[Spectral regime]  
Global oscillatory structure (coherent, incoherent, anesthetic, etc.)

\end{description}

% ------------------------------------------------------------
% L.5 RSVP Spectral Phases
% ------------------------------------------------------------
\subsection*{L.5 RSVP Spectral Phases}

\begin{description}[leftmargin=2.2cm,style=nextline]

\item[\textbf{Poisson}]  
Uncorrelated eigenvalues; low-coherence regime.

\item[\textbf{GOE/GUE}]  
Universal chaotic regime; high entropy or symmetry-reduced limit.

\item[\textbf{Coherent}]  
Structured, low-dimensional resonant wave regime (cortex-like).

\end{description}


% ======================================================================
% REFERENCES
% ======================================================================

\printbibliography

\end{document}

