\documentclass[11pt]{article}

\usepackage[margin=1in]{geometry}
\usepackage{setspace}
\setstretch{1.15}

\usepackage{amsmath, amssymb, amsthm}

\usepackage{hyperref}
\hypersetup{hidelinks}

\title{Constraint Before Optimization:\\
Toward a Thermodynamics of Trust and Meaning}
\author{Flyxion}
\date{}

\begin{document}
\maketitle

\section{Introduction}

The persistent failure of digital platforms, institutional metrics, and large-scale optimization systems has often been framed as a problem of incentives, values, or governance. This essay advances a more fundamental claim: many such failures arise from violations of informational conservation laws that precede any normative debate. When identity does not uniquely bind actions to histories, meaning cannot accumulate, reputation cannot stabilize, and optimization becomes thermodynamically incoherent.

The framework developed here treats identity as infrastructural rather than expressive, and reputation as a conserved quantity rather than a moral judgment. It draws a sharp distinction between two layers of system design: a \emph{conservation layer}, which governs whether information can persist at all, and a \emph{valuation layer}, which governs what that information should be used to optimize. Much contemporary discourse collapses these layers, attempting to solve valuation problems in regimes where conservation has already failed.

The central claim of this essay is therefore anti-instrumentalist: meaning cannot be optimized into existence once its structural substrate has been dissolved. It must be conserved by design.

\section{Conservation Without Valuation}

At the core of the framework lies a deliberate abstention from value specification. This is not a philosophical evasion, but a structural necessity. Reputation, trust, and attribution become well-defined quantities not because they are morally good, but because they are conserved under a persistent binding of identity to history.

The pipe-system metaphor is exact. Identity coherence ensures that informational pipes do not leak. It does not specify whether what flows through those pipes is virtuous or harmful. This distinction is often misunderstood as a weakness. In fact, it is the source of the framework’s strength. A system that conserves identity renders accountability inescapable. In such a system, a consistently malicious actor does not evade the equations; they satisfy them by accumulating a deep basin of negative reputation that cannot be shed.

The prevailing pathology of contemporary platforms is not that bad actors have stable identities, but that they do not. Identity fragmentation allows actors to externalize the cost of their behavior by abandoning histories. The framework demonstrates that this escape hatch is the true source of metric collapse. Value debates conducted in such environments are incoherent, because the system cannot remember who did what long enough for consequences to bind.

This clarifies the boundary of the framework. It provides a conservation law for meaning, not a moral theory. A second layer of normative specification is required to determine which latent qualities should accumulate under conservation. Crucially, that second layer is only actionable once conservation is secured.

\section{Identity Dispersion and Phase Transitions}

The framework identifies identity dispersion, defined as the ratio of apparent identifiers to actual agents, as the primary control parameter governing system stability. When dispersion exceeds a critical threshold, the variance introduced by attributional ambiguity overwhelms the sensitivity of metrics to underlying quality. At that point, Goodhart collapse becomes inevitable.

What remains empirically unspecified is the numerical value of this critical threshold and its dependence on topology, interaction rate, and domain. This is not a theoretical gap so much as an open empirical program. The framework provides the thermodynamics; it does not claim to have derived the equation of state for every platform.

However, the qualitative prediction is robust. Collapse is not gradual but bifurcational. Systems appear stable until they are not, at which point coherence dissipates rapidly. This nonlinearity explains why platforms often lose trust abruptly rather than incrementally. It also explains why restoration efforts frequently fail. Once scalar basins of identity coherence have flattened, rebuilding them requires far stronger enforcement than was originally needed to maintain them. This hysteresis effect suggests that some large platforms may indeed have crossed an irreversibility threshold.

\section{Cold Start, Homotopy, and Accessibility}

A persistent tension in any reputation system is the cold-start problem. New participants enter with shallow histories and therefore shallow basins of trust. Weakening identity constraints to ease entry solves this problem only by flattening the entire scalar field, destroying meaning for everyone.

The resolution proposed here is neither binary enforcement nor permissive ambiguity, but homotopy-coherent identity. Identity need not be rigidly identical across all contexts; it must be contractibly recoverable. Small variations—devices, pseudonyms, roles—are admissible so long as they deform back into a unique history. Large, non-contractible divergences correspond to identity fragmentation and must be prevented.

Operationally, this reframes accessibility. New entrants need not possess reputation, but they must possess persistence. Shallow basins are acceptable; fragmented basins are not. Systems should therefore lower barriers through time-based accumulation rather than through ambiguity. Persistence substitutes for privilege without sacrificing conservation.

\section{Emergent Enforcement and Decentralization}

The framework’s initial formulation treats enforcement as an external control. In decentralized systems, this assumption fails. Enforcement must emerge from protocol design and agent incentives rather than administrative decree.

The stability condition in such systems is clear: enforcement capacity must scale with reputation density. If agents benefit from a high-coherence environment, and if that coherence is required to preserve their own accumulated reputation, then they are incentivized to collectively defend the namespace. Under these conditions, enforcement becomes endogenous. Attacks on identity coherence directly damage the attacker’s own basin.

This suggests a design principle for decentralized systems. Identity duplication must be costly, persistence must be rewarded, and the cost of harming coherence must exceed the benefits of metric exploitation. Cryptographic mechanisms, stake-based identity, and gradual trust accrual all instantiate this principle in different domains.

\section{Medical Research as a Boundary Case}

Applying the framework to medical research exposes its limits with unusual clarity. Identity coherence in this domain corresponds to longitudinal binding of observations to patients, tissues, or experimental protocols. High coherence enables reproducibility and evidence accumulation. Yet even perfect identity binding does not guarantee valid outcomes. One can measure the wrong endpoints with exquisite precision.

This confirms the framework’s intended scope. Identity coherence is necessary for medical inference, not sufficient. It ensures that observations mean something stable. It does not determine what should be optimized. That decision remains clinical, ethical, and political. The framework therefore functions as a precondition for medical knowledge, not a replacement for medical judgment.

\section{Where This Goes Next}

The theory is now structurally complete. The question is not what it means, but what to do with it.

The next step is empirical calibration. Without measurements of identity dispersion, metric sensitivity, and entropy production in real systems, the framework remains descriptive. The immediate priority is therefore instrumentation: dashboards that expose dispersion coefficients, correlation decay between metrics and external quality signals, and localized entropy gradients. These measurements would allow platforms to detect impending collapse before it becomes irreversible.

The second step is protocol design. Small-scale, high-coherence systems should be built as reference implementations. Academic peer review, specialized professional communities, and medical data infrastructures are promising candidates. The goal is not mass adoption, but proof that identity conservation can be maintained under optimization pressure without authoritarian enforcement.

Further mathematical refinement is no longer the bottleneck. The categorical and homotopical machinery already exceeds what current systems can implement. Additional abstraction will not resolve the core risk, which is practical rather than theoretical: attempting to impose high-coherence constraints on systems that have already collapsed may trigger backlash, exclusion, or catastrophic hysteresis.

The long-term implication is sobering. Large platforms that have structurally dissolved their namespaces may not be recoverable. In such cases, the framework suggests that new systems must be built alongside them, not on top of them, offering continuity through trust transfer rather than restoration.

\section{Constraint Before Optimization}

\subsection{Statement of the Principle}

The principle of \emph{constraint before optimization} asserts that in any complex system, the enforcement of fundamental constraints must precede and condition all attempts at optimization. Optimization is only meaningful within a domain that has already been rendered coherent, feasible, and stable by prior constraint satisfaction. When this ordering is reversed, optimization ceases to improve system performance and instead amplifies noise, instability, and misattribution.

This principle is not domain-specific. It appears, often implicitly, across biology, engineering, systems design, and mathematical optimization. What varies between domains is not the ordering itself, but the nature of the constraints being enforced.

\subsection{Axiomatic Formulation}

The framework developed in this work adopts the following axioms.

\paragraph{Axiom 1 (Feasibility Precedes Optimality).}
A solution that violates any necessary constraint is inadmissible, regardless of its performance with respect to an objective function.

\paragraph{Axiom 2 (Constraints Define the State Space).}
Constraints determine the feasible region of a system. Optimization operates only within this region and cannot meaningfully evaluate states outside it.

\paragraph{Axiom 3 (Essential Constraints Are Non-Negotiable).}
Certain constraints correspond to structural necessities rather than preferences. These constraints cannot be traded off against optimization objectives without destabilizing the system.

\paragraph{Axiom 4 (Optimization Amplifies Existing Structure).}
Optimization processes magnify patterns already present in the constrained system. If constraints are weak or incoherent, optimization amplifies error, noise, and exploitation rather than signal.

These axioms are descriptive rather than normative. They do not prescribe what a system should value, only the conditions under which valuation is well-defined.

\subsection{Constraints as Structural Primitives}

Constraints are not secondary conditions appended to an otherwise free optimization problem. They are structural primitives that determine what it means for a system state to exist at all. In physical systems, these include conservation laws and boundary conditions. In engineered systems, they include safety margins, timing constraints, and interface specifications. In biological systems, they include bottlenecks, rate-limiting processes, and failure hierarchies.

A system that attempts to optimize without first enforcing its defining constraints does not converge toward better solutions. Instead, it explores regions of the state space that are mathematically reachable but structurally meaningless.

\subsection{Biological and Clinical Interpretation}

In biological and clinical contexts, the principle of constraint before optimization manifests as the prioritization of limiting factors. Biological systems fail in an ordered manner: certain constraints must be violated before others become relevant. Recovery therefore follows the reverse order.

The BioIntel method makes this ordering explicit by identifying the single most critical biological bottleneck and addressing it before attempting to optimize secondary markers. Attempts to optimize peripheral indicators while a primary constraint remains violated produce misleading improvements that do not generalize and often regress.

This reflects a deeper truth: biology does not optimize globally. It stabilizes locally under constraints, and only then explores degrees of freedom that remain.

\subsection{Engineering and Systems Design}

In engineering disciplines, constraint before optimization is enforced by design practice. Timing, power, safety, and interface constraints are specified prior to synthesis or compilation. Optimization tools are not permitted to violate these constraints, regardless of the performance gains that might result.

This ordering is essential. Designs that violate constraints may appear optimal in simulation but fail catastrophically in deployment. The constraints are therefore not obstacles to optimization; they are the conditions that make optimization meaningful.

\subsection{Mathematical Optimization}

Formally, an optimization problem is defined as the extremization of an objective function subject to constraints. The constraints are not auxiliary information; they define the problem itself. An incorrectly specified constraint set guarantees a meaningless solution, even if the objective function is optimized perfectly.

From this perspective, constraint specification is not a preprocessing step but the core modeling act. Errors at this stage cannot be corrected by more sophisticated optimization methods.

\subsection{Constraint Before Optimization in RSVP}

Within the RSVP framework, the principle of constraint before optimization acquires a precise interpretation. Identity coherence, attributional persistence, and entropy bounds function as constraints on the plenum. Scalar coherence fields enforce admissibility of histories; vector fields describe permissible flows; entropy measures constraint violation.

Optimization of metrics, predictions, or engagement corresponds to vector flow amplification. If scalar constraints are weak or inconsistent, vector optimization produces turbulence rather than structure. This is not a moral failure of optimization but a structural consequence of violating the axioms above.

\subsection{Failure Modes Under Constraint Violation}

When optimization precedes constraint enforcement, systems exhibit characteristic failure modes:
metrics decouple from underlying quality, attribution becomes ambiguous, feedback loops become self-reinforcing, and apparent performance gains fail to generalize or persist.

These failures are not accidental. They are the predictable outcome of amplifying objectives in a state space that has not been rendered coherent.

\subsection{Implications}

The principle of constraint before optimization implies that many contemporary debates about misaligned objectives, perverse incentives, or algorithmic bias are downstream of a more fundamental error: attempting to optimize in systems where the constraints required for meaning have already been violated.

Restoring optimization to a meaningful role therefore requires restoring constraint coherence first. Only after this restoration can questions of values, objectives, and trade-offs be addressed in a non-pathological way.

\subsection{Summary}

Constraint before optimization is not a heuristic, a preference, or a slogan. It is a structural ordering principle that governs whether optimization can succeed at all. Systems that respect this ordering accumulate meaning, evidence, and trust. Systems that violate it do not merely optimize poorly; they optimize themselves into incoherence.

\section{Optimization as Amplification Under Constraint}

\subsection{Optimization Does Not Discover Structure}

Optimization procedures do not discover structure ex nihilo. They amplify gradients that already exist within the feasible space defined by constraints. When those constraints are well-formed, optimization sharpens distinctions that correspond to latent structure. When constraints are weak, ambiguous, or internally inconsistent, optimization amplifies artifacts of measurement, noise, or adversarial exploitation.

This observation reframes optimization not as a creative process but as an accelerant. It increases the magnitude of whatever dynamics are already permitted by the system’s constraint geometry.

\subsection{Axioms of Optimization-Induced Failure}

The following axioms formalize the conditions under which optimization becomes pathological.

\paragraph{Axiom 5 (Optimization Amplifies Proxy Signals).}
Optimization increases sensitivity to whatever variables are most easily manipulable within the constrained space, regardless of whether those variables correspond to the intended target.

\paragraph{Axiom 6 (Constraint Weakening Increases Exploitability).}
As constraints weaken, the space of admissible manipulations grows faster than the space of admissible meanings.

\paragraph{Axiom 7 (Metric Sensitivity Outpaces Semantic Sensitivity).}
In poorly constrained systems, optimization increases responsiveness to metrics faster than responsiveness to underlying qualities those metrics were intended to represent.

\paragraph{Axiom 8 (Optimization Without Reference Produces Drift).}
If optimization proceeds without persistent reference to the same entities across time, the optimized quantity ceases to correspond to any stable object of evaluation.

Together, these axioms explain why optimization failures are not edge cases but generic outcomes once constraint coherence is lost.

\subsection{Goodhart Effects as Structural, Not Moral, Failures}

Goodhart’s Law is often treated as a warning about poorly chosen objectives. In this framework, it is instead understood as a consequence of optimizing in a space where the binding between measurements and referents has degraded.

When identity, attribution, or continuity constraints fail, metrics do not merely become imperfect proxies. They become detached from any conserved quantity. Optimization then maximizes a free-floating signal that no longer tracks an underlying state.

This explains why replacing one metric with another rarely solves the problem. Without restoring the constraints that bind metrics to entities, any new objective will be gamed in structurally identical ways.

\subsection{Vector Flow and Scalar Collapse in RSVP}

Within RSVP, optimization corresponds to increased magnitude of the vector field. Constraint coherence corresponds to curvature and depth in the scalar field. When scalar gradients are strong, vector flow follows meaningful paths. When scalar gradients flatten, vector flow becomes turbulent and self-referential.

This yields a precise failure mode: increased engagement, activity, or throughput coinciding with declining informational value. The system appears active while becoming epistemically inert.

Optimization does not cause collapse; it reveals it.

\subsection{Why Post-Hoc Correction Fails}

Once optimization has amplified proxy signals in a weakly constrained system, post-hoc correction becomes increasingly expensive. Each corrective intervention introduces additional rules, exceptions, or heuristics that further complicate the constraint landscape without restoring coherence.

This produces regulatory accretion rather than structural repair. The system accumulates patches instead of recovering its original binding constraints. In RSVP terms, entropy production outpaces scalar restoration, and the system enters a hysteretic regime where previous levels of enforcement are insufficient to recover coherence.

\subsection{Implications for Design and Governance}

The preceding analysis implies that responsible system design must treat optimization as a late-stage operation. Constraints must be explicitly articulated, enforced, and monitored before any performance objective is introduced.

In practical terms, this means that identity persistence, attribution fidelity, and continuity of reference are prerequisites for optimization, not optional features to be added later. Systems that violate this ordering will inevitably optimize for appearances rather than substance.

\subsection{Summary}

Optimization is not inherently dangerous, nor is it inherently beneficial. Its effects are determined entirely by the constraint structure within which it operates. When constraints preserve reference, optimization sharpens meaning. When constraints dissolve, optimization accelerates collapse.

The lesson is therefore not to abandon optimization, but to respect its dependence on constraint coherence. Optimization cannot repair a system whose foundational bindings have already failed.

\section{Constraint Hierarchies and Failure Ordering}

Complex systems do not fail arbitrarily. Their modes of breakdown exhibit a strict ordering determined by the hierarchical structure of constraints that govern admissible states. This ordering is not merely descriptive but structural: certain constraints must be satisfied before others are even meaningfully defined. The principle of \emph{constraint before optimization} arises from this asymmetry. Optimization presupposes feasibility, and feasibility presupposes constraint satisfaction.

A constraint hierarchy is defined as a partial ordering over system conditions such that violation of a higher-order constraint renders lower-order constraints either unobservable or irrelevant. In such systems, failure is non-commutative: the sequence in which constraints are violated matters, and reversing that sequence does not generally restore system function. This non-commutativity explains why recovery must proceed in the inverse order of failure and why downstream optimization cannot compensate for upstream constraint collapse.

\subsection{Axioms of Constraint Ordering}

\textbf{Axiom 1 (Feasibility Precedence).}  
For any system with state space $\Omega$ and constraint set $\mathcal{C}$, optimization over an objective function $J : \Omega \rightarrow \mathbb{R}$ is well-defined if and only if there exists a non-empty feasible region $\Omega_{\mathcal{C}} \subseteq \Omega$ satisfying all constraints in $\mathcal{C}$. If $\Omega_{\mathcal{C}} = \varnothing$, optimization is undefined.

\textbf{Axiom 2 (Hierarchical Masking).}  
If a constraint $c_i$ dominates a constraint $c_j$ in the hierarchy, then violation of $c_i$ masks violations of $c_j$. Formally, the diagnostic signal associated with $c_j$ becomes non-identifiable under violation of $c_i$.

\textbf{Axiom 3 (Ordered Failure).}  
Constraint violations propagate downward through the hierarchy. That is, if $c_i$ is violated and $c_i \prec c_j$, then the system dynamics will necessarily exhibit apparent failure in variables governed by $c_j$, even if $c_j$ itself is locally satisfied.

\textbf{Axiom 4 (Ordered Recovery).}  
Recovery propagates upward through the hierarchy. No intervention targeting $c_j$ can restore system function while $c_i \prec c_j$ remains violated.

These axioms formalize the empirical observation that systems fail in a specific order and recover in the reverse order. Attempts to optimize downstream variables in the presence of upstream constraint violations result in what appear to be paradoxical or unstable behaviors, but which are in fact consistent with the underlying structure.

\subsection{Failure Masking and Diagnostic Illusions}

A direct consequence of hierarchical masking is that system diagnostics become misleading under high-level constraint violation. Observable metrics respond to downstream variables, which are themselves functions of upstream feasibility. Optimization applied at this level often produces short-lived improvements that collapse under continued operation, as the violated constraint continues to inject entropy into the system.

This explains the common phenomenon in engineering, biology, and institutional systems where increasingly aggressive optimization worsens outcomes. The apparent paradox arises because optimization is applied to variables whose governing constraints are no longer binding. The system appears responsive while in fact drifting further from feasibility.

\subsection{RSVP Interpretation}

Within the Relativistic Scalar--Vector Plenum framework, constraint hierarchies map directly onto field structure. Scalar coherence $\Phi$ represents the satisfaction of primary constraints. Vector flow $\vec{v}$ represents optimization dynamics acting within the feasible region defined by $\Phi$. Entropy $S$ accumulates when vector flow persists in regions where scalar coherence has collapsed.

Violation of a high-order constraint corresponds to flattening of the scalar field. Once $\nabla \Phi \approx 0$, vector dynamics decouple from quality gradients and become turbulent. Optimization pressure increases entropy production rather than reducing it. This regime corresponds to metric chasing, ritualized behavior, and pathological equilibria.

\subsection{Implications}

The principle of constraint before optimization follows directly. Systems must first satisfy the constraints that define their feasible space before attempting to optimize any objective within it. Failure to respect this ordering guarantees instability, regardless of the sophistication of the optimization algorithm.

This principle is not normative but structural. It does not prescribe what should be optimized, only what must be true for optimization to be meaningful at all. In subsequent sections, this ordering will be applied to identity persistence, reputation accumulation, value learning, and phase transitions in large-scale sociotechnical systems.

\section{Identity as a Conserved Quantity}

The persistence of identity across time is a prerequisite for any system that claims to accumulate evidence, reputation, or learning. Without such persistence, observations cannot be bound to a referent, histories cannot compose, and quantities that appear to accumulate are in fact dissipating into noise. This section formalizes identity not as a label or attribute, but as a conserved informational quantity that underwrites attribution.

In physical systems, conservation laws guarantee that quantities such as mass or charge remain invariant under admissible transformations. In informational systems, identity plays an analogous role. It provides the invariant with respect to which actions, states, and outcomes can be integrated over time. When identity is conserved, accumulation is meaningful. When it is not, accumulation is illusory.

\subsection{Identity as Binding Operator}

Let $A$ denote the set of agents, $H$ the space of histories, and $I$ the space of identifiers. An identity system is a mapping $\eta : A \rightarrow I$ together with a history-binding relation $\beta : (A \times T) \rightarrow H$ such that actions performed at different times compose into a single trajectory.

Identity is conserved if and only if the mapping from actions to histories remains injective with respect to the agent. If multiple agents map to the same effective identifier, or if a single agent fragments across multiple non-reconcilable identifiers, the binding relation fails and history ceases to be well-defined.

\subsection{Axioms of Identity Conservation}

\textbf{Axiom 5 (Non-Duplication).}  
At any time $t$, no two distinct agents $a_1 \neq a_2$ may occupy the same identity fiber such that their actions are indistinguishable under the attribution map.

\textbf{Axiom 6 (Continuity).}  
For any agent $a$ and times $t_1 < t_2$, the histories $\beta(a,t_1)$ and $\beta(a,t_2)$ must be composable into a single trajectory without ambiguity.

\textbf{Axiom 7 (Attributability).}  
For any observed action or outcome $o$, there exists a unique identity $i \in I$ such that $o$ is attributable to $i$ with probability one in the absence of noise.

Together, these axioms define identity conservation. Violation of any one of them produces irreversible information loss. Once an action cannot be uniquely attributed to a history-bearing entity, no amount of downstream inference can restore that attribution.

\subsection{Entropy Production Under Identity Loss}

Identity ambiguity increases the conditional entropy of attribution. Let $X$ denote the random variable representing agents and $Y$ the observed identifiers. If $\eta$ is non-injective, then $H(X \mid Y) > 0$ necessarily. This residual uncertainty represents structural entropy, not epistemic ignorance. It cannot be eliminated by better inference, additional data, or improved modeling.

As identity dispersion increases, the system experiences a monotonic increase in entropy production. Apparent activity may increase, but informational content does not. Reputation signals flatten, evidence becomes non-composable, and learning dynamics decouple from underlying reality.

\subsection{RSVP Interpretation}

Within the RSVP framework, identity coherence corresponds to a scalar field $\Phi$ whose magnitude reflects the strength of history binding. Regions of high $\Phi$ permit accumulation of reputation density $\rho$ over time. Regions of low $\Phi$ cannot support accumulation; any injected signal rapidly thermalizes into entropy $S$.

Identity duplication or fragmentation acts as an entropy source term. It injects disorder directly into the attribution layer, bypassing any corrective effect of optimization. Vector flows $\vec{v}$ operating in such regions amplify noise rather than extracting structure.

\subsection{Consequences}

Identity conservation is not a moral preference or policy choice. It is a structural requirement for systems that claim to measure, learn, or govern. Without it, reputation systems degenerate into performative equilibria, metrics lose referential meaning, and evidence-based decision making becomes impossible.

The next section will show that value learning and alignment presuppose this conserved substrate. Without persistent reference, optimization cannot distinguish improvement from noise, and values cannot be stably learned or enforced.

\section{Why Value Learning Fails Without Persistent Reference}

Systems that attempt to learn, enforce, or align values without persistent reference inevitably collapse into proxy optimization. This failure is not due to insufficient data, imperfect objectives, or adversarial agents, but to the absence of a conserved substrate against which values can be evaluated over time. Value learning presupposes identity coherence in the same way optimization presupposes feasibility.

A value is not an abstract preference detached from history. It is a functional relation between actions and consequences evaluated across time. If the entity performing actions cannot be uniquely identified across those evaluations, the value signal cannot be integrated. What remains is optimization over instantaneous observables rather than longitudinal outcomes.

\subsection{The Reference Problem}

Let $Q$ denote a latent quality variable that a system seeks to promote, such as trustworthiness, competence, or well-being. Let $M$ be a measurable proxy for $Q$. Learning requires repeated evaluation of $M$ as a function of actions taken by a stable referent. If the referent is unstable, the mapping $M \mapsto Q$ becomes non-identifiable.

In the absence of persistent identity, optimization operates on equivalence classes of actions rather than agents. Rewards and penalties no longer bind to histories but to surface patterns. The system cannot distinguish improvement from imitation, nor learning from exploitation.

\subsection{Axioms of Reference Dependence}

\textbf{Axiom 8 (Persistence Requirement).}  
Value learning requires that the entity being evaluated persists across evaluations. Without persistence, value gradients cannot be estimated.

\textbf{Axiom 9 (Comparability).}  
Evaluation of improvement requires that measurements taken at different times refer to the same underlying entity. If referents differ, comparisons are undefined.

\textbf{Axiom 10 (Continuity of Consequence).}  
Consequences must be attributable to prior actions of the same referent. If consequences detach from actions, reinforcement collapses into noise.

These axioms formalize why alignment efforts fail in high-entropy identity regimes. Optimization presupposes a reference frame; value learning presupposes a trajectory. Remove either, and the learning problem becomes ill-posed.

\subsection{Goodhart Collapse Reinterpreted}

Goodhart’s Law is often framed as a failure of objective design: when a measure becomes a target, it ceases to be a good measure. In the present framework, Goodhart collapse is reinterpreted as a failure of reference conservation. Metrics cease to correlate with quality not because optimization is intrinsically corrupting, but because identity ambiguity severs the binding between action and outcome.

When identity is unstable, optimization pressure flows into shallow basins of imitation. The system rewards behaviors that resemble success without being causally connected to it. This produces stable equilibria characterized by high activity and low informational content.

\subsection{RSVP Interpretation}

In RSVP terms, value learning requires a nonzero scalar gradient $\nabla \Phi$ to guide vector flow. When identity coherence collapses, the scalar field flattens and gradients vanish. Vector flow $\vec{v}$ persists but becomes unguided, circulating around local noise-driven attractors.

Entropy production increases as optimization continues without constraint. The system expends increasing energy to maintain activity while extracting no additional structure. This regime corresponds to metric farming, performative alignment, and symbolic compliance.

\subsection{Implications for Alignment}

Alignment cannot be achieved by refining reward functions alone. Without persistent reference, no reward function can remain anchored to its intended target. Identity coherence is therefore a necessary precondition for alignment, not a downstream implementation detail.

This reframes alignment as a problem of infrastructure rather than intent. Before a system can learn what to value, it must be able to remember who acted and what followed. The next section formalizes how failure to maintain these conditions leads to phase transitions and irreversible collapse.

\section{Phase Transitions and Irreversibility in Complex Systems}

Many large-scale systems exhibit a characteristic asymmetry between degradation and recovery. Performance degrades gradually, often invisibly, and then collapses abruptly. Restoration, when possible at all, requires disproportionately greater intervention than was required to maintain stability. This phenomenon is not accidental. It reflects an underlying phase transition driven by violation of primary constraints.

In the context developed here, the critical transition is governed by identity coherence. As long as identity remains conserved, local failures remain local and optimization dynamics remain coupled to quality gradients. Once identity dispersion exceeds a critical threshold, the system undergoes a qualitative shift into a high-entropy regime in which meaning, attribution, and learning are no longer conserved.

\subsection{Control Parameters and Criticality}

Let $\lambda(x,t)$ denote the identity dispersion parameter, defined as the ratio between apparent identifiers and actual persistent agents within a region of the system. Let $E(x,t)$ denote enforcement strength, whether administrative, cryptographic, or emergent. System stability depends on the ratio between dispersion-driven entropy injection and coherence-preserving enforcement.

There exists a critical threshold $\lambda_c$ such that for $\lambda < \lambda_c$, scalar coherence is maintained and perturbations decay. For $\lambda \geq \lambda_c$, scalar coherence collapses and perturbations amplify. Importantly, $\lambda_c$ is not universal. It depends on interaction velocity, system size, and coupling strength between scalar and vector fields.

\subsection{Axioms of Irreversibility}

\textbf{Axiom 11 (Asymmetric Transition).}  
The transition from a low-entropy coherent regime to a high-entropy incoherent regime occurs at a lower magnitude of constraint violation than the transition required for recovery.

\textbf{Axiom 12 (Structural Memory).}  
Once coherence is lost, the system retains a memory of that loss in the form of dissipated basins. Restoring prior parameter values does not restore prior structure.

\textbf{Axiom 13 (Entropy Dominance).}  
Beyond the critical threshold, entropy production dominates coherence restoration. Optimization accelerates collapse rather than reversing it.

These axioms formalize hysteresis. The system does not respond symmetrically to increases and decreases in enforcement or dispersion. Collapse is easy; recovery is hard.

\subsection{RSVP Dynamics of Collapse}

Within the RSVP framework, phase transition corresponds to dissipation of scalar basins. As $\Phi$ flattens, reputation density $\rho$ disperses and can no longer stabilize local gradients. Vector flow $\vec{v}$ becomes turbulent, dominated by circulation rather than directed descent.

The entropy evolution equation
\[
\frac{\partial S}{\partial t} = \alpha \lvert \nabla \cdot \vec{v} \rvert + \beta \lambda(x,t) + \gamma \lvert \nabla \Phi \rvert^2
\]
reveals the mechanism of collapse. As $\lambda$ increases, entropy injection overwhelms diffusion-driven coherence restoration. Once $\nabla \Phi \rightarrow 0$, the final term vanishes, removing the system’s ability to counteract entropy growth.

At this point, optimization pressure increases activity without increasing structure. Engagement rises while meaning collapses. This regime is stable in the sense of dynamical systems but pathological in the sense of information preservation.

\subsection{Irrecoverability and Design Consequences}

The existence of hysteresis implies that late intervention is categorically more expensive than early prevention. Systems designed without identity conservation cannot be repaired through policy changes, moderation, or improved objectives alone. The underlying scalar field has already dissipated.

This explains why mature platforms often fail to recover trust despite aggressive enforcement efforts. Once attribution has dissolved, no amount of optimization can reconstruct history. New constraints must be imposed at a level stronger than those originally required, often at the cost of excluding large portions of the system.

\subsection{Implications}

Phase transitions mark the boundary between systems that can learn and systems that merely react. Identity coherence is the control parameter that determines which side of that boundary a system occupies.

This section completes the structural argument: constraint hierarchies determine failure order; identity conservation enables accumulation; value learning requires persistent reference; and violation of these conditions produces irreversible collapse. What remains is not a question of optimization, but of whether a system is still capable of meaning at all.

\section{Conclusion}

This work has argued that the central failure mode of contemporary sociotechnical systems is not poor optimization, insufficient data, or misaligned objectives, but the loss of persistent reference. When identity ceases to be conserved, systems do not merely perform worse; they undergo a qualitative transition in which attribution, learning, and governance become structurally impossible. What follows is not error, but entropy.

The analysis proceeded by establishing a strict ordering of constraints. Optimization presupposes feasibility; feasibility presupposes identity persistence; and identity persistence presupposes enforceable namespace structure. Violations at higher levels mask downstream signals, rendering corrective action ineffective and producing the illusion of responsiveness even as meaning collapses. This ordering explains why systems fail in a specific sequence and why recovery, when possible at all, must reverse that sequence.

Within this hierarchy, identity was formalized as a conserved informational quantity. Its role is not semantic but infrastructural: it binds actions to histories and permits accumulation over time. When identity fragments or duplicates, the resulting uncertainty is irreducible. No increase in data volume, modeling sophistication, or optimization pressure can restore what has been structurally dissipated. Reputation, evidence, and trust cease to be conserved, and metrics decouple from the qualities they purport to measure.

The failure of value learning under these conditions is therefore inevitable. Values cannot be learned, enforced, or aligned without persistent referents. Goodhart collapse is not fundamentally a failure of objective design, but a consequence of optimization operating in a space where attribution has already dissolved. In such regimes, intelligence degenerates into high-dimensional random walk, and governance degenerates into ritualized compliance without causal effect.

The Relativistic Scalar--Vector Plenum framework provided a unifying description of these dynamics. Scalar coherence encodes constraint satisfaction and identity persistence; vector flow encodes optimization pressure; entropy measures irreversible loss of attributional structure. Phase transitions occur when identity dispersion exceeds the system’s capacity for enforcement, producing hysteresis and irreversibility. Once scalar basins dissipate, restoring prior parameters does not restore prior meaning.

Extensions to medical research, distributed systems, and category-theoretic formalisms demonstrate that these phenomena are not platform-specific. They are structural invariants of any system that seeks to accumulate knowledge across time. Whether in longitudinal patient records, cross-platform reputation, or human--AI interaction, the same requirement recurs: identity must remain coherent enough for histories to compose.

Crucially, this framework does not specify what ought to be valued. It does not resolve ethical disagreement, define quality, or select objectives. Instead, it establishes the conditions under which valuation itself is possible. It is a theory of conservation, not preference; of feasibility, not virtue. Ethics, policy, and alignment are second-layer problems that cannot be meaningfully addressed once the substrate they depend on has dissolved.

The practical implication is therefore preventative rather than corrective. Systems must be designed to conserve identity before optimization pressure is applied. Enforcement cannot be an afterthought, and alignment cannot be retrofitted. The cost of maintaining coherence is categorically lower than the cost of attempting to reconstruct it after collapse.

What remains uncertain is not the diagnosis, but the response. Whether institutions, platforms, and research systems will recognize the thermodynamic limits described here before crossing irreversible thresholds is an open question. What is no longer open is the underlying principle.

Meaning cannot be optimized into existence. It must be conserved by design.

\begin{thebibliography}{99}

\bibitem{goodhart1975}
Goodhart, C. A. E. (1975).
Problems of Monetary Management: The U.K. Experience.
In \emph{Papers in Monetary Economics}, Reserve Bank of Australia.

\bibitem{strathern1997}
Strathern, M. (1997).
Improving Ratings: Audit in the British University System.
\emph{European Review}, 5(3), 305–321.

\bibitem{campbell1979}
Campbell, D. T. (1979).
Assessing the Impact of Planned Social Change.
\emph{Evaluation and Program Planning}, 2(1), 67–90.

\bibitem{arrow1951}
Arrow, K. J. (1951).
\emph{Social Choice and Individual Values}.
Wiley.

\bibitem{spence1973}
Spence, M. (1973).
Job Market Signaling.
\emph{Quarterly Journal of Economics}, 87(3), 355–374.

\bibitem{shapiro1983}
Shapiro, C. (1983).
Premiums for High Quality Products as Returns to Reputations.
\emph{Quarterly Journal of Economics}, 98(4), 659–680.

\bibitem{lamport1998}
Lamport, L. (1998).
The Part-Time Parliament.
\emph{ACM Transactions on Computer Systems}, 16(2), 133–169.

\bibitem{lynch1996}
Lynch, N. (1996).
\emph{Distributed Algorithms}.
Morgan Kaufmann.

\bibitem{douceur2002}
Douceur, J. R. (2002).
The Sybil Attack.
In \emph{Proceedings of the First International Workshop on Peer-to-Peer Systems}.

\bibitem{oster2016}
Oster, E. (2016).
Unobservable Selection and Coefficient Stability.
\emph{Journal of Business \& Economic Statistics}, 34(2), 187–204.

\bibitem{ioannidis2005}
Ioannidis, J. P. A. (2005).
Why Most Published Research Findings Are False.
\emph{PLoS Medicine}, 2(8), e124.

\bibitem{collins2014}
Collins, F. S., \& Varmus, H. (2015).
A New Initiative on Precision Medicine.
\emph{New England Journal of Medicine}, 372, 793–795.

\bibitem{pearl2009}
Pearl, J. (2009).
\emph{Causality: Models, Reasoning, and Inference}.
Cambridge University Press.

\bibitem{friston2010}
Friston, K. (2010).
The Free-Energy Principle: A Unified Brain Theory?
\emph{Nature Reviews Neuroscience}, 11, 127–138.

\bibitem{anderson2019}
Anderson, M. L. (2019).
Of Bits, Brains, and Behavior: A History of Cognitive Science.
Oxford University Press.

\bibitem{andersonUM1}
Anderson, M. L.
Understanding Machine One (UM1).
Unpublished lecture notes and talks, various dates.

\bibitem{goffman1959}
Goffman, E. (1959).
\emph{The Presentation of Self in Everyday Life}.
Anchor Books.

\bibitem{barwise1997}
Barwise, J., \& Seligman, J. (1997).
\emph{Information Flow: The Logic of Distributed Systems}.
Cambridge University Press.

\bibitem{maclane1998}
Mac Lane, S. (1998).
\emph{Categories for the Working Mathematician}.
Springer.

\bibitem{lurie2009}
Lurie, J. (2009).
\emph{Higher Topos Theory}.
Princeton University Press.

\bibitem{lawvere1973}
Lawvere, F. W. (1973).
Metric Spaces, Generalized Logic, and Closed Categories.
\emph{Rendiconti del Seminario Matematico e Fisico di Milano}, 43, 135–166.

\bibitem{baez2011}
Baez, J. C., \& Stay, M. (2011).
Physics, Topology, Logic and Computation: A Rosetta Stone.
In \emph{New Structures for Physics}.
Springer.

\bibitem{shannon1948}
Shannon, C. E. (1948).
A Mathematical Theory of Communication.
\emph{Bell System Technical Journal}, 27, 379–423, 623–656.

\bibitem{landauer1961}
Landauer, R. (1961).
Irreversibility and Heat Generation in the Computing Process.
\emph{IBM Journal of Research and Development}, 5(3), 183–191.

\bibitem{vermeule2017}
Vermeule, A. (2017).
\emph{Mechanisms of Democracy}.
Oxford University Press.

\bibitem{ostrom1990}
Ostrom, E. (1990).
\emph{Governing the Commons}.
Cambridge University Press.

\bibitem{yudkowsky2004}
Yudkowsky, E. (2004).
Coherent Extrapolated Volition.
Machine Intelligence Research Institute technical report.

\end{thebibliography}

\end{document}
