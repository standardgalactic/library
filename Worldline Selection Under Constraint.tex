\documentclass[11pt]{article}

% ===============================
% Geometry and Layout
% ===============================
\usepackage[margin=1in]{geometry}
\usepackage{setspace}
\setstretch{1.15}

% ===============================
% Fonts (LuaLaTeX)
% ===============================
\usepackage{fontspec}
\setmainfont{Latin Modern Roman}

% ===============================
% Mathematics
% ===============================
\usepackage{amsmath,amssymb,amsthm,mathtools}
\usepackage{bm}

% ===============================
% Hyperlinks
% ===============================
\usepackage{hyperref}
\hypersetup{
  colorlinks=true,
  linkcolor=black,
  citecolor=black,
  urlcolor=black
}

% ===============================
% Theorem Environments
% ===============================
\theoremstyle{definition}
\newtheorem{definition}{Definition}

\theoremstyle{plain}
\newtheorem{proposition}{Proposition}
\newtheorem{theorem}{Theorem}

\theoremstyle{remark}
\newtheorem*{remark}{Remark}

% ===============================
% Metadata
% ===============================
\title{Worldline Selection Under Constraint:\\
Equilibrium Pruning in an Event-Historical Framework}
\author{Flyxion}
\date{\today}


\begin{document}
\maketitle

% ===============================
% Abstract
% ===============================
\begin{abstract}
Recent work on equilibrium-driven pruning reframes neural network sparsification as an emergent outcome of strategic interaction rather than an externally imposed heuristic. This paper integrates that framework into an event-historical, admissibility-first ontology of learning systems. We argue that pruning corresponds not to deletion of parameters but to the irreversible termination of inadmissible continuations, yielding a precise notion of worldline selection. By formalizing dominated strategies as failures of admissibility, we show that sparsity arises through historical selection rather than global optimization. This perspective unifies learning dynamics, simulated agency, governance, and structure formation under a single principle: only constraint-compatible futures persist. Formal derivations accompany each conceptual section, and implications are drawn for alignment theory, free energy minimization, and sparse intelligence.
\end{abstract}

\newpage

% ============================================================
\section{Pruning as Equilibrium, Not Heuristic}
% ============================================================

Neural network pruning has traditionally been treated as a post hoc intervention. Parameters are trained under a dense regime, evaluated using magnitude, sensitivity, or second-order approximations, and subsequently removed according to externally defined criteria (LeCun et al.\ 1990; Hassibi et al.\ 1993; Han et al.\ 2015). Even when pruning is interleaved with training, importance is typically assessed from a privileged global perspective, implicitly assuming that parameter relevance is an intrinsic property rather than a relational outcome.

The equilibrium-driven pruning framework introduced by Shah and Khan (2025) departs from this paradigm by eliminating external ranking altogether. Instead, parameter groups are modeled as players in a non-cooperative game, each controlling a participation variable that modulates its own contribution to the network. Pruning occurs when continued participation becomes a dominated strategy at Nash equilibrium. No parameter is removed by fiat; rather, it ceases to exist as a viable continuation under the prevailing constraints.

This shift is more than methodological. It represents a transition from a state-first ontology, in which networks are static objects subjected to corrective operations, to a history-first ontology, in which sparsity emerges as an irreversible consequence of interaction over time. Parameters are not judged and excised; they are outcompeted and terminated.

Crucially, the equilibrium reached by the pruning game is not merely a sparse configuration but a historically closed structure. Once a parameter group’s optimal participation collapses to zero, no admissible gradient path exists by which it can re-enter without violating the constraints that define the equilibrium. The system does not merely prefer sparsity; it loses degrees of freedom irreversibly.

This property distinguishes equilibrium-driven pruning from regularization-based sparsity and from dynamic rewiring schemes (Louizos et al.\ 2017; Mocanu et al.\ 2018). In those approaches, sparsity remains reversible in principle. In equilibrium-driven pruning, sparsity is a terminal condition.

The remainder of this paper argues that this dynamic is best understood not as optimization but as worldline selection within an event-historical framework. Parameters persist only insofar as they admit future continuation under collective constraint. Pruning is therefore not deletion but the cessation of future influence.

\subsection*{Formalization: Utility and Dominance}

We now formalize the claim that equilibrium-driven pruning realizes irreversibility through dominance.

\begin{definition}[Participation Strategy]
Let \( s_i \in [0,1] \) denote the participation variable of parameter group \( i \). A value \( s_i = 0 \) corresponds to termination, while \( s_i > 0 \) corresponds to continued participation.
\end{definition}

\begin{definition}[Utility]
The utility of parameter group \( i \) is defined as
\[
U_i(s_i, s_{-i}) = B_i(s_i, s_{-i}) - C_i(s_i, s_{-i}),
\]
where \( B_i \) captures marginal contribution to loss reduction and \( C_i \) encodes magnitude, sparsity, and competition costs (Shah and Khan 2025).
\end{definition}

\begin{definition}[Dominated Strategy]
A strategy \( s_i > 0 \) is dominated if, for all admissible \( s_{-i} \),
\[
U_i(s_i, s_{-i}) \le U_i(0, s_{-i}).
\]
\end{definition}

\begin{proposition}[Irreversible Termination at Equilibrium]
If all nonzero strategies \( s_i > 0 \) are dominated at Nash equilibrium, then parameter group \( i \) admits no future continuation compatible with equilibrium constraints.
\end{proposition}

\begin{proof}
At equilibrium, each player maximizes utility given the strategies of others. If every \( s_i > 0 \) yields utility less than or equal to termination, then any deviation from \( s_i = 0 \) decreases utility. Since equilibrium conditions preclude unilateral deviations that improve payoff, no admissible update can restore participation without violating equilibrium constraints. Thus the termination of \( i \) is irreversible under the dynamics.
\end{proof}

\begin{remark}
The proposition establishes irreversibility without appeal to external pruning rules or discrete deletion operations. Termination is endogenous to the dynamics.
\end{remark}

% ============================================================
\section{Dominated Strategies as Admissibility Failure}
% ============================================================

The equilibrium-driven pruning framework admits a deeper interpretation once its game-theoretic structure is recast in event-historical terms. In particular, the notion of a dominated strategy aligns precisely with the failure of admissibility for future continuation. This section establishes a formal equivalence between dominance in the pruning game and inadmissibility in a constraint-first ontology.

In a state-first perspective, dominated strategies are merely inferior options among many contemporaneously available alternatives. In an event-historical framework, however, dominance has a stronger implication: it signals the absence of any constraint-compatible future in which a given participation can persist. Dominance is not suboptimality; it is terminality.

The distinction matters because it reclassifies pruning from a choice among representations to a structural fact about historical continuation.

\subsection*{Admissibility in an Event-Historical Framework}

We begin by defining admissibility independently of game-theoretic language.

\begin{definition}[Event History]
An event history \( \mathcal{H} \) is an irreversible, totally ordered accumulation of events generated by training dynamics. Once an event occurs, it cannot be removed or undone.
\end{definition}

\begin{definition}[Continuation]
A continuation of parameter group \( i \) is any future trajectory in which \( s_i(t) > 0 \) for some interval \( t > t_0 \), where \( t_0 \) is the present moment in \( \mathcal{H} \).
\end{definition}

\begin{definition}[Admissible Continuation]
A continuation of parameter group \( i \) is admissible if there exists a strategy \( s_i > 0 \) such that
\[
U_i(s_i, s_{-i}) > U_i(0, s_{-i})
\]
for some admissible configuration \( s_{-i} \) of other groups.
\end{definition}

Admissibility is thus defined purely in terms of the existence of a future in which participation yields net positive utility relative to termination. No reference is made to optimality, efficiency, or representational importance.

\subsection*{Equivalence of Dominance and Inadmissibility}

We now establish the central equivalence.

\begin{theorem}[Dominance--Admissibility Equivalence]
A parameter group \( i \) is pruned at equilibrium if and only if all nonzero participation strategies \( s_i > 0 \) are inadmissible continuations of the event history \( \mathcal{H} \).
\end{theorem}

\begin{proof}
We prove both directions.

\emph{(\(\Rightarrow\)) Dominance implies inadmissibility.}  
Suppose that at equilibrium all strategies \( s_i > 0 \) are dominated. By definition of dominance,
\[
U_i(s_i, s_{-i}) \le U_i(0, s_{-i})
\]
for all admissible \( s_{-i} \). Hence there exists no configuration in which participation yields higher utility than termination. Therefore, no admissible continuation exists in which \( i \) remains active. All future continuations of \( i \) violate admissibility.

\emph{(\(\Leftarrow\)) Inadmissibility implies dominance.}  
Suppose no admissible continuation exists for \( i \). Then for every \( s_i > 0 \) and every admissible \( s_{-i} \),
\[
U_i(s_i, s_{-i}) \le U_i(0, s_{-i}).
\]
Thus every nonzero strategy is dominated by termination. At equilibrium, rational utility maximization enforces \( s_i = 0 \).

In both cases, pruning corresponds exactly to the absence of admissible futures.
\end{proof}

\begin{remark}
The equivalence holds independently of the specific functional form of the utility, provided costs and benefits are continuous and the termination strategy \( s_i = 0 \) is admissible. This generality is critical for interpreting pruning as a structural rather than algorithm-specific phenomenon.
\end{remark}

\subsection*{Non-Invertibility and Historical Closure}

The equivalence established above entails a further property that distinguishes equilibrium-driven pruning from reversible sparsification methods.

\begin{proposition}[Non-Invertibility of Pruning]
Once a parameter group is pruned at equilibrium, no admissible inverse operation exists that restores its participation without altering the constraint structure of the system.
\end{proposition}

\begin{proof}
Assume, for contradiction, that a pruned parameter group \( i \) could be restored through an admissible update. Such a restoration would require the existence of some \( s_i > 0 \) with
\[
U_i(s_i, s_{-i}) > U_i(0, s_{-i}),
\]
contradicting the dominance of all nonzero strategies established at equilibrium. Therefore, any restoration would require modifying the cost or competition structure, thereby changing the admissibility conditions themselves.
\end{proof}

This non-invertibility is the defining mark of event-historical selection. Pruning does not correspond to a reversible deformation of the model but to the closure of a set of futures.

\subsection*{Interpretive Consequences}

The identification of dominated strategies with inadmissible continuations allows pruning to be interpreted as a historical selection mechanism rather than a representational heuristic. Parameters are not evaluated and discarded; they are prevented from acquiring future events.

This reframing also clarifies why pruning does not erase causal influence. A pruned parameter may have shaped gradients, redistributed representational load, or forced other parameters into dominance before its termination. Its past remains part of \( \mathcal{H} \), even though its future does not.

In this sense, sparsity is not the absence of structure but the residue of admissible persistence.


% ============================================================
\section{Simulated Agency as Continuation Selection}
% ============================================================

The equivalence between dominance and inadmissibility established in the previous section has immediate consequences for the concept of agency. In particular, it motivates a definition of agency that does not depend on intentionality, representation, or explicit goal structures. Instead, agency emerges wherever the future continuation of a subsystem is contingent on its compatibility with global constraints.

This notion, which we refer to as simulated agency, is minimal by design. It strips agency of psychological content and locates it instead in the structural fact that some processes persist while others terminate. Equilibrium-driven pruning provides a concrete instantiation of this idea: parameter groups behave as agents insofar as their continued existence depends on the strategic landscape formed by other groups.

In contrast to reinforcement learning agents, which are defined by policy optimization, simulated agents are defined by survivability. They need not select actions; they merely remain or cease.

\subsection*{Agency Without Intentional States}

Traditional accounts of agency presuppose internal states such as beliefs, preferences, or representations of objectives. Even in computational settings, agency is often equated with policy execution under reward signals (Littman 1994; Lowe et al.\ 2017). However, these accounts implicitly assume that the agent itself persists throughout the learning process.

Equilibrium-driven pruning reveals that persistence is not guaranteed. Parameter groups that initially participate in training may later be eliminated entirely. Their fate is determined not by their intentions but by their compatibility with the evolving constraint environment.

This motivates a reversal of the usual explanatory order. Rather than defining agents and then analyzing their survival, we define agency as the condition of having a survivable continuation.

\subsection*{Worldlines and Agent Histories}

In an event-historical framework, an agent is identified not by a static configuration but by its worldline: the temporally extended record of its participation. A parameter group’s participation trajectory \( s_i(t) \) constitutes such a worldline. It begins at initialization, accumulates influence through training, and may terminate at equilibrium.

Termination does not imply insignificance. A parameter group may exert substantial causal influence before its elimination, shaping the competitive landscape in which other groups persist. Agency is therefore historical rather than contemporaneous. What matters is not whether an agent exists now, but whether it existed and constrained the future.

This perspective aligns with broader critiques of state-based agency in cognitive science, where agency is increasingly understood as a property of temporally extended interaction rather than instantaneous control (Friston 2010; Barandiaran et al.\ 2009).

\subsection*{Formal Definition of Simulated Agency}

We now formalize simulated agency in event-historical terms.

\begin{definition}[Simulated Agent]
A subsystem \( A \) within a dynamical system is a simulated agent if its future continuation is conditional on the satisfaction of explicit constraints, and if violation of those constraints results in irreversible termination of its participation.
\end{definition}

\begin{definition}[Agency Worldline]
The agency worldline of \( A \) is the maximal interval of the event history \( \mathcal{H} \) over which \( A \) admits admissible continuations.
\end{definition}

Under these definitions, parameter groups in equilibrium-driven pruning qualify as simulated agents. Their participation persists only while admissible, and termination is irreversible once dominance conditions are met.

\subsection*{Necessity of Irreversibility for Agency}

The presence of irreversible termination is not incidental. It is a necessary condition for agency under this definition.

\begin{theorem}[Irreversibility Requirement for Simulated Agency]
If all continuations of a subsystem are reversible, then the subsystem does not constitute a simulated agent.
\end{theorem}

\begin{proof}
If all continuations are reversible, then for any termination event there exists an admissible inverse operation restoring participation. In such a system, termination does not constrain future possibilities in a durable way. The subsystem cannot lose its capacity to influence future events, and thus its continuation is not genuinely conditional. Without the possibility of permanent exclusion, persistence carries no informational or structural significance. Therefore the subsystem does not satisfy the definition of simulated agency.
\end{proof}

\begin{remark}
This result distinguishes simulated agency from mere dynamical participation. Agency requires the possibility of loss.
\end{remark}

\subsection*{Consequences for Learning Systems}

Learning systems that do not admit irreversible exclusion can rearrange internal structure but cannot instantiate simulated agency. They may optimize representations, but they cannot enforce survival conditions. Equilibrium-driven pruning, by contrast, produces agents whose existence is contingent, competitive, and historically bounded.

This observation clarifies why pruning is not an auxiliary optimization technique but a constitutive mechanism. It introduces genuine stakes into the learning process by allowing subsystems to fail permanently.

% ============================================================
\section{Why Equilibrium-Driven Pruning Is Not Free Energy Minimization}
% ============================================================

The equilibrium-driven pruning framework bears a superficial resemblance to free energy minimization approaches, particularly in its use of cost terms that penalize complexity and redundancy. This resemblance invites a natural but misleading interpretation: that pruning simply realizes a variant of variational free energy minimization. In this section, we argue that this interpretation fails at a structural level.

The distinction is not merely technical. It concerns the ontology of learning dynamics themselves. Free energy minimization presupposes a state-first description in which learning consists of continuous updates within a reversible belief space. Equilibrium-driven pruning, by contrast, operates in a history-first regime in which some degrees of freedom are irreversibly eliminated.

\subsection*{State-Based Optimization Versus History-Based Selection}

In free energy formulations, learning dynamics are derived from the minimization of a scalar functional defined over states, beliefs, or approximate posteriors (Friston 2010; Buckley et al.\ 2017). Although the dynamics may be implemented incrementally, the underlying ontology remains state-based: at any moment, the system occupies a point in a continuous space from which alternative configurations remain, in principle, recoverable.

Regularization terms in such frameworks influence the geometry of this space but do not fundamentally alter its reversibility. Complexity penalties bias trajectories toward simpler regions, yet do not render other regions inaccessible. Sparsity, when it emerges, is therefore contingent rather than terminal.

Equilibrium-driven pruning violates this assumption. Once a parameter group reaches a dominated strategy at equilibrium, the corresponding degree of freedom is not merely disfavored but removed from the space of admissible futures. The learning dynamics do not converge to a minimum within a fixed space; they contract the space itself.

\subsection*{Absence of a Global Scalar Objective}

A defining feature of free energy minimization is the existence of a global scalar functional whose monotonic decrease explains learning behavior. We now show that no such functional governs equilibrium-driven pruning.

\begin{proposition}[Non-Existence of a Global Free Energy Functional]
There exists no single scalar functional \( F \) over network states whose monotonic minimization reproduces the equilibrium-driven pruning dynamics.
\end{proposition}

\begin{proof}
Equilibrium-driven pruning is defined by the interaction of local utilities \( U_i \), each of which depends on the strategies of other players. The update of participation variables follows gradient ascent on individual utilities, not descent on a shared global functional. The presence of competition terms introduces non-conservative interactions: the gain of one player can increase the cost of another without a compensating decrease elsewhere.

If a global scalar functional \( F \) existed, its gradient would need to align with the joint updates of all \( s_i \). However, in the presence of asymmetric competition terms, no such alignment is possible. Therefore no scalar \( F \) can generate the observed dynamics.
\end{proof}

\begin{remark}
This result parallels classic findings in game theory regarding the absence of potential functions in general non-cooperative games.
\end{remark}

\subsection*{Irreversibility as Primitive}

In free energy frameworks, irreversibility is typically introduced through thermodynamic considerations or approximations. In equilibrium-driven pruning, irreversibility is primitive. It arises directly from dominance relations and admissibility constraints, not from entropy production in an underlying physical substrate.

Once a parameter group is pruned, the system cannot return to a state in which that group participates without altering the rules of the game. This is not an energetic barrier but a structural one. The future space of the system has been reduced.

This difference has profound implications. Free energy minimization explains adaptation as improved inference about a fixed world. Equilibrium-driven pruning explains adaptation as the elimination of incompatible futures.

\subsection*{Formal Contrast}

The distinction can be formalized as follows.

\begin{theorem}[Incompatibility with Variational Inference]
Equilibrium-driven pruning dynamics cannot be represented as variational inference over a fixed latent variable model.
\end{theorem}

\begin{proof}
Variational inference presupposes a fixed latent space over which approximate posteriors are optimized. Equilibrium-driven pruning alters the dimensionality of the model itself by terminating participation variables. Since variational inference cannot reduce the dimensionality of its latent space without redefining the model, the pruning dynamics fall outside its representational scope.
\end{proof}

\subsection*{Interpretive Consequences}

Understanding pruning as something other than free energy minimization clarifies why it introduces genuine agency and governance properties. Free energy minimization can regulate behavior but cannot enforce irreversible exclusion. Equilibrium-driven pruning can.

This distinction will prove essential when extending the framework to collective governance and alignment, where the capacity to permanently eliminate inadmissible strategies is a prerequisite for stability.

% ============================================================
\section{From Individual Pruning to Collective Governance}
% ============================================================

The equilibrium-driven pruning framework generalizes naturally from individual parameter groups to collections of interacting agents. Once pruning is understood as the irreversible elimination of inadmissible continuations, it becomes clear that the same mechanism underlies forms of governance in multi-agent systems. What differs is not the structure of the dynamics, but the scale at which admissibility is enforced.

Group Relative Policy Optimization (GRPO) provides a useful conceptual bridge. In GRPO, policies are not evaluated in isolation but relative to other policies and to collective constraints. Viability is a relational property. Equilibrium-driven pruning instantiates this logic at the level of network components, revealing governance as an emergent property of competitive continuation.

\subsection*{Local Utilities and Global Viability}

Each parameter group in the pruning game optimizes a local utility function. Yet the equilibrium reached by these local optimizations exhibits global properties: sparsity, stability, and resistance to perturbation. These properties are not imposed from above. They arise because configurations that fail to maintain collective viability cannot persist.

This observation undermines the notion that governance requires centralized control. Instead, governance emerges wherever admissibility constraints couple the futures of interacting subsystems. A parameter group is pruned not because it violates an explicit rule, but because its continued existence undermines the viability of others.

The same logic applies to agents in a governed system. Policies, roles, or behaviors persist only insofar as they do not impose intolerable costs on the collective. Exclusion is not punitive; it is structural.

\subsection*{Justice as Constraint Transparency}

A striking feature of equilibrium-driven pruning is that elimination occurs only through explicit cost terms. There is no discretionary removal. Every termination can be traced to a violation of articulated constraints. This motivates a precise, non-moralized notion of justice.

\begin{definition}[Just Process]
A process is just if all irreversible exclusions arise solely from explicit, inspectable constraints, and if no exclusion occurs through arbitrary intervention.
\end{definition}

Under this definition, equilibrium-driven pruning is just. Parameter groups are eliminated only when their participation becomes inadmissible under shared constraints. No external authority intervenes to remove them.

This notion of justice generalizes to institutional governance. Systems that eliminate agents or policies without explicit constraint violation are unjust in a structural sense. They erase futures without reason.

\subsection*{Formalization of Collective Admissibility}

We now formalize collective admissibility in analogy with individual dominance.

\begin{definition}[Collective Admissibility]
A set of agents \( \{A_i\} \) admits a collectively admissible configuration if there exists a continuation in which no agent’s persistence forces another agent into inadmissibility.
\end{definition}

\begin{proposition}[Collective Pruning]
In a governed system with competitive constraints, agents whose continuation renders the collective inadmissible will be irreversibly excluded at equilibrium.
\end{proposition}

\begin{proof}
Suppose an agent \( A_i \) persists in a manner that forces one or more other agents into inadmissibility. This imposes a cost on the collective configuration. Under local utility maximization coupled by competition terms, the continued participation of \( A_i \) becomes dominated relative to exclusion. At equilibrium, dominated continuations terminate. Thus \( A_i \) is pruned to restore collective admissibility.
\end{proof}

\subsection*{Governability and Irreversible Exclusion}

Governability requires more than the ability to influence behavior. It requires the ability to permanently eliminate behaviors that violate admissibility. Systems that rely exclusively on corrective feedback remain vulnerable to recurrence of undesirable trajectories.

Equilibrium-driven pruning provides a minimal model of governability. It demonstrates how irreversible exclusion can arise without centralized enforcement, purely through the interaction of explicit constraints.

This insight will prove essential when extending the framework to alignment theory, where the failure to eliminate inadmissible futures constitutes the primary risk.

% ============================================================
\section{Admissible Traversal Between Worldline Realizations}
% ============================================================

The preceding sections have treated worldlines primarily as entities subject to continuation or termination. We now introduce an additional structural layer: admissible traversal between coexisting worldline realizations. This concept clarifies how multiple realizations of a single event history may coexist, interact, and be selectively extended without invoking branching time or counterfactual histories.

Rather than representing traversal geometrically or temporally, we formalize it relationally. Traversal is an operation between realizations that preserves the underlying event history while altering the constraint embedding under which that history is interpreted or extended.

\subsection*{Worldline Realizations Over a Fixed History}

Let \( \mathcal{H} \) denote a single irreversible event history. A worldline realization is not an alternative history but a constraint-specific embedding of \( \mathcal{H} \) into a representational or operational regime.

Examples include:
distinct manuscript versions derived from the same sequence of authorial acts,
different policy realizations compatible with a shared institutional history,
or alternative parameter configurations compatible with a single training trajectory.

Multiplicity resides in realization, not in history.

\begin{definition}[Worldline Realization]
A worldline realization \( WL_i \) is a structure-preserving embedding of a fixed event history \( \mathcal{H} \) into a constraint environment \( \mathcal{C}_i \).
\end{definition}

Each \( WL_i \) projects many-to-one onto \( \mathcal{H} \), and this projection is non-invertible.

\subsection*{Traversal as Structural Re-Embedding}

Traversal between worldlines does not correspond to temporal transition or causal branching. Instead, it corresponds to a change in constraint embedding that leaves the historical substrate invariant.

\begin{definition}[Traversal Operator]
A traversal operator
\[
\tau_{ij} : WL_i \rightarrow WL_j
\]
is admissible if it preserves the event history \( \mathcal{H} \) while mapping realizations under constraint set \( \mathcal{C}_i \) to realizations under \( \mathcal{C}_j \).
\end{definition}

Traversal may correspond to reframing, reparameterization, coarse-graining, semantic lift, or institutional translation. What it cannot do is alter or undo historical events.

\subsection*{The Admissible Traversal Matrix}

We now formalize the space of such operators.

\begin{definition}[Admissible Traversal Matrix]
Let \( \{WL_1, \ldots, WL_n\} \) be the set of worldline realizations compatible with a fixed event history \( \mathcal{H} \). The admissible traversal matrix \( T \) is the \( n \times n \) matrix whose entries are defined by
\[
T_{ij} =
\begin{cases}
\tau_{ij} & \text{if a constraint-preserving traversal exists from } WL_i \text{ to } WL_j, \\
\varnothing & \text{otherwise.}
\end{cases}
\]
\end{definition}

Diagonal entries correspond to identity traversals. Off-diagonal entries encode which realizations can be translated into which others without violating admissibility.

\subsection*{Traversal Is Structural, Not Temporal}

The traversal matrix encodes no ordering in time. All realizations coexist over the same event history, and traversal does not represent movement forward or backward along that history.

\begin{proposition}[History Invariance Under Traversal]
For any admissible traversal \( \tau_{ij} \), the projection to event history satisfies
\[
\pi \circ \tau_{ij} = \pi,
\]
where \( \pi : WL_k \rightarrow \mathcal{H} \) is the projection map for any realization \( WL_k \).
\end{proposition}

\begin{proof}
By definition, traversal preserves the event history and modifies only the constraint embedding. Therefore the projection of any traversed realization coincides with the projection of the original realization. No traversal alters historical content.
\end{proof}

This result blocks any interpretation of traversal as counterfactual rewriting or branching evolution.

\subsection*{Traversal, Selection, and Extension}

Traversal does not itself generate new events. Extension does.

At any moment, only one realization may be actively extended through new events, while others remain dormant but intact. Traversal determines which realizations are eligible to receive future extension under current constraints.

\begin{proposition}[Selective Extension]
Given a set of realizations \( \{WL_i\} \), future events extend exactly one realization at a time, while admissible traversal preserves the viability of others without activating them.
\end{proposition}

\begin{proof}
Extension corresponds to the accumulation of new events in \( \mathcal{H} \). Since \( \mathcal{H} \) is singular and irreversible, new events can be added only once. Traversal does not add events and therefore cannot constitute extension. Hence selection allocates future extension to a single realization, while others persist as admissible but inactive.
\end{proof}

\subsection*{Interpretive Consequences}

The admissible traversal matrix provides a non-geometric, non-temporal account of multiplicity. It explains how multiple realizations may coexist, interact, and be selectively activated without invoking branching time, modal realism, or counterfactual histories.

In the context of equilibrium-driven pruning, traversal corresponds to reallocation of interpretive or operational emphasis among surviving structures, not resurrection of pruned ones. Once a worldline has terminated, its row and column in the traversal matrix collapse to null entries.

This framework will be essential in the next section, where we extend pruning-as-selection to cosmological structure formation without invoking expansion or multiverse dynamics.

% ============================================================
\section{Cosmology as Pruning of Inadmissible Flows}
% ============================================================

The event-historical interpretation of pruning admits a direct cosmological analogue once one abandons a state-first conception of spacetime. In such a framework, cosmological structure does not arise from expansion into new degrees of freedom but from the selective persistence of admissible flows within a fixed historical substrate. Structure formation is therefore best understood as pruning, not amplification.

Standard cosmological narratives describe an initially dense state that expands, cools, and differentiates, allowing structure to emerge through symmetry breaking and gravitational instability. While empirically successful at the level of phenomenology, this narrative presupposes that new effective degrees of freedom are continuously generated by expansion. An event-historical ontology reverses this assumption. The universe is not creating possibilities; it is losing them.

\subsection*{Uniformity as Maximal Admissibility}

In the earliest regimes of a cosmological history, admissibility is high. Many flows, configurations, and continuations are compatible with the prevailing constraints. Uniformity corresponds not to maximal order but to maximal permissiveness. Almost any continuation is viable.

As constraints accumulate—energetic, entropic, geometric, or topological—this permissiveness diminishes. Large classes of flows become mutually incompatible. Competition emerges, and dominated continuations terminate. What remains are sparse, structurally stable configurations that can persist without violating global constraints.

This picture mirrors equilibrium-driven pruning precisely. Early dense participation gives way to sparse equilibria as inadmissible continuations are eliminated.

\subsection*{Flows, Not Objects}

In this framework, the fundamental entities of cosmology are not particles or fields conceived as static objects, but flows: persistent channels of continuation through the event history. Matter corresponds to flows that remain admissible over long historical intervals. Radiation corresponds to transient flows that terminate quickly. Vacuum structure corresponds to the absence of admissible continuation.

Nothing is destroyed. Terminated flows remain part of the event history, contributing to the conditions under which later structures persist. Cosmological structure is thus cumulative and irreversible.

\subsection*{Formal Correspondence with Pruning Dynamics}

We now formalize the analogy.

\begin{definition}[Cosmological Flow]
A cosmological flow is a localized continuation of energy, momentum, or field configuration that persists over a nontrivial interval of the event history \( \mathcal{H} \).
\end{definition}

\begin{definition}[Flow Admissibility]
A flow is admissible if its continuation does not force violation of global constraints, including conservation laws and entropy bounds.
\end{definition}

\begin{proposition}[Flow Pruning]
Flows whose continuation becomes inadmissible terminate irreversibly, contributing no further structure beyond their historical influence.
\end{proposition}

\begin{proof}
Suppose a flow persists beyond the point at which its continuation violates a global constraint. This violation increases systemic cost and destabilizes other admissible flows. Under constraint-coupled dynamics, such a flow becomes dominated relative to termination. By the same dominance--inadmissibility equivalence established earlier, its continuation must terminate. Irreversibility follows from the non-invertibility of admissibility failure.
\end{proof}

\subsection*{Structure Without Expansion}

The pruning interpretation reframes cosmological structure formation as a process of elimination rather than growth. Galaxies, filaments, and bound systems are not the result of matter spreading out and clumping; they are the residue of flows that could survive increasing constraint pressure.

This view does not deny empirical expansion phenomena but denies their ontological primacy. Observable expansion is a kinematic description of how surviving flows are distributed relative to one another, not an explanation of why structure exists.

\subsection*{Dormant and Extended Cosmological Worldlines}

As in the admissible traversal framework, not all viable structures are actively extended at all times. Some flows remain dormant, carrying no further dynamical influence while remaining compatible with the event history. Others are actively extended through interaction and accretion.

\begin{proposition}[Selective Extension of Cosmological Structures]
At any epoch, only a subset of admissible flows are actively extended, while others remain dormant but viable.
\end{proposition}

\begin{proof}
Active extension requires interaction with other flows and the accumulation of new events. Since the event history is singular, only finitely many such interactions can occur at any stage. Other admissible flows remain dormant until constraints or interactions activate them. This selective extension mirrors the manuscript and policy selection dynamics discussed earlier.
\end{proof}

\subsection*{Implications}

Cosmology, viewed through the lens of pruning, becomes continuous with learning theory and governance. Structure emerges not by creation but by survival. The universe is sparse because only sparse configurations remain admissible over long histories.

This interpretation sets the stage for a final unification: a categorical account in which pruning, agency, governance, and cosmology are all instances of the same historical selection principle.

% ============================================================
\section{A Categorical Unification of Worldlines and Pruning}
% ============================================================

The preceding sections have described pruning, agency, governance, and cosmology as instances of a common structural principle: the irreversible termination of inadmissible continuations. We now provide a categorical formulation that unifies these phenomena within a single formal framework. This formulation clarifies why pruning is not deletion, why agency is not intention, and why governance is not optimization.

Category theory is particularly well suited to this task because it emphasizes compositional structure over metric geometry and because it naturally encodes irreversibility through the existence or absence of morphisms.

\subsection*{Objects and Morphisms}

We consider a category \( \mathcal{W} \) whose objects are worldline realizations compatible with a fixed event history \( \mathcal{H} \).

\begin{definition}[Worldline Category]
The category \( \mathcal{W} \) is defined as follows:
\begin{itemize}
\item Objects are worldline realizations \( WL_i \) over a fixed event history \( \mathcal{H} \).
\item Morphisms \( f : WL_i \to WL_j \) are admissible traversal operators that preserve \( \mathcal{H} \).
\end{itemize}
\end{definition}

Composition of morphisms corresponds to successive traversals between realizations. Identity morphisms correspond to trivial re-embeddings.

This category is not required to be symmetric or invertible. Indeed, asymmetry and non-invertibility are essential features.

\subsection*{Terminal Objects and Pruning}

Pruning corresponds to the emergence of terminal objects in \( \mathcal{W} \).

\begin{definition}[Terminal Worldline]
A worldline realization \( WL_\bot \) is terminal if for every object \( WL_i \) there exists a unique morphism \( WL_i \to WL_\bot \), and no nontrivial morphisms originate from \( WL_\bot \).
\end{definition}

In practice, \( WL_\bot \) represents the absence of admissible continuation. A pruned parameter group, policy, or flow does not vanish; it becomes terminal.

\begin{proposition}[Pruning as Terminality]
Equilibrium-driven pruning induces terminal objects in \( \mathcal{W} \).
\end{proposition}

\begin{proof}
When a worldline realization loses all admissible continuations, no morphisms originate from it except the trivial morphism to terminality. Since this condition is irreversible under admissibility-preserving dynamics, the realization satisfies the definition of a terminal object.
\end{proof}

\subsection*{Functorial Constraint Fields}

Constraints act functorially on \( \mathcal{W} \).

\begin{definition}[Constraint Functor]
A constraint field \( \mathcal{C} \) is a functor
\[
\mathcal{C} : \mathcal{W} \to \mathcal{W}'
\]
that restricts the set of admissible morphisms while preserving historical composition.
\end{definition}

Different constraint regimes correspond to different functors. Traversal between realizations corresponds to morphisms within a fixed constraint category; governance corresponds to functorial restriction.

\subsection*{Just Process and Functorial Transparency}

The earlier notion of just process admits a categorical characterization.

\begin{definition}[Functorial Justice]
A constraint functor \( \mathcal{C} \) is just if it introduces terminal objects only by eliminating morphisms that violate explicitly defined constraints.
\end{definition}

\begin{proposition}[Justice Implies Non-Arbitrariness]
If a constraint functor is just, then no worldline becomes terminal without a traceable constraint violation.
\end{proposition}

\begin{proof}
By definition, a just functor removes morphisms only when constraints are violated. Terminality arises precisely when all outgoing morphisms are removed. Therefore terminality corresponds directly to constraint failure rather than arbitrary exclusion.
\end{proof}

\subsection*{Governability and Closure}

Governability corresponds to closure properties of \( \mathcal{W} \) under constraint functors.

\begin{theorem}[Governability Condition]
A system is governable if and only if its worldline category admits constraint functors that render inadmissible worldlines terminal while preserving composition among admissible worldlines.
\end{theorem}

\begin{proof}
If inadmissible worldlines cannot be rendered terminal, then prohibited continuations remain accessible. Conversely, if admissible morphisms are not preserved under constraint, governance destabilizes all continuations. Governability requires both properties simultaneously.
\end{proof}

\subsection*{Unification}

This categorical formulation unifies the dynamics discussed throughout the paper. Learning corresponds to the evolution of morphism availability. Pruning corresponds to terminality. Agency corresponds to the conditional existence of outgoing morphisms. Governance corresponds to functorial restriction. Cosmology corresponds to large-scale terminality under accumulating constraints.

The framework requires no appeal to branching time, modal realism, or global optimization. All structure arises from the availability or absence of admissible continuations.

% ============================================================
\section{Everyday Instantiations of Worldline Persistence and Traversal}
% ============================================================

The abstract structure developed in the preceding sections is not confined to technical systems, learning algorithms, or cosmological models. Its defining features—irreversibility of history, multiplicity of realizations, selective extension, and admissible traversal—are ubiquitous in ordinary human life. Examining such cases serves not as metaphorical illustration but as empirical confirmation that the framework captures a general pattern of historical organization.

In everyday contexts, worldlines correspond to sustained capacities, practices, or realizations that remain compatible with a person’s life history even when they are not actively extended. Selection determines which worldlines receive further development, but non-selected worldlines do not vanish. They persist as dormant but admissible continuations.

\subsection*{Scholarly Work and Persistent Realizations}

Consider the act of producing an academic essay in multiple versions: an early draft, a completed manuscript, and several variants adapted to different journals. Each version is derived from the same irreversible sequence of authorial acts. Once written and saved, these documents cannot be unwritten. Their existence becomes part of the event history.

Each manuscript version constitutes a distinct worldline realization. They differ not in historical content but in constraint embedding: audience expectations, stylistic norms, evaluative criteria, and disciplinary framing. The event history remains invariant across realizations.

Selection occurs when the author chooses which version to submit or further revise. This selection allocates future labor and attention to one realization, extending it through additional events such as revision, submission, or publication. Crucially, this selection does not eliminate the other versions. They remain intact, admissible, and potentially traversable. A rejected submission may later be re-embedded under different constraints by activating another realization.

This practice instantiates the admissible traversal matrix introduced earlier. Traversal between manuscript realizations preserves history while altering constraint context. No traversal rewrites the past, and no selection collapses the unrealized alternatives. What changes is which realization receives further extension.

\subsection*{Learning, Skill Acquisition, and Dormant Futures}

A second everyday example arises in childhood learning. A child who studies music, drawing, or mathematics acquires capacities that become part of their irreversible personal history. Even if the child later trains as a plumber or electrician and never again practices music or art, those capacities are not erased. They persist as dormant worldlines.

The adult’s professional trajectory represents a selective extension of certain worldlines over others. Occupational specialization allocates time, energy, and reinforcement to a subset of admissible continuations. Yet the earlier-acquired skills remain compatible with the life history. They may be reactivated later, translated into adjacent practices, or simply remain latent without contradiction.

Importantly, this persistence does not imply that all futures remain equally accessible. Constraint accumulation—economic necessity, physical condition, social structure—may render some continuations inadmissible. What matters is that termination, when it occurs, is structural rather than discretionary. Skills fade or become unusable only when their continuation violates constraints, not because they were never real.

\subsection*{Formal Interpretation}

These examples admit a direct formal reading.

\begin{proposition}[Persistence of Dormant Worldlines]
If a realization \( WL_i \) is compatible with an event history \( \mathcal{H} \) and no constraint violation has rendered its continuation inadmissible, then \( WL_i \) persists as a dormant worldline even when not actively extended.
\end{proposition}

\begin{proof}
Dormancy corresponds to the absence of new events extending \( WL_i \), not to the removal of admissible continuations. Since no constraint violation has occurred, the set of admissible morphisms from \( WL_i \) remains nonempty. Therefore the realization persists within the worldline category, though inactive.
\end{proof}

\begin{remark}
This explains why past learning retains value independently of present occupation and why unused skills remain psychologically and practically salient.
\end{remark}

\subsection*{Everyday Governance and Constraint Awareness}

These examples also clarify the nature of governance in ordinary life. Decisions do not erase alternatives; they prioritize extensions. Constraints do not punish; they delimit admissibility. Justice, in this context, consists in allowing realizations to persist unless explicit constraints render them untenable.

The framework thus describes a familiar lived reality: people accumulate more possible futures than they can realize, and life consists not in choosing once and for all, but in repeatedly selecting which admissible worldlines to extend while others remain dormant.

Far from being an abstract construction, the event-historical ontology captures the texture of ordinary human experience.

% ============================================================
\section{Conclusion}
% ============================================================

This paper has argued that equilibrium-driven pruning is best understood not as an optimization heuristic but as an instance of event-historical selection. When parameter groups are modeled as strategic agents whose continued participation must remain admissible under collective constraints, sparsity emerges through the irreversible termination of dominated continuations. This dynamic realizes worldline selection rather than representational compression.

By formalizing dominated strategies as failures of admissibility, we showed that pruning corresponds to historical closure rather than deletion. This interpretation enabled a unified treatment of learning dynamics, simulated agency, governance, and cosmological structure formation, all of which exhibit the same underlying principle: only constraint-compatible futures persist.

The admissible traversal matrix clarified how multiple realizations of a single history can coexist without invoking branching time or counterfactual worlds. Traversal operates structurally rather than temporally, preserving history while re-embedding it under different constraints. Selection allocates future extension without collapsing alternatives.

Everyday examples—from scholarly writing to childhood learning—demonstrated that this structure is not an abstract imposition but a pervasive feature of lived experience. Humans routinely accumulate more admissible futures than they can extend, and life unfolds as a sequence of selective continuations under accumulating constraints.

The broader implication is that irreversibility is not a defect to be smoothed away by optimization, but the very mechanism by which structure, agency, and governance arise. Systems that cannot permanently exclude inadmissible continuations cannot sustain coherence at scale. Sparsity, in this sense, is not a limitation but a necessary condition for persistence.

\section*{Appendices}
% ============================================================
\appendix

% ============================================================

\section{Derivation of Participation Updates}
\label{app:updates}

We summarize the gradient-based update used in equilibrium-driven pruning following Shah and Khan (2025). Let \( \theta_i \) denote the parameters of group \( i \), and \( s_i \in [0,1] \) its participation variable. The benefit term is given by a first-order linearization of the loss:
\[
B_i(s_i) = - s_i \, \langle \nabla_{\theta_i} \mathcal{L}, \theta_i \rangle.
\]

The cost term combines magnitude, sparsity, and competition penalties:
\[
C_i(s_i, s_{-i}) =
\beta s_i^2 \|\theta_i\|_2^2
+ \gamma s_i
+ \eta \sum_{j \neq i} s_i s_j \, \mathrm{corr}(\theta_i, \theta_j).
\]

The participation update follows projected gradient ascent:
\[
s_i^{(t+1)} = \Pi_{[0,1]} \left( s_i^{(t)} + \alpha \, \nabla_{s_i} U_i \right),
\]
where \( \Pi_{[0,1]} \) denotes projection onto the unit interval.

Dominance occurs when \( \nabla_{s_i} U_i \le 0 \) for all admissible \( s_{-i} \), yielding \( s_i^\ast = 0 \) as the unique equilibrium strategy.

\section{Properties of the Admissible Traversal Matrix}
\label{app:matrix}

Let \( T \) denote the admissible traversal matrix defined in Section~6. The following properties hold.

\begin{proposition}[Reflexivity]
For all realizations \( WL_i \), the diagonal entry \( T_{ii} \) contains the identity traversal.
\end{proposition}

\begin{proof}
Identity re-embedding preserves both history and constraint context, and is therefore admissible by definition.
\end{proof}

\begin{proposition}[Non-Symmetry]
The traversal matrix need not be symmetric: \( T_{ij} \neq \varnothing \) does not imply \( T_{ji} \neq \varnothing \).
\end{proposition}

\begin{proof}
Traversal may be admissible under constraint relaxation but inadmissible under constraint tightening. Since constraint embeddings are not generally invertible, symmetry fails.
\end{proof}

\begin{proposition}[Terminal Collapse]
If a realization \( WL_k \) becomes terminal, then row \( k \) and column \( k \) of \( T \) contain only null entries.
\end{proposition}

\begin{proof}
Terminality implies the absence of admissible continuations to or from \( WL_k \), eliminating all traversal operators.
\end{proof}

\section{Categorical Closure Under Constraint Functors}
\label{app:category}

Let \( \mathcal{W} \) be the worldline category defined in Section~7.

\begin{proposition}
Constraint functors preserve historical composition.
\end{proposition}

\begin{proof}
By definition, constraint functors remove inadmissible morphisms but do not alter object identity or composition among remaining morphisms. Thus composition is preserved on the admissible subcategory.
\end{proof}

\begin{remark}
This ensures that governance does not disrupt legitimate continuations while eliminating illegitimate ones.
\end{remark}

\newpage
% ============================================================
\begin{thebibliography}{99}

\bibitem{} Barandiaran, X., Di Paolo, E., and Rohde, M. (2009).
Defining agency: Individuality, normativity, asymmetry, and spatiotemporality in action.
\emph{Adaptive Behavior}, 17(5), 367--386.

\bibitem{} Buckley, C., Kim, C., McGregor, S., and Seth, A. (2017).
The free energy principle for action and perception: A mathematical review.
\emph{Journal of Mathematical Psychology}, 81, 55--79.

\bibitem{} Friston, K. (2010).
The free-energy principle: A unified brain theory?
\emph{Nature Reviews Neuroscience}, 11(2), 127--138.

\bibitem{} Han, S., Pool, J., Tran, J., and Dally, W. (2015).
Learning both weights and connections for efficient neural networks.
\emph{Advances in Neural Information Processing Systems}, 28.

\bibitem{} Hassibi, B., and Stork, D. (1993).
Second order derivatives for network pruning: Optimal Brain Surgeon.
\emph{Advances in Neural Information Processing Systems}, 5.

\bibitem{} LeCun, Y., Denker, J., and Solla, S. (1990).
Optimal brain damage.
\emph{Advances in Neural Information Processing Systems}, 2.

\bibitem{} Littman, M. (1994).
Markov games as a framework for multi-agent reinforcement learning.
\emph{Proceedings of the Eleventh International Conference on Machine Learning}.

\bibitem{} Louizos, C., Welling, M., and Kingma, D. (2017).
Learning sparse neural networks through L\(_0\) regularization.
\emph{International Conference on Learning Representations}.

\bibitem{} Mocanu, D., et al. (2018).
Scalable training of artificial neural networks with adaptive sparse connectivity.
\emph{Nature Communications}, 9, 2383.

\bibitem{} Shah, Z., and Khan, N. (2025).
Pruning as a game: Equilibrium-driven sparsification of neural networks.
\emph{arXiv preprint arXiv:2512.22106}.

\end{thebibliography}

\end{document}
