\documentclass[11pt]{article}
\usepackage[margin=1in]{geometry}
\usepackage{setspace}
\usepackage{csquotes}

\setstretch{1.15}

\title{Merit Without Mercy: Functional Reduction, Organizational Blindness, and the End of Human Justification}
\author{Flyxion}
\date{\today}

\begin{document}
\maketitle

\begin{abstract}
This essay examines the failure of meritocracy under conditions of large-scale optimization, automation, and attention-driven economic organization. It argues that meritocracy has shifted from a principle of justice to a system of justification, one that renders human exclusion morally intelligible once participation becomes economically unnecessary. As institutions scale, individuals are reduced to financial abstractions—cost centers, productivity variables, and engagement units—producing organizational blindness, engineered turnover, and the systematic erosion of expertise and responsibility.

The analysis shows how artificial intelligence completes rather than initiates this process by automating coordination, evaluation, and communication, thereby shrinking the category of economically necessary human labor. Selection pressures decouple from social value as platform ownership, attention extraction, and advertising-based revenue models reward control over bottlenecks rather than contribution to collective well-being. Essential forms of work—care, maintenance, teaching, repair, and stewardship—become structurally invisible despite their foundational role in sustaining society.

The essay further demonstrates how meritocratic narratives moralize luck in domains such as property ownership, inheritance, and financialization, transforming contingent advantage into deserved status while legitimizing exclusion. Welfare systems, reskilling rhetoric, and appeals to fairness function not as remedies but as disciplinary mechanisms that individualize structural redundancy.

Taken together, these dynamics point to a legitimacy crisis rather than a cyclical economic disruption. When belonging is contingent on continuous proof of economic worth, and when such proof is no longer broadly possible, justification collapses. The essay concludes that a social order organized around functional reduction cannot indefinitely sustain human membership, and that merit without mercy culminates not in fairness, but in exclusion that can no longer explain itself.
\end{abstract}

\newpage
\section*{Introduction}

Meritocracy is frequently defended as a moral advance over inherited hierarchy. It promises that social position will follow effort, competence, and achievement rather than birth or patronage. Yet when meritocracy is implemented within large-scale economic systems, its meaning shifts. Merit becomes output, output becomes revenue, and human beings are evaluated as financial abstractions. Those who cannot be cleanly integrated into this logic are not merely disadvantaged; they become conceptually marginal.

This essay argues that meritocracy, once fused with optimization and automation, reduces people to money-making functions. Artificial intelligence does not initiate this reduction but completes it. The resulting system cannot coherently value children, the elderly, the disabled, or increasingly, workers themselves.

\section*{From Persons to Financial Abstractions}

In contemporary institutions, individuals are increasingly represented not as persons but as entries in spreadsheets: labor costs, productivity curves, risk profiles, and efficiency scores. Decision-making occurs at a level of abstraction where human particularity is invisible. A worker is not someone who knows a process or maintains continuity, but a variable whose contribution can be modeled and replaced.

This abstraction is not accidental. It is necessary for large-scale control. The more complex an organization becomes, the more it must simplify the humans within it in order to remain legible to itself. In doing so, it transforms social questions into accounting problems. Care becomes overhead. Stability becomes inefficiency. Loyalty becomes friction.

\section*{Turnover as an Engineered Outcome}

High turnover is often treated as a regrettable side effect of modern employment. In reality, it is frequently an explicit design goal. Jobs are decomposed into narrow, easily trainable tasks precisely so that workers can be replaced without disruption. Experience becomes a liability because it raises expectations and costs.

This structure prevents the accumulation of expertise by design. Workers are denied the opportunity to develop mastery, while organizations deprive themselves of institutional memory. What is preserved instead is procedural compliance. The system does not want understanding; it wants predictability.

In such environments, no one is meant to stay long enough to see the consequences of decisions. Responsibility dissolves across time, and accountability evaporates.

\section*{Organizational Blindness and the Loss of the Whole}

As corporations scale, they fracture understanding. No single individual is encouraged or permitted to see how the system functions in its entirety. Decisions are made by those far removed from implementation, while those closest to implementation lack authority to intervene.

This produces a distinctive form of organizational blindness. Failures are attributed to individual performance rather than structural design. Metrics substitute for knowledge, and dashboards replace judgment. When something breaks, the system searches for a replaceable part rather than questioning the architecture.

Meritocracy intensifies this blindness by framing systemic failures as personal inadequacies. If outcomes are poor, someone must lack merit. The possibility that the evaluative framework itself is flawed is excluded in advance.

\section*{Middle Management as a Sacrificial Layer}

Middle management historically served as a buffer between abstraction and reality. While imperfect, it provided translation, memory, and local interpretation. It was a site where human judgment could still intervene in mechanized systems.

Artificial intelligence now targets this layer precisely because it is interpretive rather than directly productive. Scheduling, supervision, evaluation, and coordination are reframed as optimization problems rather than social ones. Once automated, these functions no longer serve workers or institutions; they serve metrics.

The removal of middle management does not empower workers. It exposes them directly to algorithmic authority while concentrating real power further upward. The system becomes simultaneously more rigid and more opaque.

\section*{The Myth of Reskilling}

In response to technological displacement, societies often invoke retraining and reskilling as solutions. This rhetoric assumes that unemployment results from mismatched skills rather than structural redundancy. It frames exclusion as a personal failure to adapt.

Yet reskilling cannot solve a system that is actively reducing the number of roles requiring human judgment. When automation eliminates entire categories of work, there are not enough meaningful positions left to retrain into. The promise of adaptation becomes a moral alibi for exclusion.

Meritocracy relies on this myth to preserve its legitimacy. If failure can be individualized, the system itself remains beyond critique.

\section*{Children, the Elderly, and the Unjustifiable}

Children, the elderly, and the disabled expose the moral limits of functional valuation. Children are justified only as future contributors. The elderly are justified only by past contributions. Disabled individuals, whose difference cannot be temporally resolved, remain permanently suspect.

These groups persist within meritocratic societies through supplementary moral frameworks—family obligation, charity, or sentimentality—that the economic system itself does not recognize. When these supports weaken, exclusion accelerates.

What is at stake is not compassion but intelligibility. A system that cannot explain why someone matters will eventually stop supporting them.

\section*{Artificial Intelligence and Structural Redundancy}

Artificial intelligence completes the logic of functional reduction. It excels precisely where humans were last defensible: coordination, pattern recognition, evaluation, and communication. As these capacities are automated, the category of the economically necessary human shrinks.

Unemployment under these conditions is not cyclical but structural. Large populations become redundant not because they failed, but because the system no longer requires them. Meritocracy offers no response to redundancy except exclusion.

\section*{Conclusion}

A society that treats humans as money-making functions cannot sustain human dignity once those functions are obsolete. Meritocracy, stripped of its rhetoric, emerges as a mechanism for managing replaceability rather than recognizing worth.

Artificial intelligence does not dehumanize society by being intelligent. It dehumanizes society by integrating seamlessly into systems that already reduced people to instruments. Unless human value is grounded in something other than output, increased efficiency will continue to produce dispossession rather than freedom.

\section*{Welfare, Stigma, and the Moralization of Exclusion}

In societies organized around meritocratic evaluation, welfare is never morally neutral. Because legitimacy is tied to demonstrated productivity, receiving support without visible contribution is framed as a deviation requiring justification. Assistance is not understood as a shared obligation among members of a society, but as an exception granted to those who can successfully narrate their need.

This produces stigma not as an accidental side effect, but as a structural feature. Welfare systems are designed to distinguish the deserving from the undeserving, transforming material support into a moral examination. Recipients are required to prove incapacity, compliance, and gratitude, while any sign of autonomy or dignity risks being interpreted as fraud. The system demands not merely poverty, but performative abasement.

Meritocracy depends on this stigma to preserve its self-image. If exclusion could be recognized as systemic, the narrative of fair competition would collapse. By moralizing need, the system converts structural redundancy into individual failure. Unemployment becomes laziness, disability becomes inefficiency, and dependency becomes a character flaw.

As artificial intelligence increases structural unemployment, this moralization intensifies. Larger populations must be managed without being acknowledged as displaced. Welfare thus becomes both a containment strategy and a disciplinary tool, enforcing the fiction that anyone excluded from productive participation has failed to earn inclusion.

The result is a paradoxical cruelty: those most in need of support are required to surrender the very dignity that support is ostensibly meant to protect. In a meritocratic society, welfare does not contradict exclusion; it administers it.

\section*{Obligation Before Optimization: Pre-Modern Guarantees and the Limits of Merit}

Pre-modern societies were often cruel, rigid, and unequal, but they possessed a structural feature largely absent from modern meritocratic systems: social obligation was not contingent on performance. One’s inclusion in a community was typically grounded in kinship, locality, or status rather than output. While these arrangements enforced hierarchy, they also imposed duties that could not be easily revoked.

Children, the elderly, and the infirm were not required to justify their existence through productivity. Their care was understood as part of the social fabric, even when inadequately fulfilled. Obligation preceded optimization. A lord might exploit tenants, but was nevertheless bound to them; a guild might exclude outsiders, but owed protection to its members. Belonging implied responsibility.

Modern meritocracy reverses this relation. Obligation is conditional, and optimization comes first. Individuals must continually demonstrate value in order to retain access to security. When contribution falters, obligation dissolves. What remains is not reciprocal duty but discretionary aid, administered impersonally and withdrawn easily.

This shift produces a society that is formally egalitarian but substantively brittle. Because inclusion is not guaranteed by membership, but earned through ongoing performance, no one is ever secure. The promise of fairness masks the absence of commitment. Where pre-modern systems enforced obligation through tradition, modern systems evade it through abstraction.

Artificial intelligence intensifies this rupture. As productive capacity is increasingly detached from human labor, the last justification for obligation within meritocratic frameworks evaporates. The system retains optimization but loses even the pretense of mutual dependence.

What pre-modern societies lacked in justice, they partially compensated for in continuity. Meritocratic societies, in abandoning obligation without replacing it, achieve efficiency at the cost of social coherence.

\section*{Optimization, Entropy, and the Collapse of Meaning}

Optimization promises clarity. By reducing complex social processes to measurable variables, it claims to eliminate waste, inefficiency, and ambiguity. Yet when optimization becomes the dominant organizing principle, it produces a paradoxical effect: meaning collapses even as metrics proliferate. What cannot be measured is excluded, and what is excluded becomes unintelligible.

In meritocratic systems, optimization steadily narrows the definition of value. Human judgment, contextual understanding, and long-term stewardship resist quantification and are therefore treated as noise. Over time, systems evolve to reward only those behaviors that align cleanly with metrics, even when those behaviors undermine the broader purpose of the institution. Optimization thus increases local efficiency while accelerating global entropy.

This entropic dynamic is especially visible in large organizations. As feedback is filtered through layers of abstraction, information about lived reality degrades. Decisions appear rational at the top while generating dysfunction below. The system becomes increasingly confident and increasingly wrong. Because meritocratic evaluation attributes outcomes to individual performance, systemic failure remains invisible.

Artificial intelligence accelerates this process by excelling precisely at metric-driven optimization. AI systems can outperform humans within narrowly defined objective functions, but they cannot restore meaning once it has been stripped away. When deployed inside already-optimized institutions, they intensify the divergence between measured success and lived reality.

The collapse of meaning is not a cultural accident but a structural outcome. Systems optimized for efficiency lack the internal resources to recognize what they destroy. Care, dignity, and continuity cannot compete with throughput and cost reduction because they do not register in the evaluative framework.

In such conditions, exclusion appears rational. Redundancy is interpreted as inefficiency rather than as a social failure. Meritocracy, fused with optimization, transforms the erosion of meaning into a technical achievement. What remains is a society that functions smoothly while no longer knowing why it exists.

\section*{Memorability, Attention, and the Compression of Work}

In large-scale economic systems, work does not flow toward those who are most capable, but toward those who are easiest to remember. As markets expand and attention becomes scarce, the cost of searching for competent providers exceeds the cost of reusing familiar ones. Memorability thus becomes an economic advantage independent of quality.

This dynamic produces a compression of work. A small number of individuals or organizations capture a disproportionate share of demand, not because they can personally perform all the labor, but because they function as recognizable interfaces. Once an entity becomes a default choice, it absorbs marginal demand automatically. Others are not rejected for lack of merit, but never considered at all.

This mechanism operates across domains. Artists, actors, musicians, tradespeople, and managers all experience the same narrowing funnel. Expertise remains distributed, but opportunity does not. The system rewards recall over judgment and visibility over mastery.

Meritocracy cannot correct this compression because it presumes a level playing field that no longer exists. When attention is centralized, competition occurs only among those who are already visible. Everyone else competes in silence.

\section*{Why Meritocracy Converges on Oligopoly}

Meritocracy is often defended as a mechanism for rewarding excellence through open competition. Yet when combined with scale, optimization, and constrained attention, it reliably converges on oligopoly. This convergence is not an accident or a corruption of the system; it is its natural endpoint.

As markets grow, the cost of evaluating many participants becomes prohibitive. Systems respond by reducing complexity, favoring agents that are already known, standardized, and predictable. Once a small number of entities dominate visibility, competition ceases to be distributed. New entrants must outperform incumbents not merely in quality, but in memorability, price, and coordination capacity simultaneously. In practice, this barrier is insurmountable.

Meritocracy reframes this outcome as fair. Dominant actors are presented as winners of competition rather than beneficiaries of structural advantage. Their continued dominance is interpreted as proof of merit rather than evidence of compression. Meanwhile, those excluded are told they failed to compete, even when competition never meaningfully occurred.

Oligopoly is therefore not a betrayal of meritocratic ideals, but their logical conclusion under real-world constraints. When systems prioritize efficiency, recall, and cost minimization, they select for scale over skill and visibility over understanding. A few actors become legible to the system; the many disappear from consideration.

Artificial intelligence accelerates this convergence by lowering coordination costs and amplifying the advantages of incumbency. As fewer entities are needed to serve larger markets, the space for human participation narrows further. Merit becomes a post hoc justification for exclusion rather than a pathway to inclusion.

In such a system, fairness is not evaluated by who is included, but by whether exclusion can be explained. Meritocracy supplies that explanation. It transforms concentration into destiny and redundancy into failure, allowing oligopoly to present itself as justice.

\section*{Merit as Moral Justification: The Limits of Fairness}

Recent critiques of meritocracy have emphasized its corrosive moral effects. In \emph{The Tyranny of Merit}, Michael Sandel argues that meritocratic societies generate hubris among those who succeed and humiliation among those who do not. Even when opportunities are formally equal, success is interpreted as deserved and failure as personal inadequacy. This moralization of outcome undermines social solidarity and erodes democratic trust.

This critique is compelling, but incomplete. The problem with meritocracy is not only that it produces objectionable attitudes, but that it functions as a system of moral justification for exclusion. Merit does not merely explain inequality; it legitimizes it. By framing outcomes as earned, the system converts structural conditions into moral verdicts.

What Sandel treats primarily as a civic and ethical failure must also be understood as a structural one. Even a perfectly fair meritocratic competition would still converge on concentration under conditions of scale, constrained attention, and cost minimization. The winners would not merely feel superior; they would become infrastructural choke points through which opportunity flows.

Artificial intelligence sharpens this dynamic by removing any remaining ambiguity about desert. When machines outperform humans across entire categories of work, exclusion can no longer plausibly be explained by effort or ability. Meritocracy persists not because it describes reality, but because it supplies a language that renders exclusion acceptable.

In this sense, meritocracy does not fail despite its fairness. It fails because fairness is the wrong criterion. A system that requires continuous justification of one’s right to belong will inevitably abandon those it cannot efficiently explain.

\section*{Property, Equity, and the Moralization of Luck}

Nowhere is the moral distortion of meritocracy more visible than in the domain of property ownership. Individuals who are able to purchase homes often come to regard their position as deserved, interpreting financial stability as evidence of discipline, foresight, or superior judgment. Yet the ability to buy property is rarely the result of merit alone. It is typically the outcome of timing, inheritance, access to credit, local market conditions, and historical accident. In effect, it is a lottery whose winnings are later narrated as virtue.

Homeownership converts this initial advantage into a durable moral status. As property values rise, owners accumulate equity not primarily through labor, but through passive appreciation. This appreciation is then treated as earned, despite the fact that it is produced collectively. Neighborhoods increase in value because of public infrastructure, social stability, and the cumulative investments of many actors, most of whom do not share in the gains.

Renters illustrate this asymmetry starkly. In many cases, tenants collectively pay the full cost of a property’s mortgage, taxes, and maintenance over time. They contribute to its upkeep and, through their presence, to its desirability. Property managers coordinate operations, and repair workers maintain and improve the physical structure. Yet when the property is sold, the accumulated equity is captured almost entirely by the owner. The individuals whose labor and payments sustained the asset receive no share in the appreciation they helped generate.

Gentrification makes this dynamic explicit. As nearby properties are renovated and new amenities appear, values rise across an area. This increase is not the result of any single owner’s effort, but of distributed activity across a community. Nonetheless, gains are privatized while costs are socialized. Rising rents displace long-term residents, while owners interpret increased valuations as confirmation of their merit.

Meritocracy supplies the moral language that renders this arrangement acceptable. Those who own are said to have made better choices. Those who rent are framed as irresponsible, shortsighted, or lacking ambition. Structural constraints—wage stagnation, credit barriers, historical exclusion—are erased in favor of individualized narratives of success and failure.

The result is a profound misrecognition. What is largely luck is interpreted as virtue. What is collective is claimed as personal achievement. And what is structurally excluded is reframed as deserved. In this way, property ownership becomes not only an economic advantage, but a moral credential, reinforcing a hierarchy that presents itself as fair precisely because its origins are obscured.

\section*{Rent-Seeking as Moral Invisibility}

Meritocratic narratives systematically obscure rent-seeking by rendering it morally invisible. Income derived from ownership is framed as passive reward rather than active extraction. The distinction between earning through contribution and earning through control is collapsed, allowing those who capture rents to present themselves as productive participants rather than beneficiaries of structural position.

Rent-seeking thrives precisely because it does not appear as action. The owner need not understand the labor performed on their behalf, the coordination required to maintain assets, or the social conditions that sustain demand. Revenue arrives abstractly, reinforcing the illusion that it reflects merit rather than leverage. Because no visible effort is required, no moral scrutiny is triggered.

This invisibility is central to the system’s stability. If rent-seeking were recognized as appropriation rather than achievement, the moral authority of ownership would weaken. Meritocracy therefore functions as a laundering mechanism, transforming positional advantage into deserved outcome.

\section*{Inheritance, Timing, and the Freezing of Advantage}

Meritocracy presumes a continual resetting of opportunity, yet modern economies increasingly freeze advantage across generations. Inheritance, early access to appreciating assets, and favorable timing in housing or labor markets produce compounding effects that dwarf individual effort. Those who enter markets early are rewarded not for foresight, but for arrival.

Timing masquerades as wisdom. Individuals who purchased property before price inflation are later described as prudent, while those locked out by rising costs are depicted as irresponsible. The role of historical contingency disappears, replaced by narratives of discipline and sacrifice.

Inheritance accelerates this freeze. Assets transfer without labor, yet their returns are moralized as family success rather than structural privilege. Meritocracy absorbs these transfers by treating starting position as irrelevant, even as it determines outcomes.

\section*{Financialization and the Detachment of Value from Use}

As housing and other necessities become financial instruments, their value detaches from use. Homes are no longer primarily places to live, but vehicles for appreciation. Decisions are guided not by habitability or community stability, but by yield, risk, and liquidity.

This detachment intensifies exclusion. When assets are valued for their exchange potential rather than their function, those who rely on them for daily life are subordinated to abstract market considerations. Rent increases become rational, displacement becomes efficient, and vacancy becomes preferable to affordability.

Meritocracy legitimizes this transformation by interpreting market outcomes as neutral signals. If prices rise, it is taken as evidence of value rather than of scarcity manipulation. Those displaced are not wronged; they are priced out. Moral responsibility dissolves into market logic.

\section*{Concentration Without Remedy}

At this stage of development, the dominant economic system no longer presents credible internal solutions to the dynamics it produces. Capital flows upward not because of policy failure or temporary imbalance, but because concentration is the most stable configuration under conditions of scale, automation, and financial abstraction. Large corporations and technology firms do not merely benefit from this process; they embody it.

Attempts to reverse concentration through marginal reforms consistently fail because they operate within the same evaluative framework. Incentives are adjusted, regulations are tuned, and redistribution is debated, yet the underlying logic remains intact. Efficiency, competitiveness, and growth continue to serve as the primary criteria of success. Any intervention that meaningfully disrupts accumulation is rejected as unrealistic by the very metrics the system enforces.

Artificial intelligence intensifies this impasse. By reducing the need for human coordination and labor, it removes the last practical constraint on consolidation. Entire sectors can be managed by a small number of firms with minimal personnel. Employment ceases to function as a distribution mechanism for income or dignity. What remains is ownership without obligation.

In this context, appeals to innovation, entrepreneurship, or reskilling function less as solutions than as moral deferrals. They postpone recognition of structural redundancy by framing exclusion as temporary or self-inflicted. Meanwhile, capital continues to concentrate in domains that promise scale without social responsibility, most visibly in technology and finance.

The absence of solutions is not a failure of imagination but a feature of the system’s maturity. When legitimacy is tied exclusively to output and ownership, there is no internal basis for preserving inclusion once humans are no longer economically necessary. Concentration proceeds without remedy because the criteria by which remedies would be judged have already been captured.

What this reveals is not a problem awaiting correction, but a system approaching its logical conclusion. The question is no longer how to distribute rewards more fairly, but whether a society organized around accumulation can continue to justify itself once accumulation no longer requires participation.

\section*{The War on Normal People and Technological Redundancy}

In \emph{The War on Normal People}, Andrew Yang argues that technological change and automation are rendering large segments of the workforce economically redundant. His central claim is not that low-skilled workers are being displaced, but that ordinary, socially central occupations—clerical work, retail, transportation, food service, and manufacturing—are disappearing faster than new roles can replace them. What is at stake is not the fate of a marginal class, but the viability of normal economic life.

Yang frames this process as a form of systemic displacement rather than individual failure. He rejects the notion that large populations can be endlessly retrained into new forms of productive relevance, noting that automation increasingly targets routine cognitive and coordinative tasks once thought safe from mechanization. The promise that everyone can adapt through education collapses when the structure of demand itself contracts.

This analysis aligns with the broader critique of meritocracy developed here. As work disappears, meritocratic narratives intensify rather than soften. Displacement is reframed as a failure to reskill, a lack of ambition, or an unwillingness to change. Structural redundancy is individualized, preserving the fiction that outcomes remain deserved.

Where Yang focuses on income stabilization through proposals such as universal basic income, the present analysis emphasizes a deeper legitimacy crisis. Even if material needs are partially addressed, a system that ties dignity to economic necessity continues to treat large portions of the population as superfluous. Income without belonging does not restore social membership.

Seen in this light, the war Yang describes is not merely economic. It is ontological. Normal people are not only losing jobs; they are losing intelligibility within a system that no longer requires their participation. Automation does not create this condition, but it makes it unavoidable. What meritocracy once promised to distribute—security, status, and recognition—can no longer be plausibly offered to the majority.

\section*{When Selection Pressures Decouple from Social Value}

In principle, one might expect selection pressures within an economy to allocate resources toward work that is socially important or intrinsically valuable. Jobs that sustain communities, maintain infrastructure, educate children, or provide care would, under this assumption, attract proportionally greater support. Such a system would reward contribution to collective well-being rather than mere profitability.

In practice, the opposite pattern increasingly prevails. Selection pressures now favor entities that control bottlenecks rather than those that generate value directly. Companies such as \emph{0} and \emph{1} do not dominate because they perform the most socially necessary labor, but because they own scarce resources—chief among them attention. By positioning themselves as intermediaries through which communication, discovery, and visibility must pass, they capture revenue from activities they do not themselves produce.

This ownership of attention distorts economic selection. Advertising revenue scales with aggregation rather than usefulness. Once a platform reaches sufficient size, it absorbs marginal demand automatically, starving alternative forms of work of visibility and funding. The result is not competition among producers of value, but competition for access to platforms that monetize attention abstractly.

Under these conditions, socially essential work becomes systematically undervalued. Teachers, caregivers, tradespeople, and local coordinators generate benefits that are diffuse, long-term, and difficult to monetize. Their contributions do not pass cleanly through advertising markets or scale globally. As a result, selection pressures do not reward them, even as societies become increasingly dependent on their labor.

Meritocracy fails to correct this divergence because it conflates market success with value. When platforms capture disproportionate rewards, their dominance is interpreted as evidence of superior merit rather than structural advantage. The fact that attention can be monetized more easily than care or maintenance is treated as a neutral fact rather than a moral distortion.

Artificial intelligence further entrenches this decoupling. By automating content generation, targeting, and optimization, platforms can expand revenue without expanding social contribution. Selection pressures become increasingly orthogonal to human need. What is rewarded is not importance, but leverage over abstraction.

The expectation that markets naturally align reward with value thus collapses. Selection pressures still operate, but they now select for control over attention, coordination, and infrastructure rather than for work that sustains human life. The economy continues to evolve, but it evolves away from meaning rather than toward it.

\section*{Attention as an Extractive Resource}

Attention has become one of the most valuable resources in the contemporary economy, yet it differs fundamentally from traditional forms of value creation. Unlike labor, which produces goods or services, or capital investment, which enables productive capacity, attention is extracted rather than generated. It is captured from existing human activity and repackaged for sale.

Platforms that dominate attention do not primarily create the behaviors they monetize. People talk, search, socialize, argue, learn, and entertain one another regardless of platform ownership. What platforms provide is not the activity itself, but control over its visibility, routing, and measurement. This control allows attention to be aggregated, segmented, and sold to advertisers, transforming everyday life into a revenue stream.

This extractive model resembles resource capture more than production. Just as land ownership once enabled rent collection without cultivation, attention ownership enables profit without direct contribution to the underlying social value. The platform’s success depends less on improving human outcomes than on maximizing time spent, emotional arousal, and behavioral predictability.

Because attention is finite, its extraction is zero-sum. Time spent within one system is time unavailable to others. As platforms expand, they crowd out alternative forms of social organization, local coordination, and direct exchange. Attention extraction thus weakens the very social ecosystems from which it draws value.

Meritocratic narratives obscure this extraction by treating platform dominance as earned innovation. Control over attention is mistaken for contribution, and leverage is mistaken for value creation. What appears as success is often the result of enclosure rather than excellence.

\section*{Why Advertising Outperforms Production as a Revenue Model}

Advertising has emerged as the dominant revenue model not because it produces the most value, but because it scales with aggregation rather than contribution. Unlike production, which requires the creation of goods or services, advertising monetizes attention regardless of what is being done. As long as attention can be captured, segmented, and targeted, revenue can be generated without corresponding increases in social utility.

This asymmetry gives advertising-based firms a decisive advantage. Production is constrained by materials, labor, quality control, and user satisfaction. Advertising is constrained primarily by scale and data. Once a platform achieves sufficient reach, marginal costs approach zero while marginal revenue remains high. The result is a powerful incentive to grow attention capture rather than improve outcomes.

Because advertising revenue depends on engagement rather than fulfillment, it rewards systems that maximize time spent, emotional activation, and habitual use. Content that is polarizing, repetitive, or sensational often outperforms content that is accurate, careful, or constructive. The revenue model selects for intensity over insight.

This dynamic explains why advertising-driven platforms can outcompete producers of socially necessary goods. Care, education, maintenance, and craftsmanship generate diffuse benefits that cannot be easily monetized through ads. They require trust, continuity, and responsibility rather than scale. As a result, they are structurally disadvantaged despite their importance.

Meritocracy fails to recognize this distortion because it equates revenue with value. Firms that extract large advertising rents are treated as efficient innovators, while those engaged in direct provision struggle for resources. The economy evolves not toward what people need, but toward what can be monetized abstractly.

\section*{What the System Can No Longer See}

As attention extraction and advertising-driven aggregation come to dominate economic life, entire categories of value fall outside the system’s field of vision. The economy does not merely undervalue certain forms of work; it becomes structurally incapable of recognizing them at all. What cannot be captured, scaled, or monetized through abstraction effectively ceases to exist as value.

Care work, maintenance, teaching, repair, and stewardship generate benefits that are cumulative, relational, and local. Their effects unfold over time and resist quantification. They reduce future costs rather than produce immediate returns. Because these contributions do not pass cleanly through advertising markets or platform metrics, they are treated as marginal despite being foundational.

This blindness extends beyond labor to social coherence itself. Trust, institutional memory, and informal coordination sustain societies but cannot be owned or enclosed without being destroyed. When systems reward only what can be measured and monetized, they actively erode the conditions that make measurement meaningful. The economy continues to function while losing contact with the reality it depends on.

Meritocracy reinforces this blindness by interpreting invisibility as insignificance. If a contribution does not generate revenue or attention, it is assumed to lack merit. Those who perform such work are praised rhetorically while excluded materially. Recognition becomes symbolic rather than substantive.

Artificial intelligence accelerates this erasure by excelling precisely at what the system already rewards. Pattern optimization, engagement maximization, and metric alignment are automated, while unquantifiable forms of value remain unsupported. The gap between what sustains human life and what is rewarded by the economy widens.

What the system can no longer see does not disappear. It is simply unsupported, unfunded, and eventually exhausted. When care collapses, infrastructure decays, and social trust erodes, the failure appears sudden. In reality, it has been accumulating invisibly, dismissed for years as economically irrelevant.

\section*{The End of Human Justification}

This essay has argued that meritocracy, when fused with large-scale optimization, attention extraction, and automation, no longer functions as a principle of justice. It functions instead as a system of justification—one that explains why exclusion is acceptable once human participation becomes economically unnecessary. What presents itself as fairness is, in practice, a method for rendering redundancy morally legible.

Across domains, humans are reduced to functions: revenue generators, cost centers, engagement units, or risk variables. Organizational blindness follows inevitably. Systems optimized for scale lose the capacity to recognize the sources of their own stability. Care, maintenance, judgment, and continuity disappear from evaluation not because they are unimportant, but because they cannot be abstracted without being destroyed.

Artificial intelligence does not introduce this condition; it completes it. By excelling at metric-driven optimization, AI removes the last pretense that effort, adaptation, or merit can secure inclusion for the majority. What remains is ownership without obligation and concentration without justification, sustained by narratives that individualize structural outcomes.

The question that now confronts meritocratic societies is no longer how to reward contribution more efficiently, but whether contribution can remain the basis of belonging at all. A system that requires continuous proof of worth cannot survive the moment when proof is no longer broadly possible. When most people are no longer needed, justification fails.

This is not merely an economic crisis, but a legitimacy crisis. A society that cannot explain why normal people deserve security, dignity, and care—independent of their usefulness—has already abandoned its moral foundation. Efficiency may continue to increase. Wealth may continue to concentrate. But justification has ended.

What follows is not yet clear. What is clear is that a social order organized around functional reduction cannot indefinitely sustain human membership. Merit without mercy does not culminate in fairness. It culminates in exclusion that no longer knows how to explain itself.

\newpage
\begin{thebibliography}{99}

\bibitem{Sandel2020}
M. J. Sandel.
\newblock \emph{The Tyranny of Merit: What’s Become of the Common Good?}
\newblock Farrar, Straus and Giroux, 2020.

\bibitem{Yang2018}
A. Yang.
\newblock \emph{The War on Normal People: The Truth About America’s Disappearing Jobs and Why Universal Basic Income Is Our Future}.
\newblock Hachette Books, 2018.

\bibitem{Arendt1958}
H. Arendt.
\newblock \emph{The Human Condition}.
\newblock University of Chicago Press, 1958.

\bibitem{Polanyi1944}
K. Polanyi.
\newblock \emph{The Great Transformation}.
\newblock Beacon Press, 1944.

\bibitem{Graeber2018}
D. Graeber.
\newblock \emph{Bullshit Jobs: A Theory}.
\newblock Simon \& Schuster, 2018.

\bibitem{Zuboff2019}
S. Zuboff.
\newblock \emph{The Age of Surveillance Capitalism}.
\newblock PublicAffairs, 2019.

\bibitem{Doctorow2023}
C. Doctorow.
\newblock \emph{The Internet Con: How to Seize the Means of Computation}.
\newblock Verso, 2023.

\bibitem{Varian2019}
H. R. Varian.
\newblock Artificial intelligence, economics, and industrial organization.
\newblock In \emph{The Economics of Artificial Intelligence}, University of Chicago Press, 2019.

\bibitem{Autor2015}
D. H. Autor.
\newblock Why are there still so many jobs? The history and future of workplace automation.
\newblock \emph{Journal of Economic Perspectives}, 29(3):3--30, 2015.

\bibitem{AcemogluRestrepo2020}
D. Acemoglu and P. Restrepo.
\newblock Artificial intelligence and jobs.
\newblock \emph{Journal of Economic Perspectives}, 34(3):30--55, 2020.

\bibitem{Simon1971}
H. A. Simon.
\newblock Designing organizations for an information-rich world.
\newblock In \emph{Computers, Communications, and the Public Interest}, Johns Hopkins Press, 1971.

\bibitem{Scott1998}
J. C. Scott.
\newblock \emph{Seeing Like a State: How Certain Schemes to Improve the Human Condition Have Failed}.
\newblock Yale University Press, 1998.

\bibitem{Tett2015}
G. Tett.
\newblock \emph{The Silo Effect: The Peril of Expertise and the Promise of Breaking Down Barriers}.
\newblock Simon \& Schuster, 2015.

\bibitem{Piketty2014}
T. Piketty.
\newblock \emph{Capital in the Twenty-First Century}.
\newblock Harvard University Press, 2014.

\bibitem{Mazzucato2018}
M. Mazzucato.
\newblock \emph{The Value of Everything: Making and Taking in the Global Economy}.
\newblock PublicAffairs, 2018.

\bibitem{FreyOsborne2017}
C. B. Frey and M. A. Osborne.
\newblock The future of employment: How susceptible are jobs to computerisation?
\newblock \emph{Technological Forecasting and Social Change}, 114:254--280, 2017.

\bibitem{Benjamin2019}
R. Benjamin.
\newblock \emph{Race After Technology: Abolitionist Tools for the New Jim Code}.
\newblock Polity Press, 2019.

\bibitem{Rawls1971}
J. Rawls.
\newblock \emph{A Theory of Justice}.
\newblock Harvard University Press, 1971.

\bibitem{Young1958}
M. Young.
\newblock \emph{The Rise of the Meritocracy}.
\newblock Thames \& Hudson, 1958.

\bibitem{Lanier2018}
J. Lanier.
\newblock \emph{Ten Arguments for Deleting Your Social Media Accounts Right Now}.
\newblock Henry Holt and Company, 2018.

\bibitem{Srnicek2017}
N. Srnicek.
\newblock \emph{Platform Capitalism}.
\newblock Polity Press, 2017.

\bibitem{Standing2011}
G. Standing.
\newblock \emph{The Precariat: The New Dangerous Class}.
\newblock Bloomsbury Academic, 2011.

\bibitem{Klein2007}
N. Klein.
\newblock \emph{The Shock Doctrine}.
\newblock Metropolitan Books, 2007.

\bibitem{Harvey2005}
D. Harvey.
\newblock \emph{A Brief History of Neoliberalism}.
\newblock Oxford University Press, 2005.

\bibitem{FrankCook1995}
R. H. Frank and P. J. Cook.
\newblock \emph{The Winner-Take-All Society}.
\newblock Free Press, 1995.

\bibitem{Ellul1964}
J. Ellul.
\newblock \emph{The Technological Society}.
\newblock Vintage Books, 1964.

\end{thebibliography}

\end{document}
