% Configuring document class and essential packages
\documentclass[a4paper,12pt]{article}

% Including standard LaTeX packages for formatting and mathematics
\usepackage{amsmath,amssymb,amsfonts}
\usepackage{mathtools}
\usepackage{geometry}
\usepackage{enumitem}
\usepackage{hyperref}
\usepackage{natbib}
\usepackage{parskip}

% Defining custom commands for mathematical notation
\newcommand{\C}{\mathcal{C}}
\newcommand{\M}{\mathcal{M}}
\newcommand{\F}{\mathcal{F}}
\newcommand{\Ocal}{\mathcal{O}}
\newcommand{\U}{\mathcal{U}}
\newcommand{\V}{\mathcal{V}}
\newcommand{\p}{\mathbf{p}}
\newcommand{\lvec}{\mathbf{l}}
\newcommand{\E}{\mathrm{E}}
\newcommand{\Scal}{\mathcal{S}}
\newcommand{\vvec}{\vec{v}}
\DeclareMathOperator{\hocolim}{hocolim}

% Setting page geometry
\geometry{left=2.5cm,right=2.5cm,top=2.5cm,bottom=2.5cm}

% Beginning the document
\begin{document}

% Title and author
\title{Geometric Memory as Trajectory Influence on a Generative Substrate}
\author{Flyxion}
\date{September 18, 2025}
\maketitle

% Abstract
\begin{abstract}
This essay presents a geometric model of memory as trajectories influencing a generative substrate, formalized as a smooth manifold of meaning-states. Unlike static storage, memory is modeled as paths that bias future dynamics via a nonlocal-in-time flow, integrating past influences through a kernel-weighted parallel transport. The substrate evolves under a Riemannian metric, with curvature and torsion shaping convergence and reinterpretation. Plasticity updates the metric and connection based on past trajectories, embedding habitual paths as low-resistance channels. The model ties to the Relativistic Scalar Vector Plenum (RSVP) framework, mapping scalar density ($\Phi$), vector flows ($\vvec$), and entropy ($\Scal$) to memory processes. Testable predictions include geodesic reuse, curvature-habit links, and kernel-driven recall effects. A discrete version aligns with large language models (LLMs), offering a computationally tractable implementation. The model reframes memory as regenerative reconstruction, not retrieval, with implications for cognition, AI, and semantic infrastructures.
\end{abstract}

\section{Substrate and State}
Consider a smooth manifold $\M$ representing ``meaning-states'' (e.g., neural activations, LLM embeddings, or RSVP fields). A system’s state at time $t$ is $x(t) \in \M$, with local coordinates $x^i$ and a Riemannian metric $g_{ij}(x)$ quantifying semantic distances. A generative drift field $F: \M \to T\M$ encodes default evolution, reflecting grammar, physics, or policy priors. Formally:
\[
\dot{x}(t) = F(x(t)) + \text{additional terms},
\]
where $F$ drives the substrate’s intrinsic dynamics, akin to autoregressive priors in cognition or LLMs \citep{vaswani2017attention}.

\section{Memory as Trajectory Influence}
Memory is not a static snapshot but a trajectory $x(\tau)$, $\tau \in [0,t]$, that biases future evolution. Recall is a nonlocal-in-time flow:
\[
\dot{x}(t) = F(x(t)) + \int_0^t K(t,\tau; x(t), x(\tau)) \, U(x(\tau) \to x(t)) \, \dot{x}(\tau) \, d\tau + \xi(t),
\]
where:
\begin{itemize}
    \item $K(t,\tau; x(t), x(\tau))$ is an influence kernel weighting past contributions.
    \item $U(x(\tau) \to x(t)): T_{x(\tau)}\M \to T_{x(t)}\M$ is parallel transport along a connection $\Gamma^k_{ij}$, ensuring coordinate-invariant influence.
    \item $\xi(t)$ is stochastic innovation, modeling novelty or noise.
\end{itemize}
Common kernels include:
\begin{itemize}
    \item Exponential decay: $K(t,\tau) = e^{-\alpha (t-\tau)}$ (fading memory).
    \item Power-law: $K(t,\tau) = (t-\tau)^{-\beta}$ (long-tail recall).
    \item Oscillatory-consolidation: $K(t,\tau) = \cos(\omega (t-\tau)) e^{-\alpha (t-\tau)}$ (rehearsal rhythms).
    \item Selective gates: $K(t,\tau) = \sigma(\langle q(x(t)), k(x(\tau)) \rangle)$ (content-addressable recall, as in attention \citep{vaswani2017attention}).
\end{itemize}

\section{Geometry: Metric, Curvature, and Torsion}
The metric $g_{ij}(x)$ defines semantic proximity. Curvature $R^i_{jkl}$ governs geodesic deviation: high curvature induces convergence (attractors, habits) or divergence (repellors, novelty). A connection with torsion $T^k_{ij} = \Gamma^k_{ij} - \Gamma^k_{ji}$ allows ``semantic twist,'' where past influences are reinterpreted contextually \citep{penrose1989road}. Geodesic memory channels emerge where aligned trajectories reduce effective resistance, modeled as a modified connection $\tilde{\Gamma}^k_{ij}$ updated by past usage.

\section{Memory-Modulated Geometry (Plasticity)}
Past trajectories reshape the substrate:
\[
\partial_t g_{ij}(x) = -\eta_g \int_0^t K(t,\tau) \, \Pi_{ij}(x(t), x(\tau)) \, d\tau, \quad \partial_t \Gamma^k_{ij} = -\eta_\Gamma \int_0^t K(t,\tau) \, \Psi^k_{\ ij}(x(t), x(\tau)) \, d\tau,
\]
where $\Pi_{ij}$ and $\Psi^k_{\ ij}$ are Hebbian-like tensors encoding path reinforcement, and $\eta_g, \eta_\Gamma$ are learning rates. This plasticity embeds habitual paths as low-resistance geodesics, akin to neural synaptic strengthening \citep{hebb1949organization}.

\section{Energy/Lyapunov View}
Define a reconstruction functional:
\[
\E[x(\cdot)] = \int_0^T \left\{ \frac{1}{2} \langle \dot{x}(t) - F(x(t)), \dot{x}(t) - F(x(t)) \rangle_g + \int_0^t \mathcal{L}(x(t), x(\tau)) \, K(t,\tau) \, d\tau \right\} dt,
\]
where $\mathcal{L}$ penalizes deviation from past trajectories. Dynamics minimize $\E$, balancing fidelity to prior paths and generative drift \citep{friston2010free}.

\section{Discrete (Sequence) Version}
For LLMs or sequence models, let states be $x_t \in \mathbb{R}^d$ with learned metric $g_t$. The dynamics are:
\[
x_{t+1} = f(x_t) + \sum_{s=0}^t \alpha_{t,s} \, P_{t \leftarrow s} \, (x_{s+1} - x_s) + \epsilon_t,
\]
where $P_{t \leftarrow s}$ is a transport map (e.g., Jacobian of $f$ or learned linear map), $\alpha_{t,s} = K(t,s) \cdot \sigma(\langle q(x_t), k(x_s) \rangle)$ combines temporal and content-based attention, and $\epsilon_t$ is noise. Plasticity updates $g_t$ and $P_{t \leftarrow s}$ via Hebbian rules weighted by $\alpha_{t,s}$.

\section{RSVP Tie-In}
In RSVP \citep{semantic2025}:
\begin{itemize}
    \item $\Phi$: Capacity density, measuring reliable paths per region.
    \item $\vvec$: Average flow field, reflecting habitual trajectories.
    \item $\Scal$: Entropy, high for exploratory novelty, low for stabilized recall.
\end{itemize}
Memory consolidation reduces $\Scal$ along practiced channels, increasing $\Phi$ locally. This aligns with RSVP’s semantic infrastructure, preserving $\Scal$ for generativity \citep{lurie2009higher}.

\section{Testable Predictions}
\begin{enumerate}
    \item \textbf{Geodesic Reuse}: Trained transport $P_{t \leftarrow s}$ aligns with future steps; measure cosine similarity between $P_{t \leftarrow s} (x_{s+1} - x_s)$ and $x_{t+1} - x_t$.
    \item \textbf{Curvature–Habit Link}: High-recall regions show reduced sectional curvature in practiced directions; detect via PCA spectra or fMRI/EEG manifold learning.
    \item \textbf{Kernel Manipulations}: Perturbing $K$ (e.g., via sleep rhythms) shifts recency/primacy effects in recall.
    \item \textbf{Torsion and Reinterpretation}: Ambiguous cues yield path-dependent decoding; context rotates transported vectors, detectable as non-commuting transport loops.
    \item \textbf{Energy Descent}: Recall loss follows $\E$; noise flattening $\mathcal{L}$ increases creative confabulations.
\end{enumerate}

\section{Minimal Implementation Sketch}
\begin{itemize}
    \item \textbf{State}: Latent $x_t \in \mathbb{R}^d$, metric $g_t = I + W_t W_t^\top$.
    \item \textbf{Transport}: $P_{t \leftarrow s} = \exp(-\gamma (t-s)) I$ or learned matrix.
    \item \textbf{Kernel}: $K(t,s) = e^{-\alpha (t-s)}$ or $\sigma(\langle q(x_t), k(x_s) \rangle)$.
    \item \textbf{Update}: $x_{t+1} = f(x_t) + \sum_s \alpha_{t,s} P_{t \leftarrow s} (x_{s+1} - x_s)$.
    \item \textbf{Plasticity}: $W_{t+1} = W_t + \eta \sum_s \alpha_{t,s} (x_s x_s^\top)$.
    \item \textbf{Evaluation}: Measure recall fidelity (MSE to target sequence) vs. creativity (divergence from training data).
\end{itemize}

\section{Intuition}
Memory is a landscape of meanings—a manifold—where past journeys carve paths. Each step you take is pulled by those prior routes, smoothed by frequent travel and twisted by context. Recall isn’t replaying a tape; it’s regenerating a path, guided by worn trails yet free to find shortcuts, balancing fidelity with discovery.

% Bibliography
\bibliographystyle{plainnat}
\bibliography{references}

\end{document}
