% Configuring document class and essential packages
\documentclass[a4paper,12pt]{article}

% Including standard LaTeX packages for formatting and mathematics
\usepackage{amsmath,amssymb,amsfonts}
\usepackage{mathtools}
\usepackage{geometry}
\usepackage{enumitem}
\usepackage{hyperref}
\usepackage{natbib}
\usepackage{parskip}
\usepackage{tocloft}

% Defining custom commands for mathematical notation
\newcommand{\C}{\mathcal{C}}
\newcommand{\M}{\mathcal{M}}
\newcommand{\F}{\mathcal{F}}
\newcommand{\Ocal}{\mathcal{O}}
\newcommand{\U}{\mathcal{U}}
\newcommand{\V}{\mathcal{V}}
\newcommand{\p}{\mathbf{p}}
\newcommand{\lvec}{\mathbf{l}}
\newcommand{\E}{\mathrm{E}}
\newcommand{\Scal}{\mathcal{S}}
\newcommand{\vvec}{\vec{v}}
\DeclareMathOperator{\hocolim}{hocolim}

% Setting page geometry
\geometry{left=2.5cm,right=2.5cm,top=2.5cm,bottom=2.5cm}

% Beginning the document
\begin{document}

% Title and author
\title{Geometric Memory as Trajectory Influence on a Generative Substrate}
\author{Flyxion}
\date{September 18, 2025}
\maketitle

% Abstract
\begin{abstract}
This essay formalizes memory as trajectories influencing a generative substrate, modeled as a smooth manifold of meaning-states. Trajectories bias future dynamics via a nonlocal-in-time flow, integrating past influences through kernel-weighted parallel transport. The substrate evolves under a Riemannian metric, with curvature and torsion shaping convergence and reinterpretation. Plasticity updates the metric and connection, embedding habitual paths as low-resistance channels. The model integrates with the Relativistic Scalar Vector Plenum (RSVP) framework, mapping scalar density ($\Phi$), vector flows ($\vvec$), and entropy ($\Scal$) to memory processes. Semantic ledging and virtual memory extend the model, handling mismatched cues and dynamic scaling, inspired by geometric memory management \citep{kuijper2021geometric}. Testable predictions link curvature to habits, kernel perturbations to recall effects, and torsion to context-dependent reinterpretation, with applications to neuroscience (fMRI/EEG), large language models (LLMs), and semantic infrastructures. A discrete implementation aligns with transformer architectures, offering a computationally tractable framework for experimentation. Empirical validation draws from hippocampal replay \citep{oliver2017hippocampal}, representational geometry \citep{edelman1998representation}, and generative memory models \citep{ritter2023generative}.
\end{abstract}

% Table of Contents
\tableofcontents

\section{Introduction}
Traditional models of memory, often analogized to a ``file cabinet'' where static records are stored and retrieved verbatim, inadequately capture the reconstructive, dynamic nature of memory in biological and computational systems \citep{mcclelland1995why}. Empirical evidence from cognitive psychology and neuroscience reveals that recall is not a passive lookup but an active generative process: fragments of prior experiences are recombined, filtered by current context, and prone to distortion \citep{ritter2023generative}. For instance, hippocampal replay during sleep reactivates trajectories of prior events not as exact copies but as adaptive reconstructions that consolidate learning \citep{oliver2017hippocampal}. Similarly, in artificial intelligence, transformer-based large language models (LLMs) exhibit attention mechanisms that weight past states to influence predictions, mimicking reconstructive recall \citep{vaswani2017attention}.

This essay advances a geometric model of memory as trajectories on a generative substrate—a smooth manifold $\M$ of meaning-states—where past paths exert nonlocal influence on future dynamics through kernel-weighted parallel transport. The manifold's geometry provides structural constraints: the Riemannian metric $g_{ij}$ encodes semantic distances, curvature $R^i_{jkl}$ sculpts convergence into habits or divergence into novelty, and torsion $T^k_{ij}$ enables context-dependent reinterpretation \citep{penrose1989road}. Plasticity mechanisms further modulate this geometry, embedding reinforced paths as low-resistance geodesics \citep{hebb1949organization}. 

The framework aligns with representational geometry in cognitive neuroscience, where brain representations are analyzed via distance matrices to relate computation, cognition, and neural activity \citep{edelman1998representation}. It also integrates with the Relativistic Scalar Vector Plenum (RSVP) framework \citep{semantic2025}, generalizing trajectory influence across cognition, AI, and cosmology: just as cosmic flows constrain galaxy formation \citep{weinberg2008cosmology}, semantic trajectories shape memory reconstruction. Semantic ledging and geometric virtual memory \citep{kuijper2021geometric} address practical challenges like mismatched cues and scalable storage, ensuring efficiency in dynamic contexts.

This model reframes memory as an entropy-respecting generative process, balancing fidelity to prior trajectories with creative reinterpretation. Subsequent sections develop the formalism, empirical ties, and applications, culminating in testable predictions and an implementation sketch.

\section{Substrate and State}
The generative substrate is a smooth manifold $\M$ parameterizing ``meaning-states,'' encompassing neural activations, semantic embeddings in LLMs, or RSVP fields \citep{semantic2025}. A state is $x(t) \in \M$, with local coordinates $x^i$ and a Riemannian metric $g_{ij}(x)$ defining infinitesimal semantic distances. The metric quantifies conceptual proximity: for example, $g_{ij}$ is small between ``cat'' and ``feline'' embeddings, reflecting high affinity, and large between unrelated terms like ``cat'' and ``quantum'' \citep{edelman1998representation}. This geometry ensures that trajectories evolve continuously, avoiding discrete jumps inherent in symbolic models.

The intrinsic dynamics are governed by a generative drift field $F: \M \to T\M$, encoding baseline evolution:
\[
\dot{x}(t) = F(x(t)) + \text{memory influence terms}.
\]
In LLMs, $F$ corresponds to the autoregressive prior that projects likely continuations based on learned patterns \citep{vaswani2017attention}. In neural systems, it parallels cortical field dynamics, where $F$ reflects spontaneous activity shaped by synaptic weights \citep{amari1977dynamics}. The manifold $\M$ can be visualized as a semantic landscape: valleys represent stable states (e.g., well-learned concepts), ridges denote barriers (e.g., semantic conflicts), and the metric $g_{ij}$ measures the ``effort'' of traversal between them. This structure underscores that memory operates not on isolated points but on paths through a curved space, where local neighborhoods capture fine distinctions and global topology enforces coherence \citep{heusser2018geometric}.

The substrate's generative nature distinguishes it from passive storage: $\M$ is not a static archive but an active arena where states evolve autonomously under $F$, modulated by memory influences. This setup aligns with predictive coding theories, where the brain minimizes surprise by aligning sensory input with internal models \citep{friston2010free}, and extends to AI, where embeddings in models like BERT or GPT form latent manifolds traversed by attention flows \citep{devlin2019bert}.

\section{Memory as Trajectory Influence}
Memory manifests as a prior trajectory $x(\tau)$, $\tau \in [0,t]$, that biases the current evolution via a nonlocal-in-time flow:
\[
\dot{x}(t) = F(x(t)) + \int_0^t K(t,\tau; x(t), x(\tau)) \, U(x(\tau) \to x(t)) \, \dot{x}(\tau) \, d\tau + \xi(t),
\]
where $K(t,\tau; x(t), x(\tau))$ is the influence kernel, $U$ is parallel transport, and $\xi(t)$ is stochastic innovation \citep{ritter2023generative}. The integral term reconstructs the present by weighting past increments $\dot{x}(\tau)$, ensuring recall is path-dependent rather than state-based.

Influence kernels $K$ capture temporal and content-based decay:
\begin{itemize}
    \item \textbf{Exponential Decay}: $K = e^{-\alpha (t-\tau)}$ models short-term memory, with rapid fall-off mimicking hippocampal replay where recent events dominate \citep{hassabis2007patients}.
    \item \textbf{Power-Law}: $K = (t-\tau)^{-\beta}$ enables long-tail recall, aligning with empirical forgetting curves where early experiences persist indefinitely \citep{mcclelland1995why}.
    \item \textbf{Oscillatory-Consolidation}: $K = \cos(\omega (t-\tau)) e^{-\alpha (t-\tau)}$ reflects sleep rhythms, where periodic reactivation strengthens traces \citep{stickgold2005sleep}.
    \item \textbf{Selective Gated}: $K = \sigma(\langle q(x(t)), k(x(\tau)) \rangle)$ implements content-addressable memory, as in transformer attention where relevance trumps recency \citep{vaswani2017attention}.
\end{itemize}
The discrete analogue is:
\[
x_{t+1} = f(x_t) + \sum_{s=0}^t \alpha_{t,s} P_{t \leftarrow s} (x_{s+1} - x_s) + \epsilon_t,
\]
with $\alpha_{t,s} = K(t,s) \cdot \sigma(\langle q(x_t), k(x_s) \rangle)$ and $P_{t \leftarrow s}$ as transport. This formulation unifies continuous neural dynamics with discrete AI sequences, predicting that kernel choice modulates recall bias \citep{oliver2017hippocampal}.

\section{Geometry: Metric, Curvature, and Torsion}
The Riemannian metric $g_{ij}(x)$ quantifies semantic proximity, enabling geodesic distances $ds^2 = g_{ij} dx^i dx^j$ that measure conceptual affinity. For instance, in word embeddings, low $g_{ij}$ between synonyms facilitates smooth traversal, while high values enforce barriers between domains \citep{edelman1998representation}. The metric is learned or emergent, adapting to experience via plasticity.

Curvature $R^i_{jkl}$ dictates geodesic deviation, controlling trajectory convergence or divergence. Positive curvature forms basins (e.g., habitual associations like ``apple-fruit''), pulling paths together \citep{penrose1989road}; negative curvature fosters exploration (e.g., metaphorical leaps). On a 2D hyperbolic surface, exponential divergence models forgetting, where trajectories separate rapidly without reinforcement \citep{heusser2018geometric}. In neuroscience, curvature signatures appear in fMRI patterns during recall, with low-curvature regions correlating with stable memories \citep{hassabis2007patients}.

Torsion $T^k_{ij} = \Gamma^k_{ij} - \Gamma^k_{ji}$ introduces antisymmetric twists, enabling path-dependent reinterpretation. Transporting a vector around a closed loop yields a rotated result, formalizing how context reframes memories (e.g., ``bank'' as financial or riverine) \citep{amari1977dynamics}. Non-commuting loops predict context-sensitive EEG responses, where the same cue elicits divergent patterns based on preceding trajectories \citep{stickgold2005sleep}. Together, these geometric elements ensure memory is structured yet flexible: the metric grounds local similarity, curvature sculpts global flow, and torsion weaves contextual nuance.

\section{Memory-Modulated Geometry (Plasticity)}
Trajectories induce plasticity:
\[
\partial_t g_{ij}(x) = -\eta_g \int_0^t K(t,\tau) \, \Pi_{ij}(x(t), x(\tau)) \, d\tau, \quad \partial_t \Gamma^k_{ij} = -\eta_\Gamma \int_0^t K(t,\tau) \, \Psi^k_{\ ij}(x(t), x(\tau)) \, d\tau,
\]
where $\Pi_{ij}$ and $\Psi^k_{\ ij}$ are Hebbian tensors encoding co-activation \citep{hebb1949organization}. Repeated traversal lowers $g_{ij}$ along paths, embedding habits as geodesics of minimal length \citep{sutton2018reinforcement}. For example, frequent ``dog-bark'' associations compress their metric distance, accelerating recall.

This feedback loop—trajectories bias geometry, geometry biases trajectories—mirrors synaptic strengthening in neuroscience \citep{mcclelland1995why}. In AI, it corresponds to fine-tuning embeddings, where high-attention pairs warp the latent space \citep{vaswani2017attention}. Plasticity ensures the manifold evolves, transforming transient experiences into enduring structures while allowing adaptation to novelty.

\subsection{Semantic Ledging}
Inspired by Kuijper’s ledging \citep{kuijper2021geometric}, semantic ledging decomposes mismatched cues into a geometric series of aligned influences. A cue (e.g., ``dog'') activates the largest fitting trajectory (e.g., ``dog-bark''), with residuals patched by finer terms (e.g., ``pet-sound''). This bounds overhead, preventing fragmentation, and leverages torsion for reinterpretation—e.g., patching ``bank'' to fit financial or ecological contexts. Ledging ensures robust recall from noisy inputs, aligning with hippocampal pattern completion \citep{oliver2017hippocampal}.

\subsection{Semantic Virtual Memory}
Kuijper’s geometric virtual memory \citep{kuijper2021geometric} uses block trees for elastic spaces. Semantically, meaning-states occupy virtual subspaces of $\M$, instantiated via $U$. A conversation spawns a local region, transporting prior utterances without duplication. This scales recall, supports compositional reuse (e.g., merging financial/ecological ``bank'' subspaces), and isolates contexts to reduce interference. In RSVP, it enables categorical gluing of locals into globals \citep{semantic2025}, fostering scalable infrastructures.

\section{Energy/Lyapunov View}
The reconstruction functional is:
\[
\E[x(\cdot)] = \int_0^T \left\{ \frac{1}{2} \langle \dot{x}(t) - F(x(t)), \dot{x}(t) - F(x(t)) \rangle_g + \int_0^t \mathcal{L}(x(t), x(\tau)) \, K(t,\tau) \, d\tau \right\} dt,
\]
balancing drift fidelity and trajectory adherence \citep{friston2010free}. For attractors (e.g., ``cat-dog''), low $\E$ stabilizes associations; divergence via $\xi(t)$ increases $\E$, signaling reinterpretation. Gradient descent $\dot{x} = -\nabla \E$ yields Lyapunov-stable dynamics: basins form habits, saddles enable transitions \citep{hopfield1982neural}. In neuroscience, this predicts error-minimizing replay \citep{oliver2017hippocampal}; in AI, it guides sampling to minimize perplexity while preserving creativity.

\section{Discrete (Sequence) Version}
For sequences, $x_t \in \mathbb{R}^d$ with $g_t$ evolves as:
\[
x_{t+1} = f(x_t) + \sum_{s=0}^t \alpha_{t,s} \, P_{t \leftarrow s} \, (x_{s+1} - x_s) + \epsilon_t,
\]
$\alpha_{t,s} = K(t,s) \cdot \sigma(\langle q(x_t), k(x_s) \rangle)$, $P_{t \leftarrow s}$ via Jacobian. This geometric attention extends transformers \citep{vaswani2017attention}, with plasticity:
\[
W_{t+1} = W_t + \eta \sum_s \alpha_{t,s} (x_s x_s^\top).
\]
Complexity is $O(t d^2)$, mitigated by sparse kernels. In LLMs, it predicts improved long-context handling via transport, testable on benchmarks like LongBench \citep{bai2023longbench}.

\section{RSVP Tie-In}
RSVP maps \citep{semantic2025}:
\begin{itemize}
    \item $\Phi$: Path density, high in reinforced memories.
    \item $\vvec$: Thought flows, guiding narrative trajectories.
    \item $\Scal$: Entropy, modulating exploration vs. stability.
\end{itemize}
Consolidation lowers $\Scal$ along geodesics, boosting $\Phi$ \citep{mcclelland1995why}. Cosmologically, low-vorticity flows constrain galaxies \citep{weinberg2008cosmology}, paralleling memory suppression. Semantic infrastructures preserve $\Scal$ via gluing \citep{lurie2009higher}, enabling entropy-respecting AI.

\section{Applications to Cognition and AI}
In cognition, exponential kernels model short-term replay \citep{hassabis2007patients}, power-law long-term persistence \citep{mcclelland1995why}. Torsion explains reinterpretation in ambiguous recall \citep{stickgold2005sleep}. In AI, geometric attention enhances LLMs for context-sensitive generation \citep{vaswani2017attention}, e.g., dialogue systems prioritizing recent/distant paths. Virtual memory scales to multimodal inputs, reducing catastrophic forgetting \citep{kirkpatrick2017overcoming}. This unifies Hopfield attractors \citep{hopfield1982neural} with predictive coding \citep{friston2010free}, predicting hybrid architectures for robust memory.

\section{Testable Predictions}
\begin{enumerate}
    \item \textbf{Geodesic Reuse}: $P_{t \leftarrow s}$ aligns with $x_{t+1} - x_t$; cosine similarity rises with training in LLMs/EEG \citep{oliver2017hippocampal}.
    \item \textbf{Curvature–Habit Link}: Low curvature in high-recall regions; PCA spectra flatten in fMRI during habits \citep{hassabis2007patients}.
    \item \textbf{Kernel Manipulations}: Perturb $K$ shifts recency/primacy; sleep disruption alters behavioral curves \citep{stickgold2005sleep}.
    \item \textbf{Torsion and Reinterpretation}: Non-commuting loops in ambiguous cues; EEG coupling varies by context \citep{amari1977dynamics}.
    \item \textbf{Energy Descent}: Loss follows $\E$; noise boosts confabulation in LLMs/dreams \citep{friston2010free}.
    \item \textbf{Contextual Coherence}: Disrupted gluing causes incoherence; metrics drop in ablated LLMs/schizophrenia tasks \citep{edelman1998representation}.
\end{enumerate}

\section{Minimal Implementation Sketch}
State: $x_t \in \mathbb{R}^d$, $g_t = I + W_t W_t^\top$. Transport: $P_{t \leftarrow s} = \exp(-\gamma (t-s)) I$. Kernel: $K(t,s) = e^{-\alpha (t-s)}$. Update: $x_{t+1} = f(x_t) + \sum_s \alpha_{t,s} P_{t \leftarrow s} (x_{s+1} - x_s)$. Plasticity: $W_{t+1} = W_t + \eta \sum_s \alpha_{t,s} (x_s x_s^\top)$. Evaluation: MSE fidelity, KL creativity on sequences. Pseudocode as before.

\section{Discussion and Outlook}
The model excels at reconstructive dynamics but overlooks quantum coherence \citep{penrose1989road} or qualia. Future: PyTorch simulations, fMRI curvature probes, LLM torsion ablations. Implications: Entropy-respecting AI \citep{semantic2025}, unifying Hopfield/RNN/predictive coding \citep{hopfield1982neural,vaswani2017attention,friston2010free}.

\section{Intuition}
Memory is a landscape of meanings—a manifold—where past journeys carve paths. Each step is guided by worn trails, smoothed by frequent travel and twisted by context. Recall regenerates these paths, balancing fidelity to prior routes with the freedom to find shortcuts, like rivers carving channels yet meandering anew.

% Bibliography
\bibliographystyle{plainnat}
\bibliography{references}

\end{document}
