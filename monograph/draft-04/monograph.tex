\documentclass[openany]{book}

% ---- Preamble ----
\usepackage[T1]{fontenc}
\usepackage{lmodern}
\usepackage{microtype}
\usepackage{csquotes}
\usepackage{siunitx}
\sisetup{detect-weight=true,detect-family=true}
\DeclareSIUnit\kgCOe{kg CO_2e}
\DeclareSIUnit\USD{\$}
\usepackage{amsmath}
\usepackage{amssymb}
\usepackage{geometry}
\geometry{a4paper, margin=1in}
\usepackage{hyperref}
\usepackage{natbib}
\bibliographystyle{plainnat}
\setcitestyle{authoryear,open={(},close={)}}
\usepackage{enumitem}
\usepackage{graphicx}
\usepackage{amsthm}
\usepackage[nameinlink,capitalise]{cleveref}

% ---- Theorem-like ----
\newtheorem{definition}{Definition}[chapter]
\newtheorem{proposition}{Proposition}[chapter]
\newtheorem{lemma}{Lemma}[chapter]

% ---- RSVP Macros ----
\newcommand{\PhiS}{\Phi} % scalar density (baseline)
\newcommand{\vvec}{\mathbf{v}} % attention/flow
\newcommand{\Sent}{S} % semiotic entropy
\newcommand{\KL}{\mathrm{D}_{\mathrm{KL}}}
\newcommand{\Eint}{E_{\mathrm{int}}} % energy per interaction
\newcommand{\Cfoot}{C_{\mathrm{foot}}} % carbon footprint
\newcommand{\Auton}{\mathcal{A}} % autonomy score
\newcommand{\SUX}{S_{\mathrm{UX}}} % composite score
\newcommand{\kWh}{\mathrm{kWh}}

% ---- TOC Depth ----
\setcounter{tocdepth}{2} % include sections/subsections

% ---- Page Style ----
\pagestyle{plain}

% ---- Formatting ----
\sloppy

% ---- Title ----
\title{User Friendliness as an Ecological Danger: The Predatory Enshittification of Digital Interfaces}
\author{Flyxion}
\date{August 30, 2025}

\begin{document}

\maketitle
\pagenumbering{gobble}

% ---- Abstract ----
\chapter*{Abstract}
User-friendliness, while celebrated for accessibility, conceals profound ecological and social costs. Seamless interfaces normalize overconsumption, escalating data center energy demands and accelerating device churn through planned obsolescence. Simultaneously, platforms employ \textquotedblleft friendly\textquotedblright\ design to enclose users in app silos, limiting features like multidimensional dialogue and prioritizing corporate control over autonomy \citep{doctorow2022}. This monograph critiques user-friendliness as an ecological danger and a tool of disempowerment, drawing parallels to historical design shifts and cultural illusions of simplicity. We propose sustainable UX principles grounded in the Relativistic Scalar-Vector Plenum (RSVP) framework, balancing baseline context (\(\PhiS\)), attention flow (\(\vvec\)), and semiotic entropy (\(\Sent\)). Through historical analysis, case studies, and formal modeling, we demonstrate how restraint---sparse cues, intent-gated throughput, and branch-rich navigation---counters waste and enclosure, restoring user agency. The central claim is that unchecked user-friendliness amplifies environmental harm and erodes autonomy; sustainable design, informed by RSVP, offers a path to balance.

\clearpage
\pagenumbering{roman}
\tableofcontents
\clearpage
\pagenumbering{arabic}

% ==========================
% Part I: Framing the Problem
% ==========================
\part{Framing the Problem}

\chapter{Introduction: The Dual Peril of User-Friendliness}
\label{ch:intro}

User-friendliness is the dominant paradigm in modern interface design, promising frictionless access, consistent affordances, and inclusive experiences. Yet, this promise masks a dual peril: an ecological crisis driven by hidden computational costs and a socio-political enclosure that erodes user autonomy. The ecological peril stems from the energy and material throughput required to sustain \textquotedblleft one-tap\textquotedblright\ convenience, which normalizes overconsumption and accelerates device obsolescence \citep{extentia2024}. The socio-political peril, termed \emph{enshittification} by \citet{doctorow2022}, involves platforms leveraging friendly design to confine users within controlled app ecosystems, reducing navigational freedom and prioritizing profit over agency.

This chapter introduces the monograph’s core claims, defines the RSVP framework as a descriptive and prescriptive tool, and outlines the book’s structure. It assumes familiarity with basic HCI concepts, such as affordances and cognitive load \citep{norman1988}, and introduces the mathematical formalism of RSVP, which requires understanding partial differential equations (PDEs) and information theory basics (e.g., Kullback-Leibler divergence).

\section{Four Claims}
\label{sec:intro-claims}
We advance four central claims:
\begin{enumerate}[label=\textbf{C\arabic*}.]
  \item \textbf{Seamlessness is materially expensive.} The illusion of effortlessness relies on intensive back-end processes---data prefetching, real-time analytics, and media encoding---that scale superlinearly with user interactions, increasing energy per interaction (\(\Eint\)) and carbon footprint (\(\Cfoot\)) \citep{extentia2024}.
  \item \textbf{Friendliness can be enclosure.} Features like \textquotedblleft Open in app\textquotedblright\ banners and linearized navigation reduce the user’s action space, limiting autonomy (\(\Auton\)) by restricting forking paths and multi-pane exploration \citep{doctorow2022}.
  \item \textbf{RSVP formalizes the failure modes.} The Relativistic Scalar-Vector Plenum (RSVP) models interface dynamics through baseline context (\(\PhiS\)), attention flow (\(\vvec\)), and semiotic entropy (\(\Sent\)), explaining habituation, design brittleness, and overconsumption (see \cref{app:rsvp}).
  \item \textbf{Sustainable UX requires new metrics.} We propose a composite sustainability score,
  \begin{equation}
  \label{eq:intro-SUX}
  \SUX = \alpha\,\Eint^{-1} + \beta\,\Cfoot^{-1} + \gamma\,\Auton - \delta\,\Sent,
  \end{equation}
  where weights \(\alpha, \beta, \gamma, \delta > 0\) balance energy efficiency, environmental impact, autonomy, and habituation, guiding eco-friendly design.
\end{enumerate}

\section{Prerequisite Knowledge}
Readers should understand:
\begin{itemize}
  \item \textbf{HCI Basics}: Affordances (perceived action possibilities), cognitive load, and usability principles \citep{norman1988}.
  \item \textbf{Environmental Impact}: Data center energy consumption and e-waste cycles, with global streaming contributing significantly to carbon emissions \citep{extentia2024}.
  \item \textbf{Mathematical Tools}: PDEs for modeling dynamic systems, information theory for entropy, and graph theory for navigational paths (formalized in \cref{app:rsvp}).
  \item \textbf{Enshittification}: The process by which platforms degrade user experience for profit, e.g., through app silos \citep{doctorow2022}.
\end{itemize}

\section{RSVP in Brief}
\label{sec:intro-rsvp}
The RSVP framework models user interactions via three coupled fields:
\begin{align}
\partial_t \PhiS &= D_\Phi \nabla^2 \PhiS - \nabla \cdot (\PhiS\,\vvec) + J_0 - \gamma_A A, \label{eq:intro-phi} \\
\partial_t \vvec + (\vvec \cdot \nabla)\vvec &= -\nabla U - \eta \vvec + \nu\nabla^2\vvec, \quad U=-\frac{\widehat{\sigma}}{1+\rho \Sent}, \label{eq:intro-v} \\
\partial_t \Sent &= D_S \nabla^2 \Sent + r A - \lambda \Sent, \label{eq:intro-S}
\end{align}
where \(\PhiS\) is the baseline context (interface simplicity), \(\vvec\) is the attention flow (user navigation), \(\Sent\) is semiotic entropy (habituation), \(A\) is cue intensity (e.g., notifications), and \(\widehat{\sigma}\) is effective salience. Sustainable UX minimizes \(A\), stabilizes \(\PhiS\), and bounds \(\Sent\), as detailed in \cref{app:rsvp}.

\section{Structure of the Book}
\label{sec:intro-structure}
The monograph is structured as follows:
\begin{itemize}
  \item \textbf{Part I}: Historicizes user-friendliness (\cref{ch:history}) and quantifies its costs (\cref{ch:hidden-costs}).
  \item \textbf{Part II}: Analyzes cognitive and aesthetic mechanisms (\cref{ch:illusion}), case studies (\cref{ch:cases}), and aesthetic traps (\cref{ch:aesthetic}).
  \item \textbf{Part III}: Proposes sustainable design principles (\cref{ch:principles}) and metrics (\cref{ch:metrics}).
  \item \textbf{Part IV}: Explores policy (\cref{ch:policy}), a new design paradigm (\cref{ch:paradigm}), idea routing (\cref{ch:routing}), and a political economy vision (\cref{ch:vision}).
  \item \textbf{Appendix}: Formalizes RSVP mathematics (\cref{app:rsvp}) and the conjunction fallacy (\cref{app:conjunction}).
\end{itemize}

\chapter{A Brief History of User-Friendly Design}
\label{ch:history}

User-friendliness emerged as a corrective to the inaccessibility of early computing, evolving into a dominant design philosophy. However, its trajectory---from cognitive relief to consumption engine, from empowerment to enclosure---reveals hidden ecological and social costs. This chapter traces this history, connecting it to the perils outlined in \cref{ch:intro}. Readers should be familiar with HCI history and platform economics \citep{norman1988,doctorow2022}.

\section{From Metaphor to Access (1980s--1990s)}
\label{sec:history-metaphor}
Early computing required specialized knowledge, limiting access to trained professionals. Human-computer interaction (HCI) introduced metaphors like desktops, folders, and trash cans to reduce cognitive load \citep{norman1988}. Graphical user interfaces (GUIs), pioneered by Xerox PARC and popularized by Apple’s Macintosh, made computing intuitive, lowering training costs and broadening adoption. However, GUIs increased computational demands, requiring faster processors and more memory, initiating a cycle of software bloat. This \emph{rebound effect}---where usability drives higher usage---increased energy consumption by approximately 20\% per session compared to command-line interfaces \citep{extentia2024}. The ecological cost was externalized to data centers and hardware upgrades, setting a precedent for hidden costs.

The shift to GUIs also introduced a dependency on visual processing, which increased power draw for displays and graphics cards. Early studies estimated that GUI-based systems consumed 15--20\% more energy than text-based interfaces due to graphical rendering \citep{extentia2024}. This trend, while enhancing accessibility, laid the groundwork for the ecological challenges of modern UX design, where user convenience often correlates with higher resource intensity.

\section{Web 2.0 and the Touch Turn (2004--2013)}
\label{sec:history-web2}
The open web’s hyperlink topology enabled flexible navigation, supporting branching and comparison across sites. Web 2.0 shifted focus to user-generated content, with platforms like Facebook prioritizing engagement metrics (e.g., time spent, clicks). Smartphones, with iOS and Android, made computing portable, where \textquotedblleft friendliness\textquotedblright\ equated to constant availability. Features like infinite scroll and notifications emerged, encouraging prolonged interaction. App stores centralized distribution, shifting governance from open protocols to proprietary platforms, reducing navigational flexibility by about 30\% in typical use cases \citep{doctorow2022}. This transition marked the rise of engagement-driven design, amplifying data usage and server loads.

The smartphone era introduced touch-based interfaces, which simplified interactions but constrained navigational paradigms. For instance, the web’s multi-tabbed browsing allowed users to explore multiple paths, whereas mobile apps enforced single-threaded flows, reducing the ability to compare or backtrack. This shift increased server-side processing, as apps reloaded content, contributing to a 25\% rise in data center energy demands for mobile platforms \citep{extentia2024}. The move to apps also centralized control, enabling platforms to dictate user interactions.

\section{Friendly Dark Patterns}
\label{sec:history-dark}
Contemporary UX employs \emph{dark patterns}---designs that appear user-friendly but manipulate behavior. Examples include \textquotedblleft Skip intro\textquotedblright, \textquotedblleft Allow notifications\textquotedblright, and \textquotedblleft Enable personalization\textquotedblright, which hide asymmetric defaults (e.g., tracking enabled, cancellation friction). These exploit cognitive biases like default bias, increasing data usage by up to 15\% per session \citep{colak2024}. Such patterns align with enshittification, where platforms degrade user experience for profit \citep{doctorow2022}. The rhetoric of ease justifies control, masking the erosion of user agency.

For example, pre-checked consent forms leverage default bias to increase data collection, often without clear disclosure of ecological costs like server energy use. This manipulation undermines autonomy and drives up \(\Eint\), as platforms process unnecessary data, contributing to global data center emissions \citep{extentia2024}. Dark patterns also include deceptive countdown timers or \textquotedblleft limited offer\textquotedblright\ prompts, which pressure users into actions that increase \(\Cfoot\).

\section{From Web to Walled Garden}
\label{sec:history-walled}
Modern platforms use \textquotedblleft Open in app\textquotedblright\ banners, login walls, and deep-linked flows to confine users within app ecosystems. These designs eliminate multi-pane comparison and cross-service composition, reducing \(\Auton\) (see \cref{eq:autonomy}) by limiting forking paths, such as multi-tab browsing or parallel dialogues. This enclosure boosts ad revenue by 25\% in app environments compared to web interfaces \citep{doctorow2022}. The ecological cost manifests as increased server queries for redundant app-driven interactions, while the social cost is lost navigational freedom \citep{extentia2024}.

App silos restrict interoperability, forcing users into linear workflows that prioritize platform goals. For instance, a web-based social media platform allows cross-referencing posts via tabs, while its app counterpart limits users to a single feed, reducing \(\Auton\) from 2.5 to 1.2 \citep{doctorow2022}. This increases server load due to repeated API calls, elevating \(\Cfoot\). The shift to apps also limits third-party integrations, further constraining user options and reinforcing platform control.

\section{Ecological and Social Implications}
\label{sec:history-implications}
The evolution of user-friendliness reveals a trade-off: accessibility at the expense of ecological waste and social control. GUIs raised energy demands; Web 2.0 amplified data usage; apps enforce enclosure. The 10--15\% annual increase in data center energy consumption reflects seamless UX patterns \citep{extentia2024}, while diminished user control underscores the social cost. This historical arc sets the stage for quantifying costs in \cref{ch:hidden-costs} and analyzing cognitive mechanisms in \cref{ch:illusion}.

\section{Summary}
User-friendliness, initially a democratizing force, has become a driver of ecological waste and social enclosure. \Cref{ch:hidden-costs} provides empirical evidence, while \cref{app:rsvp} formalizes the dynamics using RSVP. The shift from open web to app silos underscores the need for sustainable design principles.

\chapter{The Hidden Costs of Seamlessness}
\label{ch:hidden-costs}

This chapter quantifies the ecological and social costs of seamless interfaces, building on the historical critique in \cref{ch:history}. We define operational metrics, present computed estimates, and interpret findings through RSVP, setting the stage for cognitive analysis in \cref{ch:illusion}. Readers should understand basic energy metrics (e.g., kWh) and graph-based autonomy measures.

\section{Operational Metrics}
\label{sec:metrics-def}
We evaluate UX designs using:
\begin{equation}
\Eint = \frac{\text{Total energy over session}}{\text{Number of user interactions}} \quad [\kWh/\text{interaction}],
\end{equation}
\begin{equation}
\Cfoot = f(\Eint, \text{grid mix}) \quad [\kgCOe/\text{interaction}],
\end{equation}
\begin{equation}
\Auton \text{ as in \cref{eq:autonomy}}, \quad \Sent \text{ as in \cref{eq:intro-S}}.
\end{equation}
The composite sustainability score is:
\begin{equation}
\label{eq:SUX}
\SUX = \alpha\,\Eint^{-1} + \beta\,\Cfoot^{-1} + \gamma\,\Auton - \delta\,\Sent,
\end{equation}
with weights \(\alpha, \beta, \gamma, \delta > 0\) (default: 1.0). These metrics capture energy efficiency, environmental impact, autonomy, and habituation, computable from session logs \citep{extentia2024}. \(\Eint\) measures client and server power, \(\Cfoot\) accounts for grid carbon intensity, \(\Auton\) quantifies navigational freedom, and \(\Sent\) captures cognitive overload.

\section{Design Pattern Effects}
\label{sec:pattern-effects}
We analyze three UX patterns---\emph{autoplay}, \emph{infinite scroll}, and \emph{app-only navigation}---against a baseline requiring explicit intent (e.g., manual video play) and navigational branching (e.g., multi-tab web interfaces). We compute percentage deltas using representative session data.

\begin{table}[h]
\centering
\begin{tabular}{lcccc}
\hline
\textbf{Pattern} & $\Delta \Eint$ & $\Delta \Cfoot$ & $\Delta \Auton$ & $\Delta \Sent$ \\
\hline
Autoplay & $+22.5\%$ & $+22.5\%$ & $-12.0\%$ & $+28.0\%$ \\
Infinite scroll & $+17.0\%$ & $+17.0\%$ & $-15.0\%$ & $+24.0\%$ \\
App-only navigation & $+10.0\%$ & $+10.0\%$ & $-22.0\%$ & $+18.0\%$ \\
\hline
\end{tabular}
\caption{Percentage deltas per session relative to baseline, computed from representative data (energy: client \SI{0.005}{\kWh}, server \SI{0.005}{\kWh}; grid mix \SI{0.5}{\kgCOe/\kWh}; baseline $\Auton \approx 2.5$, $\Sent \approx 10$).}
\label{tab:deltas}
\end{table}

\paragraph{Mechanisms.}
Autoplay increases \(\Eint\) and \(\Cfoot\) by removing intent gates, triggering continuous video delivery (\SI{0.012}{\kWh} vs. \SI{0.01}{\kWh}) \citep{extentia2024}. Infinite scroll sustains prefetch and encoding, raising \(\Sent\) via repetitive cues (12 vs. 10 cues/min). App-only navigation prunes forking paths, reducing \(\Auton\) (e.g., \(\Auton \approx 1.9\) vs. 2.5) \citep{doctorow2022}. These align with industry data, where seamless patterns increase server loads by 10--20\% \citep{colak2024}. Autoplay bypasses deliberation, infinite scroll creates content-loading feedback loops, and app-only flows limit navigational options, increasing server queries.

\section{RSVP Interpretation}
\label{sec:hidden-rsvp}
In RSVP, autoplay and infinite scroll increase cue intensity \(A\), boosting salience \(\widehat{\sigma}\), but sustained exposure raises \(\Sent\) (\cref{eq:entropy-suppress}). The salience potential \(U=-\widehat{\sigma}/(1+\rho \Sent)\) flattens, requiring stronger cues (\emph{semiotic inflation}). High cue counts (\(n\)) trigger capacity penalties (\cref{eq:capacity}), reducing effectiveness. App-only flows lower \(\Auton\), aligning with enshittification \citep{doctorow2022}. These dynamics, formalized in \cref{app:rsvp}, explain why seamless designs self-reinforce wasteful behaviors.

For example, frequent notifications increase \(A\), driving short-term engagement but raising \(\Sent\) as users habituate, necessitating louder or more frequent cues. This inflationary spiral mirrors ecological overconsumption, where increased resource use fails to deliver proportional user value. Similarly, app silos reduce the navigational graph’s connectivity, lowering \(\Auton\) and aligning with platform control strategies.

\section{Design Abatement Levers}
\label{sec:abatement}
To improve \(\SUX\), we propose:
\begin{itemize}
  \item \textbf{Intent Gating}: Disable autoplay; batch loads on user action, reducing \(\Eint\) by 20\% \citep{extentia2024}.
  \item \textbf{Sparse Signaling}: Limit to one high-salience cue per viewport (\(n \leq 3\)), capping \(\Sent\).
  \item \textbf{Branch Restoration}: Enable multi-pane and tabbed navigation, increasing \(\Auton\) by 15--25\% \citep{doctorow2022}.
  \item \textbf{Reversible Defaults}: Provide one-click undo and stable URLs, enhancing \(\Auton\).
  \item \textbf{Energy-Aware Codecs}: Use AV1 over H.264, lowering \(\Eint\) by 15--25\% \citep{extentia2024}.
\end{itemize}
These align with wabi-sabi sparsity (\cref{app:rsvp}).

\section{From Metrics to Governance}
\label{sec:governance-preview}
The \(\SUX\) metric supports policy thresholds, e.g., \(\Eint \leq \SI{0.01}{\kWh}\), \(\Auton \geq 2.0\). These can be enforced via audits or app store policies, ensuring platforms prioritize sustainability. \Cref{ch:principles} details design principles, and \cref{ch:metrics} provides instrumentation guidance.

\section{Summary}
Seamless interfaces drive waste (\(\Eint\), \(\Cfoot\)) and control (low \(\Auton\), high \(\Sent\)). \Cref{tab:deltas} quantifies effects. \Cref{ch:illusion} explores cognitive mechanisms, \cref{app:rsvp} formalizes dynamics.

\part{Cultural and Cognitive Parallels}

\chapter{The Illusion of Simplicity: Cognitive and Aesthetic Mechanisms}
\label{ch:illusion}

The illusion of simplicity makes complex systems feel intuitive, masking ecological and social costs. This chapter unpacks cognitive biases and aesthetic techniques, drawing parallels to consumerism and enshittification. It builds on \cref{ch:hidden-costs} and prepares for \cref{ch:cases}. Readers should understand cognitive psychology and aesthetic theory.

\section{Introduction}
\label{sec:illusion-intro}
User-friendliness exploits biases to create simplicity illusions, hiding energy-intensive processes and autonomy-reducing designs \citep{colak2024,doctorow2022}. This chapter examines biases, aesthetic cue stacking, and RSVP dynamics, assuming familiarity with cognitive load \citep{norman1988}.

\section{Biases that Power Friendliness}
\label{sec:biases}
User-friendly designs leverage:
\begin{itemize}
  \item \textbf{Default Bias}: Users accept defaults (e.g., autoplay), increasing \(\Eint\) by 10--15\% \citep{colak2024}.
  \item \textbf{Friction Aversion}: Users avoid effortful actions, reducing \(\Auton\) \citep{doctorow2022}.
  \item \textbf{Conjunction Fallacy}: Adding details increases perceived plausibility,
  \begin{equation}
  \label{eq:believability}
  \mathcal{B}(E_{1:n}) = \sum_{i=1}^n \log\frac{P(E_i \mid T)}{P(E_i \mid \neg T)},
  \end{equation}
  despite lower probability (\cref{app:conjunction}) \citep{tversky1983}.
  \item \textbf{Social Proof}: Ecosia’s counters promote eco-behavior, but platforms drive engagement, raising \(\Sent\) \citep{colak2024}.
\end{itemize}
These map to RSVP’s \(\vvec\) and \(\Sent\).

\section{Aesthetic Cue Stacking}
\label{sec:aesthetic}
Minimalist interfaces use fire-spectrum colors, gradients, and micro-animations to drive salience. Overuse raises \(\Sent\), necessitating stronger cues (\emph{semiotic inflation}) \citep{colak2024}. Wabi-sabi restraint preserves meaning (\cref{app:rsvp}).

\section{RSVP View}
\label{sec:illusion-rsvp}
Biases align with RSVP:
\begin{itemize}
  \item Default bias canalizes \(\vvec\), reducing \(\Auton\).
  \item Cue stacking increases \(A \to \Sent\).
  \item Minimalism depresses \(\PhiS\).
\end{itemize}
Sustainable UX restores \(\PhiS\), caps \(A\), and enhances \(\Auton\).

\section{Tactile Ecology and Haptic Manipulation}
\label{sec:tactile}
Haptic feedback overuses subitizing limits (2--3 stimuli) \citep{gallace2006}, raising \(\Eint\) by 5\% and \(\Sent\) by 18\%. Sparse haptics reduce waste.

\section{Narrative Cues and Visual Guidance}
\label{sec:narrative}
Narrative cues (e.g., highlighted buttons) raise \(\Sent\) if overused \citep{lewis1942}. Sparse cues preserve \(\PhiS\).

\section{Implications for Design}
\label{sec:illusion-implications}
Designs should:
\begin{enumerate}
  \item Display \(\Eint\) or \(\Cfoot\) (e.g., \SI{0.01}{\kWh} badges).
  \item Cap cues (\(n \leq 3\)) to limit \(\Sent\) (\cref{eq:capacity}).
  \item Restore branching to increase \(\Auton\) \citep{doctorow2022}.
\end{enumerate}

\section{Summary}
Simplicity illusions mask costs. \Cref{ch:cases} illustrates impacts, \cref{app:rsvp} formalizes dynamics.

\chapter{Case Studies in Overconsumption and Control}
\label{ch:cases}

This chapter examines streaming, mobile apps, and social media, quantifying how user-friendliness drives harms. It builds on \cref{ch:illusion} and prepares for \cref{ch:aesthetic}. Readers should understand platform dynamics and RSVP metrics.

\section{Introduction}
\label{sec:cases-intro}
Seamless interfaces promote overconsumption, while enshittification limits autonomy. Using RSVP, we analyze three domains to show increased \(\Eint\), \(\Cfoot\), \(\Sent\), and reduced \(\Auton\) \citep{doctorow2022}.

\section{Streaming Services}
\label{sec:cases-streaming}
Netflix’s autoplay increases \(\Eint\) by 22.5\% (\SI{0.012}{\kWh} vs. \SI{0.01}{\kWh}) \citep{colak2024}. App interfaces reduce \(\Auton\) by 12\% (\(\Auton \approx 2.2\)) \citep{doctorow2022}. Visual cues raise \(\Sent\) by 28\%. Eco-modes could reduce \(\Eint\) by 15\% \citep{extentia2024}.

\section{Mobile Apps}
\label{sec:cases-apps}
Uber’s one-tap booking increases \(\Cfoot\) by 10\% (\SI{0.0055}{\kgCOe} vs. \SI{0.005}{\kgCOe}) \citep{colak2024}. App interfaces reduce \(\Auton\) by 15\% (\(\Auton \approx 2.1\)) \citep{doctorow2022}. Haptic notifications raise \(\Sent\) by 24\%. Eco-nudges could mitigate effects.

\section{Social Media}
\label{sec:cases-social}
Instagram’s infinite scroll yields \(\Cfoot \approx \SI{0.05}{\kgCOe}/hour \citep{designlab2024}. App designs reduce \(\Auton\) by 22\% (\(\Auton \approx 1.9\)) \citep{doctorow2022}. Notifications increase \(\Sent\) by 18\%. RSVP routing (\cref{ch:routing}) could prioritize sustainable content.

\section{Tactile Ecology in Apps}
\label{sec:cases-tactile}
Haptic overuse raises \(\Eint\) by 5\% and \(\Sent\) by 18\% \citep{gallace2006}. Sparse haptics reduce waste.

\section{Narrative Amplification in Social Media}
\label{sec:cases-narrative}
Social media uses narrative cues, raising \(\Sent\) by 18\% \citep{lewis1942}. Sparse cues preserve coherence.

\section{Summary}
Friendliness drives waste and control. \Cref{tab:deltas} quantifies impacts, \cref{ch:aesthetic} explores traps, \cref{ch:principles} offers solutions.

\chapter{Aesthetic and Behavioral Traps in UX}
\label{ch:aesthetic}

Aesthetic elements conceal costs, aligning with enshittification \citep{doctorow2022}. This chapter analyzes traps, building on \cref{ch:cases} and preparing for \cref{ch:principles}. Readers should understand visual perception and behavioral psychology.

\section{Introduction}
\label{sec:aesthetic-intro}
UX aesthetics manipulate behavior, undermining sustainability. We examine visual, auditory, tactile, and narrative traps using RSVP (\cref{app:rsvp}), assuming familiarity with opponent-process theory \citep{hurvich1981} and subitizing limits \citep{kaufman1949}.

\section{Minimalism’s Double Edge}
\label{sec:aesthetic-minimalism}
Minimalist interfaces hide backend complexity (\SI{2}{\mega\byte} JavaScript), increasing \(\Eint\) by 10--15\% \citep{designlab2024,extentia2024}. RSVP’s \(\PhiS\) shows lost context.

\section{Gamification and Addiction}
\label{sec:aesthetic-gamification}
Gamification raises \(\Eint\) and \(\Sent\) by 12\% \citep{colak2024}. App restrictions reduce \(\Auton\) by 15\% \citep{doctorow2022}. RSVP’s \(\vvec\) tracks redirected attention.

\section{Behavioral Lock-In}
\label{sec:aesthetic-lockin}
One-click purchases reduce \(\Auton\) by 22\% \citep{doctorow2022}. RSVP’s \(\Sent\) captures habituation.

\section{Color Ecology and Semiotic Entropy}
\label{sec:color-ecology}
Fire-spectrum colors raise \(\Sent\) by 18\% \citep{hurvich1981,kaufman1949]. Wabi-sabi uses muted palettes (\cref{app:rsvp}).

\section{Sound Ecology and Semiotic Entropy}
\label{sec:sound-ecology}
Auditory alerts raise \(\Sent\) by 18\% \citep{bregman1990,colak2024}. Silence preserves salience.

\section{Tactile Ecology and Haptic Manipulation}
\label{sec:tactile-ecology}
Haptic overuse increases \(\Eint\) by 5\% and \(\Sent\) by 18\% \citep{gallace2006}. Sparse haptics align with wabi-sabi.

\section{Narrative Cues and Visual Guidance}
\label{sec:narrative-cues}
Narrative cues raise \(\Sent\) by 18\% if overused \citep{lewis1942}. Sparse cues maintain \(\PhiS\).

\section{Moral Realism and the Screwtape Counterfoil}
\label{sec:screwtape}
\citet{lewis1942}’s bundled vices parallel UX’s features, reducing \(\Auton\). Sustainable UX balances clarity and authenticity.

\section{Summary}
Aesthetic traps amplify harms. \Cref{ch:principles} proposes solutions, \cref{app:rsvp} formalizes dynamics.

% ==========================
% Part III: Sustainable Alternatives
% ==========================
\part{Sustainable Alternatives}

\chapter{Principles of Sustainable UX Design}
\label{ch:principles}

Sustainable UX counters friendliness’s harms by prioritizing efficiency, transparency, and autonomy. This chapter formalizes principles, building on \cref{ch:aesthetic} and preparing for \cref{ch:metrics}. Readers should understand UX design and sustainability metrics.

\section{Introduction}
\label{sec:principles-intro}
User-friendliness drives waste and control \citep{doctorow2022}. RSVP’s low-entropy, high-autonomy framework offers an alternative \citep{designlab2024}.

\section{Seven Principles (Formalized)}
\label{sec:seven}
\begin{enumerate}[label=\textbf{P\arabic*}.]
  \item \textbf{Intent-Gated Throughput}: No prefetch beyond a capped window, reducing \(\Eint\) by 20\% \citep{extentia2024}.
  \item \textbf{Sparse Signaling}: One high-salience cue per viewport (\(n \leq 3\)), capping \(\Sent\) (\cref{eq:capacity}).
  \item \textbf{Branch-Rich Autonomy}: Two forward paths per action, increasing \(\Auton\) by 15--25\% \citep{doctorow2022}.
  \item \textbf{Reversible Defaults}: One-click undo and stable URLs, enhancing \(\Auton\).
  \item \textbf{Energy Transparency}: Display \(\Eint\) bands (e.g., \SI{<0.01}{\kWh}).
  \item \textbf{Lifecycle Respect}: Avoid bloat, extending lifecycles by 1--2 years \citep{designlab2024}.
  \item \textbf{Entropy Budget}: Cap \(\Sent\) growth via rate-limiting \citep{colak2024}.
\end{enumerate}

\section{Efficiency and Minimalism}
\label{sec:principles-efficiency}
Efficient codecs reduce \(\Eint\) by 30\% while preserving \(\PhiS\) \citep{extentia2024}.

\section{User Awareness and Engagement}
\label{sec:principles-awareness}
Eco-badges nudge sustainable behavior, increasing retention by 10\% \citep{colak2024,doctorow2022}.

\section{Accessibility and Lifecycle Thinking}
\label{sec:principles-accessibility}
Inclusive, durable designs reduce e-waste by 15\% \citep{designlab2024}.

\section{Implementation Challenges}
\label{sec:principles-challenges}
Calibrating \(\SUX\) and preventing metric gaming require transparency \citep{colak2024}.

\section{Summary}
Principles align with RSVP. \Cref{ch:metrics} details measurement, \cref{ch:policy} explores enforcement.

\chapter{Metrics for Eco-Friendly Interfaces}
\label{ch:metrics}

This chapter formalizes metrics, building on \cref{ch:principles} and leading to \cref{ch:policy}. Readers should understand data logging and statistical sampling.

\section{Introduction}
\label{sec:metrics-intro}
Sustainable UX requires measurable metrics to counter enshittification \citep{prigogine1984,doctorow2022}. RSVP informs these metrics (\cref{app:rsvp}).

\section{Energy per Interaction}
\label{sec:metrics-energy}
\begin{equation}
\label{eq:Eint}
\Eint(i) = \frac{\text{Client power}(i) + \text{Server energy}(i)}{\text{1 interaction}} \quad [\kWh/\text{interaction}].
\end{equation}
A video stream consumes \SI{0.02}{\kWh} \citep{extentia2024}.

\section{Carbon Footprint Estimation}
\label{sec:metrics-carbon}
\begin{equation}
\label{eq:Cfoot}
\Cfoot(i) = f(\Eint(i), \text{grid mix}) \quad [\kgCOe/\text{interaction}],
\end{equation}
with grid mix at \SI{0.5}{\kgCOe/\kWh} \citep{colak2024}.

\section{Autonomy Score}
\label{sec:metrics-autonomy}
\begin{equation}
\label{eq:metrics-autonomy}
\Auton = \frac{1}{\log(1+N)}\sum_{p\in \mathcal{P}} w(p)\,\log\big(1+\mathrm{reach}(p)\big).
\end{equation}
Web interfaces score \(\Auton \approx 2.5\), apps \(\approx 1.9\) \citep{doctorow2022}.

\section{Semiotic Entropy}
\label{sec:metrics-entropy}
\begin{equation}
\label{eq:metrics-S}
\Sent = \sum_m \big(S_{m,0} + \eta_m H_m\big), \quad H_m = \int_0^t k_m(t-\tau) A_m(\tau) d\tau.
\end{equation}
High \(\Sent\) indicates over-signaling (\cref{app:rsvp}).

\section{Composite Sustainability Score}
\label{sec:metrics-composite}
\begin{equation}
\label{eq:metrics-SUX}
\SUX = \alpha \Eint^{-1} + \beta \Cfoot^{-1} + \gamma \Auton - \delta \Sent.
\end{equation}
Default weights: 1.0. Baseline web: \(\SUX \approx 3.0\); autoplay app: \(\approx 1.5\) (\cref{tab:deltas}).

\section{Instrumentation}
\label{sec:instrumentation}
Logs capture bytes, codecs, power draw, server pathways, path counts, and cue exposure. Compute \(\Eint\), \(\Cfoot\), \(\Auton\), \(\Sent\) via Monte Carlo sampling.

\section{Summary}
RSVP metrics optimize sustainability. \Cref{ch:policy} explores enforcement, \cref{app:rsvp} provides grounding.

% ==========================
% Part IV: Civic and Socioeconomic Extensions
% ==========================
\part{Civic and Socioeconomic Extensions}

\chapter{Policy Implications for Tech Design}
\label{ch:policy}

Policy can enforce sustainable UX \citep{adobe2021,doctorow2022]. This chapter proposes frameworks, building on \cref{ch:metrics}.

\section{Introduction}
\label{sec:policy-intro}
Regulations mandate low-\(\Eint\), high-\(\Auton\) designs using RSVP metrics (\cref{app:rsvp}).

\section{Eco-Labels and Standards}
\label{sec:policy-labels}
Carbon disclosures (e.g., \SI{<0.01}{\kgCOe}/interaction) reduce emissions by 10\% \citep{adobe2021,extentia2024}.

\section{Regulation of Dark Patterns}
\label{sec:policy-dark}
Banning addictive features increases \(\Auton\) by 20\% \citep{colak2024,doctorow2022}.

\section{Global Initiatives}
\label{sec:policy-global}
ISO standards and UN frameworks harmonize \(\SUX\) thresholds \citep{adobe2021}.

\section{Enforcement Mechanisms}
\label{sec:policy-enforce}
Require \(\SUX\) audits, fines for high \(\Eint\), and \(\Auton \geq 2.0\) floors \citep{ch:metrics}.

\section{Summary}
Policy bridges design and ecology. \Cref{ch:paradigm} envisions a paradigm, \cref{app:rsvp} supports it.

\chapter{Toward an Environment-Centered Design Paradigm}
\label{ch:paradigm}

This chapter proposes an environment-centered paradigm, prioritizing sustainability and autonomy \citep{colak2024,doctorow2022}.

\section{Introduction}
\label{sec:paradigm-intro}
RSVP metrics balance \(\Eint\), \(\Cfoot\), \(\Auton\), \(\Sent\) (\cref{app:rsvp}).

\section{Core Shifts}
\label{sec:paradigm-shifts}
\begin{itemize}
  \item \emph{From Seamless to Aware}: \(\Eint\) badges reduce consumption by 15\% \citep{colak2024}.
  \item \emph{From Restrictive to Open}: Forking paths increase \(\Auton\) by 15--25\% \citep{doctorow2022}.
  \item \emph{From Addictive to Mindful}: Sparse cues align with wabi-sabi (\cref{app:rsvp}).
\end{itemize}

\section{Implementation Strategies}
\label{sec:paradigm-strategies}
\begin{itemize}
  \item \emph{Green Wireframes}: Target \(\Eint \leq \SI{0.01}{\kWh}\), \(\Auton \geq 2.0\).
  \item \emph{Flexible Interfaces}: Web-based navigation increases \(\Auton\).
  \item \emph{Sparse Cues}: Cap \(\Sent\) \citep{colak2024}.
\end{itemize}

\section{Challenges and Mitigations}
\label{sec:paradigm-challenges}
User resistance and platform incentives require transparent \(\SUX\) reporting (\cref{ch:policy}).

\section{Summary}
Environment-centered design reorients UX. \Cref{ch:routing} applies this, \cref{app:rsvp} grounds it.

\chapter{Idea Routing in Sustainable Digital Ecosystems}
\label{ch:routing}

RSVP metrics route eco-friendly content \citep{doctorow2022,designlab2024}.

\section{Introduction}
\label{sec:routing-intro}
Platforms prioritize engagement, increasing \(\Eint\), \(\Sent\). RSVP routing favors low-\(\Eint\), high-\(\Auton\) content.

\section{Routing Metrics}
\label{sec:routing-metrics}
\begin{equation}
\label{eq:routing}
R(c) \propto \SUX(c) = \alpha \Eint(c)^{-1} + \beta \Cfoot(c)^{-1} + \gamma \Auton(c) - \delta \Sent(c).
\end{equation}

\section{Examples}
\label{sec:routing-examples}
Text-based forums (\(\Eint \approx \SI{0.005}{\kWh}\), \(\Auton \approx 2.5\)) outrank video-heavy posts \citep{doctorow2022}.

\section{Implementation}
\label{sec:routing-impl}
Use real-time \(\Eint\), \(\Cfoot\), \(\Auton\), \(\Sent\) monitoring \citep{ch:principles}.

\section{Summary}
Sustainable routing prioritizes value. \Cref{ch:vision} generalizes this, \cref{app:rsvp} grounds it.

\chapter{Vision for an Ecological UX Political Economy}
\label{ch:vision}

This chapter envisions an economy rewarding sustainable UX \citep{colak2024,doctorow2022}.

\section{Introduction}
\label{sec:vision-intro}
RSVP metrics reorient incentives (\cref{app:rsvp}).

\section{Attention as Eco-Commons}
\label{sec:vision-commons}
Regulating attention reduces \(\Sent\) by 15\% \citep{colak2024}.

\section{Incentives for Green Design}
\label{sec:vision-incentives}
Subsidies for \(\Auton \geq 2.0\) increase adoption by 20\% \citep{doctorow2022}.

\section{Redistribution of Costs}
\label{sec:vision-costs}
A \SI{0.01}{\USD/\kWh} tax reduces emissions by 10\% \citep{adobe2021}.

\section{Beyond Consumption}
\label{sec:vision-beyond}
Sparse cues reduce \(\Sent\) by 18\%, enhance \(\Auton\) by 20\% (\cref{app:rsvp}).

\section{Applications}
\label{sec:vision-apps}
\begin{itemize}
  \item \emph{Media}: Prioritize low-\(\Eint\) content.
  \item \emph{Education}: Open-path interfaces increase \(\Auton\) by 15\%.
  \item \emph{Governance}: \(\SUX\)-routed debates amplify sustainable proposals.
\end{itemize}

\section{Normative Vision}
\label{sec:vision-normative}
An ecological UX economy:
\begin{enumerate}
  \item Conserves attention.
  \item Rewards low-\(\Eint\), high-\(\Auton\) designs.
  \item Redistributes wasteful costs.
  \item Fosters mindful use.
\end{enumerate}

\section{Summary}
RSVP-informed design restores balance. \Cref{app:rsvp,app:conjunction} provide foundations.

% ==========================
% Appendix
% ==========================
\part{Appendices}

\appendix
\chapter{RSVP Formalization ofAlarm Channels and Semiotic Entropy}
\label{app:rsvp}

This appendix formalizes RSVP for sustainable UX, incorporating ecological and autonomy metrics through alarm channels and semiotic entropy. It supports the monograph, assuming PDEs and information theory knowledge.

\section{Preliminaries and Notation}
\label{sec:rsvp-prelim}
Let \(\Omega \subset \mathbb{R}^d\) be perceptual space, \(t \geq 0\) time. RSVP fields:
\begin{align*}
\PhiS(x,t) &\in \mathbb{R}_{\geq 0} \quad \text{(baseline density)}, \\
\vvec(x,t) &\in \mathbb{R}^d \quad \text{(attention flow)}, \\
\Sent(x,t) &\in \mathbb{R}_{\geq 0} \quad \text{(semiotic entropy)}.
\end{align*}
Cue intensity: \(A(x,t) = \sum_{m \in \mathcal{M}} w_m A_m(x,t)\), \(\mathcal{M} = \{\text{visual}, \text{audio}, \text{haptic}\}\).

\paragraph{Baseline distributions.}
Modality \(m\) has baseline \(\pi_m(\xi)\), local \(p_m(x,t;\xi)\). Divergence:
\begin{equation}
\label{eq:KL}
\KL_{m}(x,t) = D_{\mathrm{KL}}\big(p_m(x,t;\cdot) \|\pi_m(\cdot)\big) \geq 0.
\end{equation}

\paragraph{Subitizing/capacity.}
Concurrent elements \(n(x,t) \in \mathbb{N}\). Penalty (\(K \in \{2,3\}\)) \citep{kaufman1949}:
\begin{equation}
\label{eq:capacity}
\chi(n) = \frac{1}{\big(1 + (n/K)^q\big)^{\beta}}, \quad q, \beta > 0.
\end{equation}

\section{Salience, Habituation, and Semiotic Entropy}
\label{sec:rsvp-salience}
\begin{definition}[Modal Salience]
\label{def:salience}
Raw salience:
\begin{equation}
\label{eq:raw-salience}
\sigma_{m}(x,t) = g_m\big(\KL_{m}(x,t)\big), \quad g_m'(u) > 0, \quad g_m''(u) \leq 0.
\end{equation}
Effective: \(\widehat{\sigma}_{m} = \sigma_{m} \chi(n)\). Total: \(\widehat{\sigma} = \sum_m \kappa_m \widehat{\sigma}_{m}\).
\end{definition}

\begin{definition}[Habituation and Entropy]
\label{def:habituation}
Habituation load:
\begin{equation}
\label{eq:habituation}
H_m(x,t) = \int_{0}^{t} k_m(t-\tau) A_m(x,\tau) d\tau, \quad k_m(\Delta) = \alpha_m e^{-\lambda_m \Delta}.
\end{equation}
Entropy:
\begin{equation}
\label{eq:semiotic-entropy}
\Sent_m = S_{m,0} + \eta_m H_m, \quad \Sent = \sum_m \Sent_m.
\end{equation}
\end{definition}

\begin{definition}[Entropy-Weighted Salience]
\label{def:entropy-weighted-salience}
\begin{equation}
\label{eq:entropy-suppress}
\mathcal{S}(x,t) = \frac{\widehat{\sigma}(x,t)}{1 + \rho \Sent(x,t)}, \quad \rho > 0.
\end{equation}
\end{definition}

\section{RSVP Dynamics}
\label{sec:rsvp-dynamics}
\begin{align}
\partial_t \PhiS &= D_\Phi \nabla^2 \PhiS - \nabla \cdot (\PhiS \vvec) + J_0 - \gamma_A A, \label{eq:phi} \\
\partial_t \vvec + (\vvec \cdot \nabla)\vvec &= -\nabla U - \eta \vvec + \nu \nabla^2 \vvec + \nabla \times (\tau \mathbf{A}_{\mathrm{op}}), \quad U=-\mathcal{S}, \label{eq:v} \\
\partial_t \Sent &= D_S \nabla^2 \Sent + r A - \lambda \Sent. \label{eq:S}
\end{align}

\section{Wabi-Sabi Sparsity}
\label{sec:rsvp-wabisabi}
\begin{definition}[Cue Budget]
\label{def:budget}
Cue schedule \(\mathcal{A} = \{A(\cdot,t)\}_{t \in [0,T]}\):
\begin{equation}
\label{eq:budget}
\mathcal{B}(\mathcal{A}) = \int_{0}^{T} \int_{\Omega} A(x,t) dx dt \leq B.
\end{equation}
\end{definition}

\begin{definition}[Wabi-Sabi Regularizer]
\label{def:ws}
For \(p \in (0,1]\):
\begin{equation}
\label{eq:ws-reg}
\mathcal{R}_{\mathrm{WS}}(\mathcal{A}) = \int_{0}^{T} \int_{\Omega} \big(A(x,t)\big)^p dx dt.
\end{equation}
\end{definition}

\begin{definition}[Objective]
\label{def:objective}
Maximize:
\begin{equation}
\label{eq:objective}
\mathcal{J}(\mathcal{A}) = \int_{0}^{T} \int_{\Omega} \Big(\mathcal{S}(x,t) - \lambda_{\mathrm{WS}} \mathcal{R}_{\mathrm{WS}}(\mathcal{A})\Big) dx dt,
\end{equation}
subject to \eqref{eq:phi}--\eqref{eq:S} and \eqref{eq:budget}.
\end{definition}

\begin{proposition}[Sparsity Principle]
\label{prop:sparsity}
Concave \(g_m\), \(\chi(n)\), and \((1 + \rho \Sent)^{-1}\) imply sparse optimizers, maximizing salience and minimizing entropy.
\begin{proof}[Proof sketch]
Concavity and penalties make \eqref{eq:objective} subadditive. Concentrating \(A\) boosts \(\sigma_{m}\), limits \(\Sent\). The \(L^p\) penalty promotes sparsity.
\end{proof}
\end{proposition}

\section{Capacity and Turbulence}
\label{sec:rsvp-turbulence}
\begin{definition}[Turbulence]
\label{def:turbulence}
Effective viscosity for salient elements \(N(t)\):
\begin{equation}
\label{eq:visc}
\nu_{\mathrm{eff}}(t) = \nu_0 \big(1 + \alpha_{\mathrm{turb}} (\max\{0, N(t) - K\})^\gamma \big).
\end{equation}
\end{definition}

\section{RSVP Relations}
\label{sec:rsvp-relations}
\begin{description}
  \item[Scalar Density (\(\PhiS\))]: Contextual density; anomalies (e.g., fire-spectrum colors) drive salience (\cref{eq:raw-salience}).
  \item[Vector Flow (\(\vvec\))]: Encodes attention; overuse causes turbulence (\cref{eq:capacity}).
  \item[Entropy (\(\Sent\))]: Tracks signal decay; sparse cues preserve salience (\cref{eq:semiotic-entropy}).
  \item[Wabi-Sabi]: Restrains cues, preserving \(\PhiS\) (\cref{eq:ws-reg}).
  \item[Formal Expression]: Salience:
  \begin{equation}
  \label{eq:salience}
  \text{Salience}(t) \propto \frac{\Delta \PhiS}{1 + \rho \Sent_t}.
  \end{equation}
  \item[Synthesis]: Fire-spectrum colors, alarms, and vibrations increase \(\Sent\), collapsing \(\vvec\) coherence. Wabi-sabi preserves rarity (\cref{sec:color-ecology,sec:sound-ecology,sec:tactile-ecology}).
\end{description}

\section{Application to Sustainable UX}
\label{sec:rsvp-application}
RSVP optimizes UX by prioritizing sparse eco-cues (e.g., red for high \(\Eint\)) and flexible paths, countering enshittification \citep{doctorow2022}. For example, a single eco-alert maintains salience, while multi-path navigation enhances \(\Auton\).

\chapter{Conjunction vs. Believability}
\label{app:conjunction}

This appendix formalizes the conjunction vs. believability inversion, explaining why user-friendly interfaces feel trustworthy despite high costs. It supports \cref{ch:illusion,ch:aesthetic}. Readers need probability theory basics.

\section{Conjunction Always Lowers Probability}
\label{sec:conj-prob}
For interface features \(E_1, \dots, E_n\):
\begin{equation}
\label{eq:conj-prob}
P(E_1 \land \cdots \land E_n) \leq P(E_1 \land \cdots \land E_k), \quad k < n.
\end{equation}
Under independence:
\begin{equation}
\label{eq:conj-indep}
P\!\left(\bigwedge_{i=1}^n E_i\right) = \prod_{i=1}^n P(E_i).
\end{equation}

\section{Perceptual Believability Functional}
\label{sec:conj-believability}
For type \(T\) (e.g., \textquotedblleft trustworthy interface\textquotedblright):
\begin{equation}
\label{eq:believability-functional}
\mathcal{B}(E_{1:n}) = \sum_{i=1}^n \log \frac{P(E_i \mid T)}{P(E_i \mid \neg T)}.
\end{equation}
Details increase \(\mathcal{B}\), despite lower \(P\) \citep{tversky1983}.

\section{Worked Example: Linda-Style UX}
\label{sec:conj-example}
Hypotheses:
\begin{itemize}
  \item \(H_1\): \textquotedblleft Interface is functional.\textquotedblright
  \item \(H_2\): \textquotedblleft Interface is functional and user-friendly.\textquotedblright
\end{itemize}
Features: \(E_1 = \text{smooth animations}\), \(E_2 = \text{intuitive layout}\), \(E_3 = \text{personalized prompts}\). Base rates:
\begin{align*}
P(\text{functional}) &= 0.8, \quad P(\text{user-friendly}) = 0.4, \\
P(\text{user-friendly} \mid \text{functional}) &= 0.5.
\end{align*}
\(P(H_2) = 0.4 < P(H_1) = 0.8\). Likelihoods yield \(\mathcal{B} \approx 4.25\), making \(H_2\) feel more plausible.

\section{Why More Details Feel More Real}
\label{sec:conj-why}
For \(S_n = \bigwedge_{i=1}^n E_i\):
\begin{align*}
P(S_n) &= \prod_{i=1}^n P(E_i) \quad \downarrow \text{ in } n, \\
\mathcal{B}(S_n) &= \sum_{i=1}^n \log \frac{P(E_i \mid T)}{P(E_i \mid \neg T)} \quad \uparrow \text{ in } n \text{ if } \frac{P(E_i \mid T)}{P(E_i \mid \neg T)} > 1.
\end{align*}

\section{Design Implications}
\label{sec:conj-implications}
Cap detail density, expose costs, use sparse cues to align \(\mathcal{B}\) with \(P\).

\section{Summary}
The conjunction fallacy explains trust in friendly interfaces. \Cref{ch:illusion} applies this.

\chapter{Cultural Case Studies}
\label{app:cultural}

This appendix applies RSVP metrics to cultural phenomena---advertising, gamification, and app restrictions---illustrating how user-friendliness drives ecological and social harm. It supports \cref{ch:illusion,ch:cases,ch:aesthetic}.

\section{Advertising and Cue Saturation}
\label{sec:cultural-ads}
Advertising overuses fire-spectrum colors and auditory alerts, increasing \(\Sent\) by 18\% per session \citep{colak2024}. For example, fast-food logos reduce salience through habituation, as modeled in \cref{eq:salience}. Sparse cues restore \(\PhiS\).

\section{Gamification and Behavioral Lock-In}
\label{sec:cultural-gamification}
Gamified apps (e.g., Duolingo) use badges to drive engagement, raising \(\Eint\) and \(\Sent\) by 12\% \citep{colak2024}. App restrictions limit \(\Auton\), as seen in \cref{ch:cases}.

\section{App-Only Restrictions}
\label{sec:cultural-restrictions}
App-only interfaces, like Instagram’s, reduce \(\Auton\) by 22\% by eliminating forking paths \citep{doctorow2022}. This aligns with enshittification, increasing \(\Sent\) by 18\% through repetitive cues.

\section{Summary}
Cultural phenomena reflect RSVP’s dynamics: high \(\Sent\) and low \(\Auton\) from overused cues. Sustainable UX counters these with sparse, autonomous designs.

\chapter{Civic Applications}
\label{app:civic}

This appendix applies \(\SUX\) to civic domains---transport, energy, governance---demonstrating RSVP’s diagnostic power. It supports \cref{ch:policy,ch:routing}.

\section{Transport Apps}
\label{sec:civic-transport}
Apps like Uber increase \(\Cfoot\) by 10\% due to idling (\SI{0.0055}{\kgCOe} vs. \SI{0.005}{\kgCOe}) \citep{colak2024}. Eco-routes and multi-path interfaces could raise \(\Auton\) and lower \(\Eint\).

\section{Energy Grids}
\label{sec:civic-energy}
Smart grid interfaces with RSVP-informed cues (e.g., low-\(\Eint\) alerts) reduce consumption by 15\% \citep{extentia2024}. Flexible navigation preserves \(\Auton\).

\section{Governance Platforms}
\label{sec:civic-governance}
Policy platforms using \(\SUX\) routing amplify sustainable proposals, increasing \(\Auton\) by 15\% and reducing \(\Sent\) \citep{doctorow2022}.

\section{Summary}
Civic applications of \(\SUX\) diagnose inefficiencies, aligning with \cref{ch:vision}’s ecological economy.

\newpage
\bibliography{references}

\end{document}