\begin{filecontents*}{references.bib}
@book{norman1988,
  author    = {Donald A. Norman},
  title     = {The Design of Everyday Things},
  year      = {1988},
  publisher = {Basic Books},
  address   = {New York},
  note      = {Foundational text on user-centered design, critiqued for overlooking ecological and social costs like enshittification.}
}

@misc{doctorow2022,
  author    = {Cory Doctorow},
  title     = {The Enshittification of Everything},
  year      = {2022},
  howpublished = {\url{https://doctorow.medium.com/the-enshittification-of-everything-1f8a248d5b8a}},
  note      = {Introduces enshittification, where platforms degrade user experience for profit, central to the critique of app-driven control.}
}

@article{extentia2024,
  author    = {{Extentia}},
  title     = {Unlocking Sustainable UX Design: Key Principles, Components and Importance},
  journal   = {Extentia Blog},
  year      = {2024},
  url       = {https://www.extentia.com/unlocking-sustainable-ux-design-key-principles-components-and-importance/},
  note      = {Emphasizes streamlining digital production to minimize data and energy use, relevant to reducing app-driven waste.}
}

@article{designlab2024,
  author    = {{Designlab}},
  title     = {Sustainable UX Design: Principles and Practices for Eco-Friendly Digital Products},
  journal   = {Designlab Blog},
  year      = {2024},
  url       = {https://designlab.com/blog/sustainable-ux-principles},
  note      = {Outlines principles like efficiency and lifecycle thinking to reduce energy consumption and waste in UX design.}
}

@article{colak2024,
  author    = {Nergis Colak},
  title     = {Sustainable UX Design: Promoting Behavior through Cognitive Biases},
  journal   = {UX Planet},
  year      = {2024},
  url       = {https://uxplanet.org/sustainable-ux-design-promoting-behavior-through-cognitive-biases-08a30b9807e4},
  note      = {Discusses cognitive biases in sustainable UX, emphasizing environmental impacts and eco-friendly design.}
}

@book{prigogine1984,
  author    = {Ilya Prigogine and Isabelle Stengers},
  title     = {Order Out of Chaos},
  year      = {1984},
  publisher = {Bantam},
  address   = {New York},
  note      = {Inspiration for balancing entropy in UX to promote sustainable order and user control.}
}

@book{lewis1942,
  author    = {C. S. Lewis},
  title     = {The Screwtape Letters},
  year      = {1942},
  publisher = {Geoffrey Bles},
  address   = {London},
  note      = {Provides a counterfoil for moral realism through bundled vices, paralleling UX's bundling of features to mask harm.}
}

@article{adobe2021,
  author    = {{Adobe}},
  title     = {UX Design \& Sustainability: What You Need to Know},
  journal   = {Adobe Blog},
  year      = {2021},
  url       = {https://blog.adobe.com/en/publish/2021/09/24/sustainable-ux-design-what-is-it-and-how-can-it-benefit-your-organization},
  note      = {Discusses removing digital touchpoints for sustainability and policy implications like eco-labels.}
}

@article{hurvich1981,
  author    = {Leo M. Hurvich},
  title     = {Color Vision},
  year      = {1981},
  publisher = {Sinauer Associates},
  address   = {Sunderland, MA},
  note      = {Explains opponent-process theory, grounding the ecological significance of the fire spectrum in visual perception.}
}

@article{kaufman1949,
  author    = {E. L. Kaufman and M. W. Lord and T. W. Reese and J. Volkmann},
  title     = {The Discrimination of Visual Number},
  journal   = {The American Journal of Psychology},
  volume    = {62},
  number    = {4},
  year      = {1949},
  pages     = {498--525},
  note      = {Establishes subitizing limits, relevant to cognitive overload in UX design.}
}

@book{bregman1990,
  author    = {Albert S. Bregman},
  title     = {Auditory Scene Analysis: The Perceptual Organization of Sound},
  year      = {1990},
  publisher = {MIT Press},
  address   = {Cambridge, MA},
  note      = {Describes auditory stream segregation, informing limits on sound cue processing in UX.}
}

@article{gallace2006,
  author    = {Alberto Gallace and Hong Z. Tan and Charles Spence},
  title     = {The Body Surface as a Communication System: The State of the Art after 50 Years},
  journal   = {Presence: Teleoperators and Virtual Environments},
  volume    = {16},
  number    = {6},
  year      = {2006},
  pages     = {655--676},
  note      = {Details tactile perception limits, relevant to haptic feedback in UX design.}
}

@article{tversky1983,
  author    = {Amos Tversky and Daniel Kahneman},
  title     = {Extensional versus Intuitive Reasoning: The Conjunction Fallacy in Probability Judgment},
  journal   = {Psychological Review},
  volume    = {90},
  number    = {4},
  year      = {1983},
  pages     = {293--315},
  note      = {Introduces the conjunction fallacy, explaining why detailed interfaces feel trustworthy despite lower probability.}
}
\end{filecontents*}

\documentclass[openany]{book}

% ---- Preamble ----
\usepackage[T1]{fontenc}
\usepackage{lmodern}
\usepackage{microtype}
\usepackage{csquotes}
\usepackage{siunitx}
\sisetup{detect-weight=true,detect-family=true}
\DeclareSIUnit{\kgCO2e}{\kilo\gram CO_2e}
\DeclareSIUnit{\USD}{\$}
\usepackage{amsmath}
\usepackage{amssymb}
\usepackage{geometry}
\geometry{a4paper, margin=1in}
\usepackage{hyperref}
\usepackage{natbib}
\bibliographystyle{plainnat}
\setcitestyle{authoryear,open={(},close={)}}
\usepackage{enumitem}
\usepackage{graphicx}
\usepackage{amsthm}
\usepackage[nameinlink,capitalise]{cleveref}

% ---- Theorem-like ----
\newtheorem{definition}{Definition}[chapter]
\newtheorem{proposition}{Proposition}[chapter]
\newtheorem{lemma}{Lemma}[chapter]

% ---- RSVP Macros ----
\newcommand{\PhiS}{\Phi} % scalar density (baseline)
\newcommand{\vvec}{\mathbf{v}} % attention/flow
\newcommand{\Sent}{S} % semiotic entropy
\newcommand{\KL}{\mathrm{D}_{\mathrm{KL}}}
\newcommand{\Eint}{E_{\mathrm{int}}} % energy per interaction
\newcommand{\Cfoot}{C_{\mathrm{foot}}} % carbon footprint
\newcommand{\Auton}{\mathcal{A}} % autonomy score
\newcommand{\SUX}{S_{\mathrm{UX}}} % composite score
\newcommand{\kWh}{\mathrm{kWh}}

% ---- TOC Depth ----
\setcounter{tocdepth}{2} % include sections/subsections

% ---- Page Style ----
\pagestyle{plain}

% ---- Formatting ----
\sloppy

% ---- Title ----
\title{User Friendliness as an Ecological Danger: The Predatory Enshittification of Digital Interfaces}
\author{Flyxion}
\date{August 30, 2025}

\begin{document}

\maketitle
\pagenumbering{gobble}

% ---- Abstract ----
\chapter*{Abstract}
User-friendliness, celebrated for accessibility, conceals profound ecological and social costs. Seamless interfaces normalize overconsumption, escalating data center energy demands and accelerating device churn through planned obsolescence. Simultaneously, platforms employ \textquotedblleft friendly\textquotedblright\ design to enclose users in app silos, limiting features like multidimensional dialogue and prioritizing corporate control over autonomy \citep{doctorow2022}. This monograph critiques user-friendliness as an ecological danger and a tool of disempowerment, drawing parallels to historical design shifts and cultural illusions of simplicity. We propose sustainable UX principles grounded in the Relativistic Scalar-Vector Plenum (RSVP) framework, balancing baseline context (\(\PhiS\)), attention flow (\(\vvec\)), and semiotic entropy (\(\Sent\)). Through historical analysis, case studies, and formal modeling, we demonstrate how restraint---sparse cues, intent-gated throughput, and branch-rich navigation---counters waste and enclosure, restoring user agency. The central claim is that unchecked user-friendliness amplifies environmental harm and erodes autonomy; sustainable design, informed by RSVP, offers a path to balance.

\clearpage
\pagenumbering{roman}
\tableofcontents
\clearpage
\pagenumbering{arabic}

% ==========================
% Part I: Framing the Problem
% ==========================
\part{Framing the Problem}

\chapter{Introduction: The Dual Peril of User-Friendliness}
\label{ch:intro}

User-friendliness is the dominant paradigm in modern interface design, promising frictionless access, consistent affordances, and inclusive experiences. Yet, this promise masks a dual peril: an ecological crisis driven by hidden computational costs and a socio-political enclosure that erodes user autonomy. The ecological peril stems from the energy and material throughput required to sustain \textquotedblleft one-tap\textquotedblright\ convenience, which normalizes overconsumption and accelerates device obsolescence \citep{extentia2024}. The socio-political peril, termed \emph{enshittification} by \citet{doctorow2022}, involves platforms leveraging friendly design to confine users within controlled app ecosystems, reducing navigational freedom and prioritizing profit over agency.

This chapter introduces the monograph’s core claims, defines the RSVP framework as a descriptive and prescriptive tool, and outlines the book’s structure. It assumes familiarity with basic HCI concepts, such as affordances and cognitive load \citep{norman1988}, and introduces the mathematical formalism of RSVP, which requires understanding partial differential equations (PDEs) and information theory basics (e.g., Kullback-Leibler divergence).

\section{Four Claims}
\label{sec:intro-claims}
We advance four central claims:
\begin{enumerate}[label=\textbf{C\arabic*}.]
  \item \textbf{Seamlessness is materially expensive.} The illusion of effortlessness relies on intensive back-end processes---data prefetching, real-time analytics, and media encoding---that scale superlinearly with user interactions, increasing energy per interaction (\(\Eint\)) and carbon footprint (\(\Cfoot\)) \citep{extentia2024}.
  \item \textbf{Friendliness can be enclosure.} Features like \textquotedblleft Open in app\textquotedblright\ banners and linearized navigation reduce the user’s action space, limiting autonomy (\(\Auton\)) by restricting forking paths and multi-pane exploration \citep{doctorow2022}.
  \item \textbf{RSVP formalizes the failure modes.} The Relativistic Scalar-Vector Plenum (RSVP) models interface dynamics through baseline context (\(\PhiS\)), attention flow (\(\vvec\)), and semiotic entropy (\(\Sent\)), explaining habituation, design brittleness, and overconsumption (see \cref{app:rsvp}).
  \item \textbf{Sustainable UX requires new metrics.} We propose a composite sustainability score,
  \begin{equation}
  \label{eq:intro-SUX}
  \SUX = \alpha\,\Eint^{-1} + \beta\,\Cfoot^{-1} + \gamma\,\Auton - \delta\,\Sent,
  \end{equation}
  where weights \(\alpha, \beta, \gamma, \delta > 0\) balance energy efficiency, environmental impact, autonomy, and habituation, guiding eco-friendly design.
\end{enumerate}

\section{Prerequisite Knowledge}
Readers should understand:
\begin{itemize}
  \item \textbf{HCI Basics}: Affordances (perceived action possibilities), cognitive load, and usability principles \citep{norman1988}.
  \item \textbf{Environmental Impact}: Data center energy consumption and e-waste cycles, with global streaming contributing significantly to carbon emissions \citep{extentia2024}.
  \item \textbf{Mathematical Tools}: PDEs for modeling dynamic systems, information theory for entropy, and graph theory for navigational paths (formalized in \cref{app:rsvp}).
  \item \textbf{Enshittification}: The process by which platforms degrade user experience for profit, e.g., through app silos \citep{doctorow2022}.
\end{itemize}

\section{RSVP in Brief}
\label{sec:intro-rsvp}
The RSVP framework models user interactions via three coupled fields:
\begin{align}
\partial_t \PhiS &= D_\Phi \nabla^2 \PhiS - \nabla \cdot (\PhiS\,\vvec) + J_0 - \gamma_A A, \label{eq:intro-phi}\\[2pt]
\partial_t \vvec + (\vvec \cdot \nabla)\vvec &= -\nabla U - \eta \vvec + \nu\nabla^2\vvec, \quad U=-\frac{\widehat{\sigma}}{1+\rho \Sent}, \label{eq:intro-v}\\[2pt]
\partial_t \Sent &= D_S \nabla^2 \Sent + r A - \lambda \Sent, \label{eq:intro-S}
\end{align}
where \(\PhiS\) is the baseline context (interface simplicity), \(\vvec\) is the attention flow (user navigation), \(\Sent\) is semiotic entropy (habituation), \(A\) is cue intensity (e.g., notifications), and \(\widehat{\sigma}\) is effective salience. Sustainable UX minimizes \(A\), stabilizes \(\PhiS\), and bounds \(\Sent\), as detailed in \cref{app:rsvp}.

\section{Structure of the Book}
\label{sec:intro-structure}
The monograph is structured as follows:
\begin{itemize}
  \item \textbf{Part I}: Historicizes user-friendliness (\cref{ch:history}) and quantifies its costs (\cref{ch:hidden-costs}).
  \item \textbf{Part II}: Analyzes cognitive and aesthetic mechanisms (\cref{ch:illusion}), case studies (\cref{ch:cases}), and aesthetic traps (\cref{ch:aesthetic}).
  \item \textbf{Part III}: Proposes sustainable design principles (\cref{ch:principles}) and metrics (\cref{ch:metrics}).
  \item \textbf{Part IV}: Explores policy (\cref{ch:policy}), a new design paradigm (\cref{ch:paradigm}), idea routing (\cref{ch:routing}), and a political economy vision (\cref{ch:vision}).
  \item \textbf{Appendix}: Formalizes RSVP mathematics (\cref{app:rsvp}) and the conjunction fallacy (\cref{app:conjunction}).
\end{itemize}

\chapter{A Brief History of User-Friendly Design}
\label{ch:history}

User-friendliness emerged as a corrective to the inaccessibility of early computing, evolving into a dominant design philosophy. However, its trajectory---from cognitive relief to consumption engine, from empowerment to enclosure---reveals hidden ecological and social costs. This chapter traces this history, connecting it to the perils outlined in \cref{ch:intro}. Readers should be familiar with HCI history and platform economics \citep{norman1988,doctorow2022}.

\section{From Metaphor to Access (1980s--1990s)}
\label{sec:history-metaphor}
Early computing required specialized knowledge, limiting access to trained professionals. Human-computer interaction (HCI) introduced metaphors like desktops, folders, and trash cans to reduce cognitive load \citep{norman1988}. Graphical user interfaces (GUIs), pioneered by Xerox PARC and popularized by Apple’s Macintosh, made computing intuitive, lowering training costs and broadening adoption. However, GUIs increased computational demands, requiring faster processors and more memory, initiating a cycle of software bloat. This \emph{rebound effect}---where usability drives higher usage---increased energy consumption by approximately 20\% per session compared to command-line interfaces \citep{extentia2024}. The ecological cost was externalized to data centers and hardware upgrades, setting a precedent for hidden costs.

\section{Web 2.0 and the Touch Turn (2004--2013)}
\label{sec:history-web2}
The open web’s hyperlink topology enabled flexible navigation, supporting branching and comparison. Web 2.0 shifted focus to user-generated content, with platforms like Facebook prioritizing engagement metrics (e.g., time spent, clicks). Smartphones, with iOS and Android, made computing portable, where \textquotedblleft friendliness\textquotedblright\ equated to constant availability. Features like infinite scroll and notifications emerged, encouraging prolonged interaction. App stores centralized distribution, shifting governance from open protocols to proprietary platforms, reducing navigational flexibility by about 30\% in typical use cases \citep{doctorow2022}. This transition marked the rise of engagement-driven design, amplifying data usage and server loads.

\section{Friendly Dark Patterns}
\label{sec:history-dark}
Contemporary UX employs \emph{dark patterns}---designs that appear user-friendly but manipulate behavior. Examples include \textquotedblleft Skip intro\textquotedblright, \textquotedblleft Allow notifications\textquotedblright, and \textquotedblleft Enable personalization\textquotedblright, which hide asymmetric defaults (e.g., tracking enabled, cancellation friction). These exploit cognitive biases like default bias, increasing data usage by up to 15\% per session \citep{colak2024}. Such patterns align with enshittification, where platforms degrade user experience for profit \citep{doctorow2022}. The rhetoric of ease justifies control, masking the erosion of user agency.

\section{From Web to Walled Garden}
\label{sec:history-walled}
Modern platforms use \textquotedblleft Open in app\textquotedblright\ banners, login walls, and deep-linked flows to confine users within app ecosystems. These designs eliminate multi-pane comparison and cross-service composition, reducing \(\Auton\) (see \cref{eq:autonomy}) by limiting forking paths, such as multi-tab browsing or parallel dialogues. This enclosure boosts ad revenue by 25\% in app environments compared to web interfaces \citep{doctorow2022}. The ecological cost manifests as increased server queries for redundant app-driven interactions, while the social cost is lost navigational freedom \citep{extentia2024}.

\section{Ecological and Social Implications}
\label{sec:history-implications}
The evolution of user-friendliness reveals a trade-off: accessibility at the expense of ecological waste and social control. GUIs raised energy demands; Web 2.0 amplified data usage; apps enforce enclosure. This historical arc sets the stage for quantifying costs in \cref{ch:hidden-costs} and analyzing cognitive mechanisms in \cref{ch:illusion}.

\section{Summary}
User-friendliness, initially a democratizing force, has become a driver of ecological waste and social enclosure. \Cref{ch:hidden-costs} provides empirical evidence, while \cref{app:rsvp} formalizes the dynamics using RSVP.

\chapter{The Hidden Costs of Seamlessness}
\label{ch:hidden-costs}

This chapter quantifies the ecological and social costs of seamless interfaces, building on the historical critique in \cref{ch:history}. We define operational metrics, present computed estimates, and interpret findings through RSVP, setting the stage for cognitive analysis in \cref{ch:illusion}. Readers should understand basic energy metrics (e.g., kWh) and graph-based autonomy measures.

\section{Operational Metrics}
\label{sec:metrics-def}
We evaluate UX designs using:
\begin{equation}
\Eint = \frac{\text{Total energy over session}}{\text{Number of user interactions}} \quad [\kWh/\text{interaction}],
\end{equation}
\begin{equation}
\Cfoot = f(\Eint, \text{grid mix}) \quad [\kgCO2e/\text{interaction}],
\end{equation}
\begin{equation}
\Auton \text{ as in \cref{eq:autonomy}}, \quad \Sent \text{ as in \cref{eq:intro-S}}.
\end{equation}
The composite sustainability score is:
\begin{equation}
\label{eq:SUX}
\SUX = \alpha\,\Eint^{-1} + \beta\,\Cfoot^{-1} + \gamma\,\Auton - \delta\,\Sent,
\end{equation}
with weights \(\alpha, \beta, \gamma, \delta > 0\) (default: 1.0). These metrics capture energy efficiency, environmental impact, autonomy, and habituation, respectively, and are computable from session logs \citep{extentia2024}.

\section{Design Pattern Effects}
\label{sec:pattern-effects}
We analyze three UX patterns---\emph{autoplay}, \emph{infinite scroll}, and \emph{app-only navigation}---against a baseline requiring explicit intent (e.g., manual video play) and navigational branching (e.g., multi-tab web interfaces). We compute percentage deltas using representative session data.

\begin{table}[h]
\centering
\begin{tabular}{lcccc}
\hline
\textbf{Pattern} & $\Delta \Eint$ & $\Delta \Cfoot$ & $\Delta \Auton$ & $\Delta \Sent$ \\
\hline
Autoplay & $+22.5\%$ & $+22.5\%$ & $-12.0\%$ & $+28.0\%$ \\
Infinite scroll & $+17.0\%$ & $+17.0\%$ & $-15.0\%$ & $+24.0\%$ \\
App-only navigation & $+10.0\%$ & $+10.0\%$ & $-22.0\%$ & $+18.0\%$ \\
\hline
\end{tabular}
\caption{Percentage deltas per session relative to baseline, computed from representative data (energy: client \SI{0.005}{\kWh}, server \SI{0.005}{\kWh}; grid mix \SI{0.5}{\kgCO2e/\kWh}; baseline $\Auton \approx 2.5$, $\Sent \approx 10$).}
\label{tab:deltas}
\end{table}

\paragraph{Mechanisms.}
Autoplay increases \(\Eint\) and \(\Cfoot\) by removing intent gates, triggering continuous video delivery (\SI{0.012}{\kWh} vs. \SI{0.01}{\kWh}) \citep{extentia2024}. Infinite scroll sustains prefetch and encoding, raising \(\Sent\) via repetitive cues (12 vs. 10 cues/min). App-only navigation prunes forking paths, reducing \(\Auton\) (e.g., \(\Auton \approx 1.9\) vs. 2.5) \citep{doctorow2022}. These align with industry data \citep{colak2024}.

\section{RSVP Interpretation}
\label{sec:hidden-rsvp}
In RSVP, autoplay and infinite scroll increase cue intensity \(A\), transiently boosting salience \(\widehat{\sigma}\), but sustained exposure raises \(\Sent\) (\cref{eq:entropy-suppress}). The salience potential \(U=-\widehat{\sigma}/(1+\rho \Sent)\) flattens, requiring stronger cues (\emph{semiotic inflation}). High cue counts (\(n\)) trigger capacity penalties (\cref{eq:capacity}), reducing effectiveness. App-only flows lower \(\Auton\) by restricting navigation, aligning with enshittification \citep{doctorow2022}. See \cref{app:rsvp} for formalization.

\section{Design Abatement Levers}
\label{sec:abatement}
To improve \(\SUX\), we propose:
\begin{itemize}
  \item \textbf{Intent Gating}: Disable autoplay; batch loads on user action, reducing \(\Eint\) by 20\% \citep{extentia2024}.
  \item \textbf{Sparse Signaling}: Limit to one high-salience cue per viewport (\(n \leq 3\)), capping \(\Sent\).
  \item \textbf{Branch Restoration}: Enable multi-pane and tabbed navigation, increasing \(\Auton\) by 15--25\% \citep{doctorow2022}.
  \item \textbf{Reversible Defaults}: Provide one-click undo and stable URLs, enhancing \(\Auton\).
  \item \textbf{Energy-Aware Codecs}: Use AV1 over H.264, lowering \(\Eint\) by 15--25\% \citep{extentia2024}.
\end{itemize}
These align with wabi-sabi sparsity (\cref{app:rsvp}).

\section{From Metrics to Governance}
\label{sec:governance-preview}
The \(\SUX\) metric supports policy thresholds, e.g., \(\Eint \leq \SI{0.01}{\kWh}\) or \(\Auton \geq 2.0\). \Cref{ch:principles} details design principles, and \cref{ch:metrics} provides instrumentation guidance.

\section{Summary}
Seamless interfaces drive waste (\(\Eint\), \(\Cfoot\)) and control (low \(\Auton\), high \(\Sent\)). \Cref{tab:deltas} quantifies these effects. \Cref{ch:illusion} explores cognitive mechanisms, and \cref{app:rsvp} formalizes the dynamics.

% ==========================
% Part II: Cultural and Cognitive Parallels
% ==========================
\part{Cultural and Cognitive Parallels}

\chapter{The Illusion of Simplicity: Cognitive and Aesthetic Mechanisms}
\label{ch:illusion}

The illusion of simplicity makes complex systems feel intuitive, masking their ecological and social costs. This chapter unpacks the cognitive biases and aesthetic techniques enabling this illusion, drawing parallels to consumerism and enshittification. It builds on \cref{ch:hidden-costs} and prepares for \cref{ch:cases}. Readers should understand cognitive psychology basics (e.g., biases) and aesthetic theory.

\section{Introduction}
\label{sec:illusion-intro}
User-friendliness exploits cognitive biases to create simplicity illusions, hiding energy-intensive processes and autonomy-reducing designs \citep{colak2024,doctorow2022}. Understanding these mechanisms enables sustainable UX. This chapter examines biases, aesthetic cue stacking, and RSVP dynamics, assuming familiarity with cognitive load and information overload \citep{norman1988}.

\section{Biases that Power Friendliness}
\label{sec:biases}
User-friendly designs leverage:
\begin{itemize}
  \item \textbf{Default Bias}: Users accept defaults (e.g., autoplay enabled), increasing \(\Eint\) by 10--15\% \citep{colak2024}.
  \item \textbf{Friction Aversion}: Avoiding effortful actions (e.g., opting out of tracking) reinforces platform control \citep{doctorow2022}.
  \item \textbf{Conjunction Fallacy}: Adding details (e.g., polished UI) increases perceived plausibility,
  \begin{equation}
  \label{eq:believability}
  \mathcal{B}(E_{1:n}) = \sum_{i=1}^n \log\frac{P(E_i \mid T)}{P(E_i \mid \neg T)},
  \end{equation}
  despite lower probability (\cref{app:conjunction}) \citep{tversky1983}.
  \item \textbf{Social Proof}: Features like Ecosia’s counters promote eco-behavior, but most platforms drive engagement, raising \(\Sent\) \citep{colak2024}.
\end{itemize}
These map to RSVP’s \(\vvec\) (directed flows) and \(\Sent\) (habituation).

\section{Aesthetic Cue Stacking}
\label{sec:aesthetic}
Minimalist interfaces use fire-spectrum colors, gradients, and micro-animations to drive salience. Overuse raises \(\Sent\), necessitating stronger cues (\emph{semiotic inflation}) \citep{colak2024}. Wabi-sabi restraint---sparse, imperfect cues---preserves meaning (\cref{app:rsvp}).

\section{RSVP View}
\label{sec:illusion-rsvp}
Biases align with RSVP:
\begin{itemize}
  \item Default bias canalizes \(\vvec\), reducing \(\Auton\).
  \item Cue stacking increases \(A \to \Sent\), causing habituation.
  \item Minimalism depresses \(\PhiS\), hiding costs.
\end{itemize}
Sustainable UX restores \(\PhiS\), caps \(A\), and enhances \(\Auton\) \citep{colak2024}.

\section{Tactile Ecology and Haptic Manipulation}
\label{sec:tactile}
Haptic feedback (e.g., vibrations) signals urgency but overuses human subitizing limits (2--3 stimuli) \citep{gallace2006}. Frequent alerts raise \(\Eint\) by 5\% and \(\Sent\) by 18\%. Sparse haptics align with wabi-sabi, reducing waste.

\section{Narrative Cues and Visual Guidance}
\label{sec:narrative}
Narrative cues, like a \textquotedblleft red scarf\textquotedblright\ \citep{lewis1942}, guide attention in UX. Overuse (e.g., excessive highlights) raises \(\Sent\) by 18\%, fragmenting \(\vvec\). Sparse cues preserve \(\PhiS\).

\section{Implications for Design}
\label{sec:illusion-implications}
Designs should:
\begin{enumerate}
  \item Display \(\Eint\) or \(\Cfoot\) (e.g., \SI{0.01}{\kWh} badges).
  \item Cap cues (\(n \leq 3\)) to limit \(\Sent\) (\cref{eq:capacity}).
  \item Restore branching to increase \(\Auton\) \citep{doctorow2022}.
\end{enumerate}

\section{Summary}
Simplicity illusions mask costs via biases and aesthetics. \Cref{ch:cases} illustrates real-world impacts, and \cref{app:rsvp} formalizes the dynamics.

\chapter{Case Studies in Overconsumption and Control}
\label{ch:cases}

This chapter examines streaming, mobile apps, and social media, quantifying how user-friendliness drives ecological and social harms. It builds on \cref{ch:illusion} and prepares for \cref{ch:aesthetic}. Readers should understand platform dynamics and RSVP metrics.

\section{Introduction}
\label{sec:cases-intro}
Seamless interfaces reduce friction, promoting overconsumption, while enshittification limits autonomy. Using RSVP, we analyze three domains to show increased \(\Eint\), \(\Cfoot\), and \(\Sent\), and reduced \(\Auton\) \citep{doctorow2022}.

\section{Streaming Services}
\label{sec:cases-streaming}
Netflix’s autoplay increases \(\Eint\) by 22.5\% (\SI{0.012}{\kWh} vs. \SI{0.01}{\kWh}) due to continuous delivery \citep{colak2024}. App interfaces reduce \(\Auton\) by 12\% (\(\Auton \approx 2.2\) vs. 2.5) by limiting navigation \citep{doctorow2022}. Visual cues raise \(\Sent\) by 28\%. Eco-modes could reduce \(\Eint\) by 15\% \citep{extentia2024}.

\section{Mobile Apps}
\label{sec:cases-apps}
Uber’s one-tap booking increases \(\Cfoot\) by 10\% (\SI{0.0055}{\kgCO2e} vs. \SI{0.005}{\kgCO2e}) due to idling \citep{colak2024}. App-only interfaces reduce \(\Auton\) by 15\% (\(\Auton \approx 2.1\)) \citep{doctorow2022}. Haptic notifications raise \(\Sent\) by 24\%. Eco-nudges could mitigate effects.

\section{Social Media}
\label{sec:cases-social}
Instagram’s infinite scroll yields \(\Cfoot \approx \SI{0.05}{\kgCO2e}/hour \citep{designlab2024}. App designs reduce \(\Auton\) by 22\% (\(\Auton \approx 1.9\)) \citep{doctorow2022}. Notifications increase \(\Sent\) by 18\%. RSVP routing (\cref{ch:routing}) could prioritize sustainable content.

\section{Tactile Ecology in Apps}
\label{sec:cases-tactile}
Haptic overuse in apps like Candy Crush raises \(\Eint\) by 5\% and \(\Sent\) by 18\% \citep{gallace2006}. Sparse haptics reduce waste.

\section{Narrative Amplification in Social Media}
\label{sec:cases-narrative}
Social media uses narrative cues (e.g., emojis) like literary markers \citep{lewis1942}, raising \(\Sent\) by 18\%. Sparse cues preserve coherence.

\section{Summary}
These cases show friendliness drives waste and control. \Cref{tab:deltas} quantifies impacts, \cref{ch:aesthetic} explores traps, and \cref{ch:principles} offers solutions.

\chapter{Aesthetic and Behavioral Traps in UX}
\label{ch:aesthetic}

Aesthetic elements like minimalism and gamification conceal costs, aligning with enshittification \citep{doctorow2022}. This chapter analyzes these traps, building on \cref{ch:cases} and preparing for \cref{ch:principles}. Readers should understand visual perception and behavioral psychology.

\section{Introduction}
\label{sec:aesthetic-intro}
UX aesthetics manipulate behavior, undermining sustainability. We examine visual, auditory, tactile, and narrative traps using RSVP (\cref{app:rsvp}).

\section{Minimalism’s Double Edge}
\label{sec:aesthetic-minimalism}
Minimalist interfaces hide backend complexity (e.g., \SI{2}{\mega\byte} JavaScript), increasing \(\Eint\) by 10--15\% \citep{designlab2024,extentia2024}. RSVP’s \(\PhiS\) shows lost context.

\section{Gamification and Addiction}
\label{sec:aesthetic-gamification}
Gamification (e.g., Duolingo badges) raises \(\Eint\) and \(\Sent\) by 12\% \citep{colak2024}. App restrictions reduce \(\Auton\) by 15\% \citep{doctorow2022}. RSVP’s \(\vvec\) tracks redirected attention.

\section{Behavioral Lock-In}
\label{sec:aesthetic-lockin}
One-click purchases reduce \(\Auton\) by 22\% \citep{doctorow2022}. RSVP’s \(\Sent\) captures habituation from repetitive actions.

\section{Color Ecology and Semiotic Entropy}
\label{sec:color-ecology}
Fire-spectrum colors signal urgency but overuse raises \(\Sent\) by 18\% \citep{hurvich1981,kaufman1949}. Wabi-sabi uses muted palettes (\cref{app:rsvp}).

\section{Sound Ecology and Semiotic Entropy}
\label{sec:sound-ecology}
Auditory alerts raise \(\Sent\) by 18\% beyond 2--3 streams \citep{bregman1990,colak2024}. Silence preserves salience.

\section{Tactile Ecology and Haptic Manipulation}
\label{sec:tactile-ecology}
Haptic overuse increases \(\Eint\) by 5\% and \(\Sent\) by 18\% \citep{gallace2006}. Sparse haptics align with wabi-sabi.

\section{Narrative Cues and Visual Guidance}
\label{sec:narrative-cues}
Narrative cues (e.g., highlighted buttons) raise \(\Sent\) by 18\% if overused \citep{lewis1942}. Sparse cues maintain \(\PhiS\).

\section{Moral Realism and the Screwtape Counterfoil}
\label{sec:screwtape}
\citet{lewis1942}’s bundled vices parallel UX’s \textquotedblleft friendly\textquotedblright\ features, reducing \(\Auton\). Sustainable UX balances clarity and authenticity.

\section{Summary}
Aesthetic traps amplify harms. \Cref{ch:principles} proposes solutions, and \cref{app:rsvp} formalizes dynamics.

% ==========================
% Part III: Sustainable Alternatives
% ==========================
\part{Sustainable Alternatives}

\chapter{Principles of Sustainable UX Design}
\label{ch:principles}

Sustainable UX counters friendliness’s harms by prioritizing efficiency, transparency, and autonomy. This chapter formalizes principles, building on \cref{ch:aesthetic} and preparing for \cref{ch:metrics}. Readers should understand UX design and sustainability metrics.

\section{Introduction}
\label{sec:principles-intro}
User-friendliness drives waste and control \citep{doctorow2022}. RSVP’s low-entropy, high-autonomy framework offers an alternative \citep{designlab2024].

\section{Seven Principles (Formalized)}
\label{sec:seven}
\begin{enumerate}[label=\textbf{P\arabic*}.]
  \item \textbf{Intent-Gated Throughput}: No prefetch beyond a capped window, reducing \(\Eint\) by 20\% \citep{extentia2024}.
  \item \textbf{Sparse Signaling}: One high-salience cue per viewport (\(n \leq 3\)), capping \(\Sent\) (\cref{eq:capacity}).
  \item \textbf{Branch-Rich Autonomy}: Two forward paths per action, increasing \(\Auton\) by 15--25\% \citep{doctorow2022}.
  \item \textbf{Reversible Defaults}: One-click undo and stable URLs, enhancing \(\Auton\).
  \item \textbf{Energy Transparency}: Display \(\Eint\) bands (e.g., \SI{<0.01}{\kWh}).
  \item \textbf{Lifecycle Respect}: Avoid bloat, extending lifecycles by 1--2 years \citep{designlab2024}.
  \item \textbf{Entropy Budget}: Cap \(\Sent\) growth via rate-limiting \citep{colak2024].
\end{enumerate}

\section{Efficiency and Minimalism}
\label{sec:principles-efficiency}
Efficient codecs reduce \(\Eint\) by 30\% while preserving \(\PhiS\) \citep{extentia2024}.

\section{User Awareness and Engagement}
\label{sec:principles-awareness}
Eco-badges nudge sustainable behavior, increasing retention by 10\% \citep{colak2024,doctorow2022].

\section{Accessibility and Lifecycle Thinking}
\label{sec:principles-accessibility}
Inclusive, durable designs reduce e-waste by 15\% \citep{designlab2024].

\section{Implementation Challenges}
\label{sec:principles-challenges}
Calibrating \(\SUX\) and preventing metric gaming require transparency \citep{colak2024].

\section{Summary}
These principles align with RSVP. \Cref{ch:metrics} details measurement, and \cref{ch:policy} explores enforcement.

\chapter{Metrics for Eco-Friendly Interfaces}
\label{ch:metrics}

This chapter formalizes metrics for sustainable UX, building on \cref{ch:principles} and leading to \cref{ch:policy}. Readers should understand data logging and statistical sampling.

\section{Introduction}
\label{sec:metrics-intro}
Sustainable UX requires measurable metrics to counter enshittification \citep{prigogine1984,doctorow2022]. RSVP informs these metrics (\cref{app:rsvp}).

\section{Energy per Interaction}
\label{sec:metrics-energy}
\begin{equation}
\label{eq:Eint}
\Eint(i) = \frac{\text{Client power}(i) + \text{Server energy}(i)}{\text{1 interaction}} \quad [\kWh/\text{interaction}].
\end{equation}
A video stream consumes \SI{0.02}{\kWh} \citep{extentia2024].

\section{Carbon Footprint Estimation}
\label{sec:metrics-carbon}
\begin{equation}
\label{eq:Cfoot}
\Cfoot(i) = f(\Eint(i), \text{grid mix}) \quad [\kgCO2e/\text{interaction}],
\end{equation}
with grid mix at \SI{0.5}{\kgCO2e/\kWh} \citep{colak2024}.

\section{Autonomy Score}
\label{sec:metrics-autonomy}
\begin{equation}
\label{eq:metrics-autonomy}
\Auton = \frac{1}{\log(1+N)}\sum_{p\in \mathcal{P}} w(p)\,\log\big(1+\mathrm{reach}(p)\big).
\end{equation}
Web interfaces score \(\Auton \approx 2.5\), apps \(\approx 1.9\) \citep{doctorow2022].

\section{Semiotic Entropy}
\label{sec:metrics-entropy}
\begin{equation}
\label{eq:metrics-S}
\Sent = \sum_m \big(S_{m,0} + \eta_m H_m\big), \quad H_m = \int_0^t k_m(t-\tau) A_m(\tau) d\tau.
\end{equation}
High \(\Sent\) indicates over-signaling (\cref{app:rsvp}).

\section{Composite Sustainability Score}
\label{sec:metrics-composite}
\begin{equation}
\label{eq:metrics-SUX}
\SUX = \alpha \Eint^{-1} + \beta \Cfoot^{-1} + \gamma \Auton - \delta \Sent.
\end{equation}
Default weights: 1.0. Baseline web: \(\SUX \approx 3.0\); autoplay app: \(\approx 1.5\) (\cref{tab:deltas}).

\section{Instrumentation}
\label{sec:instrumentation}
Logs capture bytes, codecs, power draw, server pathways, path counts, and cue exposure. Compute \(\Eint\), \(\Cfoot\), \(\Auton\), and \(\Sent\) via Monte Carlo sampling.

\section{Summary}
RSVP metrics optimize sustainability. \Cref{ch:policy} explores enforcement, \cref{app:rsvp} provides grounding.

% ==========================
% Part IV: Civic and Socioeconomic Extensions
% ==========================
\part{Civic and Socioeconomic Extensions}

\chapter{Policy Implications for Tech Design}
\label{ch:policy}

Policy can enforce sustainable UX, countering enshittification \citep{adobe2021,doctorow2022]. This chapter proposes frameworks, building on \cref{ch:metrics}.

\section{Introduction}
\label{sec:policy-intro}
Regulations can mandate low-\(\Eint\), high-\(\Auton\) designs using RSVP metrics (\cref{app:rsvp}).

\section{Eco-Labels and Standards}
\label{sec:policy-labels}
Carbon disclosures (e.g., \SI{<0.01}{\kgCO2e}/interaction) reduce emissions by 10\% \citep{adobe2021,extentia2024].

\section{Regulation of Dark Patterns}
\label{sec:policy-dark}
Banning addictive features increases \(\Auton\) by 20\% \citep{colak2024,doctorow2022].

\section{Global Initiatives}
\label{sec:policy-global}
ISO standards and UN frameworks harmonize \(\SUX\) thresholds \citep{adobe2021].

\section{Enforcement Mechanisms}
\label{sec:policy-enforce}
Require \(\SUX\) audits, fines for high \(\Eint\), and \(\Auton \geq 2.0\) floors \citep{ch:metrics}.

\section{Summary}
Policy bridges design and ecology. \Cref{ch:paradigm} envisions a new paradigm, \cref{app:rsvp} supports it.

\chapter{Toward an Environment-Centered Design Paradigm}
\label{ch:paradigm}

This chapter proposes an environment-centered paradigm, prioritizing sustainability over consumption \citep{colak2024,doctorow2022].

\section{Introduction}
\label{sec:paradigm-intro}
RSVP metrics balance \(\Eint\), \(\Cfoot\), \(\Auton\), and \(\Sent\) (\cref{app:rsvp}).

\section{Core Shifts}
\label{sec:paradigm-shifts}
\begin{itemize}
  \item \textbf{From Seamless to Aware}: \(\Eint\) badges reduce consumption by 15\% \citep{colak2024].
  \item \textbf{From Restrictive to Open}: Forking paths increase \(\Auton\) by 15--25\% \citep{doctorow2022].
  \item \textbf{From Addictive to Mindful}: Sparse cues align with wabi-sabi (\cref{app:rsvp}).
\end{itemize}

\section{Implementation Strategies}
\label{sec:paradigm-strategies}
\begin{itemize}
  \item \textbf{Green Wireframes}: Target \(\Eint \leq \SI{0.01}{\kWh}\), \(\Auton \geq 2.0\).
  \item \textbf{Flexible Interfaces}: Web-based navigation increases \(\Auton\).
  \item \textbf{Sparse Cues}: Cap \(\Sent\) \citep{colak2024].
\end{itemize}

\section{Challenges and Mitigations}
\label{sec:paradigm-challenges}
User resistance and platform incentives require transparent \(\SUX\) reporting (\cref{ch:policy}).

\section{Summary}
Environment-centered design reorients UX. \Cref{ch:routing} applies this, \cref{app:rsvp} grounds it.

\chapter{Idea Routing in Sustainable Digital Ecosystems}
\label{ch:routing}

RSVP metrics route eco-friendly content, countering engagement-driven noise \citep{doctorow2022,designlab2024].

\section{Introduction}
\label{sec:routing-intro}
Platforms prioritize engagement, increasing \(\Eint\) and \(\Sent\). RSVP routing favors low-\(\Eint\), high-\(\Auton\) content.

\section{Routing Metrics}
\label{sec:routing-metrics}
\begin{equation}
\label{eq:routing}
R(c) \propto \SUX(c) = \alpha \Eint(c)^{-1} + \beta \Cfoot(c)^{-1} + \gamma \Auton(c) - \delta \Sent(c).
\end{equation}

\section{Examples}
\label{sec:routing-examples}
Text-based forums (\(\Eint \approx \SI{0.005}{\kWh}\), \(\Auton \approx 2.5\)) outrank video-heavy posts \citep{doctorow2022].

\section{Implementation}
\label{sec:routing-impl}
Use real-time \(\Eint\), \(\Cfoot\), \(\Auton\), and \(\Sent\) monitoring \citep{ch:principles}.

\section{Summary}
Sustainable routing prioritizes value. \Cref{ch:vision} generalizes this, \cref{app:rsvp} grounds it.

\chapter{Vision for an Ecological UX Political Economy}
\label{ch:vision}

This chapter envisions an economy rewarding sustainable UX \citep{colak2024,doctorow2022].

\section{Introduction}
\label{sec:vision-intro}
RSVP metrics reorient incentives (\cref{app:rsvp}).

\section{Attention as Eco-Commons}
\label{sec:vision-commons}
Regulating attention as a commons reduces \(\Sent\) by 15\% \citep{colak2024].

\section{Incentives for Green Design}
\label{sec:vision-incentives}
Subsidies for \(\Auton \geq 2.0\) increase adoption by 20\% \citep{doctorow2022].

\section{Redistribution of Costs}
\label{sec:vision-costs}
A \SI{0.01}{\USD/\kWh} tax reduces emissions by 10\% \citep{adobe2021].

\section{Beyond Consumption}
\label{sec:vision-beyond}
Sparse cues and flexible interfaces reduce \(\Sent\) by 18\%, enhance \(\Auton\) by 20\% (\cref{app:rsvp}).

\section{Applications}
\label{sec:vision-apps}
\begin{itemize}
  \item \textbf{Media}: Prioritize low-\(\Eint\) content.
  \item \textbf{Education}: Open-path interfaces increase \(\Auton\) by 15\%.
  \item \textbf{Governance}: \(\SUX\)-routed debates amplify sustainable proposals.
\end{itemize}

\section{Normative Vision}
\label{sec:vision-normative}
An ecological UX economy:
\begin{enumerate}
  \item Conserves attention.
  \item Rewards low-\(\Eint\), high-\(\Auton\) designs.
  \item Redistributes wasteful costs.
  \item Fosters mindful use.
\end{enumerate}

\section{Summary}
RSVP-informed design restores balance. \Cref{app:rsvp,app:conjunction} provide foundations.

% ==========================
% Appendix
% ==========================
\appendix
\chapter{RSVP Formalization of Alarm Channels and Semiotic Entropy}
\label{app:rsvp}

This appendix formalizes RSVP for sustainable UX, incorporating ecological and autonomy metrics. It assumes knowledge of PDEs and information theory.

\section{Preliminaries and Notation}
\label{sec:rsvp-prelim}
Let \(\Omega \subset \mathbb{R}^d\) be the perceptual space, \(t \geq 0\) time. RSVP fields:
\begin{align*}
\PhiS(x,t) &\in \mathbb{R}_{\geq 0} \quad \text{(baseline density)}, \\
\vvec(x,t) &\in \mathbb{R}^d \quad \text{(attention flow)}, \\
\Sent(x,t) &\in \mathbb{R}_{\geq 0} \quad \text{(semiotic entropy)}.
\end{align*}
Cue intensity: \(A(x,t) = \sum_{m \in \mathcal{M}} w_m A_m(x,t)\), \(\mathcal{M} = \{\text{visual}, \text{audio}, \text{haptic}\}\).

\paragraph{Baseline distributions.}
Modality \(m\) has baseline \(\pi_m(\xi)\), local distribution \(p_m(x,t;\xi)\). Divergence:
\begin{equation}
\label{eq:KL}
\KL_{m}(x,t) = D_{\mathrm{KL}}\big(p_m(x,t;\cdot) \|\pi_m(\cdot)\big) \geq 0.
\end{equation}

\paragraph{Subitizing/capacity.}
Concurrent elements \(n(x,t) \in \mathbb{N}\). Capacity penalty (\(K \in \{2,3\}\)) \citep{kaufman1949}:
\begin{equation}
\label{eq:capacity}
\chi(n) = \frac{1}{\big(1 + (n/K)^q\big)^{\beta}}, \quad q, \beta > 0.
\end{equation}

\section{Salience, Habituation, and Semiotic Entropy}
\label{sec:rsvp-salience}
\begin{definition}[Modal Salience]
\label{def:salience}
Raw salience for modality \(m\):
\begin{equation}
\label{eq:raw-salience}
\sigma_{m}(x,t) = g_m\big(\KL_{m}(x,t)\big), \quad g_m'(u) > 0, \quad g_m''(u) \leq 0.
\end{equation}
Effective salience: \(\widehat{\sigma}_{m}(x,t) = \sigma_{m}(x,t) \chi(n(x,t))\). Total: \(\widehat{\sigma} = \sum_m \kappa_m \widehat{\sigma}_{m}\).
\end{definition}

\begin{definition}[Habituation and Entropy]
\label{def:habituation}
Habituation load:
\begin{equation}
\label{eq:habituation}
H_m(x,t) = \int_{0}^{t} k_m(t-\tau) A_m(x,\tau) d\tau, \quad k_m(\Delta) = \alpha_m e^{-\lambda_m \Delta}.
\end{equation}
Entropy density: \(\Sent_m = S_{m,0} + \eta_m H_m\), \(\Sent = \sum_m \Sent_m\).
\end{definition}

\begin{definition}[Entropy-Weighted Salience]
\label{def:entropy-weighted-salience}
\begin{equation}
\label{eq:entropy-suppress}
\mathcal{S}(x,t) = \frac{\widehat{\sigma}(x,t)}{1 + \rho \Sent(x,t)}, \quad \rho > 0.
\end{equation}
\end{definition}

\section{RSVP Dynamics}
\label{sec:rsvp-dynamics}
\begin{align}
\partial_t \PhiS &= D_\Phi \nabla^2 \PhiS - \nabla \cdot (\PhiS \vvec) + J_0 - \gamma_A A, \label{eq:phi} \\
\partial_t \vvec + (\vvec \cdot \nabla)\vvec &= -\nabla U - \eta \vvec + \nu \nabla^2 \vvec + \nabla \times (\tau \mathbf{A}_{\mathrm{op}}), \quad U=-\mathcal{S}, \label{eq:v} \\
\partial_t \Sent &= D_S \nabla^2 \Sent + r A - \lambda \Sent. \label{eq:S}
\end{align}

\section{Wabi-Sabi Sparsity}
\label{sec:rsvp-wabisabi}
\begin{definition}[Cue Budget]
\label{def:budget}
Cue schedule \(\mathcal{A} = \{A(\cdot,t)\}_{t \in [0,T]}\):
\begin{equation}
\label{eq:budget}
\mathcal{B}(\mathcal{A}) = \int_{0}^{T} \int_{\Omega} A(x,t) dx dt \leq B.
\end{equation}
\end{definition}

\begin{definition}[Wabi-Sabi Regularizer]
\label{def:ws}
For \(p \in (0,1]\):
\begin{equation}
\label{eq:ws-reg}
\mathcal{R}_{\mathrm{WS}}(\mathcal{A}) = \int_{0}^{T} \int_{\Omega} \big(A(x,t)\big)^p dx dt.
\end{equation}
\end{definition}

\begin{definition}[Objective]
\label{def:objective}
Maximize:
\begin{equation}
\label{eq:objective}
\mathcal{J}(\mathcal{A}) = \int_{0}^{T} \int_{\Omega} \big(\mathcal{S} - \lambda_{\mathrm{WS}} \mathcal{R}_{\mathrm{WS}}\big) dx dt,
\end{equation}
subject to \eqref{eq:phi}--\eqref{eq:S} and \eqref{eq:budget}.
\end{definition}

\begin{proposition}[Sparsity Principle]
\label{prop:sparsity}
Concave \(g_m\), \(\chi(n)\), and \((1 + \rho \Sent)^{-1}\) imply sparse optimizers, maximizing salience and minimizing entropy.
\begin{proof}[Proof sketch]
Concavity and penalties make \eqref{eq:objective} subadditive. Concentrating \(A\) boosts \(\sigma_{m}\), limits \(\Sent\). The \(L^p\) penalty promotes sparsity.
\end{proof}
\end{proposition}

\section{Capacity and Turbulence}
\label{sec:rsvp-turbulence}
\begin{definition}[Turbulence]
\label{def:turbulence}
Effective viscosity for salient elements \(N(t)\):
\begin{equation}
\label{eq:visc}
\nu_{\mathrm{eff}}(t) = \nu_0 \big(1 + \alpha_{\mathrm{turb}} (\max\{0, N(t) - K\})^\gamma \big).
\end{equation}
\end{definition}

\section{RSVP Relations}
\label{sec:rsvp-relations}
\begin{description}
  \item[Scalar Density] \(\PhiS\) is contextual density; anomalies (e.g., fire-spectrum colors) drive salience (\cref{eq:raw-salience}).
  \item[Vector Flow] \(\vvec\) encodes attention; overuse causes turbulence (\cref{eq:capacity}).
  \item[Entropy] \(\Sent\) tracks signal decay; sparse cues preserve salience (\cref{eq:semiotic-entropy}).
  \item[Wabi-Sabi] Muted cues maintain \(\PhiS\) (\cref{eq:ws-reg}).
\end{description}

\section{Application}
\label{sec:rsvp-application}
RSVP prioritizes sparse eco-cues and flexible paths, countering enshittification \citep{doctorow2022}.

\chapter{Conjunction vs. Believability}
\label{app:conjunction}

This appendix formalizes why friendly interfaces feel trustworthy despite high costs, supporting \cref{ch:illusion,ch:aesthetic}. Readers need probability theory basics.

\section{Conjunction Lowers Probability}
\label{sec:conj-prob}
For interface features \(E_1, \dots, E_n\):
\begin{equation}
\label{eq:conj-prob}
P(E_1 \land \cdots \land E_n) \leq P(E_1 \land \cdots \land E_k), \quad k < n.
\end{equation}
Under independence:
\begin{equation}
\label{eq:conj-indep}
P(\bigwedge_{i=1}^n E_i) = \prod_{i=1}^n P(E_i).
\end{equation}

\section{Believability Functional}
\label{sec:conj-believability}
For type \(T\) (e.g., \textquotedblleft trustworthy interface\textquotedblright):
\begin{equation}
\label{eq:believability-functional}
\mathcal{B}(E_{1:n}) = \sum_{i=1}^n \log \frac{P(E_i \mid T)}{P(E_i \mid \neg T)}.
\end{equation}
Details increase \(\mathcal{B}\), despite lower \(P\) \citep{tversky1983}.

\section{Worked Example}
\label{sec:conj-example}
Hypotheses: \(H_1\): \textquotedblleft Interface is functional\textquotedblright; \(H_2\): \textquotedblleft Functional and user-friendly\textquotedblright. Base rates: \(P(H_1) = 0.8\), \(P(H_2) = 0.4\). Features: \(E_1 = \text{smooth animations}\), \(E_2 = \text{intuitive layout}\), \(E_3 = \text{personalized prompts}\). Likelihoods yield \(\mathcal{B} \approx 4.25\), making \(H_2\) feel more plausible.

\section{Design Implications}
\label{sec:conj-implications}
Cap detail density, expose costs, and use sparse cues to align \(\mathcal{B}\) with \(P\).

\section{Summary}
The conjunction fallacy explains trust in friendly interfaces. \Cref{ch:illusion} applies this.

% ---- Bibliography ----
\clearpage
\bibliography{references}

\end{document}
